% Chapter Template

\chapter{Introduction} % Main chapter title

\label{ChapterIntroduction}

%----------------------------------------------------------------------------------------
%	SECTION 1
%----------------------------------------------------------------------------------------

\section{Introduction to Cryptography}
The terms \emph{Cryptography} (from the Greek \emph{krypt\`os} (secret) and \emph{graphein} (writing)) and \emph{Cryptanalysis}, denotes two branches of a science named \emph{Cryptology}, or \emph{science of secret}. Cryptography initially refers to the art of \emph{encrypting} messages, which means writing meaningful messages in such a way to appear nonsense to anyone unaware of the encryption process. In general, cryptography aims to construct protocols to secure communication, while cryptanalysis studies the resistance of cryptographic techniques, developing \emph{attacks} to break the cryptosystems' security claims. These two complementary domains evolve in parallels, since the evolution of attack techniques allows conceiving more resistant cryptographic algorithms, and inversely the resistance of such algorithms requires the conception of more sophisticated attacks.\\

The art of cryptography is very ancient, probably as ancient as the language, but only the development of information technology made cryptology take the shape of a proper science, sometimes referred to as \emph{Modern cryptology}. The last be seen as a branch of different disciplines, such as applied mathematics, computer science, electrical engineering, and communication science. Modern cryptosystems exploit algorithms based on mathematical tools and are implemented as computer programs, or electronic circuits. Their goal is to provide security functionality for communications that use \emph{insecure channels}, for example the internet. In particular, modern cryptosystems are designed in order to ensure at least one of the four following information security properties:
\begin{itemize}
\item[a.] \emph{confidentiality}: the transmitted message must be readable only by a chosen pool of authorized entities;
\item[b.] \emph{authenticity}: the receiver can verify the identity of the sender of a message;
\item[c.] \emph{non-repudiation}: the sender of a message cannot deny having sent the message afterwards;
\item[d.] \emph{data integrity}: the receiver can be convinced that the message has not been corrupted during the transmission.


\end{itemize} 

Two branches of cryptography may be distinguished: the \emph{symmetric cryptography} and the \emph{asymmetric cryptography}. 
\subsection{Secret-Key Cryptography}
\subsection{Public-Key Cryptography}


%----------------------------------------------------------------------------------------
%	SECTION 2
%----------------------------------------------------------------------------------------
\section{Embedded Cryptography and Secure Hardware}
\subsection{The Example of the Smart Card}
\subsection{Certification of a Secure Hardware}
\subsection{Modern More Complex Devices to Certify}
\subsection{Embedded Cryptography Vulnerabilities}

