% Chapter Template

\chapter{Introduction} % Main chapter title

\label{ChapterIntroduction}

%----------------------------------------------------------------------------------------
%	SECTION 1
%----------------------------------------------------------------------------------------

\section{Introduction to Cryptography}
The terms \emph{Cryptography} (from the Greek \emph{krypt\`os} (secret) and \emph{graphein} (writing)) and \emph{Cryptanalysis}, denotes two branches of a science named \emph{Cryptology}, or \emph{science of the secret}. Cryptography initially refers to the art of \emph{encrypting} messages, which means writing meaningful messages in such a way to appear nonsense to anyone unaware of the encryption process. In general, cryptography aims to construct protocols to secure communication, while cryptanalysis studies the resistance of cryptographic techniques, developing \emph{attacks} to break the cryptosystems' security claims. These two complementary domains evolve in parallels, since the evolution of attack techniques allows conceiving more resistant cryptographic algorithms, and inversely the resistance of such algorithms requires the conception of more sophisticated attacks.\\

The art of cryptography is very ancient, probably as ancient as the language, but only the development of information technology made cryptology take the shape of a proper science, sometimes referred to as \emph{Modern cryptology}. The last be seen as a branch of different disciplines, such as applied mathematics, computer science, electrical engineering, and communication science. Modern cryptosystems exploit algorithms based on mathematical tools and are implemented as computer programs, or electronic circuits. Their goal is to provide security functionality for communications that use \emph{insecure channels}, for example the internet. In particular, modern cryptosystems are designed in order to ensure at least one of the four following information security properties:
\begin{itemize}
\item[a.] \emph{confidentiality}: the transmitted message must be readable only by a chosen pool of authorized entities;
\item[b.] \emph{authenticity}: the receiver can verify the identity of the sender of a message;
\item[c.] \emph{non-repudiation}: the sender of a message cannot deny having sent the message afterwards;
\item[d.] \emph{data integrity}: the receiver can be convinced that the message has not been corrupted during the transmission.


\end{itemize} 

Two branches of cryptography may be distinguished: the \emph{symmetric cryptography} and the \emph{asymmetric cryptography}. The first one historically appeared before and is based on the hypothesis that the two communicating entities share a common secret, or private key; for this reason this is also called \emph{secret key cryptography}. The second one, introduced around 1970, allows any entity to encrypt a message in such a way that only a unique chosen other entity could decrypt it; this is also called \emph{public key cryptography}. \\

A general principle in cryptography, nowadays widely accepted by cryptography researchers, is the one given by Kerckhoff in in 19th century: it states that cryptosystems should be secure even if everything about the system, except the key, is public knowledge. Following this principle, today many industrials and governmental agencies exploit for their security services cryptosystems based over standardized algorithms. Such algorithms are of public domain, thus have been tested and tried to be broken by a large amount of people, before, during and after the standardization process. Resistance to many attempts of attacks is actually the strengths of standard algorithms.\\

In the following part of this section a description of the two standard cryptographic primitives, \emph{i.e.} building block algorithms used to build cryptographic protocols, that will be used in this thesis will; a symmetric one, the AES, and an asymmetric one, the RSA. 
\subsection{Description of AES}
The \emph{Advanced Encryption Standard} (AES) has been standardized in 2001 by the United States governmental agency \emph{National Institute of Standards and Technology} (NIST) through the \emph{Federal Information
Processing Standards Publication 197 } (FIPS PUB 197) \cite{nist197}. It is a symmetric \emph{block cipher}, \emph{i.e.} an algorithm operating on fixed-length groups of bits.\footnote{in contrast with \emph{stream ciphers}, which operate over a single plaintext bit at time} The AES operates on blocks of 128 bits of plaintext, and can use keys of size 128, 192 or 256 bits. The encryption is done by rounds. The number of executed rounds depends on the key size (10 rounds for 128 bits, 12 for 192 and 14 pour 256). The basic unit for processing in the AES algorithm is a byte. For AES internal operations, bytes are arranged on a two-dimensional array of bytes called the \emph{state}, denoted $s$. Such a state has 4 rows and 4 columns, thus contains 16 bytes. The byte lying at the $i$-th row, $j$-th column of $s$ will be denoted by $s_{i,j}$ for $i,j\in\{0,1,2,3\}$. The 16 input bytes and the 16 output bytes are indexed column-wise as shown in Fig.~\ref{fig:AES_state}. Each element $s_{i,j}$ of a state is mathematically seen as an element of the \emph{Rjindael finite field}, defined as $GF(2^8) = \mathbb{Z}/{2\mathbb{Z}[X]}/P(X)$ where $P(X) = X^8 + X^4 + X^3 + X + 1$. Five functions are performed during the AES, named KeySchedule, AddRoundKey, SubBytes, ShiftRow and MixColumn. At high level the AES algorithm is described hereafter:
\begin{itemize}
\item[\textbf{Key Expansion:}]  derivation of round keys from secret key through the KeySchedule function
\item[\textbf{Round 0:} ] 
\begin{itemize}
\item[] AddRoundKey
\end{itemize}
\item[\textbf{Rounds 1 to penultimate:}] 
\begin{itemize}
\item[] SubBytes
\item[] ShiftRow
\item[] MixColumn
\item[] AddRoundKey
\end{itemize}
\item[\textbf{Last Round:}] 
\begin{itemize}
\item[] SubBytes
\item[] ShiftRow
\item[] AddRoundKey
\end{itemize}
\end{itemize}

\begin{figure}
\includegraphics[width = \textwidth]{../Figures/FISP_AES/state.png} 
\caption[State array input and output.]{State array input and output. Source: \cite{nist197}.}\label{fig:AES_state}
\end{figure}

A description of the five functions is provided hereafter.

\subsubsection*{KeySchedule}
The key round of the initial round of AES coincides with the secret encryption key $\boldsymbol{K} = (k_{0,0},k_{0,1},\dots,k_{0,3}, k_{1,0},\dots,k_{1,3},\dots,k_{3,3})$. The $i$-th round key is given by 
\begin{equation*}
\boldsymbol{K_i} = (k_{4i,0},k_{4i,1},\dots,k_{4i,3}, k_{4i+1,0},\dots,k_{4i+1,3},\dots,k_{4i+3,3}),
\end{equation*}
where, for $i>3$
\begin{equation*}
\begin{cases}
k_{a,b} = k_{a-4,b}\oplus k_{a-1,b} & \mbox{if } a \not\equiv 0 \mathrm{ mod } 4\\
k_{a,b} = k_{a-4,b}\oplus \Sbox(k_{a-1,(b+1) \mathrm{mod } 4} \oplus \mathrm{Rcon}(a) & \mbox{if } a \equiv 0 \mathrm{ mod } 4 \mathrm{,}
\end{cases}
\end{equation*}

where $Rcon(a) = 2^{a-1}$ in the Rjindael finite field.\footnote{where $2=(00000010)_2$ is represented by the polynomial $x$}
\subsection{Description of RSA}


%----------------------------------------------------------------------------------------
%	SECTION 2
%----------------------------------------------------------------------------------------
\section{Secure Components and Embedded Cryptography}
As we have seen in the previous section, modern cryptography proposes solutions to secure communications that asks for electronic computations and repose their security over some secret keys. Keys are represented as long bit strings, impossible to be memorized by users. Keys need then to be stored in a secure medium, and never delivered in clear over insecure channels.


\subsection{The Example of the Smart Card}

\subsection{Certification of a Secure Hardware}
\subsection{Embedded Cryptography Vulnerabilities}
parla anche degli attachi per perturbazione (tesi alexandre)


