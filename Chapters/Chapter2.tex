\chapter{Introduction to Side-Channel Attacks} % Main chapter title

\label{ChapterIntroductionSCA}

%----------------------------------------------------------------------------------------
%	SECTION 1
%----------------------------------------------------------------------------------------

\section{Introduction to Side-Channel Attacks}

\subsection{Terminology and Generalities}
\subsubsection{Target and Leakage Model}
\begin{itemize}
\item The goal of an SCA is to retrieve a secret part of a cryptographic algorithm, typically the secret key.
\item The underlying hypothesis of a SCA is that some information about internal variables (or parts of internal variables) of the implemented algorithm leak during its execution through some observable \emph{side} channels. Such leakages are collectable in the form of signals, observing such channels. 
\item No matter the size of an algorithm inputs, outputs and internal variables, an hardware implementation  always operates over variables of a bounded size. Such a bound size depends on the hardware architecture. For example, in an 8-bit architecture an RSA with 1024-bit-sized key, modulo and plaintext will be somehow implemented as multiple operations over 8-bit blocks of observable data. 
\item This fact allows an attacker to apply the \emph{divide-and-conquer} strategy: if his goal is retrieving the full 128-bit AES key or 1024-bit RSA key, he will smartly divide his problem in retrieving small parts of such keys at time, namely some \emph{subkeys}.  
\item An attack = find the right association between a trace (or a set of traces) and the value assumed by a target variable during the acquisition of such trace/traces. 
\item the target variable can directly be a target subkey, ....simple/advanced
\item model HW but also Z = HW(...)

\end{itemize}

\subsubsection{Points of Interest}
\subsubsection{Simple vs Advanced SCAs}
\subsubsection{Vertical vs Horizontal SCAs}
\subsubsection{Profiled vs Non-Profiled SCAs}
\subsubsection{Side-Channel Algebraic Attacks}
\subsubsection{Distinguishers}
\subsubsection{SCA Metrics}



%----------------------------------------------------------------------------------------
%	SECTION 2
%----------------------------------------------------------------------------------------
\section{Main Side-Channel Countermeasures}
\subsection{Random Delays and Jitter}
\subsection{Shuffling}
\subsection{Masking}



%----------------------------------------------------------------------------------------
%	SECTION 3
%----------------------------------------------------------------------------------------
\section{Higher-Order Attacks}
\subsection{Higher-Order Moments Analysis and Combining Functions}
\subsection{Profiling Higher-Order Attacks}
\subsubsection{Profiling with Masks Knowledge}
\subsubsection{Profiling without Masks Knowledge}

