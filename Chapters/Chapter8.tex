% Chapter Template

\chapter{Convolutional Neural Networks} % Main chapter title
\label{ChapterCNN}

% TODO: manca universal approximation theorem (da pagg 194 e seguenti del deeplearningbook)

In this chapter we explore a new strategy to perform profiling SCAs, addressing the misalignment issue and endorsing the Deep Learning (DL) paradigm. To this aim we revisit the publication \cite{DBLP:conf/ches/CagliDP17} appeared at CHES 2017.\\

\section{Introduction}
Often Machine Learning approach declines in multiple pre-processing phases such as data alignment, feature selections or dimensionality reduction, followed by a final model optimisation. This is the case for the SCAs routines that we considered in previous chapters: a dimensionality reduction pre-processing $\extract$ is previously learned an applied, followed by a Gaussian template parameters estimation that are used to construct the model which solves the side-channel classification task.  Deep Learning is a branch of Machine Learning whose characteristic is to avoid any preliminary pre-processing step from the model construction work-flow. For example, in Deep Learning the data dimensionality reduction is not necessarily explicitly performed by a distinct learned  function $\extract$. Thus, DL models are in charge of directly and implicitly extracting interesting features and of estimating the opportune model to solve the task. The model is searched in a family of models that are composed by a cascade of parametrisable layers, which may be optimised in a single global learning process. Such models are called \emph{Artificial Neural Network}, or simply \emph{Neural Networks} (NNs). \\

By construction, NNs are the ML answer to the drawback of work-flows we analysed in previous chapters and discussed as \emph{two-phased approach drawback} in last section of Chapter~\ref{ChapterKernel}. Actually, NNs are answers to other drawbacks pointed out in the same section. \\
In particular they are not memory-based. It implies that, after the training phase whose computational complexity is influenced by the size of the training set, they do not need to access the training set any more. By consequence, the obtained model is in general faster in processing new data, with respect to those obtained \via kernel machines: the form of the obtained model (\ie the so-called \emph{architecture}) is independent from the size of the training set. This property, together with other computational optimizations they allow that will be discussed later, make NNs more easily scalable for huge training sets.\\
Finally, we pointed out as drawback of techniques analysed in previous chapters their weakness to trace misalignment. In DL literature, a family of models called \emph{Convolutional Neural Networks} (CNNs) has been developed to treat difficulties usually met in image processing as misalignments, scaling, rotations, \dots We claim in this chapter, and verify through various experiments, that such CNNs provide an attack strategy that is robust to misalignment countermeasure. Attacks proposed in this chapter are performed against non-masked implementation. Nevertheless, since NNs are in general non-linear models, they naturally well-fit also the higher-order attack context, as discussed in \cite{maghrebi2016breaking} and \cite{DLwhitepaper}.\\
The CNNs are a generalisation of the simplest NN architecture, named \emph{Feed-forward Neural Networks} or \emph{Multi-Layer Perceptron} (MLP). The name \emph{feedforward} is due to the fact that for this kind of models information flows from  the input to the output, through the intermediate computations, without any feedback connection in which outputs of the model are fed back into itself. This is in opposition to the so-called \emph{Recurrent Neural Network} structures. A description of Neural Networks models, and classification-oriented MLP in particular, is provided in next section.

% The answers to the 3 drawbacks of previous chapter (misalignment, memory complexity and actual number of params, two phases approach)
%"In practice, however, it is often worth investing substantial computational resources during the training phase in order to obtain a compact model that is fast at processing new data" (Bishop intro chap 5)

% RICICLATO DALL'EX CAPITOLO 1
%Kernel techniques like the KDA  are as well inherited from Machine Learning domain and consist in strategies that allow to build interesting extensions of many algorithms. One of their characteristics, that turns to be a drawback to apply them in SCA context is that are memory-based: the entire set of profiling traces, \ie those acquired by observing the open samples, has to be stored and accessed in the attack phase. In this sense they are highly memory-consuming, and quite slow to apply: they do not scale well in presence of huge profiling trace sets as those that are often necessary to perform profiling SCAs. In contrast to them, models provided by Neural Networks (NNs) are Machine Learning solutions that are known to be easily scalable to huge datasets and not memory-based. We decided to explore such an approach and pointed out that it not only provided solutions to tackle such a computational performance drawback.


\section{Neural Networks and Multi-Layer Perceptrons}
As anticipated in Chapters~\ref{ChapterIntroductionSCA} and \ref{ChapterIntroML}, we are interested in the NNs' solutions for the classification task, as we remarked a strong analogy between profiling SCAs and classical ML classification task. We recall from Chapters~\ref{ChapterIntroML} that for the classification task, the learning algorithm is asked to construct a function $\MLmodel\colon \mathbb{R}^\traceLength \rightarrow \{0,1\}^{\numClasses}$, where elements of $\sensVarSet$, \ie the set of classes, are here expressed \via the \emph{one-hot encoding}. The output of such a function is said to be \emph{categorical}, \ie $\sensVarSet$ is a discrete finite set. A variant of the classification task consists in finding a function $\MLmodel\colon \mathbb{R}^\traceLength \rightarrow [0,1]^{\numClasses}$ defining a probability distribution over classes. Often for this task, NNs are exploited to create discriminative models, \ie models that directly approximate the latter function $\MLmodel$ that describes the posterior conditional probability of a label given the observed trace. This is in opposition to the Template Attack we exploited in previous chapters, that is based over the construction of generative models, \ie the approximation of the \emph{templates}, which coincide with the conditional probabilities of the traces given a label, as described in Sec.~\ref{sec:TA}. \\

Using NNs the function $\MLmodel$ is obtained by combining several simpler functions, called \emph{layers}. An NN has an \emph{input layer} (the identity over the input datum $\vLeakVec$), an output layer (the last function) and all other layers are called \emph{hidden} layers.  The output of $\MLmodel$ is a $\numClasses$-sized vector $\vNNOutput$ of scores for the $\numClasses$ labels. Such a vector might or not represent the approximation of a probability distribution. The nature of the NN's  layers, their number and their dimension in particular, is called the \emph{architecture} of the NN. All the parameters that define an architecture, together with some other parameters that govern the training phase, are its \emph{hyper-parameters}. The so-called \emph{neurons}, that give the name to the NNs, are the computational units of the network and essentially process a scalar product between the coordinates of its input and a vector of  \emph{trainable weights} (or simply \emph{weights}) that have to be \emph{trained}. Each layer processes some neurons and the outputs of the neuron evaluations will form new input vectors for the subsequent layer. \\

The {\em Multi-Layer Perceptrons} (MLPs) are a family of NN's architectures, associated with a function $\MLmodel$  that is composed of multiple linear functions and some non-linear functions, called {\em activations}. \\

We can express a typical classification-oriented MLP by the following equation:
\begin{equation}\label{eq:MLP}
\MLmodel(\vLeakVec) = \softmax\circ\lambda_n\circ\sigma_{n-1}\circ\lambda_{n-1}\circ\dots\circ \lambda_1(\vLeakVec)=\yyy \mbox{ ,}
\end{equation}
where:
\begin{itemize}
\item The $\lambda_i$ functions are typically the so-called \emph{Fully-Connected} (FC) layers and are expressible as affine functions: denoting $\vLeakVec\in\mathbb{R}^D$ the input of an FC, its output is given by $\textbf{A}\vLeakVec + \vec{b}$, being $\textbf{A}\in\mathbb{R}^{D\times C}$ a matrix of weights and $\vec{b}\in\mathbb{R}^C$ a vector of biases. These weights and biases are the trainable weights of the FC layer. They are called \emph{Fully-Connected} because each $i$-th input coordinate is \emph{connected} to each $j$-th output via the $\textbf{A}[i,j]$ weight. FC layers can be seen as a special case of the linear layers in general feedforward networks, in which not all the connections are present. The absence of some $(i,j)$-th connections can be formalized as a constraint for the matrix $\textbf{A}$ consisting in forcing to $0$ its $(i,j)$-th coordinates.

\item  The $\sigma_i$ are the so-called \emph{activation functions} (ACT): an activation function is a non-linear real function that is applied independently to each coordinate of its input. In general it does not depend on trainable weights. We denote them by $\sigma$ since in general they are functions similar to the \emph{logistic sigmoid} introduced in \ref{example:LDA} and denoted by $\sigma$ as well: indeed historically sigmoidal functions where recommended, \ie real-valued, bounded, monotonic, and differentiable functions with a non-negative first derivative. Nevertheless, the recommended function in modern neural network literature is the so-called \emph{Rectified Linear Unit} (ReLU), introduced by \cite{nair2010rectified} and defined as $\mathrm{ReLU}(\vLeakVec)[i] = \max(0,\vLeakVec[i])$. Even it this function is not sigmoidal, not being bounded, nor differentiable, near transformation the fact of being a non-linear transformation but still piecewise linear, allows to preserve many of the properties that make linear models easy to optimize with gradient-
function. 

\item $\softmax$ is the \emph{softmax}\footnote{To prevent underflow, the log-softmax is usually preferred if several classification outputs must be combined.} function (SOFT), already introduced in \ref{example:LDA}: $\softmax(\vLeakVec)[i] = \frac{e^{\vLeakVec[i]}}{\sum_{j}e^{\vLeakVec[j]}}$.
\end{itemize}
 
The choice of the softmax function as last layer of a neural network classifier is the most common one. It allows the model $MLmodel$ to be interpreted as a generalisation of the binary classifier described in \eqref{eq:binary_linear_classifier}, where the softmax takes the place of the sigmoid to make the model multi-class and the linear argument is substituted by all previous layers of $\MLmodel$. The previous layers takes in charge the feature extraction and preprocessing and are supposed to predict unnormalised log probabilities \eqref{eq:softmax_entries}. The role of the \emph{softmax} is thus to renormalise such output scores in such a way that they define a probability distribution $\MLmodel(\vLeakVec) \approx \pdf_{\given{\sensRandVar}{\vaLeakVec=\vLeakVec}}$. 

\section{Loss Function and Training}
The weights of an NN are tuned during a training phase. They are first initialized with random values and are afterwards updated  \via an
iterative approach which locally applies the (Stochastic) Gradient Descent
algorithm \cite{Goodfellow-et-al-2016} to minimize a loss function
quantifying the \emph{classification error} of the function
$\MLmodel(\vaLeakVec)$ over a training set which is a part of the profiling set. 


\subsection{Training}
The training is said to be \emph{full batch learning} if
the full training database is processed before one update. At the opposite, if
a single training input is processed at a time then the approach is named
\emph{stochastic}. In practice, one often prefers to follow an approach in
between, called \emph{mini-batch learning}, and to use small \emph{batch} (aka
group) of training inputs at a time during the learning. In this case a step of the training consists in: 
\begin{itemize}
\item selecting a batch of training traces $(\vLeakVec_i, \vec{\sensVar_i})_{i\in I}$ chosen in random order (here $I$ is a random set of indexes),
\item computing the outputs, or scores, of the current model function for the batch inputs $(\vNNOutput_i = \MLmodel(\vLeakVec_i))_{i\in I}$, 
\item evaluating the loss function,
\item compute the partial derivatives of the loss function with respect to each trainable weight (this is done through a method called \emph{backpropagation} \cite{LeCun2012}),
\item updating trainable parameters by subtracting from each a small multiple of the loss gradient (the used multiple is called \emph{learning rate}).
\end{itemize}  

The size of the mini-batch is generally
driven by several efficiency/accuracy factors which are \eg discussed in
\cite{GBC16} (\eg optimal use of the multi-core architectures, parallelisation
with GPUs, trade-off between regularisation effect and stability, etc.). \\

An iteration over all the training datasets during the Stochastic Gradient Descent is called an \emph{epoch}.
The number of epochs is an important hyper-parameter to tune because
small values may lead to underfitting and high values may lead to overfitting. In our experiments we chose to a so-called \emph{early stopping} approach in order to avoid the need of a prior tuning of the number of epochs. It will be described in \ref{XXXXXX}.\\

\subsection{Cross-Entropy}
The cross-entropy
metric is a classical (and often by default) tool to construct the loss function in a classification-oriented NNs \cite{LCH05,Goodfellow-et-al-2016}. It is smooth and
decomposable, and therefore amenable to optimization with standard
gradient-based methods. Before providing the definition of cross-entropy in \eqref{eq:crossentropy}, we precise the chosen form for the loss function. Given a batch of training data $(\vLeakVec_i)_{i\in I}$ and their scores by the current model $(\vNNOutput_i)_{i\in I}$, the loss function is defined as the following averaged value:

\begin{equation}\label{eq:lossfunction}
\mathcal{L} = \frac{1}{\lvert I \rvert} \sum_{i\in \I} -\sum_{t=1}^{|\sensVarSet|}\vec{\sensVar_i}[t]\log{\vNNOutput_i[t]}
\end{equation}   

There are two ways to interpret such a choice. 

\begin{itemize}
\item First, recalling that $\vNNOutput_i$ may be interpreted as an estimation of the conditional probability $\prob[\given{\sensRandVar}{\vaLeakVec=\vLeakVec_i}]$, the principle of maximum-likelihood suggests to drive the training in such a way that for such an estimate the probability of the correct label $\vec{\sensVar_i}$ is as high as possible. Thus, if we suppose that $\vec{\sensVar_i} = \vec{\sensVar}^j$, we want to maximize $\vNNOutput_i[j]$ (or equivalently to minimize $-\log{\vNNOutput_i[j]}$).\footnote{We remark that thanks to the softmax function used as last network layer, each coordinate of $\vNNOutput_i$ is always strictly positive.} It may be observed that such a log-likelihood rewrites as 
\begin{equation}\label{eq:log_lik}
-\log{\vNNOutput_i[j]} = -\sum_{t=1}^{|\sensVarSet|}\vec{\sensVar_i}[t]\log{\vNNOutput_i[t]} \mbox{ ,}
\end{equation}
which equals to the quantity averaged in \eqref{eq:lossfunction}.
\item The second interpretation of the chosen loss function is linked to fact that it actually represents the average of  the cross-entropy of pairs of well-chosen probability mass functions. Indeed we observe that the encoding $\vec{\sensVar}_i = (0,\ldots , 0,\underbrace{1}_{j},0,\dots,0)$ is interpretable as the pmf of $\given{\sensRandVar}{\sensRandVar = \sensVar_i}$, which corresponds to the exact probability density we want the network to approximate. Informally speaking, the cross-entropy between two probability distributions, in our case given by $\vec{\sensVar_i}$ and $\vNNOutput_i$ gives a measure of dissimilarity between them, and is defined as follows:
\begin{equation}\label{eq:crossentropy}
\entropy(\vec{\sensVar_i}, \vNNOutput_i) = \entropy(\vec{\sensVar_i}) + D_{KL}(\vec{\sensVar_i} || \vNNOutput_i) = \esper_{\vec{\sensVar_i}}[-\log{\vNNOutput_i}] = -\sum_{t=1}^{|\sensVarSet|}\vec{\sensVar_i}[t]\log{\vNNOutput_i[t]} \mbox{ ,}
\end{equation}
where $\entropy$ denotes the entropy and $D_{KL}$ denotes the Kullback-Leibler divergence \cite{christopher2006pattern}. Thus, this is an information-theoretic notion, that comes out to be equivalent to the negative log-likelihood formula given by \eqref{eq:log_lik}. 
\end{itemize}
In conclusion, depending on the point of view, minimizing the loss function \eqref{eq:log_lik}, which is the cross-entropy averaged over the traces contained in a batch, corresponds to maximize the likelihood of the right label, or to minimize the dissimilarity between the network estimation of a distribution and the right distribution that we want it to approximate. 
We chose the loss function \eqref{eq:lossfunction} for our experiments. However, other metrics may be investigated and can
potentially lead to better results \cite{MHK10,SSZU15}. \\

As justified in \ref{sec:validation}, we will for our experiments divide the side-channel profiling set into two sets: the training one and the validation one. The training set is the one processed by batch and used to update the NN's parameters. The validation set is exploited in general at the end of each epoch to monitor the training, and in particular to watch over the incoming of an overfitting phenomenon. Cross-validation has not been performed. We will use a side-channel attack set as test set as well, to finally evaluate the performance of the trained model.


%SEI ARRIVATA QUI
\section{Performance }
\subsubsection{The accuracy} is the most common metric to both monitor and evaluate an NN. It is defined as the successful classification rate reached over a dataset. The {\em training accuracy}, the \emph{validation accuracy} and the \emph{test accuracy} are the successful classification rates achieved respectively over the training, the validation and the test sets. At the end of each epoch it is useful to compute and to compare the training accuracy and the validation accuracy. For some trained models we will measure in this paper (see \emph{e.g.} Table~\ref{tab:res_CW_shift}) the following two additional quantities: 
\begin{itemize}
\item the \emph{maximal training accuracy}, corresponding to the maximum of the training accuracies computed at the end of each epoch
\item the \emph{maximal validation accuracy}, corresponding to the maximum of the validation accuracies computed at the end of each epoch.
\end{itemize}
In addition to the two quantities above, we will also evaluate the performances of our trained model, by computing a \emph{test accuracy}. Sometimes it is useful to complete this evaluation by looking at the so-called \emph{confusion matrix} (see the bottom part of Fig. \ref{fig:CW_shift_history}). Indeed the latter matrix enables, in case of misclassification, for the identification of the classes which are confused. The confusion matrix corresponds to the distribution over the couples \emph{(true label, predicted label)} directly deduced from the results of the classification on the test set. A test accuracy of $100\%$ corresponds to a diagonal confusion matrix.\\

\subsubsection{On the Need to also Consider the Guessing Entropy.} The accuracy metric is perfectly adapted to the machine learning classification problem, but corresponds in side-channel language to the success rate of a Simple Attack, \emph{i.e.} an attack where a single attack trace is available. When the attacker can acquire several traces for varying plaintexts, the accuracy metric is not sufficient alone to evaluate the attack performance.
Indeed such a metric only takes into account the label corresponding to the maximal score and does not consider the other ones, whereas an SCA through~\eqref{eq:NN_SCA} does (and therefore exploits the full information).

To take this remark into account, we will always associate the test accuracy to a side-channel metric defined as the minimal number  $N^\star$ of side-channel traces that makes the \emph{guessing entropy} (the average rank of the right key candidate) be permanently equal to 1 (see \emph{e.g.} Table~\ref{tab:res_CW_shift}). We will estimate such a guessing entropy through 10 independent attacks. \\

As we will see in the sections dedicated to our attack experiments, applying Machine Learning in a context where at the same time (1) the model to recover is complex and (2) the amount of exploitable measurements for the training is limited, may be ineffective due to some overfitting phenomena.

\subsubsection{Overfitting.} Often the training accuracy is higher than the validation one. When the gap between the two accuracies is excessive, we assist to the \emph{overfitting} phenomenon. It means that the NN is using its weights to \emph{learn by heart} the training set instead of detecting significant discriminative features. For this reason its performances are poor over the validation set, which is new to it. Overfitting occurs when an NN is excessively complex, \emph{i.e.} when it is able to express an excessively large family of functions. In order to keep the NN as complex as wished and hence limiting the overfitting, some \emph{regularization} techniques can be applied. For example, in this paper we will propose the use of the
\emph{Data Augmentation} (DA)~\cite{simard2003best} that consists in artificially adding observations to the training set. Moreover we will take advantage of the \emph{early-stopping} technique~\cite{Prechelt2012} that consists in well choosing a stop condition based on the validation accuracy or on the validation loss (\emph{i.e.} the value taken by the loss function over the validation set).



Several extensions and variants of the Stochastic Gradient Descent have been proposed in the context of DL. These variants, called \emph{optimizers}, aim to adapt the \emph{learning rate} (the step size) of the Gradient Descent during the training process. More details about the
specification of neural networks will be given in the dedicated sections \ref{sec:FNN} and
\ref{sec:cnn}, but we will not go further on the optimization approaches and the
interested reader may refer to \cite{Goodfellow-et-al-2016}. 

\section{Attack Strategy with an MLP}
Looking back to the Template Attack strategy described in \ref{sec:TA}, we observe that the main difference between such a strategy and the exploitation of an MLP constructed as just described is the fact the the TA bases on a generative model, while MLPs are used to construct discriminative ones. Indeed, in TA the templates \eqref{eq:class-conditional} are priorly estimated, while with an MLP $\MLmodel(\vLeakVec) \approx \pdf_{\given{\sensRandVar}{\vaLeakVec=\vLeakVec}}$ one approximates directly posterior probabilities \eqref{eq:a-posteriori}. Once this approximation is done, the attack strategy proceeds in the same way for both approaches. The attacker acquires the new attack traces, that he only can associate to the public parameter $\publicParRandVar$, obtaining couples  $(\vLeakVec_i, \publicParVar_i)_{i=1, \dots , \nbAttackTraces}$. Then he makes key hypothesis $\keyVar \in \keyVarSet$ and, making the assumption that each acquisition is an independent observation of $\vaLeakVec$, he associates to each hypothesis $\keyVar \in \keyVarSet$ a score $d_\keyVar$ given by \eqref{eq:joint_distr}, that in terms of MLP model $\MLmodel$ rewrites  as:

\begin{equation}\label{eq:NN_SCA}
d_{\keyVar} = \prod_{i=1}^{\nbAttackTraces} \MLmodel(\vLeakVec_i)[\sensFunction(\keyVar,\publicParVar_i)] \mbox{ .}
\end{equation}

Finally, the best key candidate $\hat{\keyVar}$ is the one maximizing such a joint probability, as in \eqref{eq:max_classifier}
\begin{equation}
\hat{\keyVar} = \argmax_{\keyVar} d_{\keyVar} \mbox{ .}
\end{equation}





%----------------------------------------------------------------------------------------
%	SECTION 2
%----------------------------------------------------------------------------------------

%\section{Misalignment of Side-Channel Traces}

%\subsection{The Necessity and the Risks of Applying Realignment Techniques}
%\subsection{Analogy with Image Recognition Issues}

%----------------------------------------------------------------------------------------
%	SECTION 3
%----------------------------------------------------------------------------------------

\section{Shift-Invariance and Convolutional Neural Network}

%----------------------------------------------------------------------------------------
%	SECTION 4
%----------------------------------------------------------------------------------------

\section{Data Augmentation for Misaligned Side-Channel Traces}
%\todo{cita o qui o all'inizio il paper di CARDIS 2017 sulla trace augmentation}
%----------------------------------------------------------------------------------------
%	SECTION 5
%----------------------------------------------------------------------------------------

\section{Experiments against Software Countermeasures}


%----------------------------------------------------------------------------------------
%	SECTION 6
%----------------------------------------------------------------------------------------

\section{Experiments against Artificial Hardware Countermeasures}\label{sec:hardware}%label citato in appendice artifical jitter

%----------------------------------------------------------------------------------------
%	SECTION 7
%----------------------------------------------------------------------------------------

\section{Experiments against Real-Case Hardware Countermeasures}