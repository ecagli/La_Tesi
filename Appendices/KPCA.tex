\chapter{Kernel PCA construction}\label{app:KPCA}

Suppose that we want to perform PCA in the image space of a function $\Phi$ that is associated to a given kernel function $K$. The kernel version for PCA has been presented in \cite{scholkopf1998nonlinear}, where all details can be found; as we said, the important step consists in expressing the operations needed for the PCA procedure in terms of the dot products between the mapped data.\\

Let us assume that data are centered in the feature space, {\em i.e.} $\sum_{i=1,\dots,\nbProfilingTraces}\Phi(\vLeakVec_{i})=0$.\footnote{Such a condition is not hard to achieve, even without explicitly pass through the feature space: it suffices substituting the kernel matrix $\kernelMatrix$ by the matrix $\tilde{\kernelMatrix} = \kernelMatrix - \boldsymbol{1}_{\nbProfilingTraces}\kernelMatrix - \kernelMatrix\boldsymbol{1}_{\nbProfilingTraces} + \boldsymbol{1}_{\nbProfilingTraces}\kernelMatrix\boldsymbol{1}_{\nbProfilingTraces}$, where $\boldsymbol{1}_{\nbProfilingTraces}$ denotes the matrix with each entry equal to $\frac{1}{\nbProfilingTraces}$. The same kind of matrix has to be computed in projecting phase, using the test data.} In this way the empirical covariance matrix $\covmat^\Phi$ of data in the feature space is given by:
\begin{equation}
\covmat^\Phi = \frac{1}{\nbProfilingTraces} \sum_{i=1}^{\nbProfilingTraces}\Phi(\vLeakVec_{i})\Phi(\vLeakVec_{i})^\intercal \mbox{ .}
\end{equation} 
We want to find eigenvalues $\lambda^\Phi \neq 0$ and eigenvectors $\AAlpha^\Phi\in \mathcal{F}\smallsetminus \{\boldsymbol{0}\}$ such that
\begin{equation}\label{eq:eigProblem}
\covmat^\Phi\AAlpha^\Phi = \lambda^\Phi \AAlpha^\Phi \mbox{ .}
\end{equation}
We remark that such an eigenvector satisfies
\begin{align}
\AAlpha^\Phi &= \frac{1}{\lambda^\Phi\nbProfilingTraces}\sum_{i=1}^{\nbProfilingTraces} \Phi(\vLeakVec_{i})\Phi(\vLeakVec_{i})^\intercal \AAlpha^\Phi\\
\label{eq:passaggio}
&=  \frac{1}{\lambda^\Phi\nbProfilingTraces}\sum_{i=1}^{\nbProfilingTraces} \left[\Phi(\vLeakVec_{i})^\intercal \AAlpha^\Phi\right] \Phi(\vLeakVec_{i}) =  \\
&= \sum_{i=1}^{\nbProfilingTraces}\underbrace{\frac{\Phi(\vLeakVec_{i})^\intercal \AAlpha^\Phi}{\lambda^\Phi\nbProfilingTraces}}_{\nu_i}\Phi(\vLeakVec_{i}) = \\
\label{eq:linearComb}
&= \sum_{i=1}^{\nbProfilingTraces}\nu_i \Phi(\vLeakVec_{i})\mbox{ ,}
\end{align}
where the step \eqref{eq:passaggio} makes use of the associativity of the matrix product and the commutativity of the scalar-matrix product. Eq.~\eqref{eq:linearComb} tells us that each eigenvector $\AAlpha^\Phi$ is expressible as a linear combination of the data mapped into the feature space $(\Phi(\vLeakVec_{i})_{i=1,\dots,\nbProfilingTraces}$, or equivalently each eigenvector $\AAlpha^\Phi$ lies in the span of $(\Phi(\vLeakVec_{i})_{i=1,\dots,\nbProfilingTraces})$. This observation authorizes to substitute to the problem \eqref{eq:eigProblem}, the following equivalent system:
\begin{equation}\label{eq:system}
\begin{cases}
\lambda^\Phi(\Phi(\vLeakVec_{1})\cdot  \AAlpha^\Phi) = \Phi(\vLeakVec_{1})\cdot \covmat^\Phi \AAlpha^\Phi \\
\vdots\\
\lambda^\Phi(\Phi(\vLeakVec_{\nbProfilingTraces})\cdot  \AAlpha^\Phi) = \Phi(\vLeakVec_{\nbProfilingTraces})\cdot \covmat^\Phi \AAlpha^\Phi
\end{cases}
\end{equation}


Joining \eqref{eq:linearComb} and \eqref{eq:system} we obtain, looking to the first equation of the system:
\begin{align}
\lambda^\Phi(\Phi(\vLeakVec_{1})\cdot \sum_{i=1}^{\nbProfilingTraces}\nu_i\Phi(\vLeakVec_{i})) &= \Phi(\vLeakVec_{1})\cdot\left[ \frac{1}{N}\sum_{i=1}^{\nbProfilingTraces}\Phi(\vLeakVec_{i})\Phi(\vLeakVec_{i})^\intercal (\sum_{i=1}^{\nbProfilingTraces}\nu_i \Phi(\vLeakVec_{i}))\right]\\
\lambda^\Phi \sum_{i=1}^{\nbProfilingTraces}\nu_i(\Phi(\vLeakVec_{1})\cdot \Phi(\vLeakVec_{i})) &=\Phi(\vLeakVec_{1})\cdot\left[ \sum_{j=1}^{\nbProfilingTraces}\frac{\nu_j}{N}\left(\sum_{i=1}^{\nbProfilingTraces}\Phi(\vLeakVec_{i})\Phi(\vLeakVec_{i})^\intercal\right)  \Phi(\vLeakVec_{j})\right]\\
\lambda^\Phi \sum_{i=1}^{\nbProfilingTraces}\nu_i(\Phi(\vLeakVec_{1})\cdot \Phi(\vLeakVec_{i})) &=\Phi(\vLeakVec_{1})\cdot\left[ \sum_{j=1}^{\nbProfilingTraces}\frac{\nu_j}{N}\sum_{i=1}^{\nbProfilingTraces}\underbrace{\Phi(\vLeakVec_{i})^\intercal \Phi(\vLeakVec_{j})}_{\Phi(\vLeakVec_{i}) \cdot \Phi(\vLeakVec_{j})}\Phi(\vLeakVec_{i})\right]\\
\lambda^\Phi \sum_{i=1}^{\nbProfilingTraces}\nu_i(\Phi(\vLeakVec_{1})\cdot \Phi(\vLeakVec_{i})) &= \sum_{j=1}^{\nbProfilingTraces}\frac{\nu_j}{N}\left[\Phi(\vLeakVec_{1})\cdot\sum_{i=1}^{\nbProfilingTraces}(\Phi(\vLeakVec_{i}) \cdot \Phi(\vLeakVec_{j}))\Phi(\vLeakVec_{i})\right]\\
\nbProfilingTraces\lambda^\Phi \sum_{i=1}^{\nbProfilingTraces}\nu_i\underbrace{(\Phi(\vLeakVec_{1})\cdot \Phi(\vLeakVec_{i}))}_{\kernelMatrix[1,i]} &= \sum_{j=1}^{\nbProfilingTraces}\nu_j\left[\sum_{i=1}^{\nbProfilingTraces}\underbrace{(\Phi(\vLeakVec_{i}) \cdot \Phi(\vLeakVec_{j}))}_{\kernelMatrix[i,j]}\underbrace{(\Phi(\vLeakVec_{1})\cdot\Phi(\vLeakVec_{i}))}_{\kernelMatrix[1,j]}\right]\mbox{ .}
\end{align}
Thus, the system \eqref{eq:system} is equivalent to the follow:

\begin{equation}
\begin{cases}
\nbProfilingTraces\lambda^\Phi \sum_{i=1}^{\nbProfilingTraces}\nu_i{\kernelMatrix[1,i]} &= \sum_{j=1}^{\nbProfilingTraces}\nu_j\left[\sum_{i=1}^{\nbProfilingTraces}{\kernelMatrix[1,j]}{\kernelMatrix[i,j]}\right]\\
\vdots \\
\nbProfilingTraces\lambda^\Phi \sum_{i=1}^{\nbProfilingTraces}\nu_i{\kernelMatrix[\nbProfilingTraces,i]} &= \sum_{j=1}^{\nbProfilingTraces}\nu_j\left[\sum_{i=1}^{\nbProfilingTraces}{\kernelMatrix[\nbProfilingTraces,j]}{\kernelMatrix[i,j]}\right]
\end{cases}
\end{equation}

%\begin{remark}
%The kernel matrix $\kernelMatrix$ is symmetric by construction: $\kernelMatrix[i,j] = \kernelMatrix[j,i]$ for each pair $i,j$.
%\end{remark}

Let $\nununu$ be the column vector containing the coefficients $\nu_i$ of \eqref{eq:linearComb}. The above system is expressible in matricial form as

\begin{equation}
\begin{cases}
\nbProfilingTraces\lambda^\Phi [\kernelMatrix\nununu][1] &= [\kernelMatrix^2\nununu][1]\\
\vdots \\
\nbProfilingTraces\lambda^\Phi [\kernelMatrix\nununu][\nbProfilingTraces] &= [\kernelMatrix^2\nununu][\nbProfilingTraces]\mbox{ ,}
\end{cases}
\end{equation}

which equals the following equation:

\begin{equation}
\nbProfilingTraces\lambda^\Phi\kernelMatrix\nununu = \kernelMatrix^2\nununu \mbox{ .}
\end{equation}

It can be shown that solving the last equation is equivalent to solve the following eigenvector problem
\begin{equation}\label{eq:eigProblemKPCA}
\gamma\nununu = \kernelMatrix\nununu \mbox{ .}
\end{equation}

Let $\gamma_1\geq\gamma_2\geq\dots\geq\gamma_{\nbProfilingTraces}$ denote the eigenvalues of $\kernelMatrix$, $\gamma_C$ being the last different from zero, and $\nununu_1,\dots, \nununu_{\nbProfilingTraces}$ the corresponding eigenvectors. For the sake of obtaining the corresponding normalized principal components in the feature space $\mathcal{F}$, denoted $\AAlpha^\Phi_1,\dots, \AAlpha^\Phi_C$, a normalization step is required, imposing for all $k=1,\dots, C$

\begin{equation}
\AAlpha^\Phi_k\cdot\AAlpha^\Phi_k = 1 \mbox{ ,}
\end{equation}

which can be translated into a condition for $\nununu_1,\dots, \nununu_C$, using \eqref{eq:linearComb} and \eqref{eq:eigProblemKPCA}:
\begin{equation}
1 = \sum_{i,j=1}^{\nbProfilingTraces}\nununu_k[i]\nununu_k[j](\Phi(\vLeakVec_{i})\cdot \Phi(\vLeakVec_{j})) = \nununu_k \cdot \kernelMatrix\nununu_k = \gamma_k(\nununu_k\cdot \nununu_k)
\end{equation}


Extracting the non-linear principal components of a datum $\vLeakVec_{}$ means projecting its image $\Phi(\vLeakVec_{})$ onto the eigenvectors $\AAlpha^\Phi_1,\dots, \AAlpha^\Phi_C$ in $\mathcal{F}$. To do so, we neither need to explicitly compute $\Phi(\vLeakVec_{})$ nor $\AAlpha^\Phi_i$. Indeed, using \eqref{eq:linearComb}:

\begin{equation}
\AAlpha^\Phi_k \cdot \Phi(\vLeakVec_{}) = \sum_{i=1}^{\nbProfilingTraces}\nununu_k[i](\Phi(\vLeakVec_{i}) \cdot \Phi(\vLeakVec_{})) =  \sum_{i=1}^{\nbProfilingTraces}\nununu_k[i]K(\vLeakVec_{i}, \vLeakVec_{}) \mbox{ .}
\end{equation}



\section{Kernel class-oriented PCA}

Suppose now that we want to perform a class-oriented PCA in the image space of a function $\Phi$ that is associated to a given kernel function $K$, i.e. we want to solve, using a kernel trick, the eigenvalue problem


\begin{equation}\label{eq:eigProblemSB}
\SB^\Phi\AAlpha^\Phi = \lambda^\Phi \AAlpha^\Phi \mbox{ ,}
\end{equation}

where $\SB^\Phi$ is the between-scatter matrix in the feature space: 

\begin{equation}
\SB^\Phi = \sum_{\sensVar\in\sensVarSet}\nbTracesPerClass(\mmmXclassPhi-\mmmXPhi)(\mmmXclassPhi-\mmmXPhi)^\intercal \mbox{ .}
\end{equation}

Here $\mmmXclassPhi = \frac{1}{\nbTracesPerClass}\sum_{i=1\colon \sensVar_i=\sensVar}\Phi(\vLeakVec_{i})$ and $\mmmXPhi = \frac{1}{\nbProfilingTraces}\sum_{i=1}^{\nbProfilingTraces}\Phi(\vLeakVec_{i})$.\\

As before, the eigenvectors $\AAlpha^\Phi_i$ are expressible as linear combination of the data images on $\mathcal{F}$, i.e. \eqref{eq:linearComb} is still true:

\begin{equation}
\AAlpha^\Phi = \sum_{i=1}^{\nbProfilingTraces}\nu_i \Phi(\vLeakVec_{i}) \mbox{ .}
\end{equation}

Moreover as before, the eigenvector problem \eqref{eq:eigProblemSB} can be translated in an eigenvector problem that gives the coefficients $\nununu$ as solutions. That is:

\begin{equation}\label{eq:eigProblemM}
\gamma\MMM = \MMM \nununu \mbox{ ,}
\end{equation}

where the matrix $\MMM$ is computed as

\begin{equation}
\MMM = \sum_{\sensVar\in\sensVarSet}\nbTracesPerClass(\MMMclass - \MMMT)(\MMMclass-\MMMT)\mbox{ ,}
\end{equation}

where $\MMMclass[j] = \frac{1}{\nbTracesPerClass}\sum_{i=1\colon \sensVar_i=\sensVar}K(\vLeakVec_{j},\vLeakVec{i})$ is an $\nbProfilingTraces$-sized column vector, and $\MMMT[j] = \frac{1}{\nbProfilingTraces}\sum_{i=1}^{\nbProfilingTraces}K(\vLeakVec_{j},\vLeakVec_{i})$.\\

Finally, one the eigenvector $\nununu$ are found, to project a datum $\vLeakVec_{}$ onto the corresponding principal component in the feature space we proceed as in the previous case:

\begin{equation} \label{eq:projection}
\AAlpha^\Phi_k \cdot \Phi(\vLeakVec_{}) =  \sum_{i=1}^{\nbProfilingTraces}\nununu_k[i]K(\vLeakVec_{i}, \vLeakVec_{}) \mbox{ .}
\end{equation}
