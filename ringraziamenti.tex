C'est avec grand plaisir que j'utilise cette pr\'emi\`ere page de mon manuscrit pour remercier toutes les personnes qui ont eu un r\^{o}le dans ce long parcours qui se termine. Je veux remercier qui a permis \`a ce parcours de commencer et de s'achever, qui m'a donn\'e confiance, qui m'a guid\'e, qui m'a stimul\'e, qui m'a encourag\'e, qui m'a fait douter, et pourquoi pas, qui m'a entrav\'e. \\

Emmanuel a sans doute recouvert une grande partie de ces r\^{o}les (mais pas $\text{tous !)}$. Sa passion pour la recherche est contagieuse et a \'et\'e un v\'eritable carburant pendant ces ann\'ees. Je le remercie pour m'avoir guid\'e, pour tous les \'echanges enrichissants que nous avons eu, pour ces enseignements, les id\'ees pr\'ecieuse qu'il a voulu partager avec moi, et ses encouragements qui n'ont jamais manqu\'e. \\

Je tiens \`a remercier de coeur C\'ecile, qui m'a accueilli et suivie au quotidien au sein du CESTI. Je la remercie pour sa disponibilit\'e illimit\'ee, sa confiance, ses id\'ees brillantes, et son esprit positif. \\

Je souhaite remercier sinc\`{e}rement Fran\c{c}ois-Xavier Standaert et Louis Goubin, qui ont \'et\'e rapporteurs de mon manuscrit, ainsi que Olivier Rioul, qui avait en principe aussi accept\'e de l'\^{e}tre. Je remercie en outre Philippe Elbaz-Vincent, Marios Choudary,  Annelie Heuser et Yannick Teglia  pour avoir accept\'e de prendre part \`a mon jury de th\`{e}se. \\

Je remercie une deuxi\`{e}me fois Fran\c{c}ois-Xavier Standaert , pour m'avoir accueilli une semaine \`a l'Universit\'e Catholique de Louvain-la-Neuve. Ce s\'ejour m'a permis de profiter d'\'echanges tr\`{e}s enrichissants avec lui, ainsi que avec Vincent Grosso, Liran Lerman et Anthony Journault, que je remercie \'egalement. Je ne peux pas cacher que le s\'ejour \`a Louvain-la-Neuve, plong\'ee dans un contexte purement acad\'emique, a \'et\'e source de forte motivation pour la poursuite de mes \'etudes. \\

Pendant ma th\`{e}se, j'ai eu diverses occasions de faire des rencontres aussi beaux que dignes de grands remerciements. Je voudrais remercier toutes les personnes que j'ai rencontr\'e \`a l'ANSSI, \`a Louvain-la-Neuve, en conf\'erence ou en \'ecole d'\'et\'e, pour en nommer une toute petite partie merci \`a Ilaria, Elizabeth, Siemen, Guillaume, Nicolas, Marjia, Adrien, Dahmun, Aurore, Romain, Colin, Jo\"el et Sonia.\\

\emph{Ancora prima del dottorato, ho incontrato alcune persone che sono state di fondamentale importanza per la mia scelta di intraprendere questo percorso, e che ringrazio di cuore: in primo luogo Lilli e Guido, i cui nomi mi piace scrivere l'uno accanto all'altro nonostante i loro ruoli diametralmente opposti nella mia crescita, ma anche Danilo e Marco.}\\

Ensuite, j'aimerais remercier tous mes (anciens et pr\'esents) coll\`{e}gues du CESTI, pour leurs enseignements et leur aide, pour les diverses discussions de la pause caf\'e, pour la belle ambiance et pour leur pr\'esence jamais manqu\'ee lors de tout \'ev\'enement important, joyaux et non, de ma vie.  Merci en particulier \`a Louis, mon co-bureau des premi\`{e}res ann\'ees, pour avoir \'et\'e pour moi le mod\`{e}le id\'eale (et injoignable !) de doctorant.  \\

\emph{Grazie ai miei amici lontani, che durante questi anni non hanno mai mancato occasione di essermi vicini. Grazie a Ila, Ago, Simo, Gio, Fede, Marta, Fra, Franci, Fede, MG, Giulia,  Elena, Davide, Fede e Fede.} Merci \`a H\'el\`{e}ne, Maren et Yoan, qui devront traduire depuis l'italien pour comprendre la raison de leur remerciement ! \\

\emph{Grazie agli amici vicini, per le belle serate, le cene, i pic-nic, le merende, e le scampagnate (per fingere che non sia vero che non facciamo altro che magna'!). Grazie a Vera, Chiara, Yvonne, Fausto, Gaietta e Cumino, Daniela, Paolo, Martino e Michele, Fanny, Nicola ed MT, Caroline.} Merci \`a J\'er\'emie, \`a Ga\"elle, Jennifer et Manoela. Merci \`a tous mes co-\'equipier de volley, personne mieux d'eux a su me faire d\'ecrocher de ma th\`{e}se le temps de quelques heures par semaine. \\

\emph{Grazie a Fede, coinquilino d'eccellenza dell'ultimo anno, aspetto con impazienza di festeggiare insieme!}\\

Un merci immense aux plus grands obstacles vivants de ma th\`{e}se : Giacomo, qui m'a permis d'int\'egrer \`a mon projet de th\`{e}se quatre d\'em\'enagements, un PACS, un chantier et deux enfants, et mes deux petits Camillo et Ottavio. Merci \`a vous trois pour votre capacit\'e de m'\'epuiser et me recharger en m\^eme temps, vous \^etes mon soutien solide  et la source de mon bonheur et de toutes mes \'energies ! \\

\emph{Grazie infine a mamma, pap\`a, Laura e Carlo, che siete i miei esempi e le fonti inesauribili di sicurezza e di  incoraggiamento. Grazie anche a Clio, Graziano, Alice, Roberto, Samuele e Lorenzo, che avete reso frizzanti e indimenticabili tutti i momenti passati in famiglia!}