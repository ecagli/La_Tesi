C'est avec grand plaisir que j'utilise cette page de mon manuscrit pour remercier toutes les personnes qui ont eu un rôle dans ce long parcours qui se termine. Je veux remercier qui a permis à ce parcours de commencer et de se terminer, qui m'a donné confiance, qui m'a guidé, qui m'a stimulé, qui m'a encouragé, qui m'a fait douter, et pourquoi pas, qui m'a entravé. \\

Emmanuel a sans doute recouvert une grande partie de ces rôles (mais pas tous !). Sa passion pour la recherche est contagieuse et a été un véritable carburant pendant ces années. Je le remercie pour m'avoir guidé, pour tous les échanges enrichissants que nous avons eu, pour ces enseignements, les idées précieuse qu'il a voulu partager avec moi, et ses encouragements qui n'ont jamais manqué. \\

Je tiens à remercier de cœur Cécile, qui m'a accueilli et suivie au quotidien au sein du CESTI. Je la remercie pour sa disponibilité illimitée, sa confiance ses idées brillantes, et son esprit positif. \\

Je souhaite remercier sincèrement François-Xavier Standaert et Louis Goubin, qui ont été rapporteurs de mon manuscrit, ainsi que Olivier Rioul, qui avait en principe aussi accepté de l'être. Je remercie en outre Philippe Elbaz-Vincent, Annelie Heuser, Yannick Teglia et Marios Choudary pour avoir accepté de prendre part à mon jury de thèse. \\

Je remercie encore François-Xavier Standaert pour m'avoir accueilli une semaine à l'Université Catholique de Louvain-la-Neuve. Même si notre collaboration n'a pas abouti au résultat envisagé par causes diverses et de force majeure, j'ai pu profiter d'échanges très enrichissants avec lui, ainsi que avec Vincent Grosso, Liran Lerman et Anthony Journault, que je remercie également. Je ne veux pas cacher que le séjour à Louvain-la-Neuve, plongée dans un contexte purement académique, a été source de forte motivation pour la poursuite de mes études. \\

Pendant ma thèse, j'ai eu diverses occasions de faire des rencontres aussi beaux que dignes de grands remerciements. Je voudrais remercier toutes les personnes que j'ai rencontré à l'ANSSI, à Louvain-la-Neuve, en conférence ou en école d'été et en particulier à Ilaria, Elizabeth, Siemen, Guillaume, Nicolas, Marjia, Adrien, Dahmun, Aurore, Romain, Colin, Joël et Sonia.

Prima del dottorato, ho incontrato alcune persone che sono state di fondamentale importanza per la mia scelta di intraprendere questo percorso, e che ringrazio di cuore: in primo luogo Lilli e Guido, i cui nomi mi piace scrivere l'uno accanto all'altro nonostante i loro ruoli diametralmente opposti nella mia crescita, ma anche Danilo e Marco.

Ensuite, j'aimerais remercier les personnes que j'ai côtoyé au quotidien pendant ces années, mes (anciens ou présents) collègues du CESTI. Merci en particulier à Louis, mon co-bureau des premières années, pour avoir été pour moi le modèle (injoignable !) de doctorant parfait.  

......................

Pour finir un merci immense aux plus grands obstacles vivants de ma thèse : Giacomo, qui m'a permis d'intégrer à mon projet de thèse quatre déménagements et deux enfants, et mes deux petits Camillo et Ottavio. Merci à vous trois pour votre capacité de m'épuiser et me recharger en même temps, vous êtes la source de toutes mes énergies et mon soutien solide !