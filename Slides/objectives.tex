\section{State of the Art, Objectives, Contributions}

\begin{frame}
\frametitle{Advanced Attacks}
\begin{itemize}
\item choose a sensitive variable $\sensRandVar = \sensFunction(\keyRandVar,\publicParRandVar)$, 
\item acquire side-channel traces $(\vLeakVec_i)_{i=1,\dots , \nbTraces}$ making entries $(\publicParVar_i)_{i=1,\dots,\nbTraces}$ vary,
\item define a \emph{leakage model} $\vaLeakVec \approx \leakageModel(\sensRandVar)$ (\eg \emph{mono-bit}, \emph{Hamming weight}, \emph{Hamming distance}, \emph{linear bit-combination}, \emph{identity}\uncover<3->{\important{, learned templates}}),
\item for every key chunk hypothesis $\keyVar \in \keyVarSet$ predict the side-channel leakage 
\begin{equation}\label{eq:predictions}
\leakageModel_{\keyVar,i} = \leakageModel(\sensFunction(\keyVar,\publicParVar_i)) \mbox{ ,}
\end{equation}
\item statistically compare the hypothetical predictions to the observed side-channel acquisitions, by means of a  \emph{distinguisher} $\distinguisher$ (\eg DoM \cite{kocher1999differential}, Multi-bit DoM \cite{bevan2002ways,messerges2002examining}, CPA \cite{brier2004correlation}, MIA \cite{gierlichs2008mutual,batina2011mutual}, ..., \uncover<4->{\important{, Maximum Likelihood (ML) or Maximum-a-Posteriori (MAP) $\rightarrow$ Template Attack}}):
\begin{equation}
\distinguisher_{\keyVar} = \distinguisher((\vLeakVec_i)_{i=1,\dots , \nbTraces}, (\leakageModel_{\keyVar,i} )_{i=1,\dots , \nbTraces}) \mbox{ ,}
\end{equation}
\item deduce the key chunk candidate from scores $\distinguisher_{\keyVar}$, in general coinciding with the key hypothesis that maximises (or minimises) the scores.
\end{itemize}

\uncover<2->{\textbf{If profiling is available...}
}
\end{frame}

\begin{frame}
\frametitle{Template Attack}
 \begin{textblock}{5}(10,2)
 \important{$\vaLeakVec\in \mathbb{R}^D$\\
  Curse of dimensionality!}
 \end{textblock}
\begin{itemize}
\item Profiling phase (using profiling traces under known $\sensRandVar$)
\begin{itemize}
\item \uncover<7->{manage de-synchronization problem}
\item \uncover<6->{mandatory dimensionality reduction}
\item \uncover<6->{Gaussian hypothesis \cite{Chari2003}}
\item \uncover<6->{Variants: \emph{pooled} version \cite{choudary2014efficient}, linear regression \cite{schindler2005stochastic}}
\item \uncover<2->{estimate \only<2-4>{$\prob[\vaLeakVec|\sensRandVar=\sensVar]$ (generative model)} \only<5-6>{\textcolor{blue}{$\prob[\vaLeakVec|\sensRandVar=\sensVar]$}} \only<7->{\textcolor{blue}{$\prob[\important{\varepsilon(\vaLeakVec)}|\sensRandVar=\sensVar]$}}for each value of $\sensVar$}
\end{itemize}
\item Attack phase ($N$ attack traces $\vLeakVec_i$, e.g. with known plaintexts $p_i$)

\only<1-4>{ \begin{itemize}
\item[] \textcolor{white}{Log-likelihood score for each key hypothesis $k$
\begin{equation*}
d_k = \sum_{i=1}^{N}\log \prob[\vaLeakVec=\vLeakVec_i | Z=f(p_i,k)]
\end{equation*}
}

\item\textcolor{white}{A-posteriori probability score for each key hypothesis $k$
\begin{align*}
\pdf_{\given{\sensRandVar}{  \vaLeakVec = \vLeakVec}}(\sensVar) &= \frac{\pdf_{\given{\vaLeakVec}{\sensRandVar = \sensVar}}(\vLeakVec)\pdf_{\sensRandVar}(\sensVar)} {\pdf_{\vaLeakVec}(\vLeakVec)}\text{Bayes' theorem}\\
d_{\keyVar} &= \prod_{i=1}^{\nbAttackTraces} \pdf_{\given{\sensRandVar}{\vaLeakVec = \vLeakVec_i}}(\sensFunction(\keyVar,\publicParVar_i) ) \mbox{ ,}
\end{align*}
}
\end{itemize}
}
\only<5-6>{\begin{itemize}
\item Log-likelihood score for each key hypothesis $k$
\begin{equation*}
d_k = \sum_{i=1}^{N}\log \textcolor{blue}{\prob[\vaLeakVec=\vLeakVec_i | Z=f(p_i,k)]}
\end{equation*}

\item A-posteriori probability score for each key hypothesis $k$
\begin{align*}
\pdf_{\given{\sensRandVar}{  \vaLeakVec = \vLeakVec}}(\sensVar) &= \frac{\pdf_{\given{\vaLeakVec}{\sensRandVar = \sensVar}}(\vLeakVec)\pdf_{\sensRandVar}(\sensVar)} {\pdf_{\vaLeakVec}(\vLeakVec)} \text{Bayes' theorem}\\
d_{\keyVar} &= \prod_{i=1}^{\nbAttackTraces} \pdf_{\given{\sensRandVar}{\vaLeakVec = \vLeakVec_i}}(\sensFunction(\keyVar,\publicParVar_i) ) \mbox{ ,}
\end{align*}
\end{itemize}
}
\only<7->{\begin{itemize}
\item Log-likelihood score for each key hypothesis $k$
\begin{equation*}
d_k = \sum_{i=1}^{N}\log \textcolor{blue}{\prob[\important{\varepsilon(
\vaLeakVec)}=\important{\varepsilon(
\vLeakVec_i)} | Z=f(p_i,k)]}
\end{equation*}


\item A-posteriori probability score for each key hypothesis $k$
\begin{align*}
\pdf_{\given{\sensRandVar}{  \vaLeakVec = \vLeakVec}}(\sensVar) &= \frac{\pdf_{\given{\vaLeakVec}{\sensRandVar = \sensVar}}(\vLeakVec)\pdf_{\sensRandVar}(\sensVar)} {\pdf_{\vaLeakVec}(\vLeakVec)} \text{Bayes' theorem}\\
d_{\keyVar} &= \prod_{i=1}^{\nbAttackTraces} \pdf_{\given{\sensRandVar}{\vaLeakVec = \vLeakVec_i}}(\sensFunction(\keyVar,\publicParVar_i) ) \mbox{ ,}
\end{align*}
\end{itemize}
}

\end{itemize}


\end{frame}


\begin{frame}
\frametitle{Objectives}

Profiling phase (using profiling traces under known $\sensRandVar$)
\begin{itemize}
\item {manage de-synchronization problem}\only<3>{\important{$\longleftarrow$}}
\item {mandatory dimensionality reduction} \only<1-2>{\important{$\longleftarrow$}}
\item {Gaussian hypothesis \cite{Chari2003}}
\item \textcolor{grey}{Variants: \emph{pooled} version \cite{choudary2014efficient}, linear regression \cite{schindler2005stochastic}}
\item estimate{$\prob[\important{\varepsilon(\vaLeakVec)}|\sensRandVar=\sensVar]$ (generative model)} for each value of $\sensVar$
\end{itemize}

\begin{block}{Objectives}
\uncover<2->{
\begin{itemize}
\item Ameliorate the template attack routine by proposing efficient dimensionality reduction techniques
\item \uncover<3->{More generally, optimise the profiling attack strategy}
\item Consider the presence of most-commonly-implemented SCA countermeasures (masking, hiding)
\end{itemize}
}
\end{block}

\end{frame}


\begin{frame}
\frametitle{State of the Art and Contributions}
\only<1>{
\begin{block}{\textit{Before}}
\begin{itemize}
\item Feature (Points of Interest - PoI) selection methods:  ...
\item Linear Feature Extraction: PCA, LCA, PP...
\item Machine Learning routines for Side-Channel Attacks: ...
\end{itemize}
\end{block}
}
\only<2->{
\begin{block}{\textit{This thesis}}
\begin{itemize}
\item \textbf{Linear Dimensionality Reduction}: 
\begin{itemize}
\item PCA, choise of components ELV
\item LDA in case of undersampling
\end{itemize}
\only<2>{
\item \textbf{Kernel Discriminant Analysis}: application of an appropriate kernel trick to LDA, in order to manage masking countermeasure
\item \textbf{Convolutional Neural Networks}: 
\begin{itemize}
\item discriminative model by means of neural network classifiers
\item convolutional layers to manage desyncrhonisation (a form of hiding)
\item Data Augmentation techniques to reduce overfitting
\end{itemize}}
\only<3->{\important{
\item \textbf{Kernel Discriminant Analysis}: application of an appropriate kernel trick to LDA, in order to manage masking countermeasure
\item \textbf{Convolutional Neural Networks}: 
\begin{itemize}
\item discriminative model by means of neural network classifiers
\item convolutional layers to manage desyncrhonisation (a form of hiding)
\item Data Augmentation techniques to reduce overfitting
\end{itemize}
}
}
\end{itemize}
\end{block}
}
\only<4>{
\begin{block}{\textit{Today}}
\begin{itemize}
\item suites de la KDA
\item ASCAD plus toutes les contributions CNN sorties dernieremets
\end{itemize}
\end{block}
}
\end{frame}



\begin{frame}
\frametitle{Contents}
\begin{itemize}
\item Introduction to LDA: as a classifier, and as a feature extractor
\item Introduction to masking countermeasure and Kernel Discriminant Analysis as a feature extractor
\item Motivations to apply deep learning techniques
\item Convolutional Neural Networks and Data Augmentation to attack jitter-based countermeasure
\end{itemize}
\end{frame}






%\begin{frame}
%\frametitle{Template Attack} 
%% aprendo des volets per ogni passaggio con riferimenti allo stato dell'arte:
%% templates : likelihood vs MAP ,  cov mats vs pooled , linear regression or not
%% dimensionality reduction : feature selection, feature extraction, masking : masques connus vs unconnus
%% misalignment : re-alignement
%
%\begin{tikzpicture}[ ->, node distance = 3cm,
%					decoration = {snake,   % <-- added
%                    pre length=3pt,post length=7pt,% <-- for better looking of arrow,
%                    }]
%
%\node [data] (db_profiling){\includegraphics[width = 0.3\textwidth]{figures/database.jpg}\\ Profiling traces};
%\node [data, right of = db_profiling] (db_profiling_realigned){\includegraphics[width = 0.3\textwidth]{figures/database.jpg}\\ \footnotesize{Realigned profiling traces}};
%\draw[arrow] (db_profiling) -- node[above]{$\rho$} (db_profiling_realigned) ;
%\node[function, above of = db_profiling, yshift=-1cm](realignment){$\rho$};
%\draw[arrow,dashed](db_profiling) -- (realignment);
%\node[function, above of= db_profiling_realigned, yshift=-1cm](extractor){$\varepsilon$};
%\node [data, right of = db_profiling_realigned] (extracted){\includegraphics[width = 0.3\textwidth]{figures/database.jpg}\\ Profiling traces' features};
%\draw[arrow] (db_profiling_realigned) -- node[above]{$\varepsilon$} (extracted) ;
%\draw[arrow,dashed](db_profiling_realigned) -- (extractor);
%\node[data, right of = extracted, xshift=0.8cm](distributions){\small{$\mathrm{Pr}(\vaLeakVec \mid Z)$}\\
%\small{$\mathrm{Pr}(Z \mid \vaLeakVec)$}};
%\draw[arrow,dashed](extracted) -- node[above]{characterisation} (distributions);
%
%\node[methods, below of= db_profiling, yshift=1cm](realignments){Realignment \cite{nagashima2007dpa,van2011improving,durvaux2012efficient}};
%\node[methods, below of= db_profiling_realigned, yshift=1cm](extraction_methods){Dimensionality reduction};
%\node[methods, below of= extracted, yshift=1cm, align=left](charac_methods){\tiny{
%Template Attack \cite{Chari2003} (generative model, Gaussian hypothesis)\\
%\emph{Pooled} variant \cite{choudary2014efficient}\\
%Stochastic variant \cite{schindler2005stochastic}\\
%Discriminative model
%}};
%\node [data, below of=realignments, yshift=1cm ] (db_attack){\includegraphics[width = 0.3\textwidth]{figures/database.jpg}\\ Attack traces};
%\node [data, right of = db_attack] (db_attack_realigned){\includegraphics[width = 0.3\textwidth]{figures/database.jpg}\\ Realigned attack traces};
%\draw[arrow] (db_attack) -- node[above]{$\rho$} (db_attack_realigned) ;
%\node [data, right of = db_attack_realigned] (extracted_attack){\includegraphics[width = 0.3\textwidth]{figures/database.jpg}\\ Attack traces' features};
%\draw[arrow] (db_attack_realigned) -- node[above]{$\varepsilon$} (extracted_attack) ;
%\node[data, right of = extracted_attack](keys){\footnotesize{$\mathrm{Pr}(K \mid \{ \vaLeakVec_i, P_i \}_{i=1,\dots,N})$}};
%\draw[arrow,dashed](extracted_attack) -- node[above]{inference} (keys);
%
%
%
%\end{tikzpicture}
%\end{frame}
%
%\begin{frame}
%\frametitle{Template Attack} 
%%% aprendo des volets per ogni passaggio con riferimenti allo stato dell'arte:
%%% templates : likelihood vs MAP ,  cov mats vs pooled , linear regression or not
%%% dimensionality reduction : feature selection, feature extraction, masking : masques connus vs unconnus
%%% misalignment : re-alignement
%%
%\begin{tikzpicture}[ ->, node distance = 2.5cm,
%				decoration = {snake,   % <-- added
%                   pre length=3pt,post length=7pt,% <-- for better looking of arrow,
%                   }]
%\uncover<4->{
%\node [data] (db_profiling){\includegraphics[width = 0.3\textwidth]{figures/database.jpg}\\ Profiling traces};
%\node [data, right of = db_profiling] (db_profiling_realigned){\includegraphics[width = 0.3\textwidth]{figures/database.jpg}\\ \footnotesize{Realigned profiling traces}};
%\draw[arrow] (db_profiling) -- node[above]{$\rho$} (db_profiling_realigned) ;
%\node[function, above of = db_profiling, yshift=-1cm](realignment){$\rho$};
%\draw[arrow,dashed](db_profiling) -- (realignment);
%\node[function, above of= db_profiling_realigned, yshift=-1cm](extractor){$\varepsilon$};
%\draw[arrow] (db_profiling_realigned) -- node[above]{$\varepsilon$} (extracted) ;
%\draw[arrow,dashed](db_profiling_realigned) -- (extractor);
%}
%
%\node [data, right of = db_profiling_realigned] (extracted){\includegraphics[width = 0.3\textwidth]{figures/database.jpg}\\ \only<1>{Profiling traces} \only<3->{Profiling traces's features}};
%\node[data, right of = extracted, xshift=0.8cm](distributions){\small{$\mathrm{Pr}(\vaLeakVec \mid Z)$}};
%\draw[arrow,dashed](extracted) -- node[above]{characterisation} (distributions);
%
%\node[methods, below of= db_profiling, yshift=1cm](realignments){Realignment \cite{nagashima2007dpa,van2011improving,durvaux2012efficient}};
%\node[methods, below of= db_profiling_realigned, yshift=1cm](extraction_methods){Dimensionality reduction};
%\node[methods, below of= extracted, yshift=1cm, align=left](charac_methods){\tiny{
%Template Attack \cite{Chari2003} (generative model)\\
%\uncover<3->{\emph{Pooled} variant \cite{choudary2014efficient}\\
%Stochastic variant \cite{schindler2005stochastic}}
%}};
%
%\only<4->{
%\node [data, below of=realignments, yshift=1cm ] (db_attack){\includegraphics[width = 0.3\textwidth]{figures/database.jpg}\\ Attack traces};
%\node [data, right of = db_attack] (db_attack_realigned){\includegraphics[width = 0.3\textwidth]{figures/database.jpg}\\ Realigned attack traces};
%\draw[arrow] (db_attack) -- node[above]{$\rho$} (db_attack_realigned) ;
%}
%\node [data, right of = db_attack_realigned] (extracted_attack){\includegraphics[width = 0.3\textwidth]{figures/database.jpg}\\ \only<1-2>{Attack traces} \only<3->{Attack traces' features}};
%\draw[arrow] (db_attack_realigned) -- node[above]{$\varepsilon$} (extracted_attack) ;
%\node[data, right of = extracted_attack](keys){Maximum Likelihood or Maximum A Posteriori};
%\draw[arrow,dashed](extracted_attack) -- node[above]{inference} (keys);
%
%
%
%\end{tikzpicture}
%\end{frame}
%