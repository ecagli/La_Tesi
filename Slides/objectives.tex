\section{State of the Art, Objectives, Contributions}

%\begin{frame}
%\frametitle{Template Attack}
% \begin{textblock}{5}(10,2)
% \important{$\vaLeakVec\in \mathbb{R}^D$\\
%  Curse of dimensionality!}
% \end{textblock}
%\begin{itemize}
%\item Profiling phase (using profiling traces under known $\sensRandVar$)
%\begin{itemize}
%\item \uncover<7->{manage de-synchronization problem}
%\item \uncover<6->{mandatory dimensionality reduction}
%\item \uncover<6->{Gaussian hypothesis \cite{Chari2003}}
%\item \uncover<6->{Variants: \emph{pooled} version \cite{choudary2014efficient}, linear regression \cite{schindler2005stochastic}}
%\item \uncover<2->{estimate \only<2-4>{$\prob[\vaLeakVec|\sensRandVar=\sensVar]$ (generative model)} \only<5-6>{\textcolor{blue}{$\prob[\vaLeakVec|\sensRandVar=\sensVar]$}} \only<7->{\textcolor{blue}{$\prob[\important{\varepsilon(\vaLeakVec)}|\sensRandVar=\sensVar]$}}for each value of $\sensVar$}
%\end{itemize}
%\item Attack phase ($N$ attack traces $\vLeakVec_i$, e.g. with known plaintexts $p_i$)
%
%\only<1-4>{ \begin{itemize}
%\item[] \textcolor{white}{Log-likelihood score for each key hypothesis $k$
%\begin{equation*}
%d_k = \sum_{i=1}^{N}\log \prob[\vaLeakVec=\vLeakVec_i | Z=f(p_i,k)]
%\end{equation*}
%}
%
%\item\textcolor{white}{A-posteriori probability score for each key hypothesis $k$
%\begin{align*}
%\pdf_{\given{\sensRandVar}{  \vaLeakVec = \vLeakVec}}(\sensVar) &= \frac{\pdf_{\given{\vaLeakVec}{\sensRandVar = \sensVar}}(\vLeakVec)\pdf_{\sensRandVar}(\sensVar)} {\pdf_{\vaLeakVec}(\vLeakVec)}\text{Bayes' theorem}\\
%d_{\keyVar} &= \prod_{i=1}^{\nbAttackTraces} \pdf_{\given{\sensRandVar}{\vaLeakVec = \vLeakVec_i}}(\sensFunction(\keyVar,\publicParVar_i) ) \mbox{ ,}
%\end{align*}
%}
%\end{itemize}
%}
%\only<5-6>{\begin{itemize}
%\item Log-likelihood score for each key hypothesis $k$
%\begin{equation*}
%d_k = \sum_{i=1}^{N}\log \textcolor{blue}{\prob[\vaLeakVec=\vLeakVec_i | Z=f(p_i,k)]}
%\end{equation*}
%
%\item A-posteriori probability score for each key hypothesis $k$
%\begin{align*}
%\pdf_{\given{\sensRandVar}{  \vaLeakVec = \vLeakVec}}(\sensVar) &= \frac{\pdf_{\given{\vaLeakVec}{\sensRandVar = \sensVar}}(\vLeakVec)\pdf_{\sensRandVar}(\sensVar)} {\pdf_{\vaLeakVec}(\vLeakVec)} \text{Bayes' theorem}\\
%d_{\keyVar} &= \prod_{i=1}^{\nbAttackTraces} \pdf_{\given{\sensRandVar}{\vaLeakVec = \vLeakVec_i}}(\sensFunction(\keyVar,\publicParVar_i) ) \mbox{ ,}
%\end{align*}
%\end{itemize}
%}
%\only<7->{\begin{itemize}
%\item Log-likelihood score for each key hypothesis $k$
%\begin{equation*}
%d_k = \sum_{i=1}^{N}\log \textcolor{blue}{\prob[\important{\varepsilon(
%\vaLeakVec)}=\important{\varepsilon(
%\vLeakVec_i)} | Z=f(p_i,k)]}
%\end{equation*}
%
%
%\item A-posteriori probability score for each key hypothesis $k$
%\begin{align*}
%\pdf_{\given{\sensRandVar}{  \vaLeakVec = \vLeakVec}}(\sensVar) &= \frac{\pdf_{\given{\vaLeakVec}{\sensRandVar = \sensVar}}(\vLeakVec)\pdf_{\sensRandVar}(\sensVar)} {\pdf_{\vaLeakVec}(\vLeakVec)} \text{Bayes' theorem}\\
%d_{\keyVar} &= \prod_{i=1}^{\nbAttackTraces} \pdf_{\given{\sensRandVar}{\vaLeakVec = \vLeakVec_i}}(\sensFunction(\keyVar,\publicParVar_i) ) \mbox{ ,}
%\end{align*}
%\end{itemize}
%}
%
%\end{itemize}
%
%
%\end{frame}

\begin{frame}
\frametitle{Notations}
\begin{block}{Notations and generalities}
\begin{itemize}
\item Side-channel traces: realizations of a random vector $\vaLeakVec \in \mathbb{R}^D$  
\item Target: a \emph{sensitive} variable $Z = f(\mathrm{plaintext,key})$ 
\end{itemize}
\end{block}
\vspace{-5pt}
\begin{block}{Profiling attack scenario}
Exploits labelled traces $(\vLeakVec{i}, z_i)_{i=1}^N$, acquired under known $Z$, to characterize the signals.\\
\end{block}
\pause
\begin{block}{Points of Interest}
Time samples that statistically depend on $Z$.
\end{block}
\pause
\begin{block}{Exhaustive search for $d$ time samples}
Complexity combinatorial in $d$.\\
\emph{Example:} $D=200, d=4 \longrightarrow \sim 70.000.000$ possibilities. 
\end{block}
\end{frame}



\begin{frame}
\frametitle{Template Attack}
 \begin{textblock}{5}(10,2)
 \important{$\vaLeakVec\in \mathbb{R}^D$\\
  Curse of dimensionality!}
 \end{textblock}
\begin{itemize}
\item Profiling phase (using profiling traces under known $\sensRandVar$)
\begin{itemize}
\item manage de-synchronization problem
\item mandatory dimensionality reduction
\item Gaussian hypothesis \cite{Chari2003}
\item Variants: \emph{pooled} version \cite{choudary2014efficient}, linear regression \cite{schindler2005stochastic}
\item estimate \only<1>{\textcolor{blue}{$\prob[\vaLeakVec|\sensRandVar=\sensVar]$} (generative model)} 
\end{itemize}
\item Attack phase ($N$ attack traces $\vLeakVec_i$, e.g. with known plaintexts $p_i$)

\begin{itemize}
\item Log-likelihood score for each key hypothesis $k$
\begin{equation*}
d_k = \sum_{i=1}^{N}\log \textcolor{blue}{\prob[\vaLeakVec=\vLeakVec_i | Z=f(p_i,k)]}
\end{equation*}


\item A-posteriori probability score for each key hypothesis $k$
\begin{align*}
\pdf_{\given{\sensRandVar}{  \vaLeakVec = \vLeakVec}}(\sensVar) &= \frac{\pdf_{\given{\vaLeakVec}{\sensRandVar = \sensVar}}(\vLeakVec)\pdf_{\sensRandVar}(\sensVar)} {\pdf_{\vaLeakVec}(\vLeakVec)} \text{Bayes' theorem}\\
d_{\keyVar} &= \prod_{i=1}^{\nbAttackTraces} \pdf_{\given{\sensRandVar}{\vaLeakVec = \vLeakVec_i}}(\sensFunction(\keyVar,\publicParVar_i) ) \mbox{ ,}
\end{align*}
\end{itemize}

\end{itemize}
\end{frame}



\begin{frame}
\frametitle{Objectives}

Profiling phase (using profiling traces under known $\sensRandVar$)
\begin{itemize}
\item {manage de-synchronization problem}\only<3>{\important{$\longleftarrow$}}
\item {mandatory dimensionality reduction} \only<1-2>{\important{$\longleftarrow$}}
\item {Gaussian hypothesis \cite{Chari2003}}
\item \textcolor{grey}{Variants: \emph{pooled} version \cite{choudary2014efficient}, linear regression \cite{schindler2005stochastic}}
\item estimate{$\prob[\important{\varepsilon(\vaLeakVec)}|\sensRandVar=\sensVar]$ (generative model)} for each value of $\sensVar$
\end{itemize}

\begin{block}{Objectives}
\uncover<2->{
\begin{itemize}
\item Ameliorate the template attack routine by proposing efficient dimensionality reduction techniques
\item \uncover<3->{More generally, optimise the profiling attack strategy - consider discriminative model}
\item Consider the presence of most-commonly-implemented SCA countermeasures (masking, hiding)
\end{itemize}
}
\end{block}

\end{frame}


\begin{frame}
\frametitle{State of the Art}
\begin{block}{Dimensionality Reduction}
\begin{columns}
\begin{column} {.6\textwidth}
Optimally exploit Points of Interest to 
\begin{itemize}
\item reduce attack complexity 
\item raise attack efficiency
\end{itemize}
\end{column}
\begin{column}{.3\textwidth}
\centering extractor  \begin{align*}
\extract \colon & \mathbb{R}^D\rightarrow \mathbb{R}^C\\
& \vLeakVec \mapsto \extract(\vLeakVec)
\end{align*}
%$\extract \colon \mathbb{R}^D\rightarrow \mathbb{R}^C$  

\end{column}
\end{columns}
\end{block}
\begin{columns}

\begin{column}{.4\textwidth}
\begin{block}{Selecting Extractors}
$\extract$ performs a sub-sampling
\begin{itemize}
\item SOD \cite{Chari2003}
\item SOST \cite{bar2010improved}
\item SNR \cite{mangard2008power}/ NICV \cite{bhasin2014side}
\item $t$-test, $F$-test,... \cite{gierlichs2006templates,choudary2014efficient}
\end{itemize}
\end{block}
\end{column}

\begin{column}{.6\textwidth}
\only<1>{

\includegraphics[width = \textwidth]{figures/SNR_1order_new.jpg} 
}
\only<2>{
\begin{block}{Projecting Extractors}
$\extract(\vLeakVec) = \vLeakVec \mbox{ with } A \in M_{\mathbb{R}}(\newTraceLength, \traceLength)$
\begin{itemize}
\item Principal Component Analysis (PCA) \cite{TAprincipal,Batina2012}
\item Linear Discriminant Analysis (LDA) \cite{Standaert2008,lessIsMore}
\item Projection Pursuits (PP) \cite{PP}
\end{itemize}
\end{block}}
\end{column}
\end{columns}
\end{frame}

\begin{frame}
\frametitle{Contributions}


\begin{itemize}
\item \textbf{Linear Dimensionality Reduction}: 
\begin{itemize}
\item PCA, choise of components ELV
\item LDA in case of undersampling
\end{itemize}
\only<2>{
\item \textbf{Kernel Discriminant Analysis}: application of an appropriate kernel trick to LDA, in order to manage masking countermeasure
\item \textbf{Convolutional Neural Networks}: 
\begin{itemize}
\item discriminative model by means of neural network classifiers
\item convolutional layers to manage desyncrhonisation (a form of hiding)
\item Data Augmentation techniques to reduce overfitting
\end{itemize}}
\only<3->{\important{
\item \textbf{Kernel Discriminant Analysis}: application of an appropriate kernel trick to LDA, in order to manage masking countermeasure
\item \textbf{Convolutional Neural Networks}: 
\begin{itemize}
\item discriminative model by means of neural network classifiers
\item convolutional layers to manage desyncrhonisation (a form of hiding)
\item Data Augmentation techniques to reduce overfitting
\end{itemize}
}
}
\end{itemize}


\end{frame}



\begin{frame}
\frametitle{Contents}
\begin{itemize}
\item Introduction to LDA: as a classifier, and as a feature extractor
\item Introduction to masking countermeasure and Kernel Discriminant Analysis as a feature extractor
\item Motivations to apply deep learning techniques
\item Convolutional Neural Networks and Data Augmentation to attack jitter-based countermeasure
\end{itemize}
\end{frame}






%\begin{frame}
%\frametitle{Template Attack} 
%% aprendo des volets per ogni passaggio con riferimenti allo stato dell'arte:
%% templates : likelihood vs MAP ,  cov mats vs pooled , linear regression or not
%% dimensionality reduction : feature selection, feature extraction, masking : masques connus vs unconnus
%% misalignment : re-alignement
%
%\begin{tikzpicture}[ ->, node distance = 3cm,
%					decoration = {snake,   % <-- added
%                    pre length=3pt,post length=7pt,% <-- for better looking of arrow,
%                    }]
%
%\node [data] (db_profiling){\includegraphics[width = 0.3\textwidth]{figures/database.jpg}\\ Profiling traces};
%\node [data, right of = db_profiling] (db_profiling_realigned){\includegraphics[width = 0.3\textwidth]{figures/database.jpg}\\ \footnotesize{Realigned profiling traces}};
%\draw[arrow] (db_profiling) -- node[above]{$\rho$} (db_profiling_realigned) ;
%\node[function, above of = db_profiling, yshift=-1cm](realignment){$\rho$};
%\draw[arrow,dashed](db_profiling) -- (realignment);
%\node[function, above of= db_profiling_realigned, yshift=-1cm](extractor){$\varepsilon$};
%\node [data, right of = db_profiling_realigned] (extracted){\includegraphics[width = 0.3\textwidth]{figures/database.jpg}\\ Profiling traces' features};
%\draw[arrow] (db_profiling_realigned) -- node[above]{$\varepsilon$} (extracted) ;
%\draw[arrow,dashed](db_profiling_realigned) -- (extractor);
%\node[data, right of = extracted, xshift=0.8cm](distributions){\small{$\mathrm{Pr}(\vaLeakVec \mid Z)$}\\
%\small{$\mathrm{Pr}(Z \mid \vaLeakVec)$}};
%\draw[arrow,dashed](extracted) -- node[above]{characterisation} (distributions);
%
%\node[methods, below of= db_profiling, yshift=1cm](realignments){Realignment \cite{nagashima2007dpa,van2011improving,durvaux2012efficient}};
%\node[methods, below of= db_profiling_realigned, yshift=1cm](extraction_methods){Dimensionality reduction};
%\node[methods, below of= extracted, yshift=1cm, align=left](charac_methods){\tiny{
%Template Attack \cite{Chari2003} (generative model, Gaussian hypothesis)\\
%\emph{Pooled} variant \cite{choudary2014efficient}\\
%Stochastic variant \cite{schindler2005stochastic}\\
%Discriminative model
%}};
%\node [data, below of=realignments, yshift=1cm ] (db_attack){\includegraphics[width = 0.3\textwidth]{figures/database.jpg}\\ Attack traces};
%\node [data, right of = db_attack] (db_attack_realigned){\includegraphics[width = 0.3\textwidth]{figures/database.jpg}\\ Realigned attack traces};
%\draw[arrow] (db_attack) -- node[above]{$\rho$} (db_attack_realigned) ;
%\node [data, right of = db_attack_realigned] (extracted_attack){\includegraphics[width = 0.3\textwidth]{figures/database.jpg}\\ Attack traces' features};
%\draw[arrow] (db_attack_realigned) -- node[above]{$\varepsilon$} (extracted_attack) ;
%\node[data, right of = extracted_attack](keys){\footnotesize{$\mathrm{Pr}(K \mid \{ \vaLeakVec_i, P_i \}_{i=1,\dots,N})$}};
%\draw[arrow,dashed](extracted_attack) -- node[above]{inference} (keys);
%
%
%
%\end{tikzpicture}
%\end{frame}
%
%\begin{frame}
%\frametitle{Template Attack} 
%%% aprendo des volets per ogni passaggio con riferimenti allo stato dell'arte:
%%% templates : likelihood vs MAP ,  cov mats vs pooled , linear regression or not
%%% dimensionality reduction : feature selection, feature extraction, masking : masques connus vs unconnus
%%% misalignment : re-alignement
%%
%\begin{tikzpicture}[ ->, node distance = 2.5cm,
%				decoration = {snake,   % <-- added
%                   pre length=3pt,post length=7pt,% <-- for better looking of arrow,
%                   }]
%\uncover<4->{
%\node [data] (db_profiling){\includegraphics[width = 0.3\textwidth]{figures/database.jpg}\\ Profiling traces};
%\node [data, right of = db_profiling] (db_profiling_realigned){\includegraphics[width = 0.3\textwidth]{figures/database.jpg}\\ \footnotesize{Realigned profiling traces}};
%\draw[arrow] (db_profiling) -- node[above]{$\rho$} (db_profiling_realigned) ;
%\node[function, above of = db_profiling, yshift=-1cm](realignment){$\rho$};
%\draw[arrow,dashed](db_profiling) -- (realignment);
%\node[function, above of= db_profiling_realigned, yshift=-1cm](extractor){$\varepsilon$};
%\draw[arrow] (db_profiling_realigned) -- node[above]{$\varepsilon$} (extracted) ;
%\draw[arrow,dashed](db_profiling_realigned) -- (extractor);
%}
%
%\node [data, right of = db_profiling_realigned] (extracted){\includegraphics[width = 0.3\textwidth]{figures/database.jpg}\\ \only<1>{Profiling traces} \only<3->{Profiling traces's features}};
%\node[data, right of = extracted, xshift=0.8cm](distributions){\small{$\mathrm{Pr}(\vaLeakVec \mid Z)$}};
%\draw[arrow,dashed](extracted) -- node[above]{characterisation} (distributions);
%
%\node[methods, below of= db_profiling, yshift=1cm](realignments){Realignment \cite{nagashima2007dpa,van2011improving,durvaux2012efficient}};
%\node[methods, below of= db_profiling_realigned, yshift=1cm](extraction_methods){Dimensionality reduction};
%\node[methods, below of= extracted, yshift=1cm, align=left](charac_methods){\tiny{
%Template Attack \cite{Chari2003} (generative model)\\
%\uncover<3->{\emph{Pooled} variant \cite{choudary2014efficient}\\
%Stochastic variant \cite{schindler2005stochastic}}
%}};
%
%\only<4->{
%\node [data, below of=realignments, yshift=1cm ] (db_attack){\includegraphics[width = 0.3\textwidth]{figures/database.jpg}\\ Attack traces};
%\node [data, right of = db_attack] (db_attack_realigned){\includegraphics[width = 0.3\textwidth]{figures/database.jpg}\\ Realigned attack traces};
%\draw[arrow] (db_attack) -- node[above]{$\rho$} (db_attack_realigned) ;
%}
%\node [data, right of = db_attack_realigned] (extracted_attack){\includegraphics[width = 0.3\textwidth]{figures/database.jpg}\\ \only<1-2>{Attack traces} \only<3->{Attack traces' features}};
%\draw[arrow] (db_attack_realigned) -- node[above]{$\varepsilon$} (extracted_attack) ;
%\node[data, right of = extracted_attack](keys){Maximum Likelihood or Maximum A Posteriori};
%\draw[arrow,dashed](extracted_attack) -- node[above]{inference} (keys);
%
%
%
%\end{tikzpicture}
%\end{frame}
%