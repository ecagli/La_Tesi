
%\begin{frame}\frametitle{Setup and Implementation}
%Target device and acquisitions: 
%
%\begin{itemize}
%\item 8-bit AVR microprocessor Atmega328P
%\item power-consumption acquired via the ChipWhisperer \cite{o2014chipwhisperer} platform
%\end{itemize}
%
%
%
%Implementation: 
%
%\begin{itemize}
%\item Begin of an AES-128
%\end{itemize}
%
%
%
%Attack: 
%
%
%\begin{itemize}
%\item Target sensitive variable: $Z = \mathrm{Sbox}(P_0 \oplus K_0)$
%\item Acquisition of $N_p\times 256$ profiling traces, under key knowledge
%\item Estimation of $C$-dimensional Gaussian templates via the projection of the profiling traces over the $C$ projecting components
%\item Template attack with $N$ attack traces
%\end{itemize}
%
%\end{frame}
%
%
%\begin{frame} \frametitle{The Problem of Selecting PCA Components - EGV}
%
%% state of the art
%% first and sixth PC DPA contest
%\begin{columns}
%\begin{column}{0.1\textwidth}
%\includegraphics[width = \textwidth]{figures/questionmark.jpg} 
%\end{column}
%\begin{column}{0.7\textwidth}
%\begin{block}{}
%{\em How many} PCs and {\em which ones} are sufficient/necessary to reduce the traces size without losing important discriminant information?
%\end{block}
%\end{column}
%\end{columns}
%\pause
%\begin{block}{Explained Global Variance (EGV)}
%$\EGV{\AAlpha_i} = \frac{\lambda_i}{\sum_{k=1}^r \lambda_k}$\\
%\cite{choudaryefficient} : 
%\begin{itemize}
%\item fix a threshold $\beta$
%\item choose the first $C$ components, where $C$ is the minimum integer such that
%\begin{equation*}
%\EGV{\AAlpha_1}+ \EGV{\AAlpha_2}+\dots + \EGV{\AAlpha_C} \geq \beta
%\end{equation*}
%\end{itemize}
%\end{block}
%
%
%
%\end{frame}
%
%
%\begin{frame} \frametitle{The Problem of Selecting PCA Components - Experimental Observation}
%\begin{block}{Experimental Observation}
%\cite{Batina2012,specht}: the first components sometimes contain more noise than information; it is worth discarding them.
%\end{block}
%\begin{figure}
%\includegraphics[width=.45\textwidth]{figures/DPAcontestPC1_new.pdf} 
%\includegraphics[width=.45\textwidth]{figures/DPAcontestPC6_new.pdf} 
%\caption{First and sixth PCs in DPA contest v4  \cite{DPAcontest} trace set}\label{fig:DPAcontest}
%\end{figure}
%\end{frame}
%
%\begin{frame} \frametitle{The Problem of Selecting PCA Components - IPR}
%\begin{block}{Assumption}
%Dealing with secured devices, the leaking side-channel information is localised in few points of the acquired trace.
%\end{block}
%\pause
%\begin{block}{Inverse Participation Ratio (IPR)}
%\cite{SCAclassProbl}: under the same assumption, use the IPR to choose informative components
%\begin{equation*}
%\mathrm{IPR}(\AAlpha_i) = \sum_{j=1}^\traceLength \AAlpha_i[j]^4 \mbox{ \em (localization score)}
%\end{equation*}
%\end{block}
%\end{frame}
%
%\begin{frame} \frametitle{The Explained Local Variance (ELV) Selection (1)}
%
%What minds to perform the choice of the PCs to keep:
%\begin{table}
%\begin{tabular}{|c|c|c|c|}
%\hline
%& EGV & IPR & \uncover<2->{\textbf{ELV}} \\
%\hline
%associated eigenvalue ($\lambda_i$) &\includegraphics[scale=0.01]{figures/yes.png}  & \includegraphics[scale=0.01]{figures/no.png} &\uncover<2->{\includegraphics[scale=0.015]{figures/yes.png}}\\
%\hline
%component form (localization of $\AAlpha_i$) &\includegraphics[scale=0.01]{figures/no.png}  & \includegraphics[scale=0.01]{figures/yes.png}&\uncover<2->{\includegraphics[scale=0.015]{figures/yes.png}} \\
%\hline
%\end{tabular}
%\end{table}
%
%\uncover<3->{
%\begin{block}{Inspection of $\lambda_i$}
%\begin{scriptsize}
%
%\begin{align*}
%\lambda_i =& \textcolor{gray}{\hat{\mathrm{var}}(\sum_{j=1}^D \XXX^\intercal[j]\AAlpha_i[j]) = \sum_{j=1}^D\sum_{k=1}^D \hat{\mathrm{cov}}(\XXX^\intercal[j]\AAlpha_i[j], \XXX^\intercal[k]\AAlpha_i[k])=}\\
%\textcolor{gray}{=}& \textcolor{gray}{\sum_{j=1}^D \AAlpha_i[j]\sum_{k=1}^D\AAlpha_i[k]\hat{\mathrm{cov}}(\XXX^\intercal[j], \XXX^\intercal[k])= \sum_{j=1}^D \AAlpha_i[j] (\covmat_{j}^\intercal \cdot \AAlpha_i)= } \\
%\textcolor{gray}{=}& \textcolor{gray}{\sum_{j=1}^D \AAlpha_i[j] \lambda_i\AAlpha_i[j] }= \sum_{j=1}^D  \lambda_i \AAlpha_i[j]^2 
%\end{align*}
%\end{scriptsize}
%
%The $j$-th time sample contribution to $\lambda_i$ is given by $\lambda_i \AAlpha_i[j]^2$
%\end{block}
%}
%
%
%\end{frame}
%
%\begin{frame} \frametitle{The ELV Selection (2)}
%\vspace*{-0.5cm}
%\uncover<1->{
%\begin{block}{Definition}
%$\mathrm{ELV}(\AAlpha_i,j) = \frac{\lambda_i \AAlpha_i[j]^2}{\sum_{k=1}^r\lambda_k} = \mathrm{EGV}(\AAlpha_i) \AAlpha_i[j]^2$  \\
%\uncover<2->{Observe that $\sum_{j=1}^D \mathrm{ELV}(\AAlpha_i,j) = \EGV{\AAlpha_i}$}
%\end{block}
%}
%\uncover<3->{
%Perform this sum in a cumulative way, sorting the ELV contributions of the time samples in decreasing order, {\em i. e.} $\mathrm{ELV}(\AAlpha_i,j^i_1)\geq \mathrm{ELV}(\AAlpha_i,j^i_2)\geq \dots \geq \mathrm{ELV}(\AAlpha_i,j^i_\traceLength)$
%
%
%\vspace*{-0.4cm}
%\begin{columns}
%\begin{column}{.5\textwidth}
%\vspace*{10pt}
%\begin{figure}
%\includegraphics[width=\textwidth]{figures/PC1_points.pdf} 
%\end{figure}
%\end{column}
%\begin{column}{.5\textwidth}
%\begin{figure}
%
%\begin{tikzpicture}[remember picture,
%    scale=1,
%    % Define styles here
%    every node/.style={transform shape}
%    block/.style={
%        rectangle,
%        draw,
%        text centered,
%        rounded corners
%        },
%    data/.style={
%        trapezium,
%        trapezium left angle=60,
%        trapezium right angle=120,
%        draw
%        },
%    component/.style={
%        circle,
%        draw
%        },
%    output/.style={
%        tape,
%        tape bend top=none,
%        draw
%        },
%    edge/.style={
%        ->,
%        >=stealth,
%        thick
%        }
%    ]
%
%    \node (only1elv) at (0,0)
%    {\includegraphics[width=\textwidth]{figures/cumulativeELV_only1.pdf} };
%    \node [component, thick, xshift=2.1cm, yshift=0.8cm] (cerchio) {};
%    \node[below left=1cm of cerchio](caption){$\mathrm{EGV}(\AAlpha_1)$};
%    \draw[->] (caption) to (cerchio.south west);
%\end{tikzpicture}
%\end{figure}
%\end{column}
%\end{columns}
%
%
%}
%
%\end{frame}
%
%
%
%\begin{frame}
%\frametitle{The ELV Selection (3)}
%\begin{columns}
%\begin{column}{0.5\textwidth}
%\uncover<1->{
%\only<1>{
%\vspace*{-0.4cm}
%\begin{center}
%\includegraphics[width = \textwidth]{figures/cumulativeELV.pdf}
%\end{center}
%}
%\only<2>{
%\vspace*{-0.4cm}
%\begin{center}
%\includegraphics[width = \textwidth]{figures/cumulativeELVallRectangle.pdf} 
%\end{center}
%}
%
%\only<3->{
%\vspace*{-0.4cm}
%\begin{center}
%\includegraphics[width = \textwidth]{figures/cumulativeELVzoomed.pdf}
%\end{center}
%}
%}
%\uncover<4->{
%\only<1-4>{
%
%\begin{block}{To select $C$ components\hspace{\textwidth}\textcolor{white}{ }}
%Sort in decreasing order the maximal ELV provided by each component $\{\max_{j=1,\dots,D}\ELV(\AAlpha_i,j)\}_{i}$ and select the $C$ first components.
%\end{block}
%}
%\only<5>{
%
%\begin{block}{Fixing a cumulative explained variance threshold $\beta$}
%Select \textbf{couples} $(\AAlpha_i, j)$ in decreasing order wrt to $\ELV(\AAlpha_i, j)$ until $\ELV(\AAlpha_{i_1}, j_1)+ \ELV(\AAlpha_{i_2}, j_2)+\dots +\ELV(\AAlpha_{i_M}, j_M)\geq \beta$.\\
%%\uncover<5>{{\em Components denoising}}
%\end{block}
%}
%}
%
%\end{column}
%
%\begin{column}{0.5\textwidth}
%\only<1-3>{
%\begin{figure}
%\includegraphics[width=0.5\textwidth]{figures/PC1.pdf} 
%\includegraphics[width=0.5\textwidth]{figures/PC4.pdf} \\
%\includegraphics[width=0.5\textwidth]{figures/PC2.pdf} 
%\includegraphics[width=0.5\textwidth]{figures/PC5.pdf} \\
%\includegraphics[width=0.5\textwidth]{figures/PC3.pdf} 
%\includegraphics[width=0.5\textwidth]{figures/PC6.pdf} 
%\caption{\begin{footnotesize}
%The first 6 PCs: 
%$\lambda_1 \approx 3.8 ,\lambda_2 \approx 3.1 , \lambda_3 \approx2.6 ,\lambda_4 \approx 1.0 ,\lambda_5 \approx 0.8 ,\lambda_6 \approx 0.6 $
%\end{footnotesize}}
%
%\end{figure}
%}
%
%
%\only<4-5>{
%\begin{figure}
%\onslide<5->{\includegraphics[width=0.5\textwidth]{figures/PC5_denoised.pdf}}\only<4>{\includegraphics[width=0.5\textwidth]{figures/PC5_cerchio.pdf}}\only<5>{\includegraphics[width=0.5\textwidth]{figures/PC5_cerchio_transp.pdf}} \\
%\onslide<5->{\includegraphics[width=0.5\textwidth]{figures/PC4_denoised.pdf}}\only<4>{\includegraphics[width=0.5\textwidth]{figures/PC4_cerchio.pdf}}\only<5>{\includegraphics[width=0.5\textwidth]{figures/PC4_cerchio_transp.pdf}} \\
%\onslide<5->{\includegraphics[width=0.5\textwidth]{figures/PC6_denoised.pdf}}\only<4>{\includegraphics[width=0.5\textwidth]{figures/PC6_cerchio.pdf}}\only<5>{\includegraphics[width=0.5\textwidth]{figures/PC6_cerchio_transp.pdf}} 
%\only<4>{\caption{The 3 components chosen by ELV selection method - $C$ fixed}}
%\only<5>{\caption{Components and time samples chosen by ELV selection method - $\beta$ fixed}}
%\end{figure}
%}
%
%
%%\only<5>{
%%\includegraphics[width=0.31\textwidth]{ figures/PC5_cerchio_transp.pdf} 
%%\includegraphics[width=0.31\textwidth]{ figures/PC4_cerchio_transp.pdf} 
%%\includegraphics[width=0.31\textwidth]{ figures/PC6_cerchio_transp.pdf} \\
%%}
%%\uncover<5>{
%%
%%
%%
%%\only<3-4>{\caption{Selected components for $C = 3$; \hspace{\textwidth} $\ELV(\AAlpha_5, 2362)\approx 0.41$, $\ELV(\AAlpha_4, 1110)\approx 0.38$, $\ELV(\AAlpha_6, 1118)\approx 0.24$}}
%%\only<5>{\caption{Selected and denoised components for $\beta = 0.08$\hspace{\textwidth}\textcolor{white}{$\ELV(\AAlpha_5, 2362)\approx 0.41$, $\ELV(\AAlpha_4, 1110)\approx 0.38$, $\ELV(\AAlpha_6, 1118)\approx 0.24$}}}
%
%\end{column}
%
%\end{columns}
%
%
%\end{frame}
%
%%\begin{frame} \frametitle{The Component Selection Issue}
%%
%%\begin{columns}
%%\begin{column}{0.1\textwidth}
%%\includegraphics[width = \textwidth]{figures/questionmark.jpg} 
%%\end{column}
%%\begin{column}{0.7\textwidth}
%%\begin{block}{}
%%{\em How many} PCs and {\em which ones} are sufficient/necessary to reduce the traces size without losing important discriminant information? 
%%\end{block}
%%\end{column}
%%\end{columns}
%% \only<1>{
%% \begin{block}{Theoretically}
%%Higher eigenvalues $\longrightarrow$ higher information.
%%\end{block}
%%\begin{block}{Experimental Observation}
%%\cite{Batina2012,specht}: the first components sometimes contain no sensitive information; it is worth discarding them.
%%\end{block}
%%
%%\begin{figure}
%%\includegraphics[width=.25\textwidth]{figures/DPAcontestPC1_new.pdf} 
%%\includegraphics[width=.25\textwidth]{figures/DPAcontestPC6_new.pdf} 
%%\vspace{-10pt}
%%\caption{First and sixth PCs in DPA contest v4  \cite{DPAcontest} trace set}\label{fig:DPAcontest}
%%\end{figure}
%%}
%%\only<2>{
%%\vspace{40pt}
%%\begin{small}
%%\begin{table}
%%\begin{tabular}{|c|c|c|c|}
%%\hline
%%& EGV \cite{choudaryefficient} & IPR \cite{SCAclassProbl}& \uncover<2->{\textbf{ELV} \cite{Cagli2016}} \\
%%\hline
%%eigenvalue $\lambda_i$ &\includegraphics[scale=0.01]{figures/yes.png}  & \includegraphics[scale=0.01]{figures/no.png} &\uncover<2->{\includegraphics[scale=0.015]{figures/yes.png}}\\
%%\hline
%%form of $\AAlpha_i$ &\includegraphics[scale=0.01]{figures/no.png}  & \includegraphics[scale=0.01]{figures/yes.png}&\uncover<2->{\includegraphics[scale=0.015]{figures/yes.png}} \\
%%\hline
%%\end{tabular}
%%\end{table}
%%\end{small}
%%
%%\includegraphics[scale=0.3]{figures/citazione1.jpg} 
%%}
%%\uncover<2->{
%%\begin{block}{Explained Local Variance}
%%$\mathrm{ELV}(\AAlpha_i,j) = \frac{\lambda_i \AAlpha_i[j]^2}{\sum_{k=1}^r\lambda_k} = \mathrm{EGV}(\AAlpha_i) \AAlpha_i[j]^2$  \\
%%%($\sum_{j=1}^D \mathrm{ELV}(\AAlpha_i,j) = \EGV{\AAlpha_i}$)
%%\end{block}
%%}
%%\uncover<3->{\begin{block}{Use of the ELV}
%%\begin{itemize}
%%\item Sort in decreasing order the maximal ELV provided by each component $\{\max_{j=1,\dots,D}\ELV(\AAlpha_i,j)\}_{i}$ and select the $C$ first components.
%%\item Select \textbf{couples} $(\AAlpha_i, j)$ in decreasing order wrt to $\ELV(\AAlpha_i, j)$ until $\ELV(\AAlpha_{i_1}, j_1)+ \ELV(\AAlpha_{i_2}, j_2)+\dots +\ELV(\AAlpha_{i_M}, j_M)\geq \beta$
%%\end{itemize}
%%\end{block}}
%%\end{frame}
%%
%%\begin{frame} \frametitle{The ELV Selection (2)}
%%\vspace*{-0.5cm}
%%\uncover<1->{
%%\begin{block}{Definition}
%%$\mathrm{ELV}(\AAlpha_i,j) = \frac{\lambda_i \AAlpha_i[j]^2}{\sum_{k=1}^r\lambda_k} = \mathrm{EGV}(\AAlpha_i) \AAlpha_i[j]^2$  \\
%%\uncover<2->{Observe that $\sum_{j=1}^D \mathrm{ELV}(\AAlpha_i,j) = \EGV{\AAlpha_i}$}
%%\end{block}
%%}
%%\uncover<3->{
%%Perform this sum in a cumulative way, sorting the ELV contributions of the time samples in decreasing order, {\em i. e.} $\mathrm{ELV}(\AAlpha_i,j^i_1)\geq \mathrm{ELV}(\AAlpha_i,j^i_2)\geq \dots \geq \mathrm{ELV}(\AAlpha_i,j^i_\traceLength)$
%%
%%
%%\vspace*{-0.4cm}
%%\begin{columns}
%%\begin{column}{.5\textwidth}
%%\vspace*{10pt}
%%\begin{figure}
%%\includegraphics[width=\textwidth]{figures/PC1_points.pdf} 
%%\end{figure}
%%\end{column}
%%\begin{column}{.5\textwidth}
%%\begin{figure}
%%
%%\begin{tikzpicture}[remember picture,
%%    scale=1,
%%    % Define styles here
%%    every node/.style={transform shape}
%%    block/.style={
%%        rectangle,
%%        draw,
%%        text centered,
%%        rounded corners
%%        },
%%    data/.style={
%%        trapezium,
%%        trapezium left angle=60,
%%        trapezium right angle=120,
%%        draw
%%        },
%%    component/.style={
%%        circle,
%%        draw
%%        },
%%    output/.style={
%%        tape,
%%        tape bend top=none,
%%        draw
%%        },
%%    edge/.style={
%%        ->,
%%        >=stealth,
%%        thick
%%        }
%%    ]
%%
%%    \node (only1elv) at (0,0)
%%    {\includegraphics[width=\textwidth]{figures/cumulativeELV_only1.pdf} };
%%    \node [component, thick, xshift=2.1cm, yshift=0.8cm] (cerchio) {};
%%    \node[below left=1cm of cerchio](caption){$\mathrm{EGV}(\AAlpha_1)$};
%%    \draw[->] (caption) to (cerchio.south west);
%%\end{tikzpicture}
%%\end{figure}
%%\end{column}
%%\end{columns}
%%
%%
%%}
%%
%%\end{frame}
%%
%%
%%
%%%\begin{frame}
%%\frametitle{The ELV Selection (2)}
%%\begin{columns}
%%\begin{column}{0.5\textwidth}
%%\uncover<1->{
%%\only<1>{
%%\vspace*{-0.4cm}
%%\begin{center}
%%\includegraphics[width = \textwidth]{figures/cumulativeELV.pdf}
%%\end{center}
%%}
%%\only<2>{
%%\vspace*{-0.4cm}
%%\begin{center}
%%\includegraphics[width = \textwidth]{figures/cumulativeELVallRectangle.pdf} 
%%\end{center}
%%}
%%
%%\only<3->{
%%\vspace*{-0.4cm}
%%\begin{center}
%%\includegraphics[width = \textwidth]{figures/cumulativeELVzoomed.pdf}
%%\end{center}
%%}
%%}
%%\uncover<4->{
%%\only<1-4>{
%%
%%\begin{block}{To select $C$ components\hspace{\textwidth}\textcolor{white}{ }}
%%Sort in decreasing order the maximal ELV provided by each component $\{\max_{j=1,\dots,D}\ELV(\AAlpha_i,j)\}_{i}$ and select the $C$ first components.
%%\end{block}
%%}
%%\only<5>{
%%
%%\begin{block}{Fixing a cumulative explained variance threshold $\beta$}
%%Select \textbf{couples} $(\AAlpha_i, j)$ in decreasing order wrt to $\ELV(\AAlpha_i, j)$ until $\ELV(\AAlpha_{i_1}, j_1)+ \ELV(\AAlpha_{i_2}, j_2)+\dots +\ELV(\AAlpha_{i_M}, j_M)\geq \beta$.\\
%%%\uncover<5>{{\em Components denoising}}
%%\end{block}
%%}
%%}
%%
%%\end{column}
%%
%%\begin{column}{0.5\textwidth}
%%\only<1-3>{
%%\begin{figure}
%%\includegraphics[width=0.5\textwidth]{figures/PC1.pdf} 
%%\includegraphics[width=0.5\textwidth]{figures/PC4.pdf} \\
%%\includegraphics[width=0.5\textwidth]{figures/PC2.pdf} 
%%\includegraphics[width=0.5\textwidth]{figures/PC5.pdf} \\
%%\includegraphics[width=0.5\textwidth]{figures/PC3.pdf} 
%%\includegraphics[width=0.5\textwidth]{figures/PC6.pdf} 
%%\caption{\begin{footnotesize}
%%The first 6 PCs: 
%%$\lambda_1 \approx 3.8 ,\lambda_2 \approx 3.1 , \lambda_3 \approx2.6 ,\lambda_4 \approx 1.0 ,\lambda_5 \approx 0.8 ,\lambda_6 \approx 0.6 $
%%\end{footnotesize}}
%%
%%\end{figure}
%%}
%%
%%
%%\only<4-5>{
%%\begin{figure}
%%\onslide<5->{\includegraphics[width=0.5\textwidth]{figures/PC5_denoised.pdf}}\only<4>{\includegraphics[width=0.5\textwidth]{figures/PC5_cerchio.pdf}}\only<5>{\includegraphics[width=0.5\textwidth]{figures/PC5_cerchio_transp.pdf}} \\
%%\onslide<5->{\includegraphics[width=0.5\textwidth]{figures/PC4_denoised.pdf}}\only<4>{\includegraphics[width=0.5\textwidth]{figures/PC4_cerchio.pdf}}\only<5>{\includegraphics[width=0.5\textwidth]{figures/PC4_cerchio_transp.pdf}} \\
%%\onslide<5->{\includegraphics[width=0.5\textwidth]{figures/PC6_denoised.pdf}}\only<4>{\includegraphics[width=0.5\textwidth]{figures/PC6_cerchio.pdf}}\only<5>{\includegraphics[width=0.5\textwidth]{figures/PC6_cerchio_transp.pdf}} 
%%\only<4>{\caption{The 3 components chosen by ELV selection method - $C$ fixed}}
%%\only<5>{\caption{Components and time samples chosen by ELV selection method - $\beta$ fixed}}
%%\end{figure}
%%}
%%
%%
%%\only<5>{
%%\includegraphics[width=0.31\textwidth]{figures/PC5_cerchio_transp.pdf} 
%%\includegraphics[width=0.31\textwidth]{figures/PC4_cerchio_transp.pdf} 
%%\includegraphics[width=0.31\textwidth]{figures/PC6_cerchio_transp.pdf} \\
%%}
%%\uncover<5>{
%%
%%
%%
%%\only<3-4>{\caption{Selected components for $C = 3$; \hspace{\textwidth} $\ELV(\AAlpha_5, 2362)\approx 0.41$, $\ELV(\AAlpha_4, 1110)\approx 0.38$, $\ELV(\AAlpha_6, 1118)\approx 0.24$}}
%%\only<5>{\caption{Selected and denoised components for $\beta = 0.08$\hspace{\textwidth}\textcolor{white}{$\ELV(\AAlpha_5, 2362)\approx 0.41$, $\ELV(\AAlpha_4, 1110)\approx 0.38$, $\ELV(\AAlpha_6, 1118)\approx 0.24$}}}
%%
%%\end{column}
%%
%%\end{columns}
%%
%%
%%\end{frame}
%%