\section{Conclusions et Perspectives}\label{sec:conclusions}

Dans cette thèse, nous nous sommes concentrés sur les attaques par profilage. L'opportunité de caractériser les fuites de la cible ouvre les portes à des approches optimales, permettant l'estimation des distribution de probabilité conditionnelle nécessaire à identifié la clé secret par maximum \emph{a-posteriori}. Cependant, l'effort d'estimer les distributions de probabilité de données largement multivariées est empêché par la \emph{malédiction de la dimension}. Nos premiers effort se sont consacré au développement de technique de réduction de dimension, et nous avons proposé deux contributions à ce sujet. Dans une troisième contribution nous abordons la malédiction de la dimension à l'aide modèles par réseaux neuronaux. \\

Le fil rouge de cette thèse a été la croissante conscience du fait que les problèmes pratiques auxquelles on est confrontés dans le domaine des attaques par canaux auxiliaires sont presque identiques dans d'autres domaines d'application, et l'approche par apprentissage automatique s'est avéré gagnant dans plusieurs d'entre eux. Nous avons participé alors à une conversion des problématiques SCA, en passant d'une approche statistique classique, à une approche par apprentissage automatique. Nous croyons que cette conversion mérite d'être poursuivie dans le future. Une prochaine étape devrais \^etre la définition d'une t\^ache d'apprentissage automatique qui soit parfaitement adapté aux contexte des attaques avancée: en effet, jusqu'à présent, nous avons utilisé des outils dédiés à la t\^ache de la classification, qui ne correspond pas parfaitement à ce type d'attaque. Des métriques et des critères d'optimisation spécialisés pour les SCA devrait être proposés. Deuxièmement, dans l'optique de renforcer la résistance des dispositifs sécurisé, des méthodes d'analyse des modèles d'apprentissage profond menant à la réussite des attaque sont nécessaires, afin d'interpréter l'action de ces modèles, identifié les caractéristiques des signaux qui plus contribuent à leur réussite, retrouver et à terme éliminer les vulnérabilité de l'implémentation qui ont permis à ces caractéristiques de fuir à travers les canaux auxiliaires. 