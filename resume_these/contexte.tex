\section{Contexte}\label{sec:contexte}
\subsection{Le CESTI}
Les pr\'esents travaux de doctorat ont \'et\'e r\'ealis\'es au sein du laboratoire CESTI (Centre d'\'Evaluation de la S\'ecurit\'e des Syst\`emes d'Information) du CEA de Grenoble. La mission d'un CESTI est d'\'evaluer les aspects s\'ecuritaires des produits qui n\'ecessitent l'obtention d'un certificat pour pouvoir \^etre commercialis\'es sur certains march\'es sensibles. Les cartes \`a puce sont un exemple notable de tels types de dispositifs. Dans le sch\'ema de certification fran�ais, c'est l'ANSSI (Agence National de la S\'ecurit\'e des Syst\`emes d'Information) qui d\'elivre le certificat, apr\`es consultation d'un rapport issu d'un des laboratoires CESTI agr\'ees. \\
     

Un dispositif s\'ecuris\'e permet, dans la grande majorit\'e des cas, d'ex\'ecuter des algorithmes cryptographiques, pour offrir des garanties de confidentialit\'e, authenticit\'e, non-r\'epudiation et int\'egrit\'e des donn\'ees pour les protocoles d'interface avec ce m�me dispositif. Quand un algorithme cryptographique est impl\'ement\'e sur un support mat\'eriel, il devient potentiellement vuln\'erable \`{a} des attaques autres que celles consid\'er\'ees en cryptanalyse classique. En effet, en plus de la faiblesse math\'ematique th\'eorique de l'algorithme, il existe des faiblesses mat\'erielles li\'ees \`{a} l'impl\'ementation. Ces attaques mat\'erielles sont \`{a} prendre en compte dans l'\'evaluation s\'ecuritaire d'un produit s�curis�. Notamment, une partie du processus d'\'evaluation consiste \`{a} mener des attaques par canaux auxiliaires (ou \emph{Side-Channel Attacks} en anglais, d'o\`u l'acronyme SCA), qui font l'objet de cette th\`{e}se, et qui exploitent des fuites d'information issues de \emph{canaux auxiliaires}, c'est-\`a-dire autres que les interfaces I/O du composant.

\subsection{Les attaques par canaux auxiliaires}

Introduites en 1996 par Paul Kocher  \cite{kocher1996timing}, les attaques par canaux auxiliaires sont bas\'ees sur l'observation des variations de certaines quantit\'es physiques du composant, comme la consommation de puissance ou le rayonnement \'electromagn\'etique, pendant l'ex\'ecution des algorithmes cryptographiques. En effet, en observant ces comportements physiques involontaires, qui sont mesur�s sous forme de signaux, des d\'eductions sur les variables internes de l'algorithme peuvent \^{e}tre faites. 
% CD sugg
L'attaquant choisit ensuite, selon l'algorithme attaqu�, les variables internes, appel�es \emph{variables sensibles}, qui seront suffisantes pour inf�rer la cl� secr�te. 
%Selon l'algorithme attaqu\'e, faire inf\'erence sur des variables internes bien choisies, les dites \emph{variables sensibles}, est suffisant pour r\'ecup\'erer une cl\'e secr\`{e}te de l'algorithme. 

