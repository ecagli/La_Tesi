\section{Contexte}\label{sec:contexte}
\subsection{Le CESTI}
Les pr\'esents travaux de doctorat ont \'et\'e r\'ealis\'es au sein du laboratoire CESTI (Centre d'\'Evaluation de la S\'ecurit\'e des Syst\`emes d'Information) du CEA de Grenoble. La mission d'un CESTI est d'\'evaluer les aspects s\'ecuritaires des composantes embarqu\'es qui n\'ecessitent l'obtention d'un certificat pour pouvoir \^etre commercialis\'es sur certains march\'es sensibles. Les cartes \`a puces sont un exemple notable de tels types de dispositifs. Dans le sch\'ema de certification français, c'est l'ANSSI (Agence National de la S\'ecurit\'e des Syst\`emes d'Information) qui d\'elivre le certificat, apr\`es consultation d'un rapport issu d'un des laboratoires CESTI agr\'ees. \\
     

Un dispositif s\'ecuris\'e permets, dans la grande majorit\'e des cas, d'ex\'ecuter des algorithmes cryptographiques, pour offrir des garanties de confidentialit\'e, authenticit\'e, non-r\'epudiation et int\'egrit\'e des donn\'ees pour les protocoles d'interface avec le dispositif-m\^{e}me. Quand un algorithme cryptographique est impl\'ement\'e sur un support mat\'erielle, il d\'evient potentiellement vuln\'erable \`{a} des attaques autres que ceux consid\'er\'e en cryptanalyse classiques. En effet, outre \`{a} la faiblesse math\'ematique th\'eorique de l'algorithme, des faiblesse mat\'erielles li\'ees \`{a} l'impl\'ementation apparaissent. Ces attaques mat\'erielles sont \`{a} prendre en compte dans une \'evaluation s\'ecuritaire. Notamment, une partie du processus d'\'evaluation consistes \`{a} mener des attaques par canaux auxiliaires (ou \emph{Side-Channel Attacks} en anglais, d'o\`u l'acronyme SCA), qui font le sujet de cette th\`{e}se, et qui exploites des fuites d'information par des \emph{canaux auxiliaires}, c'est-\`a-dire outre que les interfaces I/O du composant.

\subsection{Les Attaques par Canaux Auxiliaires}

Introduites en 1996 par Paul Kocher  \cite{kocher1996timing}, les attaques par canaux auxiliaires sont bas\'ees sur l'observation des variations de certaines quantit\'es physiques du composant, comme la consommation de puissance, ou le rayonnement \'electromagn\'etique, pendant l'ex\'ecution des algorithmes cryptographiques. En effet, en observant ces comportements physiques involontaires, qui viennent mesurés sous forme de signaux, des d\'eductions sur les variables internes de l'algorithme peuvent \^{e}tre faites. Selon l'algorithme attaqu\'e, faire inf\'erence sur des variables internes bien choisies, les ceci-dites \emph{variables sensibles}, est suffisant pour r\'ecup\'erer une cl\'e secr\`{e}te de l'algorithme. 

