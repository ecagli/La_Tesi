
\section{Objectifs et Contributions}\label{sec:obj}
Dans un contexte d'\'{e}valuation d'un certain dispositif, les \'{e}valuateurs peuvent avoir acc\`{e}s \`{a} un ou plusieurs exemplaires du dispositif \emph{ouverts}, ou \emph{\`{a} secrets connus}. Ces dispositifs donnent droit \`{a} l'\'{e}valuateur de choisir ou conna\^{i}tre la cl\'{e} secret cible d'une attaque, ou de fixer d'autres variables, de d\'{e}sactiver des contre-mesurer, ou de charger du logiciel. Cette possibilit\'{e} est exploit\'{e}e pour lancer des ex\'{e}cutions dans lesquelles l'attaquant aurait la connaissance compl\`{e}te du flux d'ex\'{e}cution, y compris les op\'{e}rations, les variables intern\'{e}es manipul\'{e}es, les access aux registres, les al\'{e}as tirer internement, ... En cette mani\`{e}re il est capable de comprendre et caract\'{e}riser les relations entre le comportement interne du composant et les observations physiques, avant de lancer l'attaque. Quand une phase de caract\'{e}risation est disponible, on parle d'attaques \emph{profil\'{e}es}, qui ont un r\^{o}le tr\`{e}s important dans l'\'{e}valuation d'un dispositif, permettant de test\'{e} celui-ci dans le sc\'{e}nario le plus favorable pour l'attaquant. Cette th\`{e}se se concentre principalement sur cette typologie d'attaques. En effet, nous traitons les probl\`{e}mes qu'un \'{e}valuateur rencontre quand, dans un sc\'{e}nario si favorable, il veut exploiter de façon optimale la phase de caract\'{e}risation, pour extraire un maximum d'information des signaux acquis dans la phase propre d'attaque.Un de ces enjeux est la s\'{e}lection des ceci-dits \emph{Points d'Int\'{e}r\^{e}t} (\emph{Points of Interest} en anglais, ou PoIs), probl\`{e}me strictement reli\'{e} au plus g\'{e}n\'{e}ral probl\`{e}me de la r\'{e}duction de dimension.

\subsection{L'Avant-Propos de cette Th\`{e}se: la Recherche des Points d'Int\'{e}r\^{e}t}\label{sec:foreword}
L'acquisition des traces venant des canaux auxiliaires se fait habituellement \`{a} l'aide d'oscilloscopes num\'{e}riques, qui effectuent un \'{e}chantillonnage des signaux analogiques et les transforment en s\'{e}quences num\'{e}riques discr\`{e}tes. Ces s\'{e}quences sont souvent appel\'{e}s \emph{traces}, et leurs composants sont le \emph{caract\'{e}ristiques} temporelles, ou les points temporels, du signal. Pour garantir une inspection profonde du dispositif, la fr\'{e}quence d'\'{e}chantillonnage doit \^{e}tre tr\`{e}s \'{e}lev\'{e}e, ce qui provoque l'acquisition de traces de grand dimension. Cependant, il est attendu que seulement un nombre limit\'{e} de points temporels soit relevant pour mener une attaque. Ce sont les PoIs, qui sont les points qui d\'{e}pendent statistiquement de la variable sensible targette de l'attaque. En litt\'{e}rature l'utilisation de certains tests d'hypoth\`{e}se statistique est d\'{e}ploy\'{e}e pour effectuer une s\'{e}lection des PoI comme phase pr\'{e}liminaire d'une attaque. Cette s\'{e}lection permettrait de r\'{e}duire la complexit\'{e} de l'attaque, en terme de temps et m\'{e}moire. L'objectif pr\'{e}liminaire de cette th\`{e}se \'{e}tait de proposer de nouvelles m\'{e}thodes pour chercher et caract\'{e}riser les PoIs, pour am\'{e}liorer et possiblement optimiser ce pr\'{e}-traitement des traces consistant en leur s\'{e}lection.


\subsection{Approche per R\'{e}duction de Dimension}\label{sec:dim_red_objective}
Au-del\`{a} de l'utilisation de statistiques univari\'{e}es pour identifier les PoIs, un diff\'{e}rent axe de recherche s'est d\'{e}velopp\'{e} dans le contexte des SCAs, important du domaine de l'apprentissage automatique (ou \emph{Machine Learning}, ML) des t\'{e}chniques plus g\'{e}n\'{e}rales pour la r\'{e}duction de la dimension des donn\'{e}es, en passant d'une approche par s\'{e}lection de caract\'{e}ristiques \`{a} une approche par \emph{extraction de caract\'{e}ristiques}. Aux alentours du 2014, les m\'{e}thodes lin\'{e}aires d'extraction de caract\'{e}ristiques ont attir\'{e} l'attention des chercheurs, en proposant l'application de techniques telles que l'\emph{Analyse aux Composantes Principales} (PCA), l'\emph{Analyse Discriminante Lin\'{e}aire} (LDA) ou les \emph{Projection Pursuits} (PP). Ces m\'{e}thodes exploitent des combinaisons lin\'{e}aires avantageuses des points temporelles des traces, pour d\'{e}finir des nouvelles caract\'{e}ristiques amenant \`{a} des attaques plus efficaces. La premi\`{e}re contribution de cette th\`{e}se fait partie de cette axe de recherche: on a abord\'{e} deux enjeux concertants l'application de PCA et LDA dans le contexte SCA: le choix des composantes, et le probl\`{e}me de la taille de l'\'{e}chantillonnage. Les r\'{e}sultats de cette \'{e}tude, publi\'{e} en 2015 \`{a} CARDIS \cite{Cagli2016}, sont r\'{e}sum\'{e}s en Sec.~\ref{sec:lin} et font le sujet du chapitre 4 de la th\`ese.\\

Aujourd'hui, tout dispositif demandant un certificat s\'{e}curitaire de haut niveau est \'{e}quip\'{e} de contre-mesures sp\'{e}cifiques contre les SCAs. Une typologie de contre-mesure tr\`{e}s efficace est le \emph{masquage}. Quand un masquage est impl\'{e}ment\'{e} correctement, toute variable interne du calcul originaire qui est sensible, est divis\'{e}e en plusieurs parties, dont la majorit\'{e} est tir\'{e} au sort pendant l'ex\'{e}cution. Ceci est fait en sort que tout sous-ensemble propre des parties est statistiquement ind\'{e}pendant de la variable sensible elle-m\^{e}me. Le calcul cryptographique est men\'{e} an acc\'{e}dant uniquement aux parties, et non pas \`{a} la variable sensible. Ceci oblige l'attaquant \`{a} analyses des distributions de probabilit\'{e} conjointes des caract\'{e}ristiques signal, en \'{e}tudiant conjointement son comportement aux instants temporels o\`u chacune des parties est manipul\'{e}e. Autrement dit, les statistiques univari\'{e}es qui sont exploitable pour identifier les PoIs en absence de masquage deviennent inefficaces si un masquage est pr\'{e}sent, car tout point temporel du signal est par lui-m\^{e}me ind\'{e}pendant de la variable sensible. En outre, les distribution jointes du signal doivent \^{e}tre analys\'{e}es aux ordres statistiques sup\'{e}rieurs pour retrouver une d\'{e}pendance statistiques des donn\'{e}es sensibles. Ceci impliques que les m\'{e}thodes lin\'{e}aires d'extraction de caract\'{e}ristiques sont aussi inefficace en ce contexte. Pour r\'{e}sumer, la s\'{e}lection ou l'extraction de caract\'{e}ristiques depuis traces prot\'{e}g\'{e}es par masquage pr\'{e}sente des difficult\'{e}s non-n\'{e}gligeables. Cette complexit\'{e} est mitig\'{e}e quand l'attaquant peut effectuer une phase de caract\'{e}risation pendant laquelle il peut acc\'{e}der aux valeurs al\'{e}atoires des parties du masquage pendant l'ex\'{e}cution. En pratique, ceci n'est pas tout le temps possible. Dans cette th\`{e}se on abord ce sujet dans le cas o\`{u} cette possibilit\'{e} est ni\'{e}e, en proposant l'exploitation de la technique de l'\emph{Analyse Discriminante par Noyau} (\emph{Kernel Discriminant Analysis}, KDA). Ceci est une extension de la LDA qui permet d'extraire des caract\'{e}ristiques de façon non-lin\'{e}aire.  Les r\'{e}sultats obtenus dans ce contexte ont \'{e}t\'{e} publi\'{e}s \`{a} CARDIS 2016 \cite{cagli2016kernel} et sont r\'{e}sum\'{e}s en section~\ref{sec:kda}. Ils font le sujet du chapitre 5 de la th\`ese.

\subsection{Vers l'Apprentissage Profond}\label{sec:NN_intro} 

En observant le chemin que nous avons suivi pendant les travaux de th\`{e}se, on remarque que nous sommes partis du probl\`{e}me d'identifier les PoIs d'un signal, ce qui est classiquement r\'{e}solu par des outils statistiques classiques, et qu'ensuite nous avons  \'{e}largi \`{a} la fois les objectifs et les m\'{e}thodologies. En effet, que ce qui influen�ait le plus la r\'{e}ussite d'une attaque \'{e}tait la qualit\'{e} de l'extraction d'information. Extraire de l'information demande d'approximer des distributions de probabilit\'{e} qui permettent de distinguer diff\'{e}rent valeurs secr\`{e}tes. Les premi�res attaques par canaux auxiliaires propos\'{e}es en litt\'{e}rature op\'{e}rait point par point, donc n\'{e}cessitait d'analyser les distributions de donn\'{e}e en seulement quelques instants temporels pris s�parement. Dans ce contexte la s\'{e}lection des PoIs jouait un r\^{o}le fondamental. Cependant, d\`{e}s qu'on fait un pas en arri\`{e}re vers l'objectif d'une attaque, et qu'on se demande comment approximer des distributions distinguables, le fait de rejeter compl\`{e}tement une grande partie des caract\'{e}ristiques du signal, en n'en s\'{e}lectionnant que quelques unes, para�t du gaspillage. Des m\'{e}thodes appropri\'{e}es pour combiner ces caract\'{e}ristiques peuvent mener \`{a} l'extraction de caract\'{e}ristiques plus discriminantes. Pour d\'{e}terminer ces combinaisons appropri\'{e}es, nous avons explor\'{e} les outils d'extraction de caract\'{e}ristiques afin de les utiliser comme pr\'{e}-traitement du signal. En un premier temps, nous avons consid\'{e}r\'{e} des outils lin\'{e}aires, ensuite des g\'{e}n\'{e}ralisations non-lin\'{e}aires pour satisfaire une condition n\'{e}cessaire pour aborder les impl\'{e}mentations prot\'{e}g\'{e}es par masquage.\\ 

Conscients du fait que ces outils sont \`{a} mi-chemin entre les statistiques multivari\'{e}es classiques et le domaine de l'apprentissage automatique, nous avons commenc\'{e} \`{a} explorer ce domaine, qui est aujourd'hui en grand d\'{e}veloppement. Le grand int\'{e}r�t pour l'apprentissage automatique est justifi\'{e} par sa capacit� \`{a} capter et analyser donn\'{e}es de grande dimension dans une large vari\'{e}t\'{e} de champs applicatifs, y compris les attaques par canaux auxiliaires. Pour cela, des mod\`{e}les de plus en plus complexes sont mis en \{oe}uvre, trop complexes pour �tre trait\'{e}s dans un cadre de statistiques formelles. L'apprentissage automatique accepte des non-optimalit\'{e}s intrins\`{e}ques mais montre aujourd'hui d'excellents r\'{e}sultats.  \\

L'\'{e}tude des outils d'apprentissage automatique nous a men\'{e} � effectuer davantage un pas en arri\`{e}re vers l'objectif d'une attaque : plut\^{o}t qu'optimiser des pr\'{e}-traitement de donn\'{e}es, afin d'obtenir des caract\'{e}ristiques montrant des distributions facilement distinguables, nous pouvons chercher des mod\`{e}les pour approximer directement ces distributions \`{a} partir des donn\'{e}es brutes. Cette approche est propre d'une branche de l'apprentissage automatique, qui s'appelle apprentissage profond. Dans l'apprentissage profond la phase de caract\'{e}risation des donn\'{e}es est effectu\'{e}e en un seul processus, qui int\`{e}gre \'{e}ventuellement les pr\'{e}-traitements n\'{e}cessaires. Ceci est fait \`{a} l'aide de mod\`{e}les multi-couches, notamment les \emph{r\'{e}seaux neuronals} (\emph{Neural Networks}, NN), sur lesquels nous nous concentrons dans la derni\`{e}re partie de la th\`{e}se. \'{e}tant des mod\`{e}les non-lin\'{e}aires, les NN peuvent être utilis\'{e}s pour adresser la contremesure de masquage. De plus, des architectures particuli\`{e}res de NN, les dits r\'{e}seaux convolutifs (CNN), conçus originairement pour la reconnaissance d'images, s'adaptent aussi bien \`{a} d'autres types de contremesures: celles qui provoquent une d\'{e}synchronisation des signaux. Nous avons \'{e}tudi\'{e} ce contexte, en proposant l'utilisation des CNNs comme solution, munie d'une autre strat\'{e}gie classique dans le domaine de l'apprentissage automatique, l'\emph{augmentation des donn\'{e}es} (DA). Le chapitre 6 de la th\`ese est d\'edi\'e \`a ce sujet. Les r\'{e}sultats obtenus ont \'{e}t\'{e} publi\'{e}s \`{a} CHES 2017 \cite{DBLP:conf/ches/CagliDP17} et sont r\'{e}sum\'{e}s en section~\ref{sec:cnn}.