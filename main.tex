%%%%%%%%%%%%%%%%%%%%%%%%%%%%%%%%%%%%%%%%%
% Masters/Doctoral Thesis 
% LaTeX Template
% Version 2.4 (22/11/16)
%
% This template has been downloaded from:
% http://www.LaTeXTemplates.com
%
% Version 2.x major modifications by:
% Vel (vel@latextemplates.com)
%
% This template is based on a template by:
% Steve Gunn (http://users.ecs.soton.ac.uk/srg/softwaretools/document/templates/)
% Sunil Patel (http://www.sunilpatel.co.uk/thesis-template/)
%
% Template license:
% CC BY-NC-SA 3.0 (http://creativecommons.org/licenses/by-nc-sa/3.0/)
%
%%%%%%%%%%%%%%%%%%%%%%%%%%%%%%%%%%%%%%%%%

%----------------------------------------------------------------------------------------
%	PACKAGES AND OTHER DOCUMENT CONFIGURATIONS
%----------------------------------------------------------------------------------------

\documentclass[
11pt, % The default document font size, options: 10pt, 11pt, 12pt
%oneside, % Two side (alternating margins) for binding by default, uncomment to switch to one side
english, % ngerman for German
singlespacing, % Single line spacing, alternatives: onehalfspacing or doublespacing
%draft, % Uncomment to enable draft mode (no pictures, no links, overfull hboxes indicated)
%nolistspacing, % If the document is onehalfspacing or doublespacing, uncomment this to set spacing in lists to single
%liststotoc, % Uncomment to add the list of figures/tables/etc to the table of contents
%toctotoc, % Uncomment to add the main table of contents to the table of contents
%parskip, % Uncomment to add space between paragraphs
%nohyperref, % Uncomment to not load the hyperref package
headsepline, % Uncomment to get a line under the header
%chapterinoneline, % Uncomment to place the chapter title next to the number on one line
%consistentlayout, % Uncomment to change the layout of the declaration, abstract and acknowledgements pages to match the default layout
]{MastersDoctoralThesis} % The class file specifying the document structure

\usepackage[applemac]{inputenc} % Required for inputting international characters
\usepackage[T1]{fontenc} % Output font encoding for international characters

\usepackage{palatino} % Use the Palatino font by default

\usepackage[hyperref, backend=bibtex,style=alphabetic,natbib=true,backref,backrefstyle=none]{biblatex} % Use the bibtex backend with the authoryear citation style (which resembles APA)

\addbibresource{all_my_bib.bib} % The filename of the bibliography

\usepackage[autostyle=true]{csquotes} % Required to generate language-dependent quotes in the bibliography

%----------------------------------------------------------------------------------------
%	MARGIN SETTINGS
%----------------------------------------------------------------------------------------

\geometry{
	paper=a4paper, % Change to letterpaper for US letter
	inner=2.5cm, % Inner margin
	outer=3.8cm, % Outer margin
	bindingoffset=.5cm, % Binding offset
	top=1.5cm, % Top margin
	bottom=1.5cm, % Bottom margin
	%showframe, % Uncomment to show how the type block is set on the page
}

\linespread{1.3}

%----------------------------------------------------------------------------------------
%	MY NEW ENVIRONMENTS AND PACKAGES
%----------------------------------------------------------------------------------------
\usepackage{enumitem}
\usepackage[export]{adjustbox}
\usepackage{epigraph}


\usepackage{amsmath,amsfonts,amsthm,amssymb}
\usepackage{xcolor,colortbl}
\usepackage{subfigure}
\usepackage{multirow}
\usepackage{booktabs}
\usepackage{csquotes}
\usepackage{listings}


\usepackage{tikz}
\usetikzlibrary{mindmap,backgrounds}
\usetikzlibrary{shapes}
\usetikzlibrary{shapes.symbols}
\usetikzlibrary{decorations.pathreplacing}
\usetikzlibrary{decorations.markings}
\usetikzlibrary{arrows}
\usetikzlibrary{patterns}
\usetikzlibrary{calc}
\usetikzlibrary{positioning}
\usetikzlibrary{matrix}
\usetikzlibrary{automata}
\usetikzlibrary{decorations.pathmorphing}
\usetikzlibrary{shapes.callouts}
\usetikzlibrary{decorations.text}

\usetikzlibrary{tikzmark}
\usetikzlibrary{decorations.pathreplacing}
\usetikzlibrary{fadings}
\pgfkeys{/tikz/.cd,
  dec_j/.store in=\dec_j,
  dec_j=0   %% initial value, set to anything so that even if you don't specify a value later, it compiles
   }


\theoremstyle{remark}
\newtheorem{remark}{Remark}[chapter]
\newtheorem{lemma}{Lemma}{\bfseries}{\itshape}
\newtheorem{definition}{Definition}
\newtheorem{assumption}{Assumption}{\bfseries}{\itshape}
\newtheorem{property}{Property}{\bfseries}{\itshape}
\newtheorem{example}{Example}{\bfseries}{\itshape}
\newtheorem{notation}{Notation}{\bfseries}{\itshape}
\newtheorem{procedure}{Procedure}{\bfseries}{}

\newcommand*{\todo}[1]{\textbf{\textcolor{red}{#1}}}

\newenvironment{psmallmatrix}
  {\left[\begin{smallmatrix}}
  {\end{smallmatrix}\right]}
%----------------------------------------------------------------------------------------
%	MY NOTATIONS
%----------------------------------------------------------------------------------------
\newcommand*{\vLeak}{x}
\newcommand*{\vLeakVec}{\vec{x}}
\newcommand*{\vaLeak}{X}
\newcommand*{\setLeak}{\vec{\mathcal{X}}}
\newcommand*{\vaLeakVec}{\vec{X}}
\newcommand{\setLeakTrain}{\setLeak_{\text{train}}}
\newcommand{\setLeakTest}{\setLeak_{\text{test}}}
\newcommand{\setLeakValidation}{\setLeak_{\text{validation}}}
\newcommand{\setLeakProfiling}{\setLeak_{\text{profiling}}}
\newcommand{\setTargetTrain}{\setTarget_{\text{train}}}
\newcommand{\setTargetTest}{\setTarget_{\text{test}}}
\newcommand{\setTargetValidation}{\setTarget_{\text{validation}}}
\newcommand{\setTargetProfiling}{\setTarget_{\text{profiling}}}

\newcommand*{\vNNOutput}{\ensuremath {\vec{y}}}

\newcommand*{\setData}{\mathcal{D}}
\newcommand*{\sizeSetData}{\lvert\setData\rvert}
\newcommand*{\setDataTrain}{\mathcal{D}_{\text{train}}}
\newcommand*{\setDataTest}{\mathcal{D}_{\text{test}}}
\newcommand*{\setDataValidation}{\mathcal{D}_{\text{validation}}}
\newcommand*{\setDataProfiling}{\mathcal{D}_{\text{profiling}}}
\newcommand*{\setDataAttack}{\mathcal{D}_{\text{attack}}}
\newcommand*{\setTarget}{\mathcal{Y}}
%\newcommand*{\setDataKDA}{\mathcal{D}_{\text{KDA}}}
%\newcommand{\setLeakKDA}{\setLeak_{\text{KDA}}}
%\newcommand{\setTargetKDA}{\setTarget_{\text{KDA}}}


\newcommand*{\sPOI}{\tau} %signal strength


\newcommand*{\publicParRandVar}{E}
\newcommand*{\publicParVar}{e}
\newcommand*{\keyRandVar}{K}
\newcommand*{\keyVar}{k}
\newcommand*{\keyStar}{k^\star}
\newcommand*{\keyVarSet}{\mathcal{K}}
\newcommand*{\sensVar}{z}
\newcommand*{\sensRandVar}{Z}
\newcommand*{\sensVarValue}[1]{s_{#1}}
\newcommand*{\sensVarGenValue}{s}
\newcommand*{\sensVarOneHot}[1]{\vec{s_{#1}}}
\newcommand*{\sensVarSet}{\mathcal{Z}}
\newcommand*{\numClasses}{\lvert\sensVarSet\rvert}
\newcommand*{\nbClasses}{\lvert\sensVarSet\rvert}
\newcommand*{\sensFunction}{f}
\newcommand{\featureSpace}{\mathcal{F}}
\newcommand*{\distinguisher}{\Delta}


\newcommand*{\yyy}{\vec{y}}
\newcommand*{\YYY}{\vec{Y}}
\newcommand*{\www}{\vec{w}}
%\newcommand*{\WWW}{\textbf{W}}
\newcommand*{\vvv}{\vec{v}}
%\newcommand*{\VVV}{\textbf{V}}

\newcommand*{\nbTraces}{N}
\newcommand*{\nbTrainingTraces}{N_t}
\newcommand*{\nbProfilingTraces}{N_p}
\newcommand*{\nbProfilingTracesPerClass}{N_{p,\sensVarGenValue}}
\newcommand*{\nbAttackTraces}{N_a}
\newcommand*{\nbTracesPerClass}[1][\sensVarGenValue]{N_{#1}}
\newcommand*{\guessingEntropy}{\mathrm{GE}}
\newcommand*{\SR}{\mathrm{SR}}
\newcommand*{\newTraceLength}{C}
\newcommand*{\traceLength}{D}
\newcommand*{\AAlpha}{\vec{\alpha}}
\newcommand*{\BBeta}{\vec{\zeta}}
\newcommand*{\covmat}{\textbf{S}}
\newcommand*{\numPoI}{\sharp\mathrm{PoI}}
\newcommand*{\leakFunction}{\phi}
\newcommand*{\LeakFunction}{\boldsymbol{\phi}}
\newcommand*{\noise}{B}
\newcommand*{\Noise}{\vec{B}}
\newcommand*{\extract}{\epsilon}
\newcommand*{\orderRate}{o}
\newcommand*{\guessingVector}{\vec{g}}

\newcommand*{\entropy}{\mathbb{H}}
\newcommand*{\MLmodel}{F}
\newcommand*{\featureSpaceSize}{S}


\newcommand*{\leakageModel}{L}
\newcommand*{\HD}{\mathrm{HD}}
\newcommand*{\HW}{\mathrm{HW}}

\newcommand*{\prob}{\mathrm{Pr}}
\newcommand*{\pdf}{p}
\newcommand*{\mumumu}{\vec{\mu}}

\newcommand*{\measuresMatrix}{\textbf{M}}
\newcommand*{\projectingMatrix}{\textbf{A}}
\newcommand*{\SW}{\textbf{S}_{\textbf{W}}}
\newcommand*{\SB}{\textbf{S}_{\textbf{B}}}
\newcommand*{\ST}{\textbf{S}_{\textbf{T}}}
\newcommand*{\var}{\mathrm{Var}}
\newcommand*{\cov}{\mathrm{Cov}}
\newcommand*{\esperEst}{\hat{\mathbb{E}}}
\newcommand*{\varEst}{\hat{\mathrm{Var}}}
\newcommand*{\esper}{\mathbb{E}}
\newcommand*{\mmm}{\vec{m}}
\newcommand*{\mmmX}{\overline{\vec{x}}}
\newcommand*{\mmmXclass}[1][\sensVarGenValue]{\vec{\mu}_{#1}}
\newcommand*{\mmmY}{\overline{\vec{y}}}
\newcommand*{\varXclass}[1][\sensVarGenValue]{\vec{\varrho}_{#1}}

\newcommand{\MMM}{\textbf{M}}
\newcommand{\MMMclass}[1][\sensVarGenValue]{\vec{M}_{#1}}
\newcommand{\MMMT}{\vec{M}_{T}}
\newcommand{\NNN}{\textbf{N}}
\newcommand{\III}{\textbf{I}}
\newcommand{\numEigenvectors}{Q}
\newcommand{\kernelMatrix}{\textbf{K}}
\newcommand{\nununu}{\vec{\nu}}
\newcommand{\mmmXPhi}{\overline{\Phi(\vec{x}})}
\newcommand{\mmmXclassPhi}[1][z]{\overline{\Phi(\vec{x})}^{#1}}


\newcommand*{\aaa}{\textbf{a}}

\newcommand*{\Sbox}{\mathrm{Sbox}}
\newcommand*{\norm}[1]{\left\lVert#1\right\rVert}
\newcommand*{\softmax}{s}

%----------------------------------------------------------------------------------------
% EDITORIAL DEFINITIONS
%----------------------------------------------------------------------------------------
%\newcommand\given[1][\Big]{\:#1\vert\:}

\makeatletter
\newcommand{\@givenstar}[2]{#1\;\middle|\;#2}
\newcommand{\@givennostar}[3][]{#1#2\;#1|\;#3#1}
\newcommand{\given}{\@ifstar\@givenstar\@givennostar}
\makeatother

\DeclareMathOperator*{\argmin}{argmin}
\DeclareMathOperator*{\argmax}{argmax}

\def\etal{\textit{et al.} }
\newcommand {\ie}{{\em i.e.} }
\def\eg{\textit{e.g.} }
\def\via{\textit{via} }
\def\resp{{\em resp.} }
\def\th{$^\text{th}$}

%----------------------------------------------------------------------------------------
%	THESIS INFORMATION
%----------------------------------------------------------------------------------------

\thesistitle{Titre de la Th\`ese} % Your thesis title, this is used in the title and abstract, print it elsewhere with \ttitle
\supervisor{Emmanuel \textsc{PROUFF}} % Your supervisor's name, this is used in the title page, print it elsewhere with \supname
\examiner{} % Your examiner's name, this is not currently used anywhere in the template, print it elsewhere with \examname
\degree{Docteur} % Your degree name, this is used in the title page and abstract, print it elsewhere with \degreename
\author{Eleonora \textsc{CAGLI}} % Your name, this is used in the title page and abstract, print it elsewhere with \authorname
\addresses{} % Your address, this is not currently used anywhere in the template, print it elsewhere with \addressname

\subject{Informatique} % Your subject area, this is not currently used anywhere in the template, print it elsewhere with \subjectname
\keywords{} % Keywords for your thesis, this is not currently used anywhere in the template, print it elsewhere with \keywordnames
\university{\href{www.sorbonne-universite.fr}{Sorbonne Universit\'e}} % Your university's name and URL, this is used in the title page and abstract, print it elsewhere with \univname
\department{\href{http://department.university.com}{Department or School Name}} % Your department's name and URL, this is used in the title page and abstract, print it elsewhere with \deptname
\group{\href{http://researchgroup.university.com}{Research Group Name}} % Your research group's name and URL, this is used in the title page, print it elsewhere with \groupname
\faculty{\href{http://faculty.university.com}{Faculty Name}} % Your faculty's name and URL, this is used in the title page and abstract, print it elsewhere with \facname

\AtBeginDocument{
\hypersetup{pdftitle=\ttitle} % Set the PDF's title to your title
\hypersetup{pdfauthor=\authorname} % Set the PDF's author to your name
\hypersetup{pdfkeywords=\keywordnames} % Set the PDF's keywords to your keywords
}

\begin{document}

\frontmatter % Use roman page numbering style (i, ii, iii, iv...) for the pre-content pages

\pagestyle{plain} % Default to the plain heading style until the thesis style is called for the body content

%----------------------------------------------------------------------------------------
%	TITLE PAGE
%----------------------------------------------------------------------------------------

\begin{titlepage}
\includegraphics[height = 20mm]{LOGO_SU.jpg}
\hfill
\includegraphics[height = 20mm]{LOGO_LIP6}
\begin{center}

\vspace*{.04\textheight}
{\scshape\huge Sorbonne Universit\'e\par}\vspace{1.0cm} % University name
\textsc{\Large Ecole Doctorale n$^\circ$ 130 }\\
\textsc{\large Informatique, T\'el\'ecommunications, \'Electronique de Paris}\\[0.3cm]
\textsc{\large Laboratoire d'Informatique de Paris 6 }\\[0.5cm]


\HRule \\[0.3cm] % Horizontal line
{\huge \bfseries \ttitle\par}\vspace{0.4cm} % Thesis title
{\LARGE \bfseries Sous-titre \par}\vspace{0.4cm} % Thesis title
\HRule \\[1.0cm] % Horizontal line

\textsc{\Large Par }\href{}{\large \authorname}\\
\textsc{Th\`ese de Doctorat de Informatique???}

\vspace*{.04\textheight}
\textsc{Dirig\'ee par }\href{}{\supname}\\
\textsc{Encadr\'ee par} C\'ecile \textsc{DUMAS}
% 
%\begin{minipage}[t]{0.4\textwidth}
%\begin{flushleft} \large
%\textsc{Par}
%\href{}{\authorname}\\ % Author name - remove the \href bracket to remove the link
%\textsc{Th\`ese de Doctorat de Informatique???}
%\end{flushleft}
%\end{minipage}
%\begin{minipage}[t]{0.4\textwidth}
%\begin{flushright} \large
%\emph{Dirig\'ee par}
%\href{}{\supname}\\% Supervisor name - remove the \href bracket to remove the link  
%\emph{Encadr\'ee par} C\'ecile \textsc{DUMAS}
%\end{flushright}
%\end{minipage}\\[3cm]

\vspace*{.04\textheight}
\emph{Pr\'esent\'ee et soutenue publiquement le JJ/MM/AAAA}
\end{center}


\begin{minipage}[t]{\textwidth}
\emph{Devant un jury compos\'e de:}\\
Pr\'esident du Jury \hfill \textsc{Aaa BBB, } \textit{CCC}\\
Rapporteur \hfill \textsc{Aaa BBB, } \textit{CCC}\\
Rapporteur \hfill \textsc{Ddd EEE, } \textit{FFF}\\
Examinateur \hfill \textsc{Ggg HHH, } \textit{III}\\
Examinateur \hfill \textsc{Lll Mmm, } \textit{Nnn}\\
Directeur de th\`ese \hfill \textsc{Emmanuel PROUFF,} \textit{ANSSI}\\
Encadrante \hfill \textsc{C\'ecile DUMAS,} \textit{CEA Grenoble}

\end{minipage}
%\large \textit{A thesis submitted in fulfillment of the requirements\\ for the degree of \degreename}\\[0.3cm] % University requirement text
%\textit{in the}\\[0.4cm]
%\groupname\\\deptname\\[2cm] % Research group name and department name
% 
%\vfill
%
%{\large \today}\\[4cm] % Date
%%\includegraphics{Logo} % University/department logo - uncomment to place it
% 
%\vfill

\end{titlepage}

%----------------------------------------------------------------------------------------
%	DECLARATION PAGE
%----------------------------------------------------------------------------------------

%\begin{declaration}
%\addchaptertocentry{\authorshipname} % Add the declaration to the table of contents
%\noindent I, \authorname, declare that this thesis titled, \enquote{\ttitle} and the work presented in it are my own. I confirm that:
%
%\begin{itemize} 
%\item This work was done wholly or mainly while in candidature for a research degree at this University.
%\item Where any part of this thesis has previously been submitted for a degree or any other qualification at this University or any other institution, this has been clearly stated.
%\item Where I have consulted the published work of others, this is always clearly attributed.
%\item Where I have quoted from the work of others, the source is always given. With the exception of such quotations, this thesis is entirely my own work.
%\item I have acknowledged all main sources of help.
%\item Where the thesis is based on work done by myself jointly with others, I have made clear exactly what was done by others and what I have contributed myself.\\
%\end{itemize}
% 
%\noindent Signed:\\
%\rule[0.5em]{25em}{0.5pt} % This prints a line for the signature
% 
%\noindent Date:\\
%\rule[0.5em]{25em}{0.5pt} % This prints a line to write the date
%\end{declaration}
%
%\cleardoublepage

%----------------------------------------------------------------------------------------
%	QUOTATION PAGE
%----------------------------------------------------------------------------------------

%\vspace*{0.2\textheight}
%
%\noindent\enquote{\itshape Thanks to my solid academic training, today I can write hundreds of words on virtually any topic without possessing a shred of information, which is how I got a good job in journalism.}\bigbreak
%
%\hfill Dave Barry

%----------------------------------------------------------------------------------------
%	ABSTRACT PAGE
%----------------------------------------------------------------------------------------

%\begin{abstract}
%\addchaptertocentry{\abstractname} % Add the abstract to the table of contents
%The Thesis Abstract is written here (and usually kept to just this page). The page is kept centered vertically so can expand into the blank space above the title too\ldots
%\end{abstract}

%----------------------------------------------------------------------------------------
%	ACKNOWLEDGEMENTS
%----------------------------------------------------------------------------------------

%\begin{acknowledgements}
%\addchaptertocentry{\acknowledgementname} % Add the acknowledgements to the table of contents
%The acknowledgments and the people to thank go here, don't forget to include your project advisor\ldots
%\end{acknowledgements}

%----------------------------------------------------------------------------------------
%	LIST OF CONTENTS/FIGURES/TABLES PAGES
%----------------------------------------------------------------------------------------

\tableofcontents % Prints the main table of contents

\listoffigures % Prints the list of figures

\listoftables % Prints the list of tables

%----------------------------------------------------------------------------------------
%	ABBREVIATIONS
%----------------------------------------------------------------------------------------

\begin{abbreviations}{ll} % Include a list of abbreviations (a table of two columns)
\textbf{AES} & \textbf{A}dvanced \textbf{E}ncryption \textbf{S}tandard\\
\textbf{ANSSI} & \textbf{A}gence \textbf{N}ational de la \textbf{S}\'ecurit\'e des \textbf{S}yst\`emes d' \textbf{I}nformation \\
\textbf{CC} & \textbf{C}ommon \textbf{C}riteria\\

\textbf{CESTI} & \textbf{C}entre d'\textbf{E}valuation de la \textbf{S}\'ecurit\'e des \textbf{T}\'echnologies de l'\textbf{I}nformation\\
\textbf{CNN} & \textbf{Convolutional} \textbf{N}eural \textbf{N}etwork\\
\textbf{CPA} & \textbf{C}orrelation \textbf{P}ower \textbf{A}nalysis \\

\textbf{DA} & \textbf{D}ata \textbf{A}ugmentation\\
\textbf{DL} & \textbf{D}eep \textbf{L}earning\\
\textbf{DoM} & \textbf{D}ifference \textbf{o}f \textbf{M}eans \\

\textbf{DPA} & \textbf{D}ifferential \textbf{P}ower \textbf{A}nalysis \\

\textbf{EAL} & \textbf{E}valuation \textbf{A}ssurance \textbf{L}evels \\
\textbf{EGV} & \textbf{E}xplained \textbf{G}lobal \textbf{Variance}\\
\textbf{ELV} & \textbf{E}xplained \textbf{L}ocal \textbf{Variance}\\


\textbf{ETR} & \textbf{E}valuation \textbf{T}echnical \textbf{R}apport\\
\textbf{GE} & \textbf{G}uessing \textbf{E}ntropy\\
\textbf{GPU} & \textbf{G}raphic \textbf{P}rocessing \textbf{U}nit\\
\textbf{HMM} & \textbf{H}idden \textbf{M}arcov \textbf{M}odel \\

\textbf{HOSCA} & \textbf{H}irer-\textbf{O}rder \textbf{S}ide \textbf{C}hannel \textbf{A}ttack\\

\textbf{IPR} & \textbf{I}nverse \textbf{P}articipation \textbf{R}atio\\

\textbf{ITSEF} & \textbf{I}nformation \textbf{T}echnology \textbf{S}ecurity \textbf{E}valuation \textbf{F}acility \\
\textbf{KDA} & \textbf{K}ernel Fisher \textbf{D}iscriminant \textbf{A}nalysis\\

\textbf{LDA} & \textbf{L}inear \textbf{D}iscriminant \textbf{A}nalysis\\
\textbf{LDC} & \textbf{L}inear \textbf{D}iscriminant \textbf{C}omponent\\

\textbf{MIA} & \textbf{M}utual \textbf{I}formation \textbf{A}nalysis\\
\textbf{MMPC} & \textbf{M}oment against \textbf{M}oment \textbf{P}rofiling \textbf{C}orrelation \\

\textbf{ML} & \textbf{M}achine \textbf{L}earning \ /  \ \textbf{M}aximum-\textbf{L}ikelihood \\
\textbf{MLP} & \textbf{M}ulti-\textbf{L}ayer \textbf{P}erceptron \\

\textbf{MSE} & \textbf{M}ean \textbf{S}quared \textbf{E}rror\\

\textbf{NIST} & \textbf{N}ational \textbf{I}nstitute of \textbf{S}tandards and \textbf{T}echnology \\
\textbf{NN} & \textbf{N}eural \textbf{N}etwork\\

\textbf{PC} & \textbf{P}rincipal \textbf{C}omponent\\

\textbf{PCA} & \textbf{P}rincipal \textbf{C}omponents \textbf{A}nalysis\\
\textbf{PoI} & \textbf{P}oint \textbf{o}f \textbf{I}nterest\\
\textbf{PP} & \textbf{P}rojection \textbf{P}ursuits\\

\textbf{PV} & \textbf{P}rincipal \textbf{V}ariable\\
\textbf{RDI} &\textbf{R}andom \textbf{D}elay \textbf{I}nterrupt \\

\textbf{SAR} &\textbf{S}ecurity \textbf{A}ssurance \textbf{R}equirements \\

\textbf{SCA} & \textbf{S}ide \textbf{C}hannel \textbf{A}ttack\\
\textbf{SFR} & \textbf{S}ecurity \textbf{F}unctional \textbf{R}equirements \\
\textbf{SNR} & \textbf{S}ignal-to-\textbf{N}oise-\textbf{R}atio \\

\textbf{SoD} & \textbf{S}um \textbf{o}f \textbf{D}ifferences\\
\textbf{SoSD} & \textbf{S}um \textbf{o}f \textbf{S}quared \textbf{D}ifferences\\
\textbf{SoST} & \textbf{S}um \textbf{o}f \textbf{S}quared \textbf{T}-statistics\\

\textbf{SPA} & \textbf{S}imple \textbf{P}ower \textbf{A}nalysis \\
\textbf{SR} & \textbf{S}uccess \textbf{R}ate\\

\textbf{SSS} & \textbf{S}mall \textbf{S}ample \textbf{S}ize problem\\
\textbf{SVM} & \textbf{S}upport \textbf{V}ector \textbf{M}achine \\

\textbf{TA} & \textbf{T}emplate \textbf{A}ttack \\
\textbf{TDNN} & \textbf{T}ime-\textbf{D}elayed \textbf{N}eural \textbf{N}etwork \\

\textbf{TOE} & \textbf{T}arget \textbf{O}f \textbf{E}valuation \\

\end{abbreviations}

%----------------------------------------------------------------------------------------
%	PHYSICAL CONSTANTS/OTHER DEFINITIONS
%----------------------------------------------------------------------------------------

%\begin{constants}{lr@{${}={}$}l} % The list of physical constants is a three column table
%
%% The \SI{}{} command is provided by the siunitx package, see its documentation for instructions on how to use it
%
%Speed of Light & $c_{0}$ & \SI{2.99792458e8}{\meter\per\second} (exact)\\
%%Constant Name & $Symbol$ & $Constant Value$ with units\\
%
%\end{constants}

%----------------------------------------------------------------------------------------
%	SYMBOLS
%----------------------------------------------------------------------------------------

%\begin{symbols}{lll} % Include a list of Symbols (a three column table)
%
%$GF(2^b)$ & Galois Field of order $2^b$\\
%$\mathbb{Z}_N^{*}$ & ...\\
%%$a$ & distance & \si{\meter} \\
%%$P$ & power & \si{\watt} (\si{\joule\per\second}) \\
%%%Symbol & Name & Unit \\
%%
%%\addlinespace % Gap to separate the Roman symbols from the Greek
%%
%%$\omega$ & angular frequency & \si{\radian} \\
%
%\end{symbols}

%----------------------------------------------------------------------------------------
%	DEDICATION
%----------------------------------------------------------------------------------------

%\dedicatory{For/Dedicated to/To my\ldots} 

%----------------------------------------------------------------------------------------
%	THESIS CONTENT - CHAPTERS
%----------------------------------------------------------------------------------------

\mainmatter % Begin numeric (1,2,3...) page numbering

\pagestyle{thesis} % Return the page headers back to the "thesis" style

% Include the chapters of the thesis as separate files from the Chapters folder
% Uncomment the lines as you write the chapters
\part{Context and State of the Art}
% Chapter Template

\chapter{Introduction} % Main chapter title

\label{ChapterIntroduction}

%----------------------------------------------------------------------------------------
%	SECTION 1
%----------------------------------------------------------------------------------------

\section{Introduction to Cryptography}
The terms \emph{Cryptography} (from the Greek \emph{krypt\`os} (secret) and \emph{graphein} (writing)) and \emph{Cryptanalysis}, denotes two branches of a science named \emph{Cryptology}, or \emph{science of the secret}. Cryptography initially refers to the art of \emph{encrypting} messages, which means writing meaningful messages in such a way to appear nonsense to anyone unaware of the encryption process. In general, cryptography aims to construct protocols to secure communication, while cryptanalysis studies the resistance of cryptographic techniques, developing \emph{attacks} to break the cryptosystems' security claims. These two complementary domains evolve in parallels, since the evolution of attack techniques allows conceiving more resistant cryptographic algorithms, and inversely the resistance of such algorithms requires the conception of more sophisticated attacks.\\

The art of cryptography is very ancient, probably as ancient as the language, but only the development of information technology made cryptology take the shape of a proper science, sometimes referred to as \emph{Modern cryptology}. The last be seen as a branch of different disciplines, such as applied mathematics, computer science, electrical engineering, and communication science. Modern cryptosystems exploit algorithms based on mathematical tools and are implemented as computer programs, or electronic circuits. Their goal is to provide security functionality for communications that use \emph{insecure channels}, for example the internet. In particular, modern cryptosystems are designed in order to ensure at least one of the four following information security properties:
\begin{itemize}
\item[a.] \emph{confidentiality}: the transmitted message must be readable only by a chosen pool of authorized entities;
\item[b.] \emph{authenticity}: the receiver can verify the identity of the sender of a message;
\item[c.] \emph{non-repudiation}: the sender of a message cannot deny having sent the message afterwards;
\item[d.] \emph{data integrity}: the receiver can be convinced that the message has not been corrupted during the transmission.


\end{itemize} 

Two branches of cryptography may be distinguished: the \emph{symmetric cryptography} and the \emph{asymmetric cryptography}. The first one historically appeared before and is based on the hypothesis that the two communicating entities share a common secret, or private key; for this reason this is also called \emph{secret key cryptography}. The second one, introduced around 1970, allows any entity to encrypt a message in such a way that only a unique chosen other entity could decrypt it; this is also called \emph{public key cryptography}. \\

A general principle in cryptography, nowadays widely accepted by cryptography researchers, is the one given by Kerckhoff in in 19th century: it states that cryptosystems should be secure even if everything about the system, except the key, is public knowledge. Following this principle, today many industrials and governmental agencies exploit for their security services cryptosystems based over standardized algorithms. Such algorithms are of public domain, thus have been tested and tried to be broken by a large amount of people, before, during and after the standardization process. Resistance to many attempts of attacks is actually the strengths of standard algorithms.\\

In the following part of this section a description of the two standard cryptographic primitives, \emph{i.e.} building block algorithms used to build cryptographic protocols, that will be used in this thesis will; a symmetric one, the AES, and an asymmetric one, the RSA. 
\subsection{Description of AES}
The \emph{Advanced Encryption Standard} (AES) has been standardized in 2001 by the United States governmental agency \emph{National Institute of Standards and Technology} (NIST) through the \emph{Federal Information
Processing Standards Publication 197 } (FIPS PUB 197) \cite{nist197}. It is a symmetric \emph{block cipher}, \emph{i.e.} an algorithm operating on fixed-length groups of bits.\footnote{in contrast with \emph{stream ciphers}, which operate over a single plaintext bit at time} The AES operates on blocks of 128 bits of plaintext, and can use keys of size 128, 192 or 256 bits. The encryption is done by rounds. The number of executed rounds depends on the key size (10 rounds for 128 bits, 12 for 192 and 14 pour 256). The basic unit for processing in the AES algorithm is a byte. For AES internal operations, bytes are arranged on a two-dimensional array of bytes called the \emph{state}, denoted $s$. Such a state has 4 rows and 4 columns, thus contains 16 bytes. The byte lying at the $i$-th row, $j$-th column of $s$ will be denoted by $s_{i,j}$ for $i,j\in\{0,1,2,3\}$. The 16 input bytes and the 16 output bytes are indexed column-wise as shown in Fig.~\ref{fig:AES_state}. Each element $s_{i,j}$ of a state is mathematically seen as an element of the \emph{Rjindael finite field}, defined as $GF(2^8) = \mathbb{Z}/{2\mathbb{Z}[X]}/P(X)$ where $P(X) = X^8 + X^4 + X^3 + X + 1$. Five functions are performed during the AES, named KeySchedule, AddRoundKey, SubBytes, ShiftRow and MixColumn. At high level the AES algorithm is described hereafter:
\begin{itemize}
\item[\textbf{Key Expansion:}]  derivation of round keys from secret key through the KeySchedule function
\item[\textbf{Round 0:} ] 
\begin{itemize}
\item[] AddRoundKey
\end{itemize}
\item[\textbf{Rounds 1 to penultimate:}] 
\begin{itemize}
\item[] SubBytes
\item[] ShiftRows
\item[] MixColumns
\item[] AddRoundKey
\end{itemize}
\item[\textbf{Last Round:}] 
\begin{itemize}
\item[] SubBytes
\item[] ShiftRows
\item[] AddRoundKey
\end{itemize}
\end{itemize}

\begin{figure}
\includegraphics[width = \textwidth]{../Figures/FISP_AES/state.png} 
\caption[State array input and output.]{State array input and output. Source: \cite{nist197}.}\label{fig:AES_state}
\end{figure}

A description of the five functions is provided hereafter.

\subsubsection*{KeySchedule}
The key round of the initial round of AES coincides with the secret encryption key $\boldsymbol{K} = (k_{0,0},k_{0,1},\dots,k_{0,3}, k_{1,0},\dots,k_{1,3},\dots,k_{3,3})$. The $i$-th round key is given by 
\begin{equation*}
\boldsymbol{K_i} = (k_{4i,0},k_{4i,1},\dots,k_{4i,3}, k_{4i+1,0},\dots,k_{4i+1,3},\dots,k_{4i+3,3}),
\end{equation*}
where, for $i>3$
\begin{equation*}
\begin{cases}
k_{a,b} = k_{a-4,b}\oplus k_{a-1,b} & \mbox{if } a \not\equiv 0 \mbox{ mod } 4\\
k_{a,b} = k_{a-4,b}\oplus \Sbox(k_{a-1,(b+1) \mbox{mod } 4}) \oplus \mathrm{Rcon}(a) & \mbox{if } a \equiv 0 \mbox{ mod } 4 \mathrm{,}
\end{cases}
\end{equation*}

where $Rcon(a) = 2^{a-1}$ in the Rjindael finite field.\footnote{where $2=(00000010)_2$ is represented by the polynomial $x$}

\subsubsection*{AddRoundKey}
\subsubsection*{SubBytes}
\subsubsection*{ShiftRows}
\subsubsection*{MixColumns}


\subsection{Description of RSA}


%----------------------------------------------------------------------------------------
%	SECTION 2
%----------------------------------------------------------------------------------------
\section{Secured Components}
\subsection{Smart Cards and Related Devices}
As we have seen in the previous section, modern cryptography proposes solutions to secure communications that ask for electronic computations and repose their security over some secret keys. Keys are represented as long bit strings, impossible to be memorized by users. Thus, keys need to be stored in a secure medium, and never delivered in clear over insecure channels. Smart cards were historically conceived as a practical solution to such a key storage issue: they consist in small devices a user can easily carry around with, which not only store secret keys, but also are able to internally perform cryptographic operations, in such a way that keys are never asked to be delivered. The registrations of a first patent by Roland Moreno in 1974 and of a second one by Michel Ugon in 1977 are often referred to date the smart card invention, finally produced for the first time in 1979. Smart cards are pocket-sized plastic-made cards equipped with a secured component, which is typically an integrated circuit containing a some computational units and some memories.\\

Today, about 40 years after its invention, they still have a huge diffusion, both in terms of applicative domains and in terms of quantity of exemplars. Indeed, they serve as credit or ATM cards, healthy cards, ID cards, public transport payment cards, fuel cards, identification and access badges, authorization cards for pay television, etc. Slightly changing the card support, we find other applications of the same kind of integrated circuits, for example the  mobile phone SIMs (\emph{Subscriber Identity Module} and the electronic passports. In terms of quantity, it seems that in 2014 8.8 billion smart cards have been sold, \emph{i.e.} the same order of magnitude of the global population. \\

In addition to smart cards, the recent growing and variation of security needs lead to the development and specification of other kinds of secured components, for example the \emph{Trusted Execution Environment} (TEE), which as a secured part of the main processor of a smartphone or tablet, and the \emph{Trusted Platform Module} 
(TPM), which is a secure element providing cryptographic functionalities to a motherboard. 

\subsection{Embedded Cryptography Vulnerabilities}
\subsubsection{Side-Channel Attacks}
Until the middle of nineties, the security of embedded cryptosystems was considered as equivalent to the mathematical security of the cryptographic algorithm. In classical cryptanalysis an attacker has usually the knowledge of the algorithm (in accordance to the Kerckhoff's principle) and of some inputs and/or outputs. Starting from this data, his goal is to retrieve the secret key. This attack model considers the algorithm computation as a black box, in the sense that no internal variable can be observed during execution, only inputs and/or outputs. With his seminal paper about Side-Channel Attacks (SCAs) in 1996, Paul Kocher showed that such a black-box model fails once the algorithm is implemented over a material component \cite{kocher1996timing}: an attacker can indeed inspect its component during the execution of the cryptographic algorithm, monitor some physical quantities, for example the execution time \cite{kocher1996timing} or the instantaneous power consumption \cite{kocher1999differential} and deduct information about internal variables of the algorithm. Depending on the attacked algorithm, making inference over some well chosen internal variable (the so-called \emph{sensitive variables} of the algorithm) is sufficient to retrieve the secret key. After these first works it was shown that other observable physical quantities contained \emph{leakages} of sensitive information, for example the electromagnetic radiation emanating from the device \cite{gandolfi2001electromagnetic,quisquater2001electromagnetic} and the acoustic emanations \cite{genkin2014rsa}. Moreover, if until few years ago it was thought that only small devices, equipped with slow microprocessors and in a small-size architecture, such as smart cards were vulnerable to this kind of Side-Channel Attacks, this recent work about acoustic emanations, together with other works exploiting electromagnetic fluctuations pointed out that much faster and bigger devices, \emph{i.e.} laptops and desktop computers, are vulnerable as well \cite{genkin2015stealing,genkin2015get,genkin2016ecdh}.

\subsubsection{A Classification of the Attacks to Secured Components}
\begin{table}[]
\centering
\caption{Classification of Attacks}
\label{fig:classification_attacks}
\begin{tabular}{|l|lll}
\cline{1-3}
\multirow{4}{*}{\textbf{Hardware Attacks}} & \multicolumn{1}{l|}{Passive} & \multicolumn{1}{l|}{Active} &                                    \\ \cline{2-4} 
                                           & \multicolumn{1}{l|}{}        & \multicolumn{1}{l|}{}       & \multicolumn{1}{l|}{Invasive}      \\ \cline{2-4} 
                                           & \multicolumn{1}{l|}{}        & \multicolumn{1}{l|}{}       & \multicolumn{1}{l|}{Semi-Invisive} \\ \cline{2-4} 
                                           & \multicolumn{1}{l|}{SCAs}    & \multicolumn{1}{l|}{FAs}    & \multicolumn{1}{l|}{Non-Invasive}  \\ \hline
\textbf{Software Attacks}                  &                              &                             &                                    \\ \cline{1-1}
\end{tabular}
\end{table}

The Side-Channel Attacks outlined in previous paragraph, and which are the main concern of this thesis,  belong to a much bigger family of attacks that can be performed to break cryptographic devices security claims. First of all software attacks and hardware attacks must be distinguished. Software attacks only exploit vulnerability coming from the way the component's code is written. Hardware attacks are still commonly classified on the base of two criteria: on one hand we can distinguish passive and active attacks, on the other hand we can distinguish invasive, semi-invasive, non-invasive attacks. 
\begin{itemize}
\item[] \textbf{Passive attacks:} in passive attack, the device is let run respecting its specifications. The attacker observes its behaviour without provoking any alteration;
\item[] \textbf{Active attacks:}  in active attacks a special manipulation is performed in order to make the normal behaviour of the device change. 
\end{itemize}


\begin{itemize}
\item[] \textbf{Invasive attacks}: in invasive attack, the device is depackaged and inspected at the level of the components technology. The circuit can be modified, broken, signals can be accessed via a probing station, etc. There is no limits to the manipulations attacker can do to the components;
\item[] \textbf{Semi-invasive attacks}: as in invasive attacks the device is depackaged, but in contrast to them, no direct electrical contact to the chip is done;
\item[] \textbf{Non-invasive attacks}: in non-invasive attacks the device is not modified and only accessible interfaces are exploited. 

In literature, the term Side-Channel Attacks commonly denotes the passive non-invasive attacks. In the same way, active non-invasive attacks are often referred to as \emph{Fault Injection Attacks}. 

\end{itemize}

\subsection{Certification of a Secure Hardware - The Common Criteria}
\begin{figure}
\includegraphics[width=\textwidth]{../Figures/ITSEF_ANSSI2.pdf} 
\caption{The actors of French Certification Scheme}
\end{figure}
In previous paragraphs we have evoked the great diffusion of the cryptographic devices, which implies consequent great risks in case of vulnerabilities of such largely diffused devices, and the existence of a wide range of attacks exploiting vulnerabilities coming from the way cryptography is embedded. These factors justify the importance and necessity to ensure reliability on the security claims of commercialised secured components and thus the arise of several guidelines and standards for their evaluation. The international standard ISO/IEC 15408, also known as \emph{Common Criteria for Information Technology Security Evaluation} (abbreviated as \emph{Common Criteria} or simply \emph{CC}) represents one of the stronger efforts in standardization, unifying in 1999 three previously existing standards:
\begin{itemize}
\item the \emph{Trusted Computer System Evaluation Criteria} (TCSEC - United States - 1983)
\item the \emph{Information Technology Security Evaluation Criteria} (ITSEC - France,Germany, Netherlands, United Kingdom - 1990)
\item the \emph{Canadian Trusted Computer Product Evaluation Criteria} (CTCPEC - Canada - 1993).
\end{itemize}

\subsubsection{The actors} The CC define four actors of the evaluation process of a secured component:
\begin{itemize}
\item \textbf{The Developer}, who conceives a product and wish sell it a certified secured product. He send a request for evaluation to the certification body and, once the request is accepted, he contacts an evaluation laboratory
\item \textbf{The ITSEF} is the \emph{IT Security Evaluation Facility}; in France it is named \emph{Centre d'Evaluation de la Securit\'e des Technologies de l'Information} (CESTI). It is an evaluation laboratory, in possession of a certification body agreement, which performs the security tests to assess the resilience of the product
\item \textbf{The Certification Body} is often a governmental organism, the \emph{Agence National de la Securit\'e des Syst\`emes d'Information} (ANSSI) in France, or the \emph{Bundesamt f\"ur Sicherheit in der Informationstechnik} (BSI) in Germany. It ensures the quality of the evaluation and delivers a certificate to the developer
\item \textbf{The end user}, who buys the product and follows its security guidelines.
\end{itemize} 

\subsubsection{The Target of Evaluation and the security objectives} 
To start the certification process, the developer compiles a document called \emph{Security Target} (ST). Such a document begins specifying the (part of the) device subjected to evaluation, the so-called \emph{Target of Evaluation} (ToE) and then lists its \emph{Security Functional Requirements} (SFR), choosing by those proposed by the CC. In practice, and to ease de redaction of the ST, the choice of the SFRs is not open, but guided by the typology of the component. In particular, the CC propose a catalogue of \emph{Protection Profiles} (PP), associated with the required SFRs; for example \emph{smart card} or \emph{TEE} designate some precise PPs. 

\subsubsection{Evaluation Assurance Level and Security Assurance Requirements}
\begin{table}[]
\centering
\caption{Evaluation Assurance Levels}
\label{tab:EAL}
\begin{tabular}{cc}
\toprule 
EAL  & Description                                \\
\midrule
EAL1 & Functionally tested                        \\
EAL2 & Structurally tested                        \\
EAL3 & Methodically tested and checked            \\
EAL4 & Methodically designed, tested and reviewed \\
EAL5 & Semi-formally designed and tested          \\
EAL6 & Semi-formally verified design and tested   \\
EAL7 & Formally verified design and tested      \\
\bottomrule
\end{tabular}
\end{table}

In CC seven  \emph{Evaluation Assurance Level} (EAL) are defined, and determine the quantity and complexity of the tasks the evaluator has to effectuate, thus specifying the insurance strength. The EAL are defined in insurance increasing order, so that the EAL1 has the lowest verification exigences while EAL7 has the highest ones. In Table~\ref{tab:EAL} the objectives given by the CC for each EAL are resumed.\\

During the process of evaluation, the SFRs of the ToE have to be verified according to the claimed EAL. To this end, the evaluation is  divided into six classes of \emph{Security Assurance Requirement} (SAR). Five of this classes are the so-called \emph{conformity} classes, and one in called the \emph{attack} class. Each class is sub-divided in several \emph{families} (excepted the attack class, which only contains one family), and the evaluators are charged to check each requirement corresponding to these families. The table~\ref{tab:SAR} resumes the SAR classes and their families. For each family a grade is assigned following precise specifications detailed in CC. Grades range in general from 1 to 6 and the obtention of a certain EAL depends on the grades obtained for each family, as reported in Table~\ref{tab:components}.


\begin{table}[]
\centering
\caption{Security Assurance Requirements}
\label{tab:SAR}
\begin{tabular}{ccc}
\toprule
Class                               & Family   & Description                           \\
\midrule
\multirow{6}{*}{Development}        & ADV\_ARC & Security architecture                 \\
                                    & ADV\_FSP & Functional specification              \\
                                    & ADV\_IMP & Implementation representation         \\
                                    & ADV\_INT & TOE Security Functions internals      \\
                                    & ADV\_SPM & Security policy modelling             \\
                                    & ADV\_TDS & TOE design                            \\
                                    \midrule
\multirow{2}{*}{Guidance Documents} & AGD\_OPE & Operational user guidance             \\
                                    & AGD\_PRE & Preparative procedures                \\
                                    \midrule
\multirow{7}{*}{Life-cycle support} & ALC\_CMC & Configuration Management capabilities \\
                                    & ALC\_CMS & Configuration Management scope        \\
                                    & ALC\_DEL & Delivery                              \\
                                    & ALC\_DVS & Development security                  \\
                                    & ALC\_FLR & Flaw remediation                      \\
                                    & ALC\_LCD & Life-cycle definition                 \\
                                    & ALC\_TAT & Tools and techniques                  \\
                                    \midrule
\multirow{7}{*}{ST evaluation}      & ASE\_CCL & Conformance claims                    \\
                                    & ASE\_ECD & Extended components definition        \\
                                    & ASE\_INT & ST introduction                       \\
                                    & ASE\_OBJ & Security objectives                   \\
                                    & ASE\_REQ & Security requirements                 \\
                                    & ASE\_SPD & Security problem definition           \\
                                    & ASE\_TSS & TOE summary specification             \\
                                    \midrule
\multirow{4}{*}{Tests}              & ATE\_COV & Coverage                              \\
                                    & ATE\_DPT & Depth                                 \\
                                    & ATE\_FUN & Functional tests                      \\
                                    & ATE\_IND & Independent testing                   \\
                                    \midrule
Vulnerability assessment            & AVA\_VAN & Vulnerability analysis   \\
\bottomrule
            
\end{tabular}
\end{table}



\begin{table}[]
\centering
\caption{Required grades for the obtention of each EAL.}
\label{tab:components}
\begin{tabular}{|c|c|c|c|c|c|c|c|}
\hline
\multirow{2}{*}{Family} & \multicolumn{7}{c|}{Assurance Components by EAL} \\ \cline{2-8} 
                        & EAL1  & EAL2  & EAL3 & EAL4 & EAL5 & EAL6 & EAL7 \\
                        \midrule
ADV\_ARC                &       & 1     & 1    & 1    & 1    & 1    & 1    \\ 
ADV\_FSP                & 1     & 2     & 3    & 4    & 5    & 5    & 6    \\ 
ADV\_IMP                &       &       &      & 1    & 1    & 2    & 2    \\ 
ADV\_INT                &       &       &      &      & 2    & 3    & 3    \\ 
ADV\_SPM                &       &       &      &      &      & 1    & 1    \\ 
ADV\_TDS                &       & 1     & 2    & 3    & 4    & 5    & 6    \\
\midrule
AGD\_OPE                & 1     & 1     & 1    & 1    & 1    & 1    & 1    \\ 
AGD\_PRE                & 1     & 1     & 1    & 1    & 1    & 1    & 1    \\ 
\midrule
ALC\_CMC                & 1     & 2     & 3    & 4    & 4    & 5    & 5    \\ 
ALC\_CMS                & 1     & 2     & 3    & 4    & 5    & 5    & 5    \\ 
ALC\_DEL                &       & 1     & 1    & 1    & 1    & 1    & 1    \\ 
ALC\_DVS                &       &       & 1    & 1    & 1    & 2    & 2    \\ 
ALC\_FLR                &       &       &      &      &      &      &      \\ 
ALC\_LCD                &       &       & 1    & 1    & 1    & 1    & 2    \\ 
ALC\_TAT                &       &       &      & 1    & 2    & 3    & 3    \\ 
\midrule
ASE\_CCL                & 1     & 1     & 1    & 1    & 1    & 1    & 1    \\ 
ASE\_ECD                & 1     & 1     & 1    & 1    & 1    & 1    & 1    \\ 
ASE\_INT                & 1     & 1     & 1    & 1    & 1    & 1    & 1    \\ 
ASE\_OBJ                & 1     & 2     & 2    & 2    & 2    & 2    & 2    \\ 
ASE\_REQ                & 1     & 2     & 2    & 2    & 2    & 2    & 2    \\ 
ASE\_SPD                &       & 1     & 1    & 1    & 1    & 1    & 1    \\ 
ASE\_TSS                & 1     & 1     & 1    & 1    & 1    & 1    & 1    \\ 
\midrule
ATE\_COV                &       & 1     & 2    & 2    & 2    & 3    & 3    \\ 
ATE\_DPT                &       &       & 1    & 1    & 3    & 3    & 4    \\ 
ATE\_FUN                &       & 1     & 1    & 1    & 1    & 2    & 2    \\ 
ATE\_IND                & 1     & 2     & 2    & 2    & 2    & 2    & 3    \\ 
\midrule
AVA\_VAN                & 1     & 2     & 2    & 3    & 4    & 5    & 5    \\ \hline
\end{tabular}
\end{table}
\chapter{Introduction to Side-Channel Attacks} % Main chapter title

\label{ChapterIntroductionSCA}

%----------------------------------------------------------------------------------------
%	SECTION 1
%----------------------------------------------------------------------------------------

\section{Introduction to Side-Channel Attacks}
\subsection{Historical Overview}

\subsection{Terminology and Generalities}
\subsubsection{Target and Leakage Model}


\subsubsection{Points of Interest}
\subsubsection{Simple vs Advanced SCAs}
\subsubsection{Vertical vs Horizontal SCAs}
\subsubsection{Profiled vs Non-Profiled SCAs}
\subsubsection{Side-Channel Algebraic Attacks}
\subsubsection{Distinguishers}
\subsubsection{SCA Metrics}



%----------------------------------------------------------------------------------------
%	SECTION 2
%----------------------------------------------------------------------------------------
\section{Main Side-Channel Countermeasures}
\subsection{Random Delays and Jitter}
\subsection{Shuffling}
\subsection{Masking}



%----------------------------------------------------------------------------------------
%	SECTION 3
%----------------------------------------------------------------------------------------
\section{Higher-Order Attacks}
\subsection{Higher-Order Moments Analysis and Combining Functions}
\subsection{Profiling Higher-Order Attacks}
\subsubsection{Profiling with Masks Knowledge}
\subsubsection{Profiling without Masks Knowledge}

 
% Chapter Template

\chapter{Introduction to Machine Learning} % Main chapter title

\label{ChapterIntroML}


\section{Basic Concepts of Machine Learning}
Machine Learning (ML) is a field of computer science that groups a variety of  methods whose aim is giving computers the ability of \emph{learning}. The more cited definition of \emph{learning} has been provided by Mitchell in 1997 \cite{Mitchell1997}: \enquote{ A computer program is said to learn from experience E with respect to some task T and performance measure P, if its performance on T, as measured by P, improves with experience E.} \\
The ML methods methods essentially come from applied statistics, and are characterised by an increased emphasis on the use of computers to statistically estimate complicated functions. This allows ML to tackle tasks that would be too difficult to solve with algorithms entirely designed and specified by human being. An ML algorithm is often said to \enquote{learn from data}, in the sense that it is able to improve an algorithm's performance at some task via a data observation experience.

\subsection{The Task, the Performance and the Experience}\label{sec:TPE}
\paragraph*{The task.} The task T is usually described in terms of how the ML system should process an \emph{example} (or \emph{data point}). An \emph{example} is one datum $\vLeakVec\in \mathbb{R}^\traceLength$, which is in turn a collection of \emph{features} $\vLeakVec[i]$, with  $i=1,\dots \traceLength$. In SCA context an example might be a side-channel trace, which is in turn a collection of time samples, that constitute its features. Some common ML tasks include these three examples: 
\begin{itemize}
\item \emph{Regression: } the computer is asked to approximate a mapping function from some input variables to some continuous output variables, \eg approximate $\MLmodel\colon \mathbb{R}^\traceLength \rightarrow \mathbb{R}$.
\item \emph{Classification: } the computer is asked to specify which class or category an input belongs to, being $\sensVarSet$ the set of the possible classes. The learning algorithm is thus asked to construct a function $\MLmodel\colon \mathbb{R}^\traceLength \rightarrow \sensVarSet$. We remark that this task is similar to the regression one, except for the form of the output, since in general $\sensVarSet$ is a discrete finite set, and not continuous. A slightly variant solution to the classification task consists in constructing a function $\MLmodel\colon \mathbb{R}^\traceLength \rightarrow \{0,1\}^{\numClasses}$, if elements of $\sensVarSet$ are expressed \via the \emph{one-hot encoding} (see \ref{sec:notations}). Another variant of the classification task consists in finding a function $\MLmodel$ defining a probability distribution over classes.
\item \emph{Verification}: the computer is asked to state whether or not two given inputs are instances of a same class or category. For example, it may be asked to state if two hand-written signatures have been produced by the same person. The learning algorithm is thus asked to construct a function $\MLmodel\colon \mathbb{R}^\traceLength \times \mathbb{R}^{\traceLength}\rightarrow \{0,1\}$. A variant of such a task consists in finding a function $\MLmodel$ defining the probability of each pair of inputs being instances of a same class. 
\end{itemize}
The functions constructed by an ML algorithm somehow describe and characterise the data form and distribution, thus are often referred to as \emph{models}.

\paragraph*{The performance measure.} The performance measure P designs a quantification of the ability of the learning algorithm. Depending on the task T, a specific performance measure P can be considered. For tasks as classification or verification the more common measure is the \emph{accuracy} of the model, \ie the proportion of inputs for which the model produces the correct output. Equivalently, the \emph{error rate} may be used as a performance measure P, \ie the proportion of inputs for which the model produces an incorrect output. For the regression task the more common performance measure P is the so-called \emph{Mean Squared Error} (MSE): it is computed by averaging over a finite set of examples, the squares of the differences between the correct outputs and the ones predicted by the model.\\
One of the crucial challenges of ML is that we are usually interested in how well a learning algorithm performs in producing a model that fits new, unseen data. For this reason, the performances of an ML algorithm are usually evaluated over a so-called \emph{test set}, \ie a set of examples that have not been used for the learning (or \emph{training}) phase. 

\paragraph*{The experience.} The experience E describes the way data and information are accessed by the learning algorithm during learning. In this context we principally distinguish two families of learning algorithms: 
\begin{itemize}
\item the \emph{supervised} learning algorithms access to a dataset of examples, each associated in general to a \emph{target} or \emph{label}. The term supervised reflects the fact that the learning is somehow guided by an instructor that knows the right answer over the learning dataset;
\item the \emph{unsupervised} learning algorithms access to a dataset, without any associated target. They try to learn useful properties of the structure of the dataset. 
\end{itemize}
In general, the nature of the task is strictly related to the kind of experience the learner is allowed; for example the classification or regression tasks are considered as supervised tasks, while examples of unsupervised tasks include \emph{clustering} and \emph{data representation} or \emph{dimensionality reduction}. For example, the Principal Component Analysis, that will be discussed in Chapter~\ref{ChapterLinear} in the context of SCA, is a dimensionality reduction algorithm that might be seen as an unsupervised algorithm that learns a representation of data. We will see in Chapter~\ref{ChapterLinear} that for SCA context a supervised version of the PCA has been proposed as well. 


% examples: Classification, Dimensionality Reduction, Verification
% Example method: Neural Network Classifier, Stacked Auto-Encoder, Siamese Network 

\subsection{Example of Linear Regression}
The regression task is not of high interest for the rest of this thesis, but is the most direct example to keep in mind in order to understand some basic ML concepts, such as the \emph{underfitting} and the \emph{overfitting} (see \ref{sec:overfitting}). Let us introduce a linear regression model to tackle the regression task: we want to construct a linear function $\MLmodel\colon \mathbb{R}^\traceLength \rightarrow \mathbb{R}$, that takes an input $\vLeakVec$ and outputs $\hat{y} = \www^\intercal \vLeakVec$, where $\www\in \mathbb{R}^\traceLength$ is a vector of \emph{parameters} that have to be learned by a learning algorithm in order to well describe some data.\footnote{An affine model may be considered as well by adding a \emph{bias}, leading to $\hat{y} = \www^\intercal \vLeakVec+w_0$. This model is equivalently obtained by adding an additional component to $\vLeakVec$, always set to $1$ and by writing back $\hat{y} = \www^\intercal \vLeakVec$ with $\www\in \mathbb{R}^{\traceLength+1}$. } 
Let $\setData_{\cdot} = (\setLeak_{\cdot}, \setTarget_{\cdot})$ denote a dataset, where $\cdot$ can stand for \textquotedbl$\text{train}$\textquotedbl \ or \textquotedbl$\text{test}$\textquotedbl \ depending on the role of the dataset in the experience, and let $\sizeSetData$ denote the size of the dataset, \ie the number of examples contained in it. Let us store the examples contained in $\setLeak_{\cdot}$ into a matrix $\measuresMatrix_{\cdot}\in \mathbb{R}^{\traceLength \times \sizeSetData}$ and the targets contained in $\setTarget_{\cdot}$ into a targets vector $\yyy_{\cdot}\in \mathbb{R}^{\sizeSetData}$. Let a learned model predict targets $y_i$ by outputting $\hat{y_i} = \www^\intercal\vLeakVec_i$ and let them be collected in turn into a predicted targets vector $\hat{\yyy}_{\cdot}$. The MSE is given by 

\begin{equation}
 \mathrm{MSE_{\cdot}} = \frac{1}{\sizeSetData} \norm{\hat{\yyy}_{\cdot}-\yyy_{\cdot}}^2_2 \mbox{ .}
\end{equation}

The performance measure for the learning algorithm is $\mathrm{MSE_{test}}$, meaning that the goal for the learning algorithm is to find a parameter vector $\www$ which minimises $\mathrm{MSE_{test}}$. Nevertheless, such an objective cannot be directly imposed, because the learning algorithm only experiences over the training set, and not over the test set. An intuitive way to act, that can be proven to be the maximum-likelihood solution to the problem, is to minimise  $\mathrm{MSE_{train}}$ instead of $\mathrm{MSE_{test}}$. This minimization can be obtained by solving an easy optimization problem. When a learning algorithm behaves as an optimization algorithm that minimises a given function, such a function is called \emph{cost function}, or \emph{loss function}, or \emph{objective function}. 
The solution to such an optimization problem can be given in closed form, by means of the \emph{pseudo-inverse} matrix $\measuresMatrix^+$ of $\measuresMatrix_{\text{train}}$, as follows:
\begin{equation}
\measuresMatrix^+ = (\measuresMatrix_{\text{train}}\measuresMatrix_{\text{train}}^\intercal)^{-1}\measuresMatrix_{\text{train}}
\end{equation}
\begin{equation}
\www = \measuresMatrix^+\yyy_{\text{train}}.
\end{equation}

%OLD 
%We want this model well describe some data and we suppose to have two available datasets of such data: $\setDataTrain = (\setLeakTrain, \setTargetTrain)$, to let the learning algorithm experience on, and $\setDataTest = (\setLeakTest, \setTargetTest)$ to evaluate its performance over some unseen data. Let us choose the MSE over the test set to evaluate such performances. Let us collect the examples $(\vLeakVec_i, y_i)_{i=1,\dots, N}$ of a dataset into a measure matrix $\measuresMatrix_{\text{train}}\in \mathbb{R}^{\traceLength \times N}$ and into a targets vector $\yyy_{\text{train}}\in \mathbb{R}^N$ (or $\yyy_\text{test}$ for the test dataset), and let the learned model predict targets $y_i$ by outputting $\hat{y_i} = \www^\intercal\vLeakVec_i$ (in turn collected into a predicted targets vector $\hat{\yyy}_{\text{train}}$, or $\hat{\yyy}_{\text{test}}$ if working with the test dataset), then the MSE is given by
%\begin{equation}
% \mathrm{MSE_{test}} = \frac{1}{m} \norm{\hat{\yyy}_{\text{test}}-\yyy_{\text{test}}}^2_2 \mbox{ .}
%\end{equation}
%Obviously, we consider the model performs well the most such an MSE is small. So the goal of the learning algorithm is to somehow minimize the $\mathrm{MSE_{test}}$. But the learning algorithm only experiences on the $\setDataTrain$ dataset. An intuitive way to act, that can be proven to be the maximum likelihood solution to the problem, is to minimize  $\mathrm{MSE_{train}}$ instead of $\mathrm{MSE_{test}}$. This minimization can be obtained by solving an easy optimization problem. When a learning algorithm behaves as an optimization algorithm that minimizes a given function, such a function is called \emph{cost function}, or \emph{objective function}.
% The solution to such an optimization problem can be given in closed form, by means of the pseudo-inverse matrix of $\measuresMatrix_{\text{train}}$:
%\begin{equation}
%\www = (\measuresMatrix_{\text{train}}\measuresMatrix_{\text{train}}^\intercal)^{-1}\measuresMatrix_{\text{train}}\yyy_{\text{train}}.
%\end{equation}


\subsection{Example of Linear Model for Classification}\label{example:LDA}
As observed in Sections \ref{sec:simple} and \ref{sec:advanced}, a strict relationship may be established between the profiling SCAs and the classification task in ML context. For this reason we introduce here a very brief overview of how classically the classification task is tackled, by means of linear models. \\
Classifying means assigning a label $\sensVar\in \sensVarSet$ to an example $\vLeakVec\in\mathbb{R}^\traceLength$ , or equivalently divide the input space $\mathbb{R}^\traceLength$ in \emph{decision regions}, whose boundaries are referred to as \emph{decision boundaries}. Making use of a linear model signifies exploiting some hyperplanes as decision boundaries. Datasets whose classes can be separated exactly by linear decision boundaries are said to be \emph{linearly separable}. Following the discussion kept by Bishop in \cite{christopher2006pattern}, two different approaches to tackle the classification task should be distinguished: the direct research for a discriminant function $\MLmodel$ that assigns to an example a label, or the prior construction of a probabilistic model. This second approach might in turn be distinguished into two options, depending on whether a \emph{generative} model  (see Sec.~\ref{sec:TA}), or a \emph{discriminative} model is constructed (\ie only conditional probability densities of outputs given the inputs are modelled). For this example we consider a probabilistic approach, constructing a generative model, which is the same approach of Template Attacks (Sec.~\ref{sec:TA}). This example will allow on one hand to introduce some interesting functions, such as the \emph{logistic sigmoid} and the \emph{softmax}, that will play a role in the construction of Neural Networks (see Chapter \ref{ChapterCNN}). On the other hand, the example justifies the large exploitations of generalised linear models in order to construct discriminative functions. Indeed, linear models come out naturally when adding some assumptions on the data distributions, as those that will be introduced below. \\

Constructing a generative probabilistic model implies modelling the class-conditional probabilities $\pdf_{\given{\vaLeakVec}{\sensRandVar = \sensVarValue{j}}}(\vLeakVec)$ for $j\in \{1,\dots,\numClasses\}$ as well as the class priors $\pdf_{\sensRandVar}(\sensVarValue{j})$ and $\pdf_{\vaLeakVec}(\vLeakVec)$. Let us first consider a 2-class context, \ie $\sensVarSet = \{\sensVarValue{1}, \sensVarValue{2}\}$. Then, the posterior probability for the class $\sensVarValue{1}$ is the following:
\begin{align}\label{eq:post_probs}
\prob(\given{\sensVarValue{1}}{\vLeakVec}) &= \frac{\prob(\given{\vLeakVec}{\sensVarValue{1}})\prob(\sensVarValue{1})}{\prob(\vLeakVec)}=\\
&=\frac{\prob(\given{\vLeakVec}{\sensVarValue{1}})\prob(\sensVarValue{1})}{\prob(\given{\vLeakVec}{\sensVarValue{1}})\prob(\sensVarValue{1}) + \prob(\given{\vLeakVec}{\sensVarValue{2}})\prob(\sensVarValue{2})}\mbox{ .}
\end{align}
To compare the two classes, we can evaluate their \emph{log-likelihood ratio} defined as:
\begin{equation}\label{eq:log-ratio}
a = \log\left[\frac{\prob(\given{\sensVarValue{1}}{\vLeakVec})}{\prob(\given{\sensVarValue{2}}{\vLeakVec})}\right] =  \log\left[\frac{\prob(\given{\vLeakVec}{\sensVarValue{1}})\prob(\sensVarValue{1})}{\prob(\given{\vLeakVec}{\sensVarValue{2}})\prob(\sensVarValue{2})}\right].
\end{equation}
Then we might assign the label the class $\sensVarValue{1}$ to $\vLeakVec$  if and only if $a>0$, which corresponds to take as decision boundary the surface defined by $\prob(\given{\vLeakVec}{\sensVarValue{1}})\prob(\sensVarValue{1}) = \prob(\given{\vLeakVec}{\sensVarValue{2}})\prob(\sensVarValue{2})$.
We remark that Eq.~\eqref{eq:post_probs} rewrites as:
\begin{equation}\label{eq:post_probs_sigmoid}
\prob(\given{\sensVarValue{1}}{\vLeakVec}) = \frac{1}{1+e^{-a}} = \sigma(a)\mbox{ ,}
\end{equation}
where the function $\sigma$ is the so-called \emph{logistic sigmoid}. This remark translates in the multi-class case, \ie $\numClasses >2$, in the following way: the posterior probability for each class $\sensVarValue{j}$ is given by
\begin{equation}\label{eq:post_probs_multi-class}
\prob(\given{\sensVarValue{j}}{\vLeakVec})  = \frac{\prob(\given{\vLeakVec}{\sensVarValue{j}})\prob(\sensVarValue{j})}{\prob(\vLeakVec)} = \frac{\prob(\given{\vLeakVec}{\sensVarValue{j}})\prob(\sensVarValue{j})}{\sum_{k=1}^{\nbClasses}\prob(\given{\vLeakVec}{\sensVarValue{k}})\prob(\sensVarValue{k} )} = \softmax (\aaa)[j]\mbox{ ,}
\end{equation}
where $\aaa$ is a $\numClasses$-dimensional vector, whose entries are given by
\begin{equation}\label{eq:softmax_entries}
\aaa[j] = \log\left[ \prob(\given{\vLeakVec}{\sensVarValue{j}})\prob(\sensVarValue{j}) \right] \mbox{ ,}
\end{equation}
and $\softmax$ is the so-called \emph{softmax} function, or \emph{normalised exponential}, that is defined, entry-wise by:
\begin{equation}\label{eq:softmax}
\softmax(\aaa)[k] = \frac{e^{\aaa[k]}}{\sum_{j=1}^{\numClasses}e^{\aaa[j]}}\mbox{ .}
\end{equation}

Let us now introduce two assumptions about the class-conditional densities:
\begin{enumerate}[label=(\roman*)]
\item \label{item:gaussian} we will suppose that they follow a Gaussian distribution with parameters $\mu_j, \Sigma_j$,
\item \label{item:covariances} and that all class-conditional densities share the same covariance matrix $\Sigma_j=\Sigma$,
\end{enumerate}
so that
\begin{equation}\label{eq:gauss_dens}
\pdf_{\given{\vaLeakVec}{\sensRandVar = \sensVarValue{j}}}(\vLeakVec)= \frac{1}{(2\pi)^{{\traceLength}/2}\lvert \Sigma\rvert^{1/2}}e^{-\frac{1}{2}(\vLeakVec- \mu_j)^\intercal\Sigma^{-1}(\vLeakVec- \mu_j)} \mbox{ .}
\end{equation}
Under these assumptions, and considering probability densities and masses instead of probability values\footnote{A formal justification of the validity of \eqref{eq:LDA-2classes} for continuous random variables is out of the scope of this section.} Eq.~\eqref{eq:log-ratio} rewrites as: 
\begin{equation}\label{eq:LDA-2classes}
a = \log\left[\frac{\pdf_{\sensRandVar}(\sensVarValue{1})}{\pdf_{\sensRandVar}(\sensVarValue{2})}\right] - \frac{1}{2}\mu_1^\intercal\Sigma^{-1}\mu_1 + \frac{1}{2}\mu_2^\intercal\Sigma^{-1}\mu_2 - \vLeakVec^\intercal\Sigma^{-1}(\mu_2-\mu_1) = \www^\intercal \vLeakVec + w_0, 
\end{equation}
where we set 
\begin{align*}
\www =& \Sigma^{-1}(\mu_1-\mu_2)\\
w_0 =  & \log\left[\frac{\pdf_{\sensRandVar}(\sensVarValue{1})}{\pdf_{\sensRandVar}(\sensVarValue{2})}\right] - \frac{1}{2}\mu_1^\intercal\Sigma^{-1}\mu_1 + \frac{1}{2}\mu_2^\intercal\Sigma^{-1}\mu_2 . 
\end{align*}
The quadratic terms in $\vLeakVec$, that appears in the exponent of the Gaussian density \eqref{eq:gauss_dens}, have cancelled thanks to the common variance assumption \ref{item:covariances}, thus we obtain that the decision boundary for the 2-class problem, given by $a=0$ is a $(\traceLength - 1)$-hyperplane of the input space.\footnote{An analogous result can be obtained in the multi-class problem.} This way of choosing linear boundaries is known under the name of \emph{Linear Discriminant Analysis}. Another way to view the same linear classification model is in terms of dimensionality reduction: intuitively, in the 2-class case\footnote{again extensible to the multi-class case} one can see the term $\www^\intercal \vLeakVec$ in \eqref{eq:LDA-2classes} as a projection of the input $\vLeakVec$ onto a one-dimensional subspace of $\mathbb{R}^\traceLength$ which is orthogonal to the decision boundary mentioned above. Then, the classification of the obtained dimensionality-reduced examples is done by the means of a real-valued threshold (that would correspond to $w_0$, in the optimal case). It can be shown that the dimensionality reduction obtained by the \emph{Fisher criterion} that we will deploy in Chapter~\ref{ChapterLinear}, to which we will refer to as LDA dimensionality reduction by a widely accepted abuse, is equivalent to the dimensionality reduction obtained in this example, under both assumptions \ref{item:gaussian} and \ref{item:covariances}.  \\
Relaxing the assumption \ref{item:covariances} and allowing each class-conditional density $\pdf(\vLeakVec\mid \sensVarValue{j})$ to have its own covariance matrix $\Sigma_j$, then the cancellations seen above will no longer occur, and the discriminant $a$ turns out to be a quadratic function of $\vLeakVec$. This gives rise to the so-called \emph{Quadratic Discriminant Analysis}, that we already mentioned in Chapter~\ref{ChapterIntroductionSCA} for its analogy with Template Attacks.\\

Assumptions \ref{item:gaussian} and \ref{item:covariances} also lead to the following expression for the posterior probability for $\sensVarValue{1}$, directly implied by \eqref{eq:post_probs_sigmoid}: 
\begin{equation}\label{eq:binary_linear_classifier}
\prob(\given{\sensVarValue{1}}{\vLeakVec}) = \sigma(\www^\intercal \vLeakVec + w_0)\mbox{ .}
\end{equation}
Thus, such a posterior probability is given by the sigmoid acting to a linear function of $\vLeakVec$. Similarly, for the multi-class case, the posterior probability of class $\sensVarValue{j}$ is given by the $j$-th entry of the softmax transformation of a linear function of $\vLeakVec$. This kind of \emph{generalised linear model} can be thus used in a probabilistic discriminant approach, where the posterior conditional probabilities are directly modelled from data without passing through the estimations of class-conditional densities and priors. Such a discriminative approach is the one that will be adopted in Chapter~\ref{ChapterCNN} when considering models constructed by Neural Networks.

%the simplest representations of a linear discriminant function is obtained by taking
%\begin{equation}
%f(\vLeakVec) = \www\vLeakVec + w_0 \mbox{ ,}
%\end{equation}
%where $-w_0$ plays the role of a threshold. Here  $\vLeakVec$ is assigned to $\sensVarValue{1}$ if $f(\vLeakVec)>0$, \ie $\www\vLeakVec>-w_0$, otherwise $\vLeakVec$ is assigned to $\sensVarValue{2}$.



% la decision boundary  è un D-1 iperpiano
% introducendo i vettori di target...
% y(x) = wx + w0...oppure in grosso per piu di due classi: la decision boundary è un iperpiano di dimensione (numClasses - 1)
% per imparare i pesi W un'idea è minimizzare il sum error square sti cazzi, un'altra è passare attraverso la riduzione di dimensione e scegliere la riduzione che amplifica la separabilità. 
%Infatti....fisher dà la nozione di ottimalità col quoziente, che porta alla Fisher Discriminant Analysis (o anche LDA, discussa nel Capitolo 6) ... Nel caso numClasses = 2 le due si equivalgono. 
% E interessante discutere brevemente anche di un semplice modello generativo, basato su hp gaussiana, perche tale ipotesi, insieme ad un'altra, porta a costruire anch'essa un modello lineare, giustificando questa scelta in molti contesti...P(x|C)...log del quoziente, funzione sigmoid, logit...softmax...linear piu sigmoid, dove di nuovo la soluzione LDA è la parte lineare (verifica). Se togliamo l'ipotesi sulle covarianze allora resta la parte quadratica e è la QDA (in pratica la parte di classificazione usata nei template attacks)  


\subsection{Underfitting, Overfitting, Capacity,  and Regularization}\label{sec:overfitting}
\paragraph*{Underfitting and Overfitting.} As already said, the main challenge of ML is that the learning algorithms are in general allowed to experience over training data, but the models they output are asked to fit some unseen test data. Observing the training data, an ML algorithm sets the model parameters in order to raise the performances over the training set, or equivalently to minimise the so-called \textit{training error}. Nevertheless, at the end of the learning process, the model performance is evaluated over the test set, by measuring the so-called \textit{test error}. Thus, two factors determine how well an ML algorithm acts: its ability to reduce the training error, and its ability to reduce the gap between the training and the test error. When the former ability is not satisfactory we assist to the \emph{underfitting} phenomenon: the model is not able to obtain a low training error, or the ML algorithm is not able to determine model parameters that make training error to be low. On the other hand, if the latter ability is not satisfactory we assist to the \emph{overfitting} phenomenon: the gap between the training and the test error, called \emph{generalisation gap}, is too large. \\

\paragraph*{Capacity.}The property of a model that controls its underfitting or overfitting behaviour is the \emph{capacity}. Roughly speaking, the capacity of a model quantifies the complexity of the functions it can represent: a model with higher capacity can be parametrised in such a way to represent a higher complex function. For example, a linear regression model is able to represent all linear functions. To raise its capacity, quadratic, cubic or general polynomial terms might be included, passing from a linear regression model to a \emph{polynomial regression} one. It allows the model to represent respectively quadratic, cubic or polynomial functions as well.\footnote{Another common way to enlarge the capacity of a linear regression model $y = \www^\intercal \vLeakVec$,  consists in choosing some \emph{basis functions} $\varphi_1, \varphi_2,\dots, \varphi_B$ and replace $\vLeakVec$ with the values $\varphi_1(\vLeakVec), \varphi_2(\vLeakVec),\dots, \varphi_B(\vLeakVec)$. The form of the basis functions will determine the capacity of the model. Basis function regression includes the linear and the polynomial case.} \\

The polynomial regression provides a striking example to understand the underfitting and overfitting phenomena. Consider a problem in which the examples $(x_i,y_i)_{i=1,\dots,N}$ lies in $\mathbb{R} \times \mathbb{R}$ and the true underlying function is quadratic, perturbed by a small noise. Let the training set contain 4 data points, \ie $N=4$. Figure ~\ref{fig:poly_reg} shows the results of a linear, quadratic and cubic regression in such a case: in the figure, red circles represents the 4 training points, the blue line gives the learned model and the green points are test example. Above the plots the evaluation of the MSE over the training and test sets is given. We can observe that the linear predictor is underfitting, since the line passes quite far from both training and test points and its training error is quite high. On the contrary, the cubic predictor is overfitting: it perfectly fits the 4 training points (it is the Lagrange polynomial interpolating such 4 points) but shows a huge error in predicting new examples. The quadratic regression is obviously in this case the model exhibiting the optimal capacity to solve such a problem. \\

A very rough way to have an intuition about the capacity of a model is counting the number of its parameters: the capacity in general grows with the number of parameters. Some formal ways to quantify the capacity of a model have been provided in ML literature. The most well-known is the \emph{Vapnik-Chervonenkis dimension}: it measures the capacity of a classifier as the cardinality of the largest set of points the model can classify without errors, for any possible assignment of labels. In practice, quantifying the capacity of a model, especially for complex models as those constructed by neural networks, is very hard and discouraged. On the other hand, these kinds of quantifications have enabled statistical learning theory to formalise and prove some important intuitions, for example the fact that the generalization gap is upper-bounded by a quantity that grows with the model capacity and that shrinks as the number of training examples increases. In Fig.~\ref{fig:cubic_more_data} we observe how the cubic model used for regression on quadratic distributed data ameliorates its performances and reduces the generalization gap despite its excessive capacity, when trained with more examples. This observation basically justifies on one hand the attitude adopted in the branch of ML called \emph{Deep Learning}, and basically based over multi-layer neural networks, consisting in considering very complex models, having confidence in the big size of the typically considered training sets. On the other hand it justifies the interest of \emph{Data Augmentation} (DA) techniques \cite{simard2003best} to respond to an eventual lack of data. Some DA techniques will be proposed in Chapter~\ref{ChapterCNN} for the SCA context.

\begin{figure}
\subfigure[Linear]{\label{fig:linear_regression}
\includegraphics[width=.5\textwidth]{../Figures/linear_regression.pdf}}
\subfigure[Quadratic]{\label{fig:quadratic_regression}
\includegraphics[width=.5\textwidth]{../Figures/quadratic_regression.pdf}}
\subfigure[Cubic]{\label{fig:cubic_regression}
\includegraphics[width=.5\textwidth]{../Figures/cubic_regression.pdf}}
\subfigure[Cubic, more training data]{\label{fig:cubic_more_data}
\includegraphics[width=.5\textwidth]{../Figures/cubic_regression_more.pdf}}
\caption[Examples of underfitting and overfitting over a regression problem.]{Examples of underfitting and overfitting over a regression problem. Linear \subref{fig:linear_regression}, quadratic \subref{fig:quadratic_regression} and cubic regression for a truly noised quadratic problem. Red circles are the training examples, green points are the test ones, the blue line represents the learned solution. Linear \subref{fig:linear_regression} regression underfits data, cubic \subref{fig:cubic_regression}  regression overfits data. \subref{fig:cubic_more_data} Cubic regression for a noised quadratic problem and more training examples. The cubic model trained over more data is better adapted to the truly quadratic data, and overfitting is attenuated.}\label{fig:poly_reg}
\end{figure}

\paragraph*{Regularization.}
In a real-case problem, the optimal capacity necessary to learn from given data is unknown. In such a case, trying to fit data with a too low capacity model assures the failure, thus it is always more interesting to oversize the capacity of the learning model. Choosing an oversized model, we risk to incur in overfitting. The so-called \emph{regularization} techniques respond to such a risk, as a widely adopted alternative to DA: in general they consist in adding constraints to the learning algorithm in order to guide it in choosing a model among a wide set of eventually fitting models. Going back to the polynomial regression example, one can try to fit data with a cubic polynomial (thus oversizing the model capacity) and induce the optimiser algorithm to choose the smallest-degree polynomial fitting data via a regularization. This can be obtained adding a penalty that depends on the polynomial degree to the cost function. Applying regularization may make the algorithm be less accurate in learning training data, but more likely to correctly operate on new examples.

\subsection{Hyper-Parameters and Validation}\label{sec:validation}
The \emph{hyper-parameters} of a model are all the parameters that are priorly set and that are not learned by the learning algorithm. They define the general form of the model. In the polynomial regression example the model had a single hyper-parameter: the degree of the polynomial. It is evident from the example that such a parameter is somehow forced not to be optimised by the means of the learning algorithm: trying to reduce the training MSE, the algorithm would choose a sufficient high degree to interpolate all training points (typically $N-1$ if $N$ is the number of training examples). This would cause overfitting, as shown in Fig.~\ref{fig:cubic_regression}. In general among all parameters of a model, the hyper-parameters are chosen as those that can not be learned from data because it would cause overfitting, as in the example, or because they are too difficult to optimise. \\
A way to choose a setting for hyper-parameters consists in performing a \emph{validation} phase. To do so, the training set is split into two disjoint sets, one still called \emph{training set} and the other one called \emph{validation set}. We can say that as the training set is used to learn the parameters, the validation set is used to somehow learn the hyper-parameters. Indeed during or after the training over the training set, the validation set is used to compute a sort of estimation of the test error, which quantifies the generalisation ability of the model. In practice the performances of the (partially) trained model are evaluated over the validation set computing a validation error and hyper-parameters are updated accordingly, in order to reduce the generalisation gap of the model. Once the model has been validated, \ie the hyper-parameters are definitely set, the real test error is evaluated over the test set. Usually the validation error is an underestimation of the test error, since hyper-parameters have been set to reduce it. \\
The validation process just described may strongly depend on the way the training set have been split to create the validation one. In order to avoid to validate a model in a strongly data-dependent way, a slightly different process is encouraged in ML community, named the \emph{cross-validation}, which we describe in Appendix~\ref{app:cross-validation}.


\subsection{No Free Lunch Theorem}\label{sec:NFL}
A so-called \emph{No Free Lunch Theorem} has been formulated for optimisation and ML algorithms around 1997 \cite{wolpert1997no}. It states that any learning algorithm has the same test error if averaged over all possible distributions of data. This means that there cannot exist a universal best ML algorithm: any of them performs in the same way, when performances are averaged over all possible tasks. Thus, making research over some kind of data, for example SCA traces, means trying to understand what kinds of ML algorithms perform well over such particular kind of data and point out the eventual interesting hyper-parameters of ML models that are responsible of the main performance variations. 




%----------------------------------------------------------------------------------------
%	SECTION 3
%----------------------------------------------------------------------------------------
\section{Overview of Machine Learning in Side-Channel Context}
In 1991 Rivest pointed out for the first time a strong link between the fields of Machine Learning and Cryptanalysis \cite{rivest1991cryptography}. Starting from observing that the goal of cryptanalysis is identifying an unknown encryption function, indexed by a secret key, and that a classic problem in ML consists as well in learning an unknown function, he drew a strong correspondence between terminology and concepts of the two fields.\\

In the context of Side-Channel Cryptanalysis, ML algorithms started to be investigated in 2011 \cite{machineLearningSCA}. In this paper the authors formulated for the first time an attack in terms of classification problem and proposed the Support Vector Machine (SVM) \cite{cortes1995support,weston1998multi} as technique to solve it. They also equipped the SVM with a kernel function to allow it to succeed even in case data would not be linearly separable. Such an approach is similar to the one we will describe in Chapter~\ref{ChapterKernel}, to obtain Kernel Discriminant Analysis dimensionality reduction technique from the Linear Discriminant Analysis. Further works analysed the use of SVM in SCA context, proposing concrete attack scenarios \cite{intelligentMachineOmicide,effTA_SVM}. 
The technique of Random Forest \cite{lior2014data} drew attention of the SCA community as well. As the SVM, it has been used as a classifier and has been evaluated in different works \cite{lerman2015machine,lerman2015template,lerman2014power}. As in recent years the privileged tools to tackle classification problem in ML area are the Neural Networks, whose multi-layer configuration has given name to the so-called \emph{Deep Learning} domain, such tools have as well been analysed in SCA context. Networks in the form of Multi-Layer Perceptrons (MLP) have been proposed as classifiers for side-channel traces in a series of works \cite{martinasek2013optimization,martinasek2013innovative,martinasek2015profiling,martinasek2016profiling}, while Convolutional Neural Network was firstly introduced in \cite{maghrebi2016breaking}. A part of this thesis contributions consists in the application of the convolutional paradigm as a way to defeat misalignment countermeasures in side-channel attacks (see Chapter~\ref{ChapterCNN}).

%\subsection{Profiled Attack as a Classification Problem}
%\todo{remark that LDA is first of all a linear method for classification and has been introduced in SCA many years ago as preprocessing for Gaussian TA}
%\subsubsection{Support Vector Machine}
%\subsubsection{Random Forest}
%\subsubsection{Neural Networks}

%\begin{table}[]
%\centering
%\caption{Examples of hyper-parameters}
%\label{tab:hyperparameters}
%\begin{tabular}{c|c}
%\multicolumn{1}{c}{\textbf{Training Hyper-Parameters}} & \multicolumn{1}{c}{\textbf{Architecture Hyper-Parameters}}    \\
%\hline
%training set size                             & number of layers                                     \\
%batch size                                    & nature of each layer{\scriptsize  (\eg FC, ACT,$\dots$)} \\
%number of epochs                              & number of units     \rdelim\}{1}{3mm}[{\scriptsize for FC layers}]                 \\
%optimizer algorithm              &  number of filters               \rdelim\}{4}{3mm}[{\scriptsize for CONV layers}]                          \\
%(initial) learning rate              & kernel size                           \\
%                                              & stride                                               \\
%                                                  & padding                                              \\
%                                              & activation function                  \rdelim\}{1}{3mm}[{\scriptsize for ACT layers}]             \\                   
%                                          
%                                              & pooling function        \rdelim\}{3}{3mm}[{\scriptsize for POOL layers}]                                              \\
%                                              & kernel size \\
%                                              & stride                                              
%\end{tabular}
%\end{table}
\part{Contributions}

% Chapter Template

\chapter{Linear Dimensionality Reduction} 
\label{ChapterLinear}
\setlength{\epigraphwidth}{0.7\textwidth}
\epigraph{\textit{\textquotedbl One side will make you grow taller, and the
other side will make you grow shorter.\textquotedbl \\
\textquotedbl One side of what? The other side of what?\textquotedbl \ thought Alice to herself.\\
\textquotedbl Of the mushroom,\textquotedbl \ said the Caterpillar, just as if she had asked it aloud; and
in another moment it was out of sight.}}{--- \textup{Lewis Carroll --- \textquotedbl Alice's Adventures in Wonderland\textquotedbl}}


In this chapter, we explore solutions for dimensionality reduction of side-channel traces exploiting linear combinations of time samples. The results presented in this chapter have been published in the proceedings of CARDIS 2015 \cite{Cagli2016}.

%----------------------------------------------------------------------------------------
%	SECTION 1
%----------------------------------------------------------------------------------------
\section{Introduction}\label{sec:intro_chapter_linear}
Linear dimensionality reduction methods produce a low-dimensional linear mapping of the original high-dimensional data that preserves some original feature of interest. An abundance of methods has been developed throughout statistics, machine learning, and applied fields for over a century, and these methods have become indispensable tools for analysing high dimensional, noisy data, such as side-channel traces.  Accordingly, linear dimensionality reduction can be used for visualizing or exploring structures in data, denoising or compressing data, extracting meaningful feature spaces, and more. A very complete survey about this great variety of linear dimensionality reduction technique has been published in 2015 by Cunningham and Ghahramani \cite{cunningham2015linear}. They proposed a generalized optimization framework for all linear dimensionality techniques, survey a dozen different techniques and mention some important extensions such as kernel mappings. \\

Among the surveyed methods in \cite{cunningham2015linear} we find the two ones that are mainly considered in SCA literature: the Principal Component Analysis (PCA) and the Linear Discriminant Analysis (LDA). The PCA has been applied both in an {\em unsupervised} way (\ie non-profiling attacks) \cite{Batina2012,karsmakers2009side}, and in a {\em supervised} way (\ie profiling attacks) \cite{TAprincipal,choudaryefficient,choudary2014efficient,disassembler,Standaert2008}. As already remarked in \cite{disassembler}, and not surprisingly, the complete knowledge assumed in the supervised approach hugely raises performances. The main competitor of PCA in the profiling attacks context is the LDA, that thanks to its classification-oriented flavour (see Sec.~\ref{example:LDA}), is known to be more meaningful and informative \cite{lessIsMore,Standaert2008} than the PCA method  for side channels. Nevertheless, the LDA is often set aside because of its practical constraints; it is subject to the so-called {\em Small Sample Size problem (SSS)}, i.e. it requires a number of observations (traces) which must be higher than the dimension (size) $\traceLength$ of them. In some contexts it might be an excessive requirement, which may become unacceptable in many practical situations where the amount of observations is very limited and the traces size is huge.\\

In 2014 Durvaux et al. proposed the use of another technique for linear dimensionality reduction in SCA context \cite{PP}, the so-called Projection Pursuits (PPs), firstly introduced in 1974 by Friedman and Tukey \cite{friedman1974projection}. This method essentially works by randomly picking time samples, randomly setting the projecting coefficients, and tracking the improvement (or the worsening) of the projection when modifying
them with small random perturbations. The main drawback of the PPs  pointed out by the authors of \cite{PP} for the SCA context is their heuristic nature,
since the convergence of the method is not guaranteed and its complexity is
context-dependent. The main advantage is the fact that
PPs can deal with any objective function, which may be adjusted to fit the problem
of higher-order SCA. Thus this technique appears advantageous in higher-order context, where it is used as a PoI selection tool. Its version for the first-order attacks, which produces a linear dimensionality reduction, is less interesting than the non-heuristic PCA and LDA. For this reason we will left PPs technique apart in this chapter, and describe their higher-order version in Chapter~\ref{ChapterKernel}.\\

 In SCA literature, one of the open issues for PCA application concerns the choice of the principal components that must be kept after the dimension reduction: as already remarked by Specht et al.  \cite{specht}, some papers declare that the leading components are those that contain almost all the useful information \cite{TAprincipal,choudary2014efficient}, while others propose to discard the leading components \cite{Batina2012}. In a specific attack context, Specht et al. compares the results obtained by choosing different subsets of consecutive components, starting from some empirically chosen index. They conclude that for their data the optimal result is obtained by selecting a single component, the fourth one, but they give no formal argumentation about this choice. Such a result is obviously very case-specific. Moreover, the possibility of keeping non-consecutive components is not considered in their analysis. \\

 
In Sec.~\ref{sec:PCA} the classical PCA technique is described, then the previous applications of PCA in SCA context are recalled, highlighting the difference between its unsupervised and supervised declination. Finally our contribution to  \textquotedbl the choice of components open issue is described\textquotedbl \  is described:  it is based on the Explained Local Variance (ELV) notion, that we will define and argument in the same section. The reasoning behind the ELV selection methodology is essentially based on the observation that, for secure implementations, the leaking information, if existing, is spread over a few time samples of each trace. This observation has already been met by Mavroeidis et al. in \cite{SCAclassProbl}, where the authors  also proposed a components selection method. As we will see in Sec.~\ref{sec:ELV}, the main difference between their proposal and ours is that in in \cite{SCAclassProbl} the information given by the eigenvalues associated to the PCA components is completely discarded, while the ELV methodology takes advantage of such information as well.  We will argue about the generality and the soundness of this methodology and we will show that it can raise the PCA performances, making them close to those of the LDA, even in the supervised context. This makes PCA an interesting alternative to LDA in those cases where the LDA is inapplicable due to the SSS problem. The ELV selection tool has been tested in a successive experimental work \cite{choudary2018efficient}. Unfortunately, the authors of this work could not observe an improvement (nor a worsening) using our new selector, because in their specific case its selection of components were equivalent to the classical one, that will be referred to as EGV in the following.\\

 The LDA technique will be described in Sec.~\ref{sec:LDA}, together with the description of the SSS problem and some solutions coming from the Pattern and Face Recognition communities \cite{eigenfaces,Chen2000,huang,Yu01adirect}. Through some experiments depicted in Sec.~\ref{sec:experiments} we will conclude about the effectiveness of the PCA-ELV solution. Finally, in Sec.~\ref{sec:misalignment} we will experimentally argue about the weakness of all these techniques when data are misaligned.

%----------------------------------------------------------------------------------------
%	SECTION 2
%----------------------------------------------------------------------------------------

\section{Principal Component Analysis} \label{sec:PCA}
\subsection{Principles and algorithm description}
\begin{figure}
\centering
\includegraphics[width=.5\textwidth]{../Figures/PCA_LDA_geometric/PCAprojection_unsupervised.pdf} 
\caption{PCA: some 2-dimensional data (blue crosses) projected into their 1-dimensional principal subspace (represented by the green line).}\label{fig:PCAunsupervised}
\end{figure}
The Principal Component Analysis (PCA) is a technique for data dimensionality reduction. The PCA algorithm can be deduced from two different points of view, a statistical one and a geometrical one. In the former one, PCA aims to project orthogonally the data onto a lower-dimensional linear space, the so-called \emph{principal subspace}, such that the variance of the projected data is maximized. In the latter one, PCA aims to project data onto a lower-dimensional linear space in such a way that the average projection cost, defined as the mean square distance between the data and their projections, is minimized. In the following it is shown how the PCA algorithm is deduced by the statistical definition. The reader interested by the equivalence between the two approaches can refer to \cite[Ch.\ 12]{christopher2006pattern}. An example of 2-dimensional data projected over their 1-dimensional principal subspace is depicted in Fig.~\ref{fig:PCAunsupervised}.\\

Let $(\vLeakVec)_{i=1..\nbTraces}$ be a set of $\traceLength$-dimensional measurements (or observations, or data), i.e. realizations of a $\traceLength$-dimensional zero-mean random vector $\vaLeakVec{}$, and collect them as columns of an $\traceLength \times \nbTraces$ matrix $\measuresMatrix$, so that the empirical covariance matrix of $\vaLeakVec{}$ can be computed as 
\begin{equation}\label{eq:covmat}
\covmat = \frac{1}{\nbTraces}\measuresMatrix\measuresMatrix^\intercal \mbox{ .}
\end{equation}

Let us first assume that we have priorly fixed the dimension $\newTraceLength<\traceLength$ of the principal subspace we are looking for. 

\paragraph*{Compute the First Principal Component}
Suppose in a first time that $\newTraceLength = 1$, i.e. that we want to represent our data by a unique variable $Y_1 =  \AAlpha_1^\intercal \vaLeakVec{}$, i.e. projecting data over a single $\traceLength\times 1$ vector $\AAlpha_1$, in such a way the variance of the obtained data is maximal. The vector $\AAlpha_1$ that provides such a linear combination is called {\em first principal component}. 
To avoid misunderstanding we will call {\em $j$-th principal component} (PC) the projecting vector $\AAlpha_j$, while we will refer to the variable $Y_j = \AAlpha_j^\intercal \vaLeakVec$ as the {\em $j$-th Principal Variable (PV)}. 
Realizations of the PVs are given by the measured data projected over the $j$-th PC, for example we can collect, in a vector $\yyy_1^\intercal = \AAlpha_1^\intercal \measuresMatrix $, $\nbTraces $ realizations of $Y_1$:
\begin{equation}
y_1[i] = \AAlpha_1^\intercal \vLeakVec_i \mbox{ for } i=1,\dots , \nbTraces \mbox{ .}
\end{equation}

The mean of these realizations will be zero as they are linear combinations of zero-mean variables, and the variance turns to be estimable as
\begin{equation}\label{eq:PCAobjective}
\frac{1}{\nbTraces}\yyy_1\yyy_1^\intercal = \frac{1}{\nbTraces}\AAlpha_1^\intercal\measuresMatrix\measuresMatrix^\intercal\AAlpha_1 = \AAlpha_1^\intercal\covmat\AAlpha_1 \mbox{ .}
\end{equation}
To compute $\AAlpha_1$ we look for the vector that maximises the variance estimate in \eqref{eq:PCAobjective}.\\

The maximisation problem by itself is not well posed, because the variance value is not limited until a restriction is not imposed to the modulo $\|\AAlpha_1| = \sqrt{\AAlpha_1^\intercal\AAlpha_1}$. In order to let the maximization problem have a solution, a restriction is thus imposed: $\AAlpha_1^\intercal\AAlpha_1 = 1$. 
This constrained  optimization problem is handled by making use of Lagrange multipliers:
\begin{equation}
\Lambda(\AAlpha_1, \lambda) = \AAlpha_1^\intercal\covmat\AAlpha_1 - \lambda(\AAlpha_1^\intercal\AAlpha_1-1) \mbox{ ,}
\end{equation}
and by computing the partial derivative of $\Lambda$ with respect to $\AAlpha_1^\intercal$:
\begin{equation}
\frac{\partial\Lambda}{\partial\AAlpha_1^\intercal} = 2\covmat\AAlpha_1-2\lambda\AAlpha_1 \mbox{ .}
\end{equation}
Thus, stationary points of $\Lambda$ verify:
\begin{equation}\label{eq:eigProblemPCA}
\covmat\AAlpha_1 = \lambda\AAlpha_1 \mbox{ ,}
\end{equation}
which implies that $\AAlpha_1$ must be an eigenvector of $\covmat$, with $\lambda$ its correspondent eigenvalue. Multiplying both sides of Eq.~(\ref{eq:eigProblemPCA}) by $\AAlpha_1^\intercal$ on the left, we remark that
\begin{equation}
\AAlpha_1^\intercal\covmat\AAlpha_1 = \lambda\AAlpha_1^\intercal\AAlpha_1 = \lambda, 
\end{equation}
which means that the variance of the obtained variable $\yyy_1$ equals $\lambda$. For this reason $\AAlpha_1$ must be the leading eigenvector of $\covmat$, the one corresponding to the maximal eigenvalue.

\paragraph*{Compute the Second and Following Principal Components}
The PCs others than the first are defined in an incremental fashion by choosing new directions orthogonal to those already considered and such that the sum of the projected variances over each direction is maximal. Explicitly, if we look for two PCs, \ie $\newTraceLength = 2$,  we look for a $2$-dimensional variable $\YYY = \begin{psmallmatrix} \AAlpha_1^\intercal \\ \AAlpha_2^\intercal \end{psmallmatrix} \vaLeakVec$ such that the trace of its covariance matrix, \ie the sum of variances $\var(Y_1)+\var(Y_2)$, is maximal. It can be shown that the same result would be obtained by maximising the so-called \emph{generalized variance} of $\YYY$, which is defined as the determinant of its covariance matrix, instead of its trace. \\

Consider, as in previous case, the Lagrangian of the problem
\begin{equation}\label{eq:lagrangian_2comps}
\Lambda = \AAlpha_1^\intercal \covmat \AAlpha_1 + \AAlpha_2^\intercal\covmat \AAlpha_2 - \lambda_1(\AAlpha_1^\intercal\AAlpha_1 -1) - \lambda_2(\AAlpha_2^\intercal\AAlpha_2 -1) \mbox{ .}
\end{equation}

The partial derivatives of \eqref{eq:lagrangian_2comps} with respect to $\AAlpha_1^\intercal$ and $\AAlpha_2^\intercal$ are null under the following conditions:
\begin{align}
\covmat  \AAlpha_1 &= \lambda_1\AAlpha_1 \\
\covmat  \AAlpha_2 &= \lambda_2\AAlpha_2 \mbox{ .}
\end{align}

It means that $\AAlpha_1$ and $\AAlpha_2$ must be eigenvectors of $\covmat$ with corresponding eigenvalues given by $\lambda_1$ and $\lambda_2$. Moreover, as before, $\lambda_1$ and $\lambda_2$ respectively equal the estimated variances of the variable components $Y_1$ and $Y_2$, and since the goal is maximizing the sum of these variables we choose $\AAlpha_1$ and $\AAlpha_2$ as the two leading vectors of $\covmat$. Let us remark that the estimated covariance between $Y_1$ and $Y_2$ is given by $\AAlpha_1^\intercal\covmat \AAlpha_2$ which equals zero, since $\AAlpha_1^\intercal\AAlpha_2 = 0$ by orthogonality. In particular the principal variables are uncorrelated, which is a remarkable property of the PCA. \\

In the general case of a $\newTraceLength$-dimensional projection space, it can be shown by induction that the PCs would correspond to the $\newTraceLength$ leading eigenvectors of the covariance matrix $\covmat$.



%----------------------------------------------------------------------------------------
%	SECTION 3
%----------------------------------------------------------------------------------------
\subsection{Original vs Class-Oriented PCA}
\begin{figure}[t]
\subfigure[]{\label{fig:PCAunsupervised_lab}
\includegraphics[width=.5\textwidth]{../Figures/PCA_LDA_geometric/PCAprojection.pdf}}
\subfigure[]{\label{fig:PCAsupervised_lab}
\includegraphics[width=.5\textwidth]{../Figures/PCA_LDA_geometric/PCA_class_projection.pdf}}
\caption{PCA: some 2-dimensional labelled data (blue crosses and red circles) projected into their 1-dimensional principal subspaces (represented by the green line). \subref{fig:PCAunsupervised_lab} classical unsupervised PCA, \subref{fig:PCAsupervised_lab} class-oriented PCA. In \subref{fig:PCAsupervised_lab} black stars represents the 2 classes centroids (sample means).}\label{fig:2class-toys}
\end{figure}
The classical version of the PCA method is unsupervised.
% in the sense that is does not take into account the information about the value assumed by the target variable during the acquisition of data. 
On the other hand a profiling attacker is not only provided with a set of traces $(\vLeakVec_i)_{i=1..\nbTraces}$, but he also has the knowledge of the target values handled during each acquisition. We denote by $(\vLeakVec_i, \sensVar_i)_{i=1..\nbProfilingTraces}$ the labelled set of traces . In Fig.~\ref{fig:2class-toys} the same data of Fig.~\ref{fig:PCAunsupervised} have been grouped into 2 classes. Even if before projection the two groups are clearly separable, after projecting data over the first principal component given by the classical PCA algorithm, the separability is lost (Fig.\ref{fig:PCAunsupervised_lab}. In the supervised context, and for the sake of distinguishing the target value assumed by the target $\sensRandVar$ in new executions, the idea of the {\em Class-Oriented} PCA is to consider as {\em equivalent} all the traces belonging to the same class. Modelling traces of a same class as traces showing the same characteristic form plus a random noise, the denoised characteristic form can be estimated by the sample means of the traces in the class. Let us recall from \eqref{eq:mmmXclass} that the empirical mean of traces belonging to the same class $\sensVarGenValue$ is given by
\begin{equation*}
\mmmXclass = \frac{1}{\nbTracesPerClass}\sum_{i\colon \sensVar_i=\sensVarGenValue} \vLeakVec_i \mbox{ ,}
\end{equation*}
where  $\nbTracesPerClass$ is the number of traces belonging to class $\sensVarGenValue$. The class-oriented version of the PCA  consists in applying the PCA dimensionality reduction to the set $(\mmmXclass)_{\sensVarGenValue \in \sensVarSet}$, instead of applying it directly to the traces $\vLeakVec_i$. This implies that the empirical covariance matrix will be computed using only the $\nbClasses$ average traces. Equivalently, in case of \textit{balanced} acquisitions ($\nbTracesPerClass$ constant for each class $\sensVarGenValue$), it amounts to replace the empirical covariance matrix $\covmat$ of data in \eqref{eq:covmat}  by the so-called {\em between-class} or  {\em inter-class scatter matrix}, given by:
\begin{equation}\label{eq:SB}
\SB = \sum_{\sensVarGenValue\in\sensVarSet}\nbTracesPerClass(\mmmXclass-\mmmX)(\mmmXclass-\mmmX)^\intercal \mbox{ .}
\end{equation}
Remark that $\SB$ coincides, up to a multiplicative factor, to the covariance matrix obtained using the class-averaged traces. In this way we focus the attention on information that discriminate the characteristic forms of classes, \ie target values. Figure~\ref{fig:PCAsupervised_lab} shows how the 2-class toy data are projected over the first class-oriented principal component: in the figure, black stars represent the class sample means. Projected data are slightly better separated than in Fig.~\ref{fig:PCAunsupervised_lab}.



\subsection{Computational Consideration}\label{sec:trick}
Performing PCA (and LDA, as explained later) always implies to compute the eigenvector of some symmetric matrix $\covmat$, obtained by the multiplication of a constant with a matrix and the transposed same matrix, \eg $\covmat = \frac{1}{\nbTraces}\measuresMatrix\measuresMatrix^\intercal$ . Let $\measuresMatrix$ have dimension $\traceLength\times\nbTraces$, and suppose $\nbTraces\ll \traceLength$. This condition is almost always satisfied when performing class-oriented PCA, because in such a case $\nbTraces$ corresponds to the number of classes $\nbClasses$, and $\traceLength$ is the traces' size. Anyway, for high-dimensional data, \ie $\traceLength$ high, it can be satisfied even when performing classical PCA. Thus, in such a common case, the $\traceLength\times\traceLength$ matrix $\covmat$ is far from being a full-rank matrix, since $\mathrm{rank}(\covmat)\leq \nbTraces\ll \traceLength$. For this reason, we expect to find at most $\nbTraces$ eigenvectors; often columns of $\measuresMatrix$ are linearly dependent, because for example they are forced to have zero mean, so actually the rank of $\covmat$ is strictly less than $\nbTraces$ and we expect to obtain at most $\nbTraces-1$ eigenvectors.

A practical problem in case of high-dimensional data, is represented by the computation and the storage of the $\traceLength\times\traceLength$ matrix $\covmat$. This problem can be bypassed by exploiting the following lemma coming from linear algebra, as proposed by Archambeau \etal \cite{TAprincipal}:
\begin{lemma}
For any $\traceLength\times\nbTraces$ matrix $\measuresMatrix$, the function $\vLeakVec\mapsto \measuresMatrix\vLeakVec$ is a one-to-one mapping that maps eigenvectors of $\measuresMatrix^\intercal\measuresMatrix$ ($\nbTraces\times\nbTraces$) onto those of $\measuresMatrix\measuresMatrix^\intercal$ ($\traceLength\times\traceLength$).
\end{lemma}
This lemma allows to compute and store the smaller $\nbTraces\times\nbTraces$ matrix $\tilde{\covmat} = \frac{1}{\nbTraces}\measuresMatrix^\intercal\measuresMatrix$, to compute its ($\nbTraces\times 1$)-sized eigenvectors $\BBeta_i$ and the relative eigenvalues $\lambda_i$, and then to convert them into eigenvectors of $\covmat$, given by $\AAlpha_i =\measuresMatrix \BBeta_i$. Observing that by definition $\tilde{\covmat}\BBeta_i =\frac{1}{\nbTraces}\measuresMatrix^\intercal\measuresMatrix\BBeta_i =  \lambda_i\BBeta_i$ the lemma is easy to verify: 
\begin{equation}
\covmat\AAlpha_i = \frac{1}{\nbTraces}\measuresMatrix\measuresMatrix^\intercal \measuresMatrix \BBeta_i = \lambda_i\measuresMatrix\BBeta_i = \lambda_i\AAlpha_i \mbox{ .}
\end{equation}
However, it is not guaranteed that eigenvectors $\AAlpha_i$ obtained in this way have norm equal to 1. This is why a normalization step usually follows.



\subsection{The Choice of the Principal Components}\label{sec:ELV}
The introduction of the PCA method in SCA context (either in its classical or class-oriented version)  has raised some non-trivial questions: \textit{how many} principal components and \textit{which ones} are sufficient/necessary to reduce the trace size (and thus the attack processing complexity) without losing important discriminative information?\\

Until 2015, the sole attempt to give an answer to the questions above was made in \cite{choudary2014efficient}, linked to the concept of {\em explained variance} (or {\em explained global variance}, EGV for short) of a PC $\AAlpha_i$:
\begin{equation}\label{eq:EGV}
\mathrm{EGV}(\AAlpha_i) =  \frac{\lambda_i}{\sum_{k=1}^r\lambda_k} \mbox{ ,}
\end{equation}
where $r$ is the rank of the covariance matrix $\covmat$, and $\lambda_j$ is the eigenvalue associated to the $j$-th PC $\AAlpha_j^\intercal$. $\mathrm{EGV}(\AAlpha_i)$ is the variance of the data projected over the $i$-th PC (which equals $\lambda_i$) divided by the total variance of the original data (given by the trace of the covariance matrix $\covmat$, \ie by the sum of all its non-zero eigenvalues). By definition of EGV, the sum of all the EGV values is equal to $1$; for this reason this quantity is often multiplied by $100$ and expressed as percentage.
Exploiting the EGV to choose among the PCs consists in fixing a wished {\em cumulative explained variance} $\beta$ and in keeping $\newTraceLength$ different PCs, where $\newTraceLength$ is the minimum integer such that
\begin{equation}
\mbox{EGV}(\AAlpha_1) +\mbox{EGV}(\AAlpha_2) + \dots +\mbox{EGV}(\AAlpha_\newTraceLength) \geq \beta \mbox{ .}
\end{equation}
However, if the attacker has a constraint for the reduced dimension $\newTraceLength$, the EGV notion simply suggests to keep the first $\newTraceLength$ components, taking for granted that the optimal way to chose PCs is in their natural order. This assumption is not always confirmed in SCA context: in some works, researchers have already remarked that the first components sometimes contain more noise than information \cite{Batina2012,specht} and it is worth discarding them. For the sake of providing a first example of this behaviour on publicly accessible traces, we applied a class-oriented PCA on 3000 traces from the DPA contest v4 \cite{DPAcontest}; we focused over a small 1000-dimensional window in which, in complete knowledge about masks and other countermeasures, information about the first Sbox processing leaks (during the first round). In Fig.~\ref{fig:DPAcontest} the first and the sixth PCs are plotted. It may be noticed that the first component indicates that one can attend a high variance by exploiting the regularity of the traces, given by the clock signal, while the sixth one has high coefficients localised in a small time interval, very likely to signalize the instants in which the target sensitive variable leaks.

\begin{figure}
\includegraphics[width=.45\textwidth]{../Figures/CARDIS2015/DPAcontestPC1_new.pdf} 
\includegraphics[width=.45\textwidth]{../Figures/CARDIS2015/DPAcontestPC6_new.pdf} 
\caption[First and sixth PCs in DPA contest v4 trace set.]{First and sixth PCs in DPA contest v4 trace set (between time samples 198001 and 199000)}\label{fig:DPAcontest}
\end{figure}
A single method adapted to SCA context has been proposed until 2015 to automatically choose PCs \cite{SCAclassProbl} while dealing with the issue raised in Fig.~\ref{fig:DPAcontest}. It was based on the following assumption:
\begin{assumption}\label{assum:local}
The leaking side-channel information is localised in few points of the acquired trace.
\end{assumption}
This assumption is reasonable in SCA contexts where the goal of the security developers is to minimize the number of leaking points.
Under this assumption, the authors of \cite{SCAclassProbl} use for side-channel attack purposes the {\em Inverse Participation Ratio} (IPR), a measure widely exploited in Quantum Mechanics domain (see for example \cite{guhr1998random}). They propose to use such a score to evaluate the eigenvectors {\em localization}. It is defined as follows:
\begin{equation}
\mathrm{IPR}(\AAlpha_i) = \sum_{j=1}^\traceLength \AAlpha_i[j]^4 \mbox{ .}
\end{equation}
The authors of \cite{SCAclassProbl} suggest to collect the PCs in decreasing order with respect to the IPR score.\\

The selection methods provided by the evaluation of the EGV and of the IPR are somehow complementary: the former one is based only on the eigenvalues associated to the PCs and does not consider the form of the PCs themselves; the latter completely discards the information given by the eigenvalues of the PCs, considering only the distribution of their coefficients. In the next section we describe a new method, part of the contributions published in \cite{Cagli2016}, that builds a bridge between the EGV and the IPR approaches. As we will argue, our method, based on the so-called {\em explained local variance}, does not only lead to the construction of a new selection criterion, but also permits  to modify the PCs, choosing individually the coefficients to keep and those to discard. 

\subsubsection{Explained Local Variance Selection Method}\label{sec:ELVmethod}
The method we develop in this section is based on a compromise between the variance provided by each PC (more precisely its EGV) and the number of time samples necessary to achieve a consistent part of such a variance. To this purpose we  introduce the concept of {\em Explained Local Variance} (ELV).
\begin{figure}
\includegraphics[width=0.5\textwidth]{../Figures/CARDIS2015/cumulativeELVallRectangle.pdf} 
\includegraphics[width=0.5\textwidth]{../Figures/CARDIS2015/cumulativeELVzoomedRectangle.pdf} 
\caption[Cumulative ELV trend of principal components.]{Cumulative ELV trend of principal components. On the right a zoom of the plot on the left. Data acquisition described in Sec.~\ref{sec:experiments}.}\label{fig:ELVcumulative}
\end{figure}
%
Let us start by giving some intuition behind our new concept. Thinking to the observations ${\vLeakVec}$, or to the class-averages ${\mmmX}$ in class-oriented PCA case, as realizations of a random variable $\vaLeakVec$, we have that $\lambda_i$ is an estimator for the variance of the random variable $\vaLeakVec^\intercal\AAlpha_i$. Developing, we obtain
\begin{align}\label{eq:ELV}
\lambda_i =& \hat{\var}(\sum_{j=1}^\traceLength \vaLeakVec^\intercal[j]\AAlpha_i[j]) = \sum_{j=1}^\traceLength\sum_{k=1}^\traceLength \hat{\cov}(\vaLeakVec^\intercal[j]\AAlpha_i[j], \vaLeakVec^\intercal[k]\AAlpha_i[k])=\\
=& \sum_{j=1}^\traceLength \AAlpha_i[j]\sum_{k=1}^\traceLength\AAlpha_i[k]\hat{\cov}(\vaLeakVec^\intercal[j], \vaLeakVec^\intercal[k])= \sum_{j=1}^\traceLength \AAlpha_i[j] (\covmat_{j}^\intercal \AAlpha_i)=  \\
=& \sum_{j=1}^\traceLength \AAlpha_i[j] \lambda_i\AAlpha_i[j]= \sum_{j=1}^\traceLength  \lambda_i \AAlpha_i[j]^2 \label{eq:toJustify}
\end{align}
where $\covmat_{j}^\intercal$ denotes the $j$-th row of $\covmat$ and \eqref{eq:toJustify} is justified by the fact that $\AAlpha_i$ is an eigenvector of $\covmat$, with $\lambda_i$ its corresponding eigenvalue. The result of this computation is quite obvious, since $\parallel \AAlpha_i\parallel=1$, but it evidences the contribution of each time sample in the information held by the PC. This makes us introduce the following definition:
\begin{definition}


The {\em Explained Local Variance} of a PC $\AAlpha_i$ in a sample $j$, is defined by
\begin{equation}
\mathrm{ELV}(\AAlpha_i,j) = \frac{\lambda_i \AAlpha_i[j]^2}{\sum_{k=1}^r\lambda_k} = \mathrm{EGV}(\AAlpha_i) \AAlpha_i[j]^2  \mbox{ .}
\end{equation}
\end{definition}
\begin{figure}
\includegraphics[width=0.31\textwidth]{../Figures/CARDIS2015/PC1.pdf} 
\includegraphics[width=0.31\textwidth]{../Figures/CARDIS2015/PC2.pdf} 
\includegraphics[width=0.31\textwidth]{../Figures/CARDIS2015/PC3.pdf} \\
\includegraphics[width=0.31\textwidth]{../Figures/CARDIS2015/PC4.pdf} 
\includegraphics[width=0.31\textwidth]{../Figures/CARDIS2015/PC5.pdf} 
\includegraphics[width=0.31\textwidth]{../Figures/CARDIS2015/PC6.pdf} 
\caption[The first six PCs. Acquisition campaign on an 8-bit AVR Atmega328P.]{The first six PCs. Acquisition campaign on an 8-bit AVR Atmega328P (see Sec.~\ref{sec:experiments}).}\label{fig:6components}
\end{figure}
Let $\mathcal{J}=\{j^i_1, j^i_2, \dots, j^i_{\traceLength}\}\subset\{1,2,\dots,\traceLength\}$ be a set of indexes sorted such that $\mathrm{ELV}(\AAlpha_i,j^i_1)\geq \mathrm{ELV}(\AAlpha_i,j^i_2)\geq \dots \geq \mathrm{ELV}(\AAlpha_i,j^i_\traceLength)$.
It may be observed that the sum over all the $\mathrm{ELV}(\AAlpha_i,j)$, for $j\in[1,\dots,\traceLength],$   equals $\mathrm{EGV}(\AAlpha_i)$. If we operate such a sum in a cumulative way following the order provided by the sorted set $\mathcal{J}$, we obtain a complete description of the trend followed by the component $\AAlpha_i$ to achieve its EGV. As we can see in Fig.~\ref{fig:ELVcumulative}, where such cumulative ELVs are represented, the first 3 components are much slower in achieving their final EGV, while the $4^\text{th}$, the $5^\text{th}$ and the $6^\text{th}$ achieve a large part of their final EGVs very quickly ({\em i.e.} by adding the ELV contributions of much less time samples). For instance, for $i=4$, the sum of the $\mathrm{ELV}(\AAlpha_4, j^4_k)$, with $k\in[1,\dots,30]$, almost equals $\mathrm{EGV}(\AAlpha_4)$, whereas the same sum for $i=1$ only achieves about the 15\% of $\mathrm{EGV}(\AAlpha_1)$. Actually, the EGV of the $4^\text{th}$, the $5^\text{th}$ and the $6^\text{th}$ component only essentially depends on a very few time samples. This observation, combined with Assumption \ref{assum:local}, suggests that they are more suitable for SCA than the three first ones. To validate this statement, it suffices to look at the form of such components (Fig.~\ref{fig:6components}): the leading ones are strongly influenced by the clock, while the latest ones are well localised over the leaking points.\\

Operating a selection of components {\em via} ELV, in analogy with the EGV, requires to fix the reduced space dimension $\newTraceLength$, or a threshold $\beta$ for the cumulative ELV. In the first case, the maximal ELVs of each PC are compared, and the $\newTraceLength$ components achieving the highest values of such ELVs are chosen. In the second case, all pairs (PC, time sample) are sorted in decreasing order with respect to their ELV, and summed until the threshold $\beta$ is achieved. Then only PCs contributing in this sum are selected. \\

We remark that the ELV is a score associated not only to the whole components, but to each of their coefficients. This interesting property can be exploited to further remove, within a selected PC, the non-significant points, {\em i.e.} those with a low ELV. In practice this is done by setting these points to zero. That is a natural way to exploit the ELV score in order to operate a kind of {\em denoising} for the reduced data, making them only depend  on the significant time samples. In Sec.~\ref{sec:experiments} (scenario 4) we test the performances of an attack varying the number of time samples involved in the computation of the reduced data, and showing that such a denoising processing might impact significantly. 



%----------------------------------------------------------------------------------------
%	SECTION 4
%----------------------------------------------------------------------------------------

\section{Linear Discriminant Analysis}\label{sec:LDA}

\begin{figure}
\centering
\includegraphics[width=.5\textwidth]{../Figures/PCA_LDA_geometric/LDAprojection.pdf} 
\caption{LDA: some 2-dimensional labelled data (blue crosses and red circles) projected onto their 1-dimensional discriminant component (represented by the green line).}\label{fig:LDAprojection}
\end{figure}

\subsection{Fisher's Linear Discriminant and Terminology Remark}
Fisher's Linear Discriminant \cite{Fukunaga} is another statistical tool for dimensionality reduction, which is commonly used as a preliminary step before classification. Indeed it seeks for linear combinations of data that characterize or separate two or more classes, not only spreading class centroids as much as possible, like the class-oriented PCA does, but also minimizing the so-called {\em intra-class variance}, i.e. the variance shown by data belonging to the same class. The terms Fisher's Linear Discriminant and Linear Discriminant Analysis (LDA) are often used interchangeably, and in particular in SCA literature the Fisher's Linear Discriminant is almost always referred to as LDA, \eg \cite{lessIsMore,Standaert2008}. As we anticipated in Chapter~\ref{ChapterIntroML} - Example~\ref{example:LDA}, this widely-accepted abuse is due to the fact that under the assumptions leading to the LDA classification tools (\ie Gaussian class-conditional densities, sharing a common covariance matrix), the solution provided by the Fisher's Linear Discriminant (that does not require such assumptions) is the same as the solution provided by the LDA. From now on we will use the more common terminology LDA to refer to Fisher's Linear Discriminant. \\

\subsection{Description} Applying LDA consists in maximizing the so-called {\em Rayleigh quotient}:
 \begin{equation}\label{eq:LDA}
 \AAlpha_1=\mathrm{argmax}_{\AAlpha} \frac{\AAlpha^\intercal \SB \AAlpha}{\AAlpha^\intercal \SW \AAlpha} \mbox{ ,}
 \end{equation}
where $\SB$ is the {\em between-class scatter matrix} already defined in \eqref{eq:SB} and $\SW$ is called 
{\em within-class} (or intra-class) {\em scatter matrix}:

\begin{equation}
\SW = \sum_{\sensVarGenValue\in\sensVarSet}\sum_{i=1}^{\nbProfilingTraces}(\vLeakVec_i-\mmmXclass)(\vLeakVec_i-\mmmXclass)^\intercal \mbox{.}
\end{equation}


\begin{remark}
Let $\covmat$ be the the global covariance matrix of data, also called {\em total scatter matrix}, defined in \eqref{eq:covmat}; we have the following relationship between $\SW,\SB$ and $\covmat$:
\begin{equation}
\covmat = \frac{1}{\nbProfilingTraces}(\SW + \SB) \mbox{ .}
\end{equation}
\end{remark}

It can be shown that the vector $\AAlpha_1$ which maximizes \eqref{eq:LDA} must satisfy $\SB\AAlpha_1 = \lambda \SW \AAlpha_1$, for a constant $\lambda$, \textit{i.e.} has to be an eigenvector of $\SW^{-1} \SB$. Moreover, for any eigenvector $\AAlpha$ of $\SW^{-1}\SB$, with associated eigenvalue $\lambda$, the Rayleigh quotient equals such a $\lambda$:
\begin{equation}\label{eq:lambdas}
\frac{\AAlpha^\intercal \SB \AAlpha}{\AAlpha^\intercal \SW \AAlpha} = \lambda \mbox{ .}
\end{equation}
Then, among all eigenvectors of $\SW^{-1} \SB$, $\AAlpha_1$ must be the leading one. \\

The computation of the eigenvectors of $\SW^{-1} \SB$ is known under the name of {\em generalized eigenvector problem}. The difficulty here comes from the fact that $\SW^{-1} \SB$ is not guaranteed to be symmetric. Due to this non-symmetry,  $\AAlpha_1$ and the others eigenvectors do not form an orthonormal basis for $\mathbb{R}^\traceLength$, but they are anyway useful for classifications scopes. Let us refer to them as {\em Linear Discriminant Components} (LDCs for short); as for PCs we consider them sorted in decreasing order with respect to their associated eigenvalue, which gives a score for their informativeness, see \eqref{eq:lambdas}. Analogously to the PCA, the LDA provides a natural dimensionality reduction: one can project the data over the $\newTraceLength$ first LDCs. In Fig.~\ref{fig:LDAprojection} the 2-class toy data used as example above, projected over their leading discriminant component, are depicted. The two classes are kept well separated in the 1-dimensional subspace. As for PCA, this choice might not be optimal when applying this reduction to side-channel traces. For the sake of comparison, we test in Sec.~\ref{sec:experiments} all the selection methods proposed for the PCA (EGV, IPR and ELV) in association to the LDA as well.\\

In the following subsection we will present a well-known problem that affects the LDA in many practical contexts, and describe four methods that circumvent such a problem, with the intention to test them over side-channel data.


\subsection{The Small Sample Size Problem}\label{sec:SSS}
In the special case in which the matrix $\SB$ is invertible, the generalized eigenvalue problem is convertible in a regular one, as in \cite{Standaert2008}. On the contrary, when $\SB$ is singular, the simultaneous diagonalization is suggested to solve such a problem \cite{Fukunaga}. In this case one can take advantage by the singularity of $\SB$ to apply the computational trick described in Sec.~\ref{sec:trick}, since at most $r = \mathrm{rank}(\SB)$ eigenvectors can be found.\\

If the singularity of $\SB$ does not affect the LDA dimensionality reduction, we cannot say the same about the singularity of $\SW$:  SCA and Pattern Recognition literatures point out the same drawback of the LDA, known as the {\em Small Sample Size problem} (SSS for short). It occurs when the total number of acquisitions $\nbProfilingTraces$ is less than or equal to the size $\traceLength$ of them.
%\footnote{It can happen for example when attacking an RSA implementation, where the acquisitions are often huge (of the order of 1,000,000 points) and the number of measurements may be small when the SNR is good, implying that a good GE can be achieved with a small $N$.} 
The direct consequence of this problem is the singularity of $\SW$ and the non-applicability of the LDA. \\

If the LDA has been introduced relatively lately in the SCA literature, the Pattern Recognition community looks for a solution to the SSS problem at least since the early nineties. We browsed some of the proposed solutions and chose some of them to introduce, in order to test them over side channel traces.

\subsubsection{Fisherface Method}
The most popular among the solutions to SSS is the so-called {\em Fisherface} method\footnote{The name is due to the fact that it was proposed and tested for face recognition scopes.} \cite{eigenfaces}. It simply relies on the combination between PCA and LDA: a standard PCA dimensionality reduction is performed to data, making them pass from dimension $\traceLength$ to dimension $\nbProfilingTraces-\nbClasses$, which is the general maximal rank for $\SW$. In this reduced space, $\SW$ is very likely to be invertible and the LDA therefore applies.

\subsubsection{$\SW$ Null Space Method}
It has been introduced by Chen et al. in \cite{Chen2000} and exploits an important result of Liu et al. \cite{liu1992generalized} who showed that Fisher's criterion \eqref{eq:LDA} is equivalent to:
 \begin{equation}
 \AAlpha_1=\mathrm{argmax}_{\AAlpha} \frac{\AAlpha^\intercal \SB \AAlpha}{\AAlpha^\intercal \SW \AAlpha + \AAlpha^\intercal \SB \AAlpha} \mbox{ .}
 \end{equation}
The authors of \cite{Chen2000} point out that such a formula is upper-bounded by 1, and that it achieves its maximal value, \textit{i.e.} 1, if and only if  $\AAlpha$ is in the null space of $\SW$. Thus they propose to first project data onto the null space of $\SW$ and then to perform a PCA, \textit{i.e.} to select as LDCs the first $\nbClasses - 1$ eigenvectors of the between-class scatter matrix of data into this new space. More precisely, let $Q = [\vvv_1, \dots, \vvv_{\traceLength - \mathrm{rank}(\SW)}]$ be the matrix of vectors that span the null space of $\SW$. \cite{Chen2000} proposes to transform the data $\vLeakVec$ into $\vLeakVec^\prime = QQ^\intercal\vLeakVec$. Such a transformation maintains the original dimension $\traceLength$ of the data, but let the new within-class matrix $\SW' = QQ^\intercal\SW QQ^\intercal$ be the null $\traceLength \times \traceLength$ matrix. Afterwards, the method looks for the eigenvectors of the new between-class matrix $\SB' = QQ^\intercal\SB QQ^\intercal$. Let $U$ be the matrix containing its first $\nbClasses - 1$ eigenvectors: the LDCs obtained via the $\SW$ null space method are the columns of $QQ^\intercal U$.

\subsubsection{Direct LDA}
As the previous, this method, introduced in \cite{Yu01adirect}, privileges the low-ranked eigenvectors of $\SW$, but proposes to firstly project the data onto the rank space of $\SB$, arguing the fact that vectors of the null space of $\SB$ do not provide any between-class separation of data. Let $D_B=V^\intercal\SB V$ be the diagonalization of $\SB$, and let $V^\star$ be the matrix of the eigenvectors of $\SB$ that are not in its null space, \textit{i.e.} whose eigenvalues are different from zero. Let also $D_B^\star$ denotes the matrix ${V^\star}^\intercal\SB V^\star$; transforming the data $\vLeakVec$ into ${D_B^\star}^{1/2}{V^\star}^\intercal\vLeakVec$ makes the between-class variance to be equal to   the $(\nbClasses - 1) \times (\nbClasses- 1)$ identity matrix. After this transformation the within-class variance assumes the form $\SW' = {D_B^\star}^{1/2}{V^\star}^\intercal\SW ' V^\star {D_B^\star}^{1/2}$. After storing the $\newTraceLength$ lowest-rank eigenvectors in a matrix $U^\star$, the LDCs obtained via the Direct LDA method are the columns of $V^\star{D_B^\star}^{1/2}{U^\star}^\intercal$. 


\subsubsection{$\ST$ Spanned Space Method}
The last variant of LDA that we consider has been proposed in \cite{huang} and is actually a variant of the Direct LDA: instead of removing the null space of $\SB$ as first step, this method removes the null space of $\ST = \SB + \SW$. Then, denoting $\SW'$ the within-class matrix in the reduced space, the reduced data are projected onto its null space, i.e. are multiplied by the matrix storing by columns the eigenvectors of $\SW'$ associated to the null eigenvector, thus reduced again. A final optional step consists in verifying whether  the between-class matrix presents a non-trivial null-space after the last projection and, in this case, in effectuating a further projection removing it.

\begin{remark}
Let us remark that, from a computational complexity point of view (see \cite{huang} for a deeper discussion), the least time-consuming procedure among the four proposed is the Direct LDA, followed by the Fisherface and the $\ST$ Spanned Space Method, that have a similar complexity. The $\SW$ Null Space Method is in general much more expensive, because the task of removing the $\SW$ null space requires the actual computation of the $(\traceLength\times \traceLength)$-dimensional matrix $\SW$, {\em i.e.} the computational trick  described in Sec.~\ref{sec:trick} is not applicable. On the contrary, the other three methods take advantage of such a procedure, reducing drastically their complexity.
\end{remark}



%----------------------------------------------------------------------------------------
%	SECTION 6
%----------------------------------------------------------------------------------------


\section{Experimental Results}\label{sec:experiments}

In this section we compare the different extractors (\ie functions applying a data dimensionality reduction, see Sec.~\ref{sec:extractors}) provided by the PCA and the LDA in association with the different techniques  of components selection. Defining an universal criterion to compare the different extractors would not make sense since the latter one should encompass a lot of parameters, sometimes opposite, that vary according to the context (amount of noise, specificity of the information leakage, nature of the side channel, etc.). For this reason we choose to split our comparisons into four scenarios. Each scenario has a single varying parameter that, depending on the attacker context, may wish to be minimized. Hereafter the definition of the four scenarios: 
\begin{itemize}
\item[][Scenario 1] varying parameter: number of attack traces $\nbAttackTraces$, 
\item[][Scenario 2] varying parameter: number of profiling traces $\nbProfilingTraces$, 
\item[][Scenario 3] varying parameter: number of projecting components selected $\newTraceLength$,
\item[][Scenario 4] varying parameter: number of original time samples implied into the trace preprocessing computation $\numPoI$ .
\end{itemize}
 
For scenarios in which $\nbProfilingTraces$ is fixed, the value of $\nbProfilingTraces$ is chosen high enough to avoid the SSS problem, and the extensions of LDA presented in Sec.~\ref{sec:SSS} are not evaluated.
 This choice of $\nbProfilingTraces$ will imply that the LDA is always performed in a favourable situation, which makes expect the LDA to be particularly efficient for these experiments. Consequentely, for the scenarios in which $\nbProfilingTraces$ is high, our goal is to study whether the PCA can be made almost as efficient as the LDA thanks to the component selection methods discussed in Sec.~\ref{sec:ELV}. 



\subsection{The testing adversary.}  
Our testing adversary attacks an 8-bit AVR microprocessor Atmega328P and acquires power-consumption traces via the ChipWhisperer platform \cite{o2014chipwhisperer}.\footnote{This choice has been done to allow for reproducibility of the experiments.} The target device stores a secret 128-bit key and performs the first steps of an AES: the loading of 16 bytes of the plaintext, the AddRoundKey step and the AES Sbox. It has been programmed twice: two different keys are stored in the device memory during the acquisition of the profiling and of the attack traces, to simulate the situation of two identical devices storing a different secret. The size $\traceLength$ of the traces equals $3996$. The sensitive target  variable is the output of the first Sbox processing, but, since the key is fixed also during the profiling phase, and both Xor and Sbox operations are bijective, we expect to detect three interesting regions (as those high-lighted by PCs 4, 5 and 6, in Fig.~\ref{fig:6components}): the reading of the first byte of the plaintext, the first AddRoundKey and the first Sbox. We consider an {\em identity classification} leaking function (i.e. we make minimal assumption on the leakage function, see Sec.~\ref{sec:sensVar}), which implies that the 256 possible values of the Sbox output yields to 256 classes. For each class we assume that the adversary acquires the same number $\nbTraces$ of traces, \textit{i.e.} $\nbProfilingTraces =\nbTraces\times 256$. After the application of the extractor $\extract$, the trace size is reduced to $\newTraceLength$. Then the attacker performs a Template Attack (see Sec.~\ref{sec:TA}), using $\newTraceLength$-variate Gaussian templates.


\begin{figure}[t]
\subfigure[]{\label{fig:1.1}
\includegraphics[width=0.5\textwidth]{../Figures/CARDIS2015//Criterion1.pdf}}
\subfigure[]{\label{fig:1.2}
\includegraphics[width=0.5\textwidth]{../Figures/CARDIS2015//Criterion1Good.pdf}}
\caption[Guessing Entropy as function of the number of attack traces]{Guessing Entropy as function of the number of attack traces for different extraction methods. All Guessing Entropies are estimated as the average rank of the right key over 100 independent experiments.}\label{fig:scenario1}
\end{figure}
\subsection{Scenario 1.}
To analyse the dependence between the extraction methods presented in Sections~\ref{sec:PCA} and \ref{sec:LDA} and the number of attack traces $\nbAttackTraces$ needed to achieve a given GE (Guessing Entropy, see Sec.~\ref{sec:metrics}), we fixed the other parameters as follows: $\nbTraces=50$ ($\nbProfilingTraces=50\times 256$), $\newTraceLength = 3$ and $\numPoI = 3996$ (all points are allowed to participate in the building of the PCs and of the LDCs). The experimental results, depicted in Fig.~\ref{fig:scenario1}\subref{fig:1.1}-\subref{fig:1.2}, show that the PCA standard method has very bad performances in SCA, while the LDA outperforms the others. Concerning the class-oriented PCA, we observe that its performance is close to that of LDA when combined with the selection methods ELV (which performs best) or IPR, while it is similar to the one of classic PCA when associated with the EGV selection.



\subsection{Scenario 2.}

Now we test the behaviour of the extraction methods as function of the number $N_z$ of available profiling traces per class. The number of components $\newTraceLength$ is still fixed to 3, $\numPoI=3996$ again and the number of attack traces is $\nbAttackTraces=100$. This scenario has to be divided into two parts: if $N_z\leq 15$, then $\nbProfilingTraces<\traceLength$ and the SSS problem occurs. Thus, in this case we test the four extensions of LDA presented in Sec.~\ref{sec:SSS}, associated to either the standard selection, to which we abusively refer as EGV,\footnote{It consists in keeping the $\newTraceLength$ first LDCs (the $C$ last for the Direct LDA)}
or to the IPR selection.  We compare them to the class-oriented PCA associated to EGV, IPR or ELV. The ELV selection is not performed for the techniques extending LDA, since for some of them the projecting LDCs are not associated to some eigenvalues in a meaningful way. On the contrary, if $\nbTraces \geq 16$ there is no need to approximate the LDA technique, so the classical one is performed. Results for this scenario are shown in Fig.~\ref{fig:scenario2}. It may be noticed that the combinations class-oriented PCA + ELV/IPR select exactly the same components, for our data, see Fig.~\ref{fig:pcaclass} and do not suffer from the lack of profiling traces. They are slightly outperformed by the $\SW$ Null Space method associated with the EGV, see Fig.\ref{fig:swnullspace}. The Direct LDA (Fig.~\ref{fig:direct}) method also provides a good alternative, while the other tested methods do not show a stable behaviour. The results in absence of the SSS problem (Fig.\ref{fig:notSSS}) confirm that the standard PCA is not adapted to SCA, even when provided with more profiling traces. It also shows that among class-oriented PCA and LDA, the class-oriented PCA converges faster.



\begin{figure}
\subfigure[]{\label{fig:fisherface}
\includegraphics[width=0.5\textwidth]{../Figures/CARDIS2015/Criterion2SSS_Fisherface.pdf}}
\subfigure[]{\label{fig:direct}
\includegraphics[width=0.5\textwidth]{../Figures/CARDIS2015/Criterion2SSS_DirectLDA.pdf}}
\subfigure[]{\label{fig:stspannedspace}
\includegraphics[width=0.5\textwidth]{../Figures/CARDIS2015/Criterion2SSS_STSpannedSpace.pdf}}
\subfigure[]{\label{fig:swnullspace}
\includegraphics[width=0.5\textwidth]{../Figures/CARDIS2015/Criterion2SSS_SWnullSpace.pdf}}
\subfigure[]{\label{fig:pcaclass}
\includegraphics[width=0.5\textwidth]{../Figures/CARDIS2015/Criterion2SSS_PCAclass.pdf}}
\subfigure[]{\label{fig:notSSS}
\includegraphics[width=0.5\textwidth]{../Figures/CARDIS2015/Criterion2notSSS.pdf}}
\caption[Guessing Entropy as function of the number of profiling traces.]{Guessing Entropy as function of the number of profiling traces. Figures \subref{fig:fisherface}-\subref{fig:swnullspace}: methods extending the LDA in presence of SSS problem; Figure \subref{fig:pcaclass}: class-oriented PCA in presence of the SSS problem; Figure \subref{fig:notSSS}: number of profiling traces high enough to avoid the SSS problem.}\label{fig:scenario2}
\end{figure}


\subsection{Scenario 3.}
Let  $\newTraceLength$ be now variable and let the other parameters be fixed as follows: $\nbAttackTraces = 100, N_z=200, \numPoI = 3996$. Looking at Fig.~\ref{fig:3}, we might observe that the standard PCA might actually well perform in SCA context if provided with a larger number of kept components; on the contrary, a little number of components suffices to the LDA. Finally, keeping more of the necessary components does not worsen the efficiency of the attack, which allows the attacker to choose $\newTraceLength$ as the maximum value supported by his computational means. 

\begin{figure}
\centering
\includegraphics[width=.5\textwidth]{../Figures/CARDIS2015/Criterion3.pdf}
\caption{Guessing Entropy as function of the trace size after reduction.}\label{fig:3}
\end{figure}

\begin{remark}
In our experiments the ELV selection method only slightly outperforms the IPR. Nevertheless, since it relies on more sound and more general observations, {\em i.e.} the maximization of explained variance concentrated into few points, it is likely to be more robust and less case-specific. For example, in Fig.~\ref{fig:notSSS} it can be remarked that while the class-oriented PCA + ELV line keeps constant on the value 1 of guessing entropy, the class-oriented PCA + IPR is sometimes higher than 1.
\end{remark}

\subsection{Scenario 4.}

\begin{figure}
\includegraphics[width=0.5\textwidth]{../Figures/CARDIS2015/Criterion4.pdf}
\includegraphics[width=0.5\textwidth]{../Figures/CARDIS2015/Criterion4cutted.pdf} 
\caption{Guessing Entropy as function of the number of time samples contributing to the extractor computation.}\label{fig:4}
\end{figure}

This is the single scenario in which we allow the ELV selection method to not only select the components to keep but also to modify them, keeping only some coefficients within each component, setting the other ones to zero. We select pairs \textit{(component, time sample)} in decreasing order of the ELV values, allowing the presence of only $\newTraceLength = 3$ components and $\numPoI$ different times samples, \ie each component must have only 3 entries different from 0.
Looking at Fig.~\ref{fig:4} we might observe that the LDA allows to achieve the maximal guessing entropy with only 1 PoI in each of the 3 selected components. 
Actually, adding PoIs worsen its performances, which is coherent with the assumption that the vulnerable information leaks in only a few points. Such points are excellently detected by the LDA. Adding contribution from other points raises the noise, which is then compensated by the contributions of further noisy points, in a very delicate balance. Such a behaviour is clearly visible in standard PCA case: the first 10 points considered raise the level of noise, that is then balanced by the last 1000 points.


\subsection{Overview of this Study and Conclusions}


This study focused on two well-known techniques to construct extractors for side-channel traces, the PCA and the LDA. The LDA method is more adequate than the PCA one, thanks to its class-distinguishing asset, but more expensive and not always available in concrete situations. We deduced from a general consideration about side-channel traces, \ie the fact that for secured implementations the vulnerable leakages are concentrated into few points, a new methodology  for selecting components, called ELV. We showed that the class-oriented PCA, equipped with the ELV, achieves performances close to those of the LDA, becoming a cheaper and valid alternative to the LDA. Being our core consideration very general in side-channel context, we believe that our results are not case-specific.  \\
A second part of the proposed study analysed experimentally some alternatives to the LDA in presence of SSS problem proposed in Pattern Recognition literature. Such experiments showed that the Direct LDA and the $\SW$ Null Space Method are promising techniques, exhibiting performances close to the ones given by the ELV-equipped class-oriented PCA. A synthetic overview of the performed comparisons in scenarios 1,2 and 3 is depicted in Table~\ref{table:results}.

\begin{table}
\centering
\begin{tabular}{|c|c|c|c|c|c|}
\hline
&&\multicolumn{4}{|>{\columncolor[gray]{0.7}}c|}{Parameter to minimize}\\
\hline
\multicolumn{1}{|>{\columncolor[gray]{0.7}}c|}{Method}&\multicolumn{1}{|>{\columncolor[gray]{0.7}}c|}{Selection}& $\nbAttackTraces$ &  $\nbProfilingTraces$ (SSS) &  $\nbProfilingTraces'$ ($\neg$SSS) &  $\newTraceLength$ \\
\hline
PCA standard & EGV & {\bf -} &  &{\bf -} &{\bf -} \\
\hline
PCA standard &\multicolumn{1}{|>{\columncolor[gray]{0.8}}c|}{ELV} & \multicolumn{1}{|>{\columncolor[gray]{0.9}}c|}{{\bf -}} & &\multicolumn{1}{|>{\columncolor[gray]{0.9}}c|}{{\bf -}} &\multicolumn{1}{|>{\columncolor[gray]{0.9}}c|}{{\bf -}} \\
\hline
PCA standard & IPR &{\bf -} & &{\bf -} &{\bf +} \\
\hline
PCA class & EGV & {\bf -} &{\bf -} &{\bf -} &{\bf -} \\
\hline
PCA class & \multicolumn{1}{|>{\columncolor[gray]{0.8}}c|}{ELV} &\multicolumn{1}{|>{\columncolor[gray]{0.9}}c|}{{\bf +}} &\multicolumn{1}{|>{\columncolor[gray]{0.9}}c|}{$\bigstar$}&\multicolumn{1}{|>{\columncolor[gray]{0.9}}c|}{$\bigstar$} &\multicolumn{1}{|>{\columncolor[gray]{0.9}}c|}{{\bf +}} \\
\hline 
PCA class & IPR & {\bf {\bf +}} &$\bigstar$ &{\bf +} &{\bf -} \\
\hline 
LDA & EGV &$\bigstar$ & & {\bf +} & $\bigstar$\\
\hline 
LDA & \multicolumn{1}{|>{\columncolor[gray]{0.8}}c|}{ELV} & \multicolumn{1}{|>{\columncolor[gray]{0.9}}c|}{{\bf +}} &  & \multicolumn{1}{|>{\columncolor[gray]{0.9}}c|}{{\bf +}} & \multicolumn{1}{|>{\columncolor[gray]{0.9}}c|}{$\bigstar$}\\
\hline 
LDA & IPR & {\bf +} & &{\bf +} & $\bigstar$ \\

\hline 
\multicolumn{1}{|>{\columncolor[gray]{0.8}}c|}{$\SW$ Null Space}  & EGV & &\multicolumn{1}{|>{\columncolor[gray]{0.9}}c|}{$\bigstar$ } & & \\
\hline 
\multicolumn{1}{|>{\columncolor[gray]{0.8}}c|}{$\SW$ Null Space}  & IPR & &\multicolumn{1}{|>{\columncolor[gray]{0.9}}c|}{{\bf +}} & & \\
\hline 
\multicolumn{1}{|>{\columncolor[gray]{0.8}}c|}{Direct LDA} & EGV & & \multicolumn{1}{|>{\columncolor[gray]{0.9}}c|}{$\bigstar$}& & \\
\hline 
\multicolumn{1}{|>{\columncolor[gray]{0.8}}c|}{Direct LDA} & IPR & &\multicolumn{1}{|>{\columncolor[gray]{0.9}}c|}{{\bf +}}& & \\
\hline
\multicolumn{2}{|>{\columncolor[gray]{0.8}}c|}{Fisherface} & &\multicolumn{1}{|>{\columncolor[gray]{0.9}}c|}{{\bf -}} & & \\
\hline 
\multicolumn{2}{|>{\columncolor[gray]{0.8}}c|}{$\ST$ Spanned Space}  & &\multicolumn{1}{|>{\columncolor[gray]{0.9}}c|}{{\bf -}} & & \\
\hline
\end{tabular}
\caption[Linear Methods. Overview of the extractors' performances.]{Overview of the extractors' performances in tested situations. Depending on the adversary purpose, given by the parameter to minimize, a $\bigstar$ denotes the best method, a ${\bf +}$ denotes a method with performances close to those of the best one and a ${\bf -}$ is for methods that show lower performances. Techniques introduced in \cite{Cagli2016} are highlighted by a grey background.}\label{table:results}
\end{table}

%----------------------------------------------------------------------------------------
%	SECTION 7
%----------------------------------------------------------------------------------------
\section{Misaligning Effects}\label{sec:misalignment}
The fact that trace misalignment leads to a drastic drop of the \emph{dimensionality reduction/ template attack} routine is well-known. When we are in presence of a misalignment, caused by the implementation of a countermeasure or  by the lack of a good trigger signal for the acquisition, the application of some previous re-synchronization techniques is recommended (see for instance \cite{choudary2014template}, where the same PCA and LDA techniques are exploited as template attack preprocessing, after a prior resnchronization). In this section we experimentally show how the approach based on linear dimensionality reduction described in this chapter is affected by traces misalignment. To this aim, we simply take the same data and parameters exploited for Scenario 1 in Sec.~\ref{sec:experiments}, and artificially misalign them through the technique proposed in Appendix~\ref{appendix:artificial_jitter} with parameters \texttt{sigma}$= 6$, \texttt{B}$= 4$. Then we tried to attack them through the 9 methodologies tested in Scenario 1. It may be noticed in Fig.~\ref{fig:PCA_LDA_misalignment} that none of the 9 techniques is still efficient, included the optimal LDA+EGV that lead to minimal guessing entropy with the synchronized traces using less than 7 attack traces. In this case it cannot lead to successful attack in less than 3000 traces.
\begin{figure}
\includegraphics[width=\textwidth]{../Figures/desynchro_results_PCA_LDA.pdf} 
\caption{Degradation of linear-reduction-based template attacks in presence of misalignment.}\label{fig:PCA_LDA_misalignment}
\end{figure} 
% Chapter Template

\chapter{Kernel Discriminant Analysis} % Main chapter title
\label{ChapterKernel}

We tackle the dimensionality reduction problem in the context of profiling attacks against implementations protected by masking countermeasure. For such attacks, the attacker might have or not access to the  random values drawn at every execution and used to mask the sensitive variables. If he have such a knowledge, then the dimensionality reduction problem turns to be equivalent to the case of unprotected implementations. Thus ,the classic statistics for the PoIs research (and we will concentrate over the SNR, see Sec.~\ref{sec:extractors}) and the linear dimensionality reduction techniques described in the previous chapter are still applicable and efficient. On the contrary, when such a knowledge is denied, linear techniques are completely inefficient: a non-linear function of the PoIs has to be considered in order to construct discriminant features from side-channel observations. In this chapter we propose to make use of Kernel Discriminant Analysis (KDA) technique to construct such a non-linear processing. To this aim we revisit the contents and the experimental results of the paper presented at CARDIS 2016 international conference \cite{cagli2016kernel}. After such a publication, the KDA has been compared to other non-linear dimensionality reduction techniques in \cite{manifold}, where manifold learning solutions such as ISOMAP, Locally Linear Embedding (LLE) and Laplacian Eigenmaps (LE) are proposed. Moreover, a use of the KDA in an unsupervised way to perform a higher-order SCA (as a key candidate distinguisher and not as a dimensionality reduction technique) has been proposed at CARDIS 2017 \cite{zhou2017novel}.


%----------------------------------------------------------------------------------------
%	SECTION 1
%----------------------------------------------------------------------------------------

\section{Motivation}
When a masking countermeasure is properly applied, it ensures that every sensitive variable $\sensRandVar$  is randomly split  into multiple shares $M_1,M_2,\dots,M_d$ in such a way that a relation $Z = M_1 \star \dots \star M_d$ holds for a group operation $\star$ (\emph{e.g.} the exclusive or for the Boolean masking). The value $d$ plays the role of a security parameter and the method is usually referred to as $(d-1)$th-order masking (since it involves $d-1$ random values). In many cases, especially for software implementations, the shares are manipulated at different times, and no time sample therefore shows dependency on $\sensRandVar$: in order to recover such  information the attacker is obliged to join information held by each of the $d$ shares, executing a so-called $d$th-order SCA. In the great majority of the published higher-order attacks, the PoI selection during the pre-attack characterization phase is either put aside or made  under the hypothesis that the random shares are known. Actually, the latter knowledge brings the problem back to the unprotected case. 
%, and the methods recalled in previous paragraphs can be directly applied. 
Here we relax this hypothesis and we consider  situations where the values of the random shares are unknown to the adversary. We however assume that the adversary can characterize the leakage before attacking the implementation, by controlling the value of the target variable $\sensRandVar$. These two assumptions put our study in the classical context of {\em template attacks without knowledge of the masks}. \\

\subsection{Getting information from masked implementations}\label{sec:HO}
The SNR measures, point by point, the information held by the first-order moment of the acquisition, \emph{i.e.} the mean, to which we can refer to as a \emph{1st-order information}. In masked implementations, such information is null: in any time sample the mean is independent from $\sensRandVar$ due to the randomization provided by the shares, namely $f(\sensVar) = \esper[\vaLeakVec\vert \sensRandVar=\sensVar]$ is constant, which implies that the SNR is asymptotically  null over the whole trace.\\
 
When a $(d-1)$th-order masking is applied, the information about the shared sensitive target $\sensRandVar$ lies in some $d$th-order statistical moments of the acquisition,\footnote{whence the name {\em $d$th-order attacks}} meaning that for some $d$-tuples of samples $(t_1,\dots ,t_d)$ the function $f(z) = \esper[\vaLeakVec[t_1]\vaLeakVec[t_2]\cdots \vaLeakVec[t_d]\vert \sensRandVar=\sensVar]$ (based on a $d$th-order raw moment) is not constant (equivalently, $f(\sensVar) = \esper[(\vaLeakVec[t_1]-\esper[\vaLeakVec[t_1]])\cdots (\vaLeakVec[t_d]-\esper[\vaLeakVec[t_d]])\vert \sensRandVar=\sensVar]$ is not constant, using the central moment). We can refer to such information as $d$th-order information.
In order to let the SNR reveal it, and consequently let such information be directly exploitable, the attacker must pre-process the traces through an extractor $\extract$ that renders the mean of the extracted data dependent on $\sensRandVar$, \emph{i.e.} such that $f(\sensVar) = \esper[\extract(\vaLeakVec)\vert \sensRandVar=\sensVar])$ is not constant. In this way, the $d$th-order information is brought back to a $1$st-order one.

\begin{property}[SCA efficiency necessary condition]\label{property:poly}
Let us denote by $\sensRandVar$ the SCA target and let us assume that $\sensRandVar$ is represented by a tuple of shares $M_i$ manipulated at $d$ different times. Denoting $t_1,\dots,t_d$ the time samples\footnote{not necessary distinct} where each share is handled, the output of an effective extractor needs to have at least one coordinate whose polynomial representation over the variables given by the coordinates of $\vaLeakVec$ contains at least one term divisible by the the $d$th-degree monomial $\prod_{i=1,\dots,d}\vaLeakVec[t_i]$ (see \emph{e.g.} \cite{carlet2014achieving} for more information).
\end{property}


\begin{remark}\label{remark:normalized}
The use of central moments has been experimentally shown to reveal more information than the use of the raw ones \cite{chari1999towards,ProuffRB09}. Thus we will from now on suppose that the acquisitions have previously been normalized, so that $\esperEst(\vLeakVec_i) = \boldsymbol{0}$ and $\varEst(\vLeakVec_i) = \boldsymbol{1}$. In this way a centred product coincides with a non-centred one. 
\end{remark}

We motivate hereafter through a simplified but didactic  example, the need for a computational efficient dimensionality reduction technique as preliminary step to perform an higher-order attack.


\subsection{Some strategies to perform higher-order attack}\label{sec:example}
We consider here an SCA targeting an $8$-bit sensitive variable $\sensRandVar$ which has been priorly split into $d$ shares and we assume that a reverse engineering and signal processing have priorly been executed to isolate the manipulation of each share  in a region of $\ell$ time samples. This implies that our SCA  now amounts to extract a $\sensRandVar$-dependent information from leakage measurements whose size has been reduced to $d\times \ell$ time samples. To extract such information the State-of-the-Art proposes three approaches to the best of our knowledge.\\

The first one consists in considering $d$ time samples  at a time, one per region, and in testing if they jointly contain information about $\sensRandVar$ (\emph{e.g.} by estimating the mutual information  \cite{Reparaz2012} or by processing a Correlation Power Attack (CPA) using their centred product \cite{chari1999towards}, etc.). Computationally speaking, this approach requires to evaluate $\ell^d$ $d$-tuples (\emph{e.g.} 6.25 million $d$-tuples for $d=4$ and $\ell=50$), thus its complexity grows exponentially with $d$. \\

The second approach, that avoids the exhaustive enumeration of the $d$-tuples of time samples, consists in estimating the conditional pdf of the whole region: 
to this scope, a Gaussian mixture model is proposed in literature \cite{lemke2007gaussian,Lomne2014} and the parameters of such a Gaussian mixture can be estimated through the expectation-maximization (EM) procedure. In \cite{lemke2007gaussian} 4 variants of the procedure are proposed according to a trade-off between the efficiency and the accuracy of the estimations;  the most rough leads to the estimation of  $256^{(d-1)}(\ell d)$  parameters (\emph{e.g.} $\approx 3.4 $ billion parameters for $d=4$ and $\ell=50$), while the finest one requires the estimation of $256^{(d-1)}(1 + \frac{3\cdot \ell d}{2} + \frac{(\ell d)^2}{2}-1)$ parameters (\emph{e.g.} $\approx 87$ trillion parameters). Once again, the complexity of the approach grows exponentially with the order $d$.\\

The third approach, whose complexity does not increase exponentially with $d$, is the application of the higher-order version of the PP tool for the PoI selection, for which we give an outline hereafter. As will be discussed in Sec.~\ref{sec:comparisonPP}, its heuristic nature is the counterpart of the relatively restrained complexity of this tool. \\

%

\subsubsection{Higher-Order Version of Projection Pursuits}
The $d$th-order version of PP makes use of the so-called \emph{Moment against Moment Profiled Correlation} (MMPC) as objective function. The extractor $\extract^{PP}$ has the following form:
\begin{equation} 
\extract^{PP}(\vLeakVec) = (\AAlpha\cdot\vLeakVec)^d \mbox{ ,}
\end{equation}
where $\AAlpha$ is a sparse projecting vector with $d$ non-overlapping windows of coordinates set to 1, where the method has identified points of interest. Actually, as will be discussed in Sec.~\ref{sec:comparisonPP}, authors of \cite{PP} propose to exploit $\AAlpha$ as a pointer of PoIs, but do not encourage the use of $\extract^{PP}$ as an attack preprocessing. The procedure is divided into two parts: a global research called {\em Find Solution} and a local optimization called {\em Improve Solution}. At each step of {\em Find Solution}, $d$ windows are randomly selected to form a primordial $\AAlpha$, thus a primordial $\extract^{PP}$. A part of the training traces are then processed via $\extract^{PP}$ and used to estimate the $d$th-order statistical moments $\mmm^d_\sensRandVar = \esperEst_i[(\extract^{PP}(\vLeakVec_i))_{i:\sensVar_i=\sensVar}])$,  for each value of $\sensRandVar$. Then the Pearson correlation coefficient $\hat{\rho}$ between such estimates and the same estimates issued from a second part of the training set is computed. If $\hat{\rho}$ is higher than some threshold $T_{det}$, the windows forming $\AAlpha$ are considered interesting\footnote{A further validation is performed over such windows, using other two training sets to estimate $\hat{\rho}$, in order to reduce the risk of false positives.}\label{fn:4trainingSets} and \emph{Improve Solution} optimises their positions and lengths, via small local movements. Otherwise, the $\AAlpha$ is discarded and another $d$-tuple of random windows is selected.\\

The threshold $T_{det}$ plays a fundamental role in this crucial part of the algorithm: it has to be small enough to promote interesting windows (avoiding false negatives) and high enough to reject uninteresting ones (avoiding false positives). A hypothesis test is used to choose a value for $T_{det}$ in such a way that the probability of $\hat{\rho}$ being higher than $T_{det}$ given that no interesting windows are selected is lower than a chosen significance level $\beta$.\footnote{Interestingly, the threshold $T_{det}$ depends on size of $\sensVarSet$ and not on the size of the training sets of traces. This fact disables the classic strategy that consists in enlarging the sample, making $T_{det}$ lower, in order to raise the statistical power of the test (\emph{i.e.} $\mathrm{Prob}[\hat{\rho}>T_{det}\vert \rho=1]$). Some developments of this algorithm have been proposed \cite{durvauximproved}, also including the substitution of the MMPC objective function with a \emph{Moments against Samples} one, that would let $T_{det}$ decrease when increasing the size of the training set.} \\

\subsection{Foreword of this study}

The exploitation of KDA technique in the way we propose in this chapter aims to exploit interesting $d$-tuples of time samples as the first strategy described in Sec.~\ref{sec:example}. It however improves it in several aspects. In particular, its complexity does not increase exponentially with $d$. Moreover, it may be remarked that such a first approach allows the attacker to extract interesting $d$-tuples of points, but does not provide any hint to conveniently combine them. This is an important limitation since  finding a convenient way to combine time samples would raise the SCA efficiency. This remark has already been made for the unprotected case and for the particular case of implementations protected by first-order masking \cite{boosting}. Nevertheless in the SCA literature no specific method has been proposed for the general case $d>2$.  This study aims to propose a new answer to this question, while showing that it compares favourably to the PP approach.

%\subsection{Our Contribution}
%The ideas developed in the paper are based on the so-called Kernel Discriminant Analysis (KDA) \cite{hofmann2008kernel,scholkopf1999fisher}, which essentially consists in applying a so-called \emph{kernel trick} to the LDA. The trick is a stratagem that allows performing the LDA over a higher-dimensional \emph{feature space} (whose dimension can even raise exponentially with $d$) in which information about $Z$ lies in single dimensions and can be enhanced through linear combinations, keeping the computational cost independent from the dimension of the feature space (thus independent from $d$).\footnote{Even if the complexity is independent of $d$, the amount of information extracted is still decreasing exponentially with $d$, as expected when $(d-1)$-th order masking is applied \cite{chari1999towards}.} This study is in line with other recent works aiming to apply machine learning techniques to the side-channel context, such as \cite{machineLearningSCA,lerman2015template} which also exploit kernel functions to deal with non-linearly separable data.\\

%We show that the application of the KDA comes with several issues that we identify and analyse.
%We afterwards apply it to attack  masked operations implemented over an 8-bit AVR microprocessor Atmega328P. The experiments performed in $2$nd-order, $3$rd-order and $4$th-order contexts confirm the effectiveness of the KDA: in all cases the dimensionality reductions provided by the KDA lead to efficient and successful key recovery template attacks.\\



%The paper is organised as follows: in Sec.~\ref{sec:general} the notations are established, together with the formalization of the treated problem.  Moreover, some State-of-the-Art methods, efficient for the $1$st-order context, are recalled. In Sec.~\ref{sec:KDA}, the KDA method is described. A discussion about issues related to the application of the KDA to the SCA is conducted in Sec.~\ref{sec:practice} on the basis of experimental results. Finally, a comparison with the PP approach is proposed in Sec.~\ref{sec:PP}.

%----------------------------------------------------------------------------------------
%	SECTION 2
%----------------------------------------------------------------------------------------

\section{Feature Space, Kernel Function and Kernel Trick}


As described in Sec.~\ref{sec:HO}, the hard part of the construction of an effective extractor  is the detection of $d$ time samples $t_1,\dots,t_d$ where the shares leak. A naive solution, depicted in Fig.~\ref{fig:scheme1}, consists in applying to the traces the centred product preprocessing for each $d$-tuple of time samples. Formally it means immerse the observed data in a higher-dimensional space, via a non-linear function
\begin{equation}\label{eq:featureSpace}
\Phi \colon \mathbb{R}^\traceLength \rightarrow \featureSpace = \mathbb{R}^{{D+d-1}\choose{d}} \mbox{ .}
\end{equation}
 Using the Machine Learning language the higher-dimensional space $\featureSpace$ will be called \emph{feature space}, because in such a space the attacker finds the features that discriminate different classes. Procedures involving a feature space defined as in \eqref{eq:featureSpace} imply the construction, the storage and the management of ${D+d-1}\choose{d}$-sized traces; such a combinatorially explosion of the size of $\featureSpace$ is undoubtedly an obstacle from a computational standpoint.
 
\begin{figure}
\centering
{
\begin{tikzpicture}
\matrix (m) [matrix of math nodes, row sep=3em,
column sep=3em, text height=1.5ex, text depth=0.25ex]
{ \mathbb{R}^\traceLength & \featureSpace & \mathbb{R}^\newTraceLength \\};
\path[->]
(m-1-1) edge node[above] {$\Phi$} (m-1-2);
         %edge [bend left=30] (m-2-2)
         %edge [bend right=15] (m-2-2);
\path[->]
($(m-1-2.north east)-(0,0.1)$) edge node[above] {$\extract^{\mathrm{PCA}}$} ($(m-1-3.north west)-(0,0.1)$);
\path[->]
($(m-1-2.south east)+(0,0.15)$) edge node[below] {$\extract^{\mathrm{LDA}}$} ($(m-1-3.south west)+(0,0.15)$);

\end{tikzpicture} 
}
\caption{Performing LDA and PCA over a high-dimensional feature space.}\label{fig:scheme1}
\end{figure} 
 
 In Machine Learning a stratagem known as \emph{kernel trick} is available for some linear classifiers, such as Support Vector Machine (SVM), PCA and LDA, to turn them into non-linear classifiers, providing an efficient way to implicitly compute them into a high-dimensional feature space. This section gives an intuition about how the kernel trick acts. It explains how it can be combined with the LDA, leading to the so-called KDA algorithm, that enables an attacker to construct some non-linear extractors that concentrate in few points the $d$-th order information held by the side-channel traces, without requiring computations into a high-dimensional feature space, see Fig.~\ref{fig:scheme2}. 

\begin{figure}
\centering
{
\begin{tikzpicture}
\matrix (m) [matrix of math nodes, row sep=3em,
column sep=3em, text height=1.5ex, text depth=0.25ex]
{ \mathbb{R}^\traceLength & \featureSpace & \mathbb{R}^\newTraceLength \\};
\path[->]
(m-1-1) edge node[above] {$\Phi$} (m-1-2);
         %edge [bend left=30] (m-2-2)
         %edge [bend right=15] (m-2-2);
\path[->]
($(m-1-2.north east)-(0,0.1)$) edge node[above] {$\extract^{\mathrm{PCA}}$} ($(m-1-3.north west)-(0,0.1)$);
\path[->]
($(m-1-2.south east)+(0,0.15)$) edge node[below] {$\extract^{\mathrm{LDA}}$} ($(m-1-3.south west)+(0,0.15)$);

\path[->]
(m-1-1) edge [bend left=50] node[above] {$\extract^{\mathrm{KPCA}}$} (m-1-3)
(m-1-1) edge [bend right=50] node[below] {$\extract^{\mathrm{KDA}}$} (m-1-3);

\end{tikzpicture} 
}
\caption{Applying KDA and KPCA permits to by-pass computations in $\featureSpace$.}\label{fig:scheme2}
\end{figure}

The central tool of a kernel trick is the \emph{kernel function} $K \colon \mathbb{R}^\traceLength \times \mathbb{R}^\traceLength \rightarrow \mathbb{R}$, that has to satisfy the following property, in relation with the function $\Phi$:

\begin{equation}\label{eq:kernelProperty}
K(\vLeakVec_i,\vLeakVec_i) = \Phi(\vLeakVec_i)\cdot \Phi(\vLeakVec_i) \mbox{ ,}
\end{equation}
for each $i,j= 1,\dots, \numTraces$, where $\cdot$ denote the dot product.


Every map $\Phi$ has an associated kernel function given by \eqref{eq:kernelProperty}, for a given set of data. The converse is not true: all and only the functions $K\colon\mathbb{R}^D\times \mathbb{R}^D \rightarrow \mathbb{R}$ that satisfy a convergence condition known as {\em Mercer's condition} are associated to some map $\Phi:\mathbb{R}^D	\rightarrow \mathbb{R}^F$, for some $F$. Importantly, a kernel function is interesting only if it is computable directly from the rough data $\vLeakVec$, without evaluating the function $\Phi$. \\

The notion of kernel function is illustrated in the following example.

\begin{example}\label{ex:polyKernel}
Let $D=2$. Consider the function
\begin{align}
&K\colon\mathbb{R}^2\times \mathbb{R}^2 \rightarrow \mathbb{R} \nonumber \\ 
&K\colon(\vLeakVec_i,\vLeakVec_j) \mapsto ( \vLeakVec_i\cdot \vLeakVec_j)^2 \mbox{ ,} \label{eq:example1}
\end{align}

After defining $\vLeakVec_i = [a,b]$ and $\vLeakVec_j = [c,d]$, we get the following development of K:
\begin{equation}
K(\vLeakVec_i,\vLeakVec_j) = (ac + bd)^2 = a^2c^2 + 2abcd + b^2d^2 \mbox{ ,}
\end{equation}

which is associated to the following map from $\mathbb{R}^2$ to $\mathbb{R}^3$:
\begin{equation}
\Phi(u,v) =  [u^2, \sqrt{2}uv, v^2]
\end{equation}

Indeed $\Phi(\vLeakVec_i)\cdot \Phi(\vLeakVec_j) = a^2c^2 + 2abcd + b^2d^2 = K(\vLeakVec_i,\vLeakVec_j)$\enspace. This means that to compute the dot product between some data mapped into the $3$-dimensional space $\featureSpace$ there is no need to apply $\Phi$: applying $K$ over the $2$-dimensional space is equivalent. This trick allows the short-cut depicted in Fig.~\ref{fig:scheme2}.

\end{example}



In view of the necessary condition exhibited by Property~\ref{property:poly},  the function $K(\vLeakVec_i,\vLeakVec_j) = (\vLeakVec_i \cdot \vLeakVec_j)^d$, hereafter named \emph{$d$th-degree polynomial kernel function}, is the convenient choice for an attack against implementations protected with $(d-1)$th-order masking. It corresponds to a function $\Phi$ that brings the input coordinates into a feature space $\featureSpace$ containing all possible $d$-degree monomials in the coordinates of $\vLeakVec$, up to constants. This is, up to constants, exactly the $\Phi$ function of \eqref{eq:featureSpace}.\footnote{Other polynomial kernel functions may be more adapted if the acquisitions are not free from $d^\prime$th-order leakages, with $d^\prime<d$. Among non-polynomial kernel functions, we effectuated some experimental trials with the most common Radial Basis Function (RBF), obtaining no interesting results. This might be caused by the infinite-dimensional size of the underlying feature space, that makes the discriminant components estimation less efficient.}\\ 


\subsection{Local Kernel Functions as Similarity Metrics}
\todo{give an intuition and anticipate siamese networks}
%----------------------------------------------------------------------------------------
%	SECTION 3
%----------------------------------------------------------------------------------------

% SEI ARRIVATA FINO A QUI E POI HAI UN PROBLEMA DI NOTAZIONI \sss[z_i]{j}
\section{Kernel Discriminant Analysis}
As shown in \ref{eq:kernelProperty}, a kernel function $K$ allows us to compute the dot product between elements mapped into the feature space $\featureSpace$ \eqref{eq:kernelProperty}, and more generally any procedure that is only composed of such products. Starting from this remark, the authors of  \cite{scholkopf1998nonlinear} have shown that the PCA and LDA procedures can be adapted to satisfy the latter condition, which led them to define the KPCA and KDA algorithms. The latter one is described in the following procedure.

\begin{procedure}[KDA for $d$th-order masked side-channel traces]\label{procedure:KDA}\\
Given a set of labelled profiling side-channel traces $(\vLeakVec_i, \sensVar_i)_{i=1, \dots , \nbProfilingTraces}$ and the kernel function $K(\vLeakVec,\yyy)= (\vLeakVec \cdot \yyy)^d$:
\begin{itemize}

\item[1)] Construct a matrix $\MMM$ (acting as \emph{between-class scatter matrix}):

\begin{equation}
\MMM = \sum_{\sensVar\in\sensVarSet}\nbTracesPerClass(\MMMclass - \MMMT)(\MMMclass-\MMMT)^\intercal\mbox{ ,}
\end{equation}

where $\MMMclass$ and $\MMMT$ are two $N$-size column vectors whose entries are given by:
\begin{align}
\MMMclass[\sensVar][j] = \frac{1}{\nbTracesPerClass}\sum_{i:\sensVar_i=\sensVar}^{\nbTracesPerClass}K(\vLeakVec_j,\vLeakVec_i)\\
\MMMT[j] = \frac{1}{\nbProfilingTraces}\sum_{i=1}^{\nbProfilingTraces}K(\vLeakVec_{j},\vLeakVec_{i}) \mbox{ .}
\end{align}


\item[2)] Construct a matrix $\NNN$ (acting as \emph{within-class scatter matrix}):

\begin{equation}\label{eq:N}
\NNN = \sum_{\sensVar\in\sensVarSet}\kernelMatrix_\sensVar(\III - \III_{\numTraces})\kernelMatrix_\sensVar^\intercal\mbox{ ,}
\end{equation}
where $\III$ is a $\nbTracesPerClass \times \nbTracesPerClass$ identity matrix, $\III_{\nbTracesPerClass}$ is a $\nbTracesPerClass \times \nbTracesPerClass$ matrix with all entries equal to $\frac{1}{\nbTracesPerClass}$ and $\kernelMatrix_{\sensVar}$ is the $\nbProfilingTraces \times \nbTracesPerClass$

%ARRIVATA QUI A LEGGERE E A SISTEMARE LE NOTAZIONI

 sub-matrix of $\kernelMatrix = (K(\sss[z_i]{i},\sss[z_j]{j}))_{\substack{i=1,\dots,\numTraces[] \\ j=1,\dots,\numTraces[]}}$ storing only columns indexed by the indices $i$ such that $z_i=z$. 

\item[2bis)] Regularize the  matrix $\NNN$ for computational stability:
\begin{equation}\label{eq:mu}
\NNN = \NNN + \mu  \III \quad \mbox{ see \ref{sec:mu};}
\end{equation}

\item[3)]\label{point:eigs} Find the non-zero eigenvalues $\lambda_1, \dots, \lambda_\numEigenvectors$ and the corresponding eigenvectors $\nununu_1, \dots, \nununu_\numEigenvectors$ of $\NNN^{-1}\MMM$; 


\item[4)] Finally, the projection of a new trace $\sss[]{}$ over the $\ell$-th non-linear $d$-th order discriminant component can be computed as:
\begin{equation}\label{eq:projection}
\extract^{\mathrm{KDA}}_{\ell}(\vec{x}) = \sum_{i=1}^{\NPoI}\boldsymbol{\nu}_\ell[i]K(\sss[z_i]{i}, \sss[]{}) \mbox{ .}
\end{equation} 

\end{itemize}
\end{procedure}
For the reasons discussed in the introduction of this section, the right-hand side of \eqref{eq:projection} may be viewed as an efficient way to process the $\ell$-coordinate of the vector $\extract^{LDA}(\Phi(\sss[]{})) = \textbf{w}_{\ell} \cdot \Phi(\sss[]{})$,
without evaluating $\Phi(\sss[]{})$. The entries of $\textbf{w}_{\ell}$ are never computed, and will thus be referred to as \emph{implicit coefficients} (see Remark 2). It may be observed that each discriminant component $\extract^{\mathrm{KDA}}_{\ell}(\cdot)$ depends on the training set $(\sss[z_i]{i})_{i=1,\dots,\NPoI}$, on the kernel function $K$ and on the regularization parameters $\mu$.

\begin{remark}[The implicit coefficients]\label{rem:implicit}
As already said, the KDA, when the $d$th-degree  polynomial kernel function is chosen as kernel function, operates implicitly in the feature space of all products of $d$-tuples of time samples. In order to investigate the effect of projecting a new trace $\sss[]{}$ over a component $\extract^{\mathrm{KDA}}_{\ell}(\sss[]{})$, we can compute for a small $d$ the implicit coefficients that are assigned to the $d$-tuples of time samples through \eqref{eq:projection}. For $d=2$ we obtain that in such a feature space the projection is given by the linear combination computed via the coefficients shown below: 
\begin{equation}\label{eq:implicit}
\extract^{\mathrm{KDA}}_{\ell}(\sss[]{}) = \sum_{j=1}^\traceLength \sum_{k=1}^\traceLength[ (\sss[]{}[j]\sss[]{}[k])\underbrace{(\sum_{i=1}^{\NPoI}\boldsymbol{\nu}_{\ell}[i] \sss[]{i}[j]\sss[]{i}[k])}_{\mbox{implicit \\ coefficients}}]
\end{equation}

\end{remark}


\subsection{Computational complexity analysis}
The order $d$ of the attack does not significantly influence the complexity of the KDA algorithm. Let $\NPoI$ be the size of the training trace set and let $D$ be the trace length, then the KDA requires:
\begin{itemize}
\item $\frac{\NPoI^2}{2}D$ multiplications, $\frac{\NPoI^2}{2}(D-1)$ additions and $\frac{\NPoI^2}{2}D$ raising to the $d$-th power, to process the kernel function over all pairs of training traces
\item $(D+C)$ multiplications, $(D+C-2)$ additions and 1 raising to the $d$-th power for the projection of each new trace over $C$ KDA components,
\item the cost of the eigenvalue problem, that is $O(\NPoI^3)$.
\end{itemize} 


%----------------------------------------------------------------------------------------
%	SECTION 4
%----------------------------------------------------------------------------------------
\section{Experiments over Atmega328P}
\subsection{The Regularization Problem}
\subsection{The Multi-Class Trade-Off}
\subsection{Multi-Class vs 2-class Approach}
\subsection{Asymmetric Preprocessing/Attack Approach}
\subsubsection{Comparison with Projection Pursuits}\label{sec:comparisonPP}


%----------------------------------------------------------------------------------------
%	SECTION 5
%----------------------------------------------------------------------------------------
\section{Drawbacks of Kernel Methods}
\subsubsection{Misalignment Effects}
\subsubsection{Memory Complexity and Actual Number of Parameters}
\subsubsection{Two-Phases Approach: Preprocessing-Templates}
% che tra l'altro per farlo dividi le tracce di profilaggio in due

% Chapter Template

\chapter{Convolutional Neural Networks} % Main chapter title
\label{ChapterCNN}

\epigraph{\textit{Aiutiamoli a fare da soli!}\newline
\footnotesize{\textit{Let's help them to do themselves!}}}{--- \textup{Maria Montessori}}

% TODO: manca universal approximation theorem (da pagg 194 e seguenti del deeplearningbook)

In this chapter we explore a new strategy to perform profiling SCAs, addressing the misalignment issue and endorsing the Deep Learning (DL) paradigm. To this aim we present results published in \cite{DBLP:conf/ches/CagliDP17}, where Convolutional Neural Networks are proposed to help against misalignment-oriented countermeasures. Actually, the term Time-Delay Neural Network (TDNN) would be more appropriated than Convolutional Neural Network. Indeed the TDNNs  \cite{lang1990time} consist in the Convolutional Neural Networks applied to one-dimensional data, as side-channel traces are. Nevertheless, the fame that CNNs reached in last years, and especially since 2012, where a CNN architecture (the \textquotedbl AlexNet\textquotedbl) \cite{KSH12}  won the \emph{ImageNet Large Scale Visual Recognition Challenge}, a large-impact image recognition contest, leads to the disappearing of term TDNN from DL literature. Today, to specify the architecture of a TDNN in the most common DL libraries, one needs to exploit functionalities related to the CNNs' architecture, specifying \eg that one of the input dimensions equals 1. For these reasons we kept the term CNN for our discussion. 

\section{Motivation}
The context we choose to study DL techniques, and CNNs in particular, is the one of cryptographic implementations protected by countermeasures aiming at enhancing misalignment or desynchronisation in side-channel acquisitions. The latter countermeasures are either implemented in hardware (random hardware interruption or non deterministic processors \cite{irwin2002instruction,may2001non}, unstable clock \cite{moore2002improving,moore2003balanced}) or in software (insertion of random delays through dummy operations \cite{coron2009efficient,coron2010analysis}, shuffling \cite{veyrat2012shuffling}). Techniques analysed in previous chapters were applied in contexts where acquisitions were perfectly synchronous, and are not able to well extend to desynchronised context, as briefly observed in Secs.~\ref{sec:misalignment} and \ref{sec:KDAdrawbacks}.\\

Desynchronisation might be seen as a noise component of the acquisitions, as done in the leakage model proposed in \cite{chari1999towards}. Anyway, it raises the noise that hides sensitive information in the traces. From a statistical point of view, a theoretically satisfying answer to such a noise raise is the solely augmentation of the number of acquisitions: if the attack strategy, in terms of exploited statistical tools, keeps unchanged, increasing the acquisitions by a factor which is somehow linear in the misalignment effect, as discussed in \cite{mangard2004hardware}, might suffice to let the attack be as effective as in the synchronous case. In practice, such an augmentation might be unacceptable for many reasons. First, an attacker or evaluator might have a time or memory bound for the acquisition campaign. Second, the attacked device might implement a security defence denying an unlimited number of executions. Third, attack routines might suffer, in terms of complexity, more than linearly from a raising of the number of  data to be treated, \eg the KDA search for a non-linear feature extraction has a complexity that grows in a cubic way with the number of traces.  \\

The second approach proposed in the SCA literature to deal with misaligned trace sets consists in applying a realignment preprocessing before the attack. Two realignment techniques families might be distinguished: a signal-processing oriented one (\eg \cite{nagashima2007dpa,van2011improving}), more adapted to hardware countermeasures, and a probabilistic-oriented one (\emph{e.g.} \cite{durvaux2012efficient}), conceived for the detection of dummy operations, \emph{i.e.} software countermeasures. \\

We found in Convolutional Neural Networks the possibility of performing a profiling attack in an end-to-end form, directly extracting sensitive information from rough data, without applying any realignment preprocessing. We believe that realignments, a well as dimensionality reduction techniques, as discussed in Sec.~\ref{sec:KDAdrawbacks}, bring with them the risk of corrupting useful information in data. Indeed a realignment process acts modifying signals with the goal of obtaining some well-synchronised dataset, making traces be somehow similar to each other. On one hand it is not trivial to evaluate the accuracy of a realignment, thus to establish if a performed preprocessing is satisfying. On the other hand, the goal of a realignment is not extracting sensitive and discriminant information from traces. Even if we were able to affirm that a resynchronisation is somehow  perfect, by means of some special metrics, nothing guaranties that in the attempt of realigning the trace set the useful information is not discarded. Nowadays, CNNs and DL tools in general are standing out, thanks to their good scalability to \textquotedbl big-data\textquotedbl context. One of their strength is that they are easily parallelisable, and can easily exploit computational facilities as GPUs (or the so-called \emph{TPU - Tensor Processing Units} developed by purpose for NNs), allowing computational accelerations. As we have seen in Sec.~\ref{sec:overfitting}, the higher amount of data are available, the higher capacity is admissible for a ML model, without incurring in overfitting, and higher capacity corresponds to the possibility of learning more complex problems. From this point of view, the success of NNs in last years is mainly due to the always increasing amount of available data, and to their scalability. However, even in contexts where a lack of data may occur, \eg side-channel contexts in which the number of acquisitions may be limited, a stratagem exists in ML literature, under the name of Data Augmentation (DA), that may allow high capacity NNs avoid overfitting and perform well. 

\section{Introduction}
Machine Learning approaches often decline in multiple preprocessing phases such as data realignment, feature selections or dimensionality reduction, followed by a final model optimisation. This is the case even for the SCA routines that we considered in previous chapters, or for SCAs that apply realignment preprocessing. Deep Learning is a branch of Machine Learning whose aim is to avoid any preliminary preprocessing step from the model construction work-flow. For example, in Deep Learning the data dimensionality reduction is not necessarily explicitly performed by a distinct learned  function $\extract$. On the contrary, they directly and implicitly extract interesting features, possibly realign data, and estimate the opportune model to solve the task. The model is searched in a family of models that are composed by a cascade of parametrisable layers, which may be optimised in a single global learning process. Such models are called \emph{Artificial Neural Network}, or simply \emph{Neural Networks} (NNs). 

\subsection*{Solution for the KDA Drawbacks}
By construction, NNs are the ML answer to the drawback of work-flows we analysed in previous chapters and discussed as \emph{two-phased approach drawback} in last section of Chapter~\ref{ChapterKernel}. Actually, NNs are answers to other drawbacks pointed out in the same section. \\

In particular NNs are not memory-based. This implies that, after the training phase whose computational complexity is influenced by the size of the training set, they do not need to access the training set any more. By consequence, the obtained model is in general faster in processing new data, than techniques obtained \via kernel machines, for which the training traces themselves are part of the model parameters. This property belongs to the characteristics allowing NNs to be easily scalable to huge training sets.\\

Finally, we pointed out as drawback of techniques analysed in previous chapters their weakness to trace misalignment. Since the CNNs has been developed to treat difficulties as misalignments, scaling, rotations, etc. usually met in image processing,  we claim in this chapter, and verify through various experiments, that such CNNs provide an attack strategy that can keeps effective in presence of misalignment countermeasure. 

\subsection*{Organisation of the Chapter}
In Sections \ref{sec:MLP} and \ref{sec:learningAlgorithm}, notions of DL are introduced. In particular the common classification-oriented \emph{Multi-Layer Perceptron} model is described together with the common practices to train it. The way we exploit NNs to perform SCAs is described in Sec.~\ref{sec:attackNN}, while the performance metrics we will use for experiments are given in Sec.~\ref{sec:performances_NN}. A description of the CNN models is provided in Sec.~\ref{sec:CNN} while the Data Augmentation techniques that we will exploit are introduced in Sec.~\ref{sec:DA}.  Finally, three sections are dedicated to experiments. We tested the same CNN architecture against three different targets: in Sec.~\ref{sec:soft} we test it against a software countermeasure. In Sec.~\ref{sec:hard} it is tested against a simulated hardware countermeasure, and, in Sec.~\ref{sec:AES}, against a real-case cryptographic implementation protected by an enhanced jitter. 


% The answers to the 3 drawbacks of previous chapter (misalignment, memory complexity and actual number of params, two phases approach)
%"In practice, however, it is often worth investing substantial computational resources during the training phase in order to obtain a compact model that is fast at processing new data" (Bishop intro chap 5)

% RICICLATO DALL'EX CAPITOLO 1
%Kernel techniques like the KDA  are as well inherited from Machine Learning domain and consist in strategies that allow to build interesting extensions of many algorithms. One of their characteristics, that turns to be a drawback to apply them in SCA context is that are memory-based: the entire set of profiling traces, \ie those acquired by observing the open samples, has to be stored and accessed in the attack phase. In this sense they are highly memory-consuming, and quite slow to apply: they do not scale well in presence of huge profiling trace sets as those that are often necessary to perform profiling SCAs. In contrast to them, models provided by Neural Networks (NNs) are Machine Learning solutions that are known to be easily scalable to huge datasets and not memory-based. We decided to explore such an approach and pointed out that it not only provided solutions to tackle such a computational performance drawback.


\section{Neural Networks and Multi-Layer Perceptrons}\label{sec:MLP}
In Chapters~\ref{ChapterIntroductionSCA} and \ref{ChapterIntroML} we highlighted a strong analogy between profiling SCAs and classical ML classification task. Thus, we are interested in the NNs' solutions for the classification task. We recall from Chapters~\ref{ChapterIntroML} that for the classification task, the learning algorithm is asked to construct a function $\MLmodel\colon \mathbb{R}^\traceLength \rightarrow \{0,1\}^{\numClasses}$, where elements of $\sensVarSet$, \ie the set of classes, are here expressed \via the \emph{one-hot encoding} (\ref{sec:notations}). The output of such a function is said to be \emph{categorical}, \ie $\sensVarSet$ is a discrete finite set. A variant of the classification task consists in finding a function $\MLmodel\colon \mathbb{R}^\traceLength \rightarrow [0,1]^{\numClasses}$ defining a probability distribution over classes. We will prefer this last formulation. Often for this task, NNs are exploited to create discriminative models, \ie models that directly approximate the latter function $\MLmodel$ which is actually viewed as the posterior conditional probability of a label given the observed trace. This is the use we propose in this chapter, and it is in opposition to the Template Attack we exploited in previous chapters. Indeed, as described in Sec.~\ref{sec:TA}, a TA is based over the construction of generative models, \ie the approximation of the \emph{templates}, which coincide with the conditional probabilities of the traces given a label. \\

Using NNs the function $\MLmodel$ is obtained by combining several simpler functions, called \emph{layers}. An NN has an \emph{input layer} (the identity over the input datum $\vLeakVec$), an output layer (the last function) and all other layers are called \emph{hidden} layers.  The output of $\MLmodel$ is a $\numClasses$-sized vector $\vNNOutput$ of scores for the $\numClasses$ labels. In general, such a vector might or not represent the approximation of a probability distribution. In our case it will. The nature of the NN's  layers, their number and their dimension in particular, is called the \emph{architecture} of the NN. All the parameters that define an architecture, together with some other parameters that govern the training phase, are its \emph{hyper-parameters}. The so-called \emph{neurons}, that give the name to the NNs, are the computational units of the network and essentially process a scalar product between the coordinates of its input and a vector of  \emph{trainable weights} (or simply \emph{weights}) that have to be \emph{trained}. Each layer processes some neurons and the outputs of the neuron evaluations will form new input vectors for the subsequent layer. As we will see, the trainable weights of a NN are in general those defining the linear operations, which are scalar products processed by the neurons. Neurons can be implemented to operate in parallel and are very efficient to be processed and differentiated on GPUs. \\


The {\em Multi-Layer Perceptrons} (MLPs), or \emph{Feed-forward Neural Networks}, are a family of NN's architectures, associated with a function $\MLmodel$  that is composed of multiple linear functions and some non-linear functions, called {\em activations}. The name \emph{feedforward} refers to the fact that the information flows from  the input to the output, through the intermediate computations, without any feedback connection in which outputs of the model are fed back into itself.  This is in opposition to the so-called \emph{Recurrent Neural Network} structures. The CNNs are a generalisation of the MLPs.  \\

We can express a typical classification-oriented MLP by the following form:
\begin{equation}\label{eq:MLP}
\MLmodel(\vLeakVec) = \softmax\circ\lambda_n\circ\sigma_{n-1}\circ\lambda_{n-1}\circ\dots\circ \lambda_1(\vLeakVec)=\yyy \mbox{ ,}
\end{equation}
where:
\begin{itemize}
\item The $\lambda_i$ functions are typically the so-called \emph{Fully-Connected} (FC) layers and are expressible as affine functions: denoting $\vLeakVec\in\mathbb{R}^D$ the input of an FC, its output is given by $\textbf{A}\vLeakVec + \vec{b}$, being $\textbf{A}\in\mathbb{R}^{D\times C}$ a matrix of weights and $\vec{b}\in\mathbb{R}^C$ a vector of biases. These weights and biases are the trainable weights of the FC layer. They are called \emph{Fully-Connected} because each $i$-th input coordinate is \emph{connected} to each $j$-th output via the $\textbf{A}[i,j]$ weight. FC layers can be seen as a special case of the linear layers in general feedforward networks, in which not all the connections are present. The absence of some $(i,j)$-th connections can be formalized as a constraint for the matrix $\textbf{A}$ consisting in forcing to $0$ its $(i,j)$-th coordinates.

\item  The $\sigma_i$ are the so-called \emph{activation functions} (ACT): an activation function is a non-linear real function that is applied independently to each coordinate of its input. In general it does not depend on trainable weights. We denote them by $\sigma$ since in general they are functions similar to the \emph{logistic sigmoid} introduced in \ref{example:LDA}, which is denoted by $\sigma$ as well: indeed historically sigmoidal functions, \ie real-valued, bounded, monotonic, and differentiable functions with a non-negative first derivative, were recommended. Nevertheless, the recommended function in modern neural network literature is the so-called \emph{Rectified Linear Unit} (ReLU), introduced by \cite{nair2010rectified} and defined as $\mathrm{ReLU}(\vLeakVec)[i] = \max(0,\vLeakVec[i])$. Even if this function is not sigmoidal, not being bounded, nor differentiable, the fact of being a non-linear transformation but still piecewise linear, allows to preserve many of the properties that make linear models easy to optimise with gradient-based method.
 

\item $\softmax$ is the \emph{softmax}\footnote{To prevent underflow, the log-softmax is usually preferred if several classification outputs must be combined.} function (SOFT), already introduced in \ref{example:LDA}: $\softmax(\vLeakVec)[i] = \frac{e^{\vLeakVec[i]}}{\sum_{j}e^{\vLeakVec[j]}}$.
\end{itemize}
 
The choice of the softmax function as last layer of a neural network classifier is the most common one. It allows the model $\MLmodel$ to be interpreted as a generalisation of the binary classifier described in \eqref{eq:binary_linear_classifier}, where the softmax takes the place of the sigmoid to make the model multi-class and the linear argument is substituted by all previous layers of $\MLmodel$. The previous layers take in charge any preprocessing and are supposed to predict the unnormalised log probabilities \eqref{eq:softmax_entries}. The role of the \emph{softmax} is thus to renormalise such output scores in such a way that they define a probability distribution $\MLmodel(\vLeakVec) \approx \pdf_{\given{\sensRandVar}{\vaLeakVec=\vLeakVec}}$. 

\section{Learning Algorithm}\label{sec:learningAlgorithm}
The weights of an NN are tuned during a training phase. They are first initialized with random values. Afterwards, they are updated  \via an
iterative approach which locally applies the (Stochastic) Gradient Descent
algorithm \cite{Goodfellow-et-al-2016} to minimize a loss function
quantifying the \emph{classification error} of the function
$\MLmodel(\vaLeakVec)$ over a training set.

\subsection{Training}\label{sec:training}
The training of an NN is said to be \emph{full batch learning} if
the full training database is processed before one update of the weights. At the opposite, if
a single training input is processed at a time, then the approach is named
\emph{stochastic}. In practice, one often prefers to follow an approach in
between, called \emph{mini-batch learning}, and to use small \emph{batches}, \ie
groups of training inputs, at a time during the learning. In this case a step of the training consists in: 
\begin{itemize}
\item selecting a batch of training traces $(\vLeakVec_i, \sensVar_i)_{i\in I}$ chosen in random order (here $I$ is a random set of indexes),
\item computing the outputs, or scores, of the current model function for the input batch $(\vNNOutput_i = \MLmodel(\vLeakVec_i))_{i\in I}$, 
\item evaluating the loss function, which in general involves values $\vNNOutput_i$ and $\sensVar_i$
\item computing the partial derivatives of the loss function with respect to each trainable weight (this is done through a method called \emph{backpropagation} \cite{LeCun2012}),
\item updating trainable parameters by subtracting from each a small multiple of the loss gradient (the used multiple is called \emph{learning rate}).
\end{itemize}  

The size of the mini-batch is generally
driven by several efficiency/accuracy factors which are \eg discussed in
\cite{GBC16} (\eg optimal use of the multi-core architectures, parallelisation
with GPUs, trade-off between regularisation effect and stability, etc.). \\

An iteration over all the training dataset during the Stochastic Gradient Descent is called an \emph{epoch}.
The number of epochs is an important hyper-parameter. Intuitively, running a too low number of epochs may lead to underfitting and high values, while running a too high number of epochs may lead to overfitting. In our experiments, we chose to apply the so-called \emph{early stopping} strategy \cite{Prechelt2012}  in order to avoid the need of a prior tuning of the number of epochs. It consists in choosing a stop criterion that will be involved during the training. In general, the choice is done on the basis of a stagnancy or  worsen of the validation accuracies or losses across epochs.\\

\subsection{Cross-Entropy}
The cross-entropy
metric is a classical (and often by default) tool to define the \emph{loss function} in a classification-oriented NN \cite{LCH05,Goodfellow-et-al-2016}. It is smooth and
decomposable, and therefore amenable to optimisation with standard
gradient-based methods. Before providing the definition of cross-entropy in \eqref{eq:crossentropy}, we precise the chosen form for the \emph{loss function}. Given a batch of training data $(\vLeakVec_i, \sensVar_i)_{i\in I}$ and their respective scores returned by the current model $(\vNNOutput_i)_{i\in I}$, the \emph{loss function} is defined as the following averaged value:

\begin{equation}\label{eq:lossfunction}
\mathcal{L} = -\frac{1}{\lvert I \rvert} \sum_{i\in I} \sum_{t=1}^{|\sensVarSet|}\vec{\sensVar_i}[t]\log{\vNNOutput_i[t]} \mbox{ ,}
\end{equation}   
where the vector $\vec{\sensVar_i}$ denotes the one-hot encoding of the value of the realisation $\sensVar_i=\sensVarValue{j}$, \ie the vector $\sensVarOneHot{j} = (0,\ldots , 0,\underbrace{1}_{j},0,\dots,0)$ (as defined in Sec.~\ref{sec:notations}).
There are two ways to interpret such a choice. 

\begin{itemize}
\item First, recalling that $\vNNOutput_i$ may be interpreted as an estimation of the conditional probability $\prob[\given{\sensRandVar}{\vaLeakVec=\vLeakVec_i}]$, the maximum-likelihood principle suggests to drive the training in such a way that for such an estimate the probability of the correct label $\sensVar_i$ is as high as possible. Thus, if we suppose that $\sensVar_i = \sensVarValue{j}$, we want to maximize $\vNNOutput_i[j]$ (or equivalently to minimize $-\log{\vNNOutput_i[j]}$).\footnote{We remark that thanks to the softmax function used as last network layer, each coordinate of $\vNNOutput_i$ is always strictly positive.} It may be observed that, thanks to the one-hot encoding, in which all entries of $\sensVarOneHot{j}$ are null but the $j$th one, such a log-likelihood rewrites as 
\begin{equation}\label{eq:log_lik}
-\log{\vNNOutput_i[j]} = -\sum_{t=1}^{|\sensVarSet|}\vec{\sensVar_i}[t]\log{\vNNOutput_i[t]} \mbox{ ,}
\end{equation}
which equals the quantity averaged in \eqref{eq:lossfunction}.
\item The second interpretation of the chosen loss function is linked to the fact that it actually represents the average of  the cross-entropy of pairs of well-chosen probability mass functions. Indeed interpreting $\vec{\sensVar_i} = (0,\ldots , 0,\underbrace{1}_{j},0,\dots,0)$ as the pmf of $\given{\sensRandVar}{\sensRandVar = \sensVarValue{j}}$, which corresponds to the exact probability density we want the network to approximate. Informally speaking, the cross-entropy between two probability distributions, in our case the probability mass functions defined by $\vec{\sensVar_i}$ and $\vNNOutput_i$, gives a measure of the dissimilarity between them, and is defined as follows:
\begin{equation}\label{eq:crossentropy}
\entropy(\vec{\sensVar_i}, \vNNOutput_i) = \entropy(\vec{\sensVar_i}) + D_{KL}(\vec{\sensVar_i} || \vNNOutput_i) = \esper_{\vec{\sensVar_i}}[-\log{\vNNOutput_i}] = -\sum_{t=1}^{|\sensVarSet|}\vec{\sensVar_i}[t]\log{\vNNOutput_i[t]} \mbox{ ,}
\end{equation}
where $\entropy$ denotes the entropy and $D_{KL}$ denotes the Kullback-Leibler divergence \cite{christopher2006pattern}. Thus, this is an information-theoretic notion, that comes out to be equivalent to the negative log-likelihood formula given by \eqref{eq:log_lik}. 
\end{itemize}
In conclusion, depending on the point of view, minimizing the loss function \eqref{eq:log_lik}, which is a cross-entropy averaged over the traces contained in a batch, corresponds to maximising the likelihood of the right label, or to minimize the dissimilarity between the network estimation of a distribution and the right distribution that we want it to approximate. 
We chose the loss function \eqref{eq:lossfunction} for our experiments. However, other metrics may be investigated and can
potentially lead to better results \cite{MHK10,SSZU15}. \\

As justified in Sec.~\ref{sec:validation}, for the experiments proposed in this chapter we will divide the side-channel profiling set into two subsets: the training one and the validation one. The training set will be processed by batch and used to update the NN's parameters. The validation set is exploited in general at the end of each epoch to monitor the training, and in particular to watch over the incoming of an overfitting phenomenon. Remarkably, cross-validation has not been performed to improve the accuracy of our observation. Instead, we used a side-channel attack set to  evaluate both the ability of the trained model to tackle the classification task, and the performance of the obtained attack strategy.

\section{Attack Strategy with an MLP}\label{sec:attackNN}
The strategy we adopt to perform a SCA, with an MLP, is almost identical to the classical Template Attack described in \ref{sec:TA}. The main difference will be that TA is based on generative models, while MLPs are used to construct a discriminative one. Indeed, in TA the templates \eqref{eq:class-conditional} are priorly estimated, while an MLP directly approximates the posterior probabilities \eqref{eq:a-posteriori} $\MLmodel(\vLeakVec) \approx \pdf_{\given{\sensRandVar}{\vaLeakVec=\vLeakVec}}$. Once this approximation is done, the attack strategy proceeds in the same way for both approaches. The attacker acquires the new attack traces, that he only can associate to the public parameter $\publicParRandVar$, obtaining couples  $(\vLeakVec_i, \publicParVar_i)_{i=1, \dots , \nbAttackTraces}$. Then he makes key hypotheses $\keyVar \in \keyVarSet$ and, making the assumption that each acquisition is an independent observation of $\vaLeakVec$, he associates to each hypothesis $\keyVar \in \keyVarSet$ a score $d_\keyVar$ given by \eqref{eq:joint_distr}, that in terms of MLP model $\MLmodel$ rewrites  as:

\begin{equation}\label{eq:NN_SCA}
d_{\keyVar} = \prod_{i=1}^{\nbAttackTraces} \MLmodel(\vLeakVec_i)[\sensFunction(\keyVar,\publicParVar_i)] \mbox{ .}
\end{equation}

Finally, the best key candidate $\hat{\keyVar}$ is the one maximising the joint probability, as in \eqref{eq:max_classifier}
\begin{equation}
\hat{\keyVar} = \argmax_{\keyVar} d_{\keyVar} \mbox{ .}
\end{equation}


\section{Performance Estimation}\label{sec:performances_NN}
\subsection{Maximal Accuracies and Confusion Matrix} 
The accuracy is the most common metric to both monitor and evaluate an NN. As already seen in Sec.~\ref{sec:validation}, the accuracy is defined as the successful classification rate reached over a dataset. The {\em training accuracy}, the \emph{validation accuracy} and the \emph{test accuracy} are the successful classification rates achieved respectively over the training, the validation and the test sets. At the end of each epoch it is useful to compute and to compare the training accuracy and the validation accuracy. For our study, we found interesting to consider the following two additional quantities: 
\begin{itemize}
\item the \emph{maximal training accuracy}, corresponding to the maximum of the training accuracies computed at the end of each epoch,
\item the \emph{maximal validation accuracy}, corresponding to the maximum of the validation accuracies computed at the end of each epoch.
\end{itemize}
In addition to the two quantities above, we will also evaluate the performances of our trained model, by computing a \emph{test accuracy}. Sometimes it is useful to complete this evaluation by looking at the so-called \emph{confusion matrix} (as the one appearing in the bottom part of Fig. \ref{fig:CW_shift_history}). Indeed the latter matrix enables for the identification of the classes which are confused, in case of misclassification. The confusion matrix corresponds to the distribution over the couples \emph{(true label, predicted label)} directly deduced from the results of the classification on the test set. A test accuracy of $100\%$ corresponds to a diagonal confusion matrix.\\

\subsection{Side-Channel-Oriented Metrics} The accuracy metric is perfectly adapted to the machine learning classification problem, but corresponds in side-channel language to the success rate of a Simple Attack, as already discussed in Chapter~\ref{ChapterIntroductionSCA}. When the attacker can acquire several traces with varying plaintexts, the accuracy metric is not sufficient alone to evaluate the attack performance.
Indeed such a metric only takes into account the label corresponding to the maximal score and does not consider the other ones, whereas an SCA through~\eqref{eq:NN_SCA} does. To take this remark into account, we will always associate the test accuracy to a side-channel metric defined as the minimal number  $N^\star$ of side-channel traces that makes the \emph{guessing entropy} (see \ref{sec:metrics}) be permanently equal to 1. In our experiments, we will estimate such a guessing entropy through 10 independent attacks. \\
%
%As we will see in the sections dedicated to our attack experiments, applying Machine Learning in a context where at the same time (1) the model to recover is complex and (2) the amount of exploitable measurements for the training is limited, may be ineffective due to some overfitting phenomena.
%
%\subsection{•} Often the training accuracy is higher than the validation one. When the gap between the two accuracies is excessive, we assist to the \emph{overfitting} phenomenon. It means that the NN is using its weights to \emph{learn by heart} the training set instead of detecting significant discriminative features. For this reason its performances are poor over the validation set, which is new to it. Overfitting occurs when an NN is excessively complex, \emph{i.e.} when it is able to express an excessively large family of functions. In order to keep the NN as complex as wished and hence limiting the overfitting, some \emph{regularization} techniques can be applied. For example, in this paper we will propose the use of the
%\emph{Data Augmentation} (DA)~\cite{simard2003best} that consists in artificially adding observations to the training set. Moreover we will take advantage of the \emph{early-stopping} technique~\cite{Prechelt2012} that consists in well choosing a stop condition based on the validation accuracy or on the validation loss (\emph{i.e.} the value taken by the loss function over the validation set).




%----------------------------------------------------------------------------------------
%	SECTION 2
%----------------------------------------------------------------------------------------

%\section{Misalignment of Side-Channel Traces}

%\subsection{The Necessity and the Risks of Applying Realignment Techniques}
%\subsection{Analogy with Image Recognition Issues}

%----------------------------------------------------------------------------------------
%	SECTION 3
%----------------------------------------------------------------------------------------

\section{Convolutional Neural Networks}\label{sec:CNN}

The Convolutional Neural Networks (CNNs) complete the classical  MLP model with two additional types of layers, in charge of making them robust to misalignment: the so-called {\em convolutional} layer based on a convolutional filtering, and the \emph{pooling} layer. We describe these two particular layers hereafter.


\paragraph*{Convolutional (CONV) layers} 
% OLD VERSION WITH VECTORS
%Convolutional Layers (CONV) are linear layers that share weights across space. A representation is given in Fig.~\ref{fig:CNN_layers}-\subref{fig:conv_layer}; since CNNs have been introduced for images \cite{lecun1995convolutional}, such representation differs from the most common one in which layer interfaces are arranged in a 3D-fashion (height, weight and depth). In Fig.~\ref{fig:conv_layer} we show a 2D-CNN (length and depth) adapted to 1D-data as side-channel traces are. To apply a CONV to an input,
%$n_{\text{filter}}$ small column vectors, called \emph{convolutional filters}, of
%size $W$ (aka \emph{kernel size}) are slid over 
%the input by some amount of units, called \emph{stride}. 
%The column vectors form a
%window, called \emph{patch} in the Machine Learning language, which defines a
%linear transformation of $W$ consecutive points of the data into new
%vectors of size $V$, arranged in such a way that $V$ is the depth of the layer output. The length dimension of the output of a convolutional layer depends on several parameters: the input length, the stride, and the \emph{padding}. The two most common ways to pad the input are called \emph{same padding} and \emph{valid padding}: with the \emph{same padding} the input is padded with some zeros at the beginning and at the end, in such a way that, for a stride equal to 1, the output has the same
%length than the input, for a stride equal to 2 the output length is exactly halved, for a stride equal to 3 it is exactly divided by 3, etc.  The \emph{valid padding} consists on the contrary to avoid any kind of padding. Only proper data points are used as input, and output length is adjusted: typically, for a stride equal to 1, the output length equals $\traceLength - W +1$, where $\traceLength$ is the input length. The
%coordinates of the window (viewed as a $W\times n_{\text{filters}}$ matrix) are among the trainable weights of the model. They slid over the input, so they are multiplied by different parts of the datum, but the they are constrained to keep unchanged while sliding, \ie to behave in the same way no matter the position of the input entries on the global input datum. This constraint aims to allow the CONV layer to
%learn shift-invariant features, \ie characteristics of the datum for which the position is not discriminant. Shift-invariant feature are largely present in image recognition context, which drove the development of CNNs. For examples the eyes, the nose and the mouth of a person in a picture, are discriminant features for the person no matter their position in the image. The ability at learning shift-invariant features makes CNNs robust to
%geometrical deformations~\cite{lecun1995convolutional} or to temporal deformation when considering side-channel signals. For this reason they are adequate to counteract misalignment-based countermeasures.
%%


Convolutional Layers (CONV) are linear layers that share weights across space. A representation is given in Fig.~\ref{fig:conv_layer}; since CNNs have been introduced for images \cite{lecun1995convolutional}, such representation differs from the most common one in which layer interfaces are arranged in a 3D-fashion (height, weight and depth). In Fig.~\ref{fig:conv_layer} we show a 2D-CNN (length and depth) adapted to 1D-data as side-channel traces are. To apply a CONV to an input of size $\traceLength\times V$, where the initial depth $V$ is one, for 1D-data,
$n_{\text{filter}}$ small matrices, called \emph{convolutional filters}, of
size $W\times V$ (where $W$ is called \emph{kernel size}) are slid over 
the length dimension of the input by some amount of units, called \emph{stride}. 
The filters form a window, called \emph{patch} in the Machine Learning language, which defines a
linear transformation of $W\times V$ consecutive points of the data into new
matrices of size $1\times n_{\text{filter}}$, arranged in such a way that$ n_{\text{filter}}$ is the depth of the layer output. The length dimension of the output of a convolutional layer depends on several parameters: the input length, the stride, and the \emph{padding}. The two most common ways to pad the input are called \emph{same padding} and \emph{valid padding}: with the \emph{same padding} the input is padded with some zeros at the beginning and at the end, in such a way that, for a stride equal to 1, the output has the same
length than the input, for a stride equal to 2 the input length is exactly halved, for a stride equal to 3 it is exactly divided by 3, etc.  The \emph{valid padding} consists on the contrary to avoid any kind of padding. Only proper data points are used as input, and output length is adjusted: typically, for a stride equal to 1, the output length equals $\traceLength - W +1$, where $\traceLength$ is the input length. The
coordinates of the window are among the trainable weights of the model. They slid over the input, so they are multiplied by different parts of the datum, but  they are constrained to keep unchanged while sliding, \ie to behave in the same way no matter the position of the input entries on the global input datum. This constraint aims to allow the CONV layer to
learn shift-invariant features, \ie characteristics of the datum for which the position is not discriminant. Shift-invariant features are largely present in image recognition context, which drove the development of CNNs. For examples the eyes, the nose and the mouth of a person in a picture, are discriminant features for the person no matter their position in the image. The ability at learning shift-invariant features makes CNNs robust to
geometrical deformations~\cite{lecun1995convolutional} or to temporal deformation when considering side-channel signals. For this reason they are adequate to counteract misalignment-based countermeasures.


\paragraph*{Pooling (POOL) layers} 
In the most typical example of convolutional layer, \ie a layer with stride equal to 1 and \emph{same padding}, the output size equals the input size multiplied by $n_{\text{filter}}$. If many of this kind of convolutional layers are stacked, it leads to a complexity exponential growing due to the increasing of data size through layers. To avoid such complexity explosion, the insertion of pooling (POOL) layers is recommended.  POOL layers 
are non-linear layers that reduce the spatial size (see Fig. \ref{fig:pool_layer}). As the CONV layers, they make a filter slide across the input. The filter is 1-dimensional, characterised by a length $W$, and usually the stride is chosen equal to its length; for example in Fig.\ref{fig:pool_layer} both the length and the stride equal 3, so that the selected segments of the input do not overlap. In contrast with convolutional layers, the pooling filter does not contain trainable weights; it only slides across the input to select a segment, then a pooling function is applied: the most common pooling functions are the \emph{max-pooling}, which outputs the maximum values within the segment, and the \emph{average-pooling}, which outputs the average of the coordinates of the segment. 

%OLD
%\begin{figure}[t]
%\centering
%\subfigure[]{\label{fig:conv_layer}
%\includegraphics[width=.4\textwidth]{../Figures/CHES2017/conv_filt.pdf}}
%\subfigure[]{\label{fig:pool_layer}
%\includegraphics[width=.4\textwidth]{../Figures/CHES2017/max_pooling.pdf}}
%\caption{\subref{fig:conv_layer} Convolutional filtering: $W=2$, $V=4$, $\mathrm{stride}=1$. \subref{fig:pool_layer} Max-pooling layer: $W = \mathrm{stride} = 3$.}\label{fig:CNN_layers}
%\end{figure}

%\begin{figure}[t]
%\centering
%\includegraphics[width=.4\textwidth]{../Figures/CHES2017/conv_filt.pdf}
%\caption{Convolutional filtering: $W=2$, $V^{\prime}=4$, $\mathrm{stride}=1$. }\label{fig:conv_layer}
%\end{figure}
%

\begin{figure}
\includegraphics[scale=1, center]{../tikz_per_manuscritto/conv_filter_2_1.pdf} \\
\includegraphics[scale=1, right]{../tikz_per_manuscritto/conv_filter_2_3.pdf} 

\caption[Convolutional layer.]{Two convolutional layers. Top: $W=2$, $V=1$, $n_{\text{filter}}=3$, $\mathrm{stride}=1$. Bottom: $W=2$, $V=3$, $n_{\text{filter}}=4$. }\label{fig:conv_layer}
\end{figure}

\begin{figure}[t]
\centering
\includegraphics[width=.4\textwidth]{../Figures/CHES2017/max_pooling.pdf}
\caption[Max-pooling layer.]{Max-pooling layer: $W = \mathrm{stride} = 3$. }\label{fig:pool_layer}
\end{figure}

\paragraph*{Discussion}
The reason why a CONV always applies several filters (\ie $n_{\text{filter}}>1$) is
that we expect each filter to extract a different kind of feature from
the input. These extracted features are arranged
side-by-side over an additional data dimension, the so-called
\emph{depth}.\footnote{Ambiguity: Neural networks with more that one non-linear layer are called
\emph{Deep Neural Networks}, where the \emph{depth} corresponds to the number of
layers.} The hope is that during training, automatically, each filter specialises over the detection/recognition/modalisation of a different discriminant feature, and the collection of all discriminant features allows the last network layer concluding a successful classification.  As one goes along convolutional layers, higher-level abstraction
features are expected to be extracted.  The face recognition problem provides a simplified didactic example for this concept: we may think to some first layers' filters that specialise in detecting some local patterns of borders and surfaces. Then we may think to a deeper layer that compose such local features and modelise the angles of eyes' borders. the pupils, their color. Then some deeper layers may compose such feature and modelise  the whole eye, which is a more complex feature, and some deeper layers may compose eyes together with noses' features coming from other filters and, going on in this compositional process, modelise the whole face, and assign to it a very abstract feature, \ie the name of the person, which is the goal of the classification task. The fact that many natural data in the works have such a compositional flavour is one of the justifications inventors of CNNs provide to explain the success of such a technique.\footnote{See for example Yann LeCun's class available at \url{https://www.college-de-france.fr/site/yann-lecun/course-2016-02-12-14h30.htm}}  Actually, analysing and understanding the very first low-level features extracted by a self-trained CNN is a very hard task, and such an impossibility to explain from where discriminant features come out is, in my opinion, one of the characteristics of the DL domain that leads it to be kept unconsidered and disliked by a still quite large community of scientists. 

\paragraph*{Common architecture}
The main block of a CNN is a CONV layer $\gamma$ directly followed by an ACT layer $\sigma$. The former locally extracts information from the input thanks to its filters and the latter increases the capacity of the model thanks to its non-linearity. After some $ ( \sigma \circ \gamma)$  blocks, a POOL layer $\delta$ is usually added to reduce the number of neurons: $\delta \circ [ \sigma\circ \gamma]^{n_2} $. This new block is repeated in the neural network until obtaining an output of reasonable size. Then, some FCs are introduced in order to obtain a global result which depends on the entire input, and not only on local features. To sum-up, a common convolutional network can be characterized by the following formula:\footnote{where each layer of the same type appearing in the composition is not to be intended as exactly the same function (\eg with same input/output dimensions), but as a function of the same form.} 
\begin{equation}\label{equ:CCN}
  \softmax \circ [\lambda]^{n_1} \circ[\delta \circ [\sigma \circ \gamma  ]^{n_2} ]^{n_3}  .
\end{equation}

 Layer by layer the network increases the spatial depth through convolution filters, adds non-linearity through activation functions and reduces the spatial (or temporal, in the side-channel traces case) size through pooling layers. Once a deep and narrow representation has been obtained, one or more FC layers are connected to it, followed by a softmax function. An example of CNN architecture is represented in Fig.~\ref{fig:archi_conv}. 
\begin{figure}[h]
\centering
\includegraphics[width=\textwidth]{../Figures/CHES2017/convnet_arch.pdf}
\caption{Common CNN architecture.}
\label{fig:archi_conv}
\end{figure} 



\section{Data Augmentation}\label{sec:DA}
As explained in Sec.~\ref{sec:overfitting}, ML models are prone to overfitting, especially when their capacity (see Sec.~\ref{sec:overfitting}) is very high, as it is often the case with deep networks. Thus, it is sometimes necessary to deal with the overfitting phenomenon, by applying some regularization techniques. As we will see in Secs. \ref{sec:soft} and \ref{sec:hard} this will be the case in our experiments: indeed we will propose a quite deep CNN architecture,  flexible enough to successfully manage the misalignment problems, but trained over some relatively small training sets. This fact, combined with the high capacity of our CNN architecture, implies that the model will \emph{learn by heart} each element of the training set, without catching the truly discriminant features of the traces.\\

Instead of applying a proper regularization techniques, we choose to concentrate priorly on the Data Augmentation strategy \cite{simard2003best}, mainly for two reasons. First, it is a common practice in side-channel context to increase the number of acquisitions to counteract the misalignment effect. In other terms, misalignment may provoke a \textquotedbl lack of data\textquotedbl phenomenon on adversary's side. In the ML domain, such a lack is classically addressed thanks to the DA technique, and its benefits are widely proved. For example, many image recognition competition winners made use of such a technique (\eg the winner of ILSVRC-2012 \cite{KSH12}). Second, the DA is controllable, meaning that the deformations applied to the data are chosen, thus fully characterized. It is therefore possible to fully determine the addition of complexity induced to the classification problem. In our opinion, other techniques add constraints to the problem in a more implicit and uncontrollable way, \eg the dropout \cite{HSKSS12}  or the $\ell_2$-norm regularization \cite{christopher2006pattern}.\\

Data augmentation consists in artificially generating new training traces by deforming those previously acquired. The deformation is done by the application of transformations that preserve the  label information (\ie the value of the handled sensitive variable in our context). We choose two kinds of deformations, that we denote by \emph{Shifting} and \emph{Add-Remove}. 

\paragraph*{Shifting Deformation ($\mathrm{SH}_{T^\star}$)} It simulates a random delay effect of maximal amplitude $T^\star$, by randomly selecting  a shifting window of the acquired trace, as shown in Fig. \ref{fig:SH}. Let $\traceLength$ denote the original size of the traces. We fix the size of the input layer of our CNN to $\traceLength^\prime = \traceLength - T^\star$. Then the technique $\mathrm{SH}_{T^\star}$  consists (1) in drawing a uniform random $t \in[0,T^\star]$, and (2) in selecting the $\traceLength^\prime$-sized window starting from the $t$-th point. For our study, we will compare the $\mathrm{SH}_T$ technique for different values $T \leq T^\star$, without changing the architecture of the CNN (in particular the input size $\traceLength^\prime$). Notably, $T \lneq T^\star$ implies that $T^\star-T$ time samples will never have the chance to be selected. As we suppose that the information is localized in the central part of the traces, we choose to center the shifting windows, discarding the heads and the tails of the traces (corresponding to the first and the last $\frac{T^\star-T}{2}$ points). 

\paragraph*{Add-Remove Deformation ($\mathrm{AR}$)}  It simulates a clock jitter effect (Fig. \ref{fig:AR}). We will denote by $\mathrm{AR}_R$ the operation that consists in two steps:
\begin{itemize}
\item[(1)] in inserting $R$ time samples, whose positions are chosen uniformly at random and whose values are the arithmetic mean between the previous time sample and the following one,
\item[(2)] in suppressing $R$ time samples, chosen uniformly at random.\\
\end{itemize}

The two deformations can be composed: we will denote by $\mathrm{SH}_T\mathrm{AR}_R$ the application of a $\mathrm{SH}_T$ followed by a $\mathrm{AR}_R$.

%
%\begin{figure}[t]
%\includegraphics[width=.5\textwidth, height=0.13\textheight]{../Figures/CHES2017/Shifting_window.pdf}
%\includegraphics[width=.5\textwidth, height=0.13\textheight]{../Figures/CHES2017/AR_example.pdf}
%\caption{Left: Shifting technique for DA. Right: Add-Remove technique for DA (added points marked by red circles, removed points marked by black crosses).}\label{fig:DA}
%\end{figure}


\begin{figure}[t]
\centering
\includegraphics[width=.5\textwidth]{../Figures/CHES2017/Shifting_window.pdf}
\caption{Shifting technique for DA.}\label{fig:SH}
\end{figure}

\begin{figure}[t]
\centering
\includegraphics[width=.5\textwidth]{../Figures/CHES2017/AR_example.pdf}
\caption[Add-Remove technique for DA.]{Add-Remove technique for DA (added points marked by red circles, removed points marked by black crosses).}\label{fig:AR}
\end{figure}

\paragraph*{Discussion}
The deformations we propose as Data Augmentation techniques are inspired by the way we modelise the countermeasures' effects. Actually, we propose to turn the misalignment problem into a virtue, enlarging the profiling trace set \via a random shift of the acquired traces and the AR distortion that together simulate a clock jitter effect. Paradoxically, instead of trying to realign the traces, we propose to further misalign them (a much easier task!). In real-case secure devices evaluation contexts, the acquisition campaign may sometimes represent a bottleneck in terms of time. Further proposals and analyses of  DA techniques, maybe inspired by other forms of noise present in side-channel acquisitions, might be interesting tracks for future researches. Actually, the idea of applying DA in profiling side-channel context appeared independently from our work, in another publication in 2017 \cite{pu2017trace}, under the name of \emph{Trace Augmentation}. In this paper, the augmentation is obtained with a shifting equivalent to our SH deformation, and it is applied as preliminary step for the profiling phase of a Gaussian TA. The authors' goal is to make Gaussian templates more robust to the discrepancy between profiling acquisitions and attack ones. Surprisingly, in the paper, authors observe that this augmentation provides benefits to the attack routine both in case where some discrepancies are present, and in the ideal case. Data Augmentation seems thus to be a good practice independently of the presence or not of specific countermeasures, nor the exploitation or not of DL techniques.



\section{Experiments against Software Countermeasures}

\begin{figure}
\includegraphics[width=.5\textwidth]{../Figures/CHES2017/CW_shift_traces.pdf} 
\includegraphics[width=.5\textwidth]{../Figures/CHES2017/CW_double_shift_traces.pdf} 
\caption[Leakages hidden by Random Delay Interruption.]{Left: one leakage protected by single uniform RDI. Right: two leaking operations protected by multiple uniform RDI.}\label{fig:CW_shift_traces}
\end{figure}

In this section we present two preliminary  experiments, performed in order to validate the shift-invariance claimed by the CNN architecture, recalled in Sec.~\ref{sec:CNN}. In the first one, a single leaking operation was observed through the side-channel acquisitions, shifted in time by the insertion of a random number of dummy operations. We will refer to such a countermeasure as Random Delay Interrupt (RDI).  In the second one we targeted two leaking operations each delayed by RDI. We remark that this kind of countermeasure is nowadays considered defeated, \eg thanks to resynchronisation by \emph{cross-correlation} \cite{nagashima2007dpa}. In this sense, the experiment we present in this section is not expected to be representative of real application cases. The complexity of the state-of-the-art resynchronisation techniques strongly depends on the variability of the shift. When the latter variability is low, \ie when attacks are judged to be applicable, multiple random delays are recommended. It has even been proposed to adapt the probabilistic distributions of the random delays to achieve good compromises between the countermeasure efficiency and the chip performance overhead \cite{coron2009efficient,coron2010analysis}. Attacks have already been shown even against this multiple-RDI kind of countermeasures, \eg \cite{durvaux2012efficient}. The latter attack exploits some Gaussian templates to classify the leakage of each instruction; the classification scores are used to feed a Hidden Markov Model (HMM) that describes the complete chip execution, and the Viterbi algorithm is applied to find the most probable sequence of states for the HMM and to remove the random delays. We remark that this HMM-based attack exploits Gaussian templates to feed the HMM model, and the accuracy of such templates is affected by other misalignment reasons, \eg clock jitter. We believe that our  CNN approach proposal for operation classification, is a valuable alternative to  the Gaussian template one, and might even provide benefits to the HMM performances, by \eg improving the robustness of the attack in presence of both RDI and jitter-based countermeasures. This robustness w.r.t. of misalignment caused by the clock jitter will be analysed in Sec. \ref{sec:hard}.

\subsection{One Leaking Operation}\label{sec:soft}
For this experiment, we implemented, on an Atmega328P microprocessor, a uniform RDI \cite{tunstall2007efficient} to protect the leakage produced by a single target operation. Our RDI simply consists in a loop of  $r$ \emph{nop} instructions, with $r$  drawn uniformly in $[0,127]$. Some acquired traces are reported in the left side of Fig. \ref{fig:CW_shift_traces}, the target peak being highlighted with a red ellipse. They are composed of $3,996$ time samples, corresponding to an access to the AES-Sbox look-up table stored in NVM. For the training, we acquired only $1,000$ traces and 700 further traces were acquired as validation data. Our CNN has been trained to classify the traces according to the Hamming weight of the Sbox output; namely, our labels are the nine values taken by $\sensRandVar = \HW(\Sbox(P\oplus \keyRandVar))$. This choice has been done to let each class contain more than only a few (i.e. about $1,000/256$) training traces.
For Atmega328P devices, the Hamming weight is known to be particularly relevant to model the leakage occurring during register writing (see for example Chapters~\ref{ChapterLinear} and \ref{ChapterKernel} or \cite{BelaidCFGKP15}). Since $\sensRandVar$ is assumed to take nine values and the position of the leakage depends on a random $r$ ranging over 128 values, it is clear that the $1,000$ training traces do not encompass the full $9 \times 128=1,152$ possible combinations $(z,r)\in [0,8]\times[0,127]$. We undersized the training set by purpose, in order to establish whether the CNN technique, equipped with DA, is able to catch the meaningful shift-invariant features without having been provided with all the possible observations.\\


\begin{figure}[t]
\includegraphics[width=.30\textwidth]{../Figures/CHES2017/DAshift0_2000traces_9classes_sgd/acc_DAshift0_2000traces_9classes_sgd.pdf} 
\includegraphics[width=.30\textwidth]{../Figures/CHES2017/DAshift100_2000traces_9classes_sgd/acc_DAshift100_2000traces_9classes_sgd.pdf} 
\includegraphics[width=.30\textwidth]{../Figures/CHES2017/DAshift500_2000traces_9classes_sgd/acc_DAshift500_2000traces_9classes_sgd.pdf}\\ 
\includegraphics[width=.30\textwidth]{../Figures/CHES2017/DAshift0_2000traces_9classes_sgd/CM_DAshift0_2000traces_9classes_sgd.pdf} 
\includegraphics[width=.30\textwidth]{../Figures/CHES2017/DAshift100_2000traces_9classes_sgd/CM_DAshift100_2000traces_9classes_sgd.pdf} 
\includegraphics[width=.30\textwidth]{../Figures/CHES2017/DAshift500_2000traces_9classes_sgd/CM_DAshift500_2000traces_9classes_sgd.pdf}
\caption[Software misalignment: accuracies vs epochs and confusion matrices obtained with our CNN for different DA techniques.]{One leakage protected via uniform RDI: accuracies vs epochs and confusion matrices obtained with our CNN for different DA techniques. From left to right: $\mathrm{SH}_0$, $\mathrm{SH}_{100}$, $\mathrm{SH}_{500}$. }\label{fig:CW_shift_history}
\end{figure}

For the training of our CNN, we applied the $SH_T$ data augmentation, selecting $T^\star = 500$ and $T \in \{ 0,100, T^\star\}$; this implies that the input dimension of our CNN is reduced to $3,496$. Our implementation is based on Keras library \cite{keras} (version 1.2.1), and we run the trainings over an ordinary computer equipped with a gamers market GPU, a GeForce GTS 450. For the CNN architecture, we chose the following structure: 
\begin{equation}\label{eq:archi}
  \softmax \circ [\lambda]^1 \circ[\delta \circ [\sigma \circ \gamma  ]^1 ]^4,   
\end{equation}
\emph{i.e.} \eqref{equ:CCN} with $n_1 = n_2 = 1$ and $n_3 = 4$.
To accelerate the training we applied a technique proposed in 2015 \cite{batch_norm}, consisting in the  introduction of a so-called \emph{Batch Normalization} layer \cite{batch_norm} after each pooling $\delta$. The network transforms the $3,496 \times 1$ inputs in a $1 \times 256$ list of abstract features, before entering the last FC layer $\lambda:\mathbb{R}^{256}\rightarrow \mathbb{R}^9$. Even if the ReLU activation function \cite{nair2010rectified} is classically recommended for many applications in literature (see Sec.~\ref{sec:MLP}), we obtained in most cases better results using the hyperbolic tangent, defined as:
\begin{equation}
\mathrm{tanh}(x) = \frac{e^x-e^{-x}}{e^x+e^{-x}} \mbox{ .}
\end{equation}
We trained our CNN by batches of size $32$.  In total the network contained $869,341$ trainable weights. The training and validation accuracies achieved after each epoch are depicted in Fig.\ref{fig:CW_shift_history} together with the confusion matrices that we obtained from the test set. Applying the early-stopping principle recalled in Sec.~\ref{sec:training},  we automatically stopped the training after $120$ epochs without decrement of the loss function evaluated over the validation set, and kept as final trained model the one that showed the minimal value for the loss function evaluation. Concerning the learning rate (see Sec.~\ref{sec:training}), we fixed the beginning one to $0.01$ and reduced it multiplying it by a factor of $\sqrt{0.1}$ after $5$ epochs without validation loss decrement.


\begin{table}[t]
\centering
\caption[Results of our CNN, for different DA techniques, in presence of an uniform RDI countermeasure protecting.]{Results of our CNN, for different DA techniques, in presence of an uniform RDI countermeasure protecting. For each technique, $4$ values are given: in position $a$ the maximal training accuracy, in position $b$ the maximal validation accuracy, in position $c$ the test accuracy, in position $d$ the value of $N^\star$ (see Sec.~\ref{sec:performances_NN} for definitions).}
\label{tab:res_CW_shift}
\begin{tabular}{|c|c|c|c|c|c|c|c|}
\hline
\multicolumn{2}{|c|}{} & \multicolumn{2}{c|}{$\mathrm{SH}_{0}$}                                    & \multicolumn{2}{c|}{$\mathrm{SH}_{100}$} & \multicolumn{2}{c|}{$\mathrm{SH}_{500}$} \\ \hline
$a$        & $b$       & \cellcolor[HTML]{EFEFEF}100\%  & \cellcolor[HTML]{EFEFEF}25.9\%           & 100\%               & 39.4\%             & \textbf{98.4\%}     & \textbf{76.7\%}    \\ \hline
$c$        & $d$       & \cellcolor[HTML]{EFEFEF}27.0\% & \cellcolor[HTML]{EFEFEF}\textgreater1000 & 31.8\%              & 101                & \textbf{78.0\%}     & \textbf{7}         \\ \hline
\end{tabular}
\end{table}

Table \ref{tab:res_CW_shift} summarizes the obtained results.  For each trained model we can compare the maximal training accuracy achieved during the training with the maximal validation accuracy, defined in Sec.~\ref{sec:performances_NN}.  This comparison gives an insight about the risk of overfitting for the training.\footnote{The validation accuracies are estimated over a 700-sized set, while the test accuracies are estimated over $100,000$ traces. Thus the latter estimation is more accurate, and we recall that the test accuracy is to be considered as the final CNN classification performance.} Case $\mathrm{SH}_0$ corresponds to a training performed without DA technique. When no DA is applied, the overfitting effect is dramatic: the training set is $100\%$-successfully classified after about $22$ epochs, while the test accuracy only achieves $27\%$. The $27\%$ is around the rate of uniformly distributed bytes showing an Hamming weight of $4$.\footnote{We recall that the Hamming weight of uniformly distributed data follows a binomial law with coefficients $(8,0.5)$.} Looking at the corresponding confusion matrix we remark that the CNN training has been biased by the binomial distribution of the training data, and almost always predicts the class $4$. This essentially means that no discriminative feature has been learned in this case, which is confirmed by the fact that the trained model leads to an unsuccessful attack ($N^\star>1,000$). Remarkably, the more artificial shifting is added by the DA, the more the overfitting effect is attenuated; for $SH_T$ with \eg $T=500$ the training set is never completely learnt and the test accuracy achieves $78\%$, leading to a guessing entropy of 1 with only $N^{\star}=7$ traces. \\

These results confirm that our CNN model is able to characterize a wide range of points in a way that is robust to RDI. 

\subsection{Two Leaking Operations}
Here we study whether our CNN classifier suffers from the presence of multiple leaking operations with the same power consumption pattern. This situation occurs for instance any time the same operation is repeated several successive times over different pieces of data (\eg the SubByte operation for a software AES implementation is often performed by 16 successive look-up table accesses). To start our study we performed the same experiments as in Sec.~\ref{sec:soft} over a second traces set, where two look-up table accesses leak, each preceded by a random delay. Some examples of this second traces set are given in the right side of Fig.~\ref{fig:CW_shift_traces}, where the two leaking operations being highlighted by red and green ellipses. We trained the same CNN as in Sec.~\ref{sec:soft}, once to classify the first leakage, and a second time to classify the second leakage, applying $\mathrm{SH}_{500}$ as DA technique. Results are given in Table~\ref{tab:label}. They show that even if the CNN transforms spatial (or temporal) information into abstract discriminative features, it still holds an ordering notion: indeed if no ordering notion would have been held, the CNN could no way discriminate the first peak from the second one. 


\begin{table}[]
\centering
\caption[Results of our CNN in presence of uniform RDI protecting two leaking operations.]{Results of our CNN in presence of uniform RDI protecting two leaking operations. See the caption of Table~\ref{tab:res_CW_shift} for a legend.}
\label{tab:label}
\begin{tabular}{|c|c|c|c|c|c|}
\hline
\multicolumn{2}{|c|}{} & \multicolumn{2}{c|}{First operation} & \multicolumn{2}{c|}{Second operation} \\ \hline
$a$        & $b$       & 95.2\%            & 79.7\%           & 96.8\%            & 81.0\%            \\ \hline
$c$        & $d$       & 76.8\%            & 7                & 82.5\%            & 6                 \\ \hline
\end{tabular}
\end{table}




%----------------------------------------------------------------------------------------
%	SECTION 6
%----------------------------------------------------------------------------------------


\section{Experiments against Artificial Hardware Countermeasures}\label{sec:hard}

A classical hardware countermeasure against side-channel attacks consists in introducing instability in the clock. This implies the cumulation of a deforming effect that affects each single acquired clock cycle, and provokes traces misalignment on the adversary side. Indeed, since clock cycles do not have the same duration, they are sampled during the attack by a varying number of time samples. As a consequence, a simple translation of the acquisitions is not sufficient in this case to align with respect to an identified clock cycle. Several realignment techniques are available to manage this kind of deformations, \eg \cite{van2011improving}. In this context, our goal is to show that  we can get rid of the realignment pre-processing, letting the CNN deep structure take it in charge implicitly. 

\subsection{Performances over Artificial Augmented Clock Jitter}\label{sec:artificial}
In this section we present the results that we obtained over two datasets named \emph{DS\_low\_jitter} and \emph{DS\_high\_jitter}. Each one contains $10,000$ labelled traces, used for the training phase (more precisely, $9,000$ are used for the training, and $1,000$ for the validation), and $100,000$ attack traces. The traces are composed of $1,860$ time samples. The two datasets have been obtained by artificially adding a simulated jitter effect over some synchronised original traces. The original traces were measured on the same Atmega328P microprocessor used in the previous section. We verified that they originally encompass leakage on 34 instructions: 2 \emph{nops}, 16 loads from the NVM and 16 accesses to look-up tables. For our attack experiments, it is assumed that the target is the first look-up table access, \ie the 19th clock cycle. As in the previous section, the target is assumed to take the form $\sensRandVar=\HW(\Sbox(P\oplus \keyRandVar))$. To simulate the jitter effect we used the technique described in Appendix~\ref{appendix:artificial_jitter}, fixing parameters $\texttt{sigma}=4$, $\texttt{B}=2$ for the \emph{DS\_low\_jitter} dataset,  and $\texttt{sigma}=6$, $\texttt{B}=4$ for the \emph{DS\_high\_jitter} dataset. In the same Appendix~\ref{appendix:artificial_jitter},  some traces of  \emph{DS\_low\_jitter} and  \emph{DS\_high\_jitter} are depicted (respectively in Fig.~\ref{fig:jitter_traces22} and in Fig.~\ref{fig:jitter_traces66}): the cumulative effect of the jitter is observable by remarking that the desynchronisation raises with time. For both datasets we did not operate any PoI selection, but entered the entire traces into our CNN.

%
%\begin{figure}
%\centering
%\subfigure[]{\label{fig:jitter_traces22}
%\includegraphics[width=\textwidth]{../Figures/CHES2017/jitter_2_2_framed.png} }
%\subfigure[]{\label{fig:jitter_traces66}
%\includegraphics[width=\textwidth]{../Figures/CHES2017/jitter_6_6_framed.png} }
%\caption[Hardware misalignment: \emph{DS\_low\_jitter} and \emph{DS\_high\_jitter} datasets.]{\subref{fig:jitter_traces22}  some traces of the \emph{DS\_low\_jitter} dataset, a zoom of the part highlighted by the red rectangle is given in the bottom part. \subref{fig:jitter_traces22} some traces of the \emph{DS\_high\_jitter} dataset. The interesting clock cycle is highlighted by the grey rectangular area.}\label{fig:jitter_traces}
%\end{figure}


%
%\begin{figure}
%\centering
%\includegraphics[width=.7\textwidth]{../Figures/CHES2017/jitter_2_2_framed.png} 
%\caption{Some traces of the \emph{DS\_low\_jitter} dataset, a zoom of the part highlighted by the red rectangle is given in the bottom part. }\label{fig:jitter_traces22}
%\end{figure}
%
%\begin{figure}
%\centering
%\includegraphics[width=.7\textwidth]{../Figures/CHES2017/jitter_6_6_framed.png} 
%\caption{Some traces (and the relative) of the \emph{DS\_high\_jitter} dataset. The interesting clock cycle is highlighted by the grey rectangular area.}\label{fig:jitter_traces66}
%\end{figure}


We used the same CNN architecture \eqref{eq:archi} as in previous section. We assisted again to a strong overfitting phenomenon and we successfully reduced it by applying the DA strategy introduced in Sec.~\ref{sec:DA}. This time we applied both the \emph{shifting} deformation $\mathrm{SH}_T$ with $T^\star = 200$ and $T\in\{0,20,40\}$ and the \emph{add-remove} deformation $\mathrm{AR}_R$ with $R\in \{0,100,200\}$, training the CNN model using the nine combinations $\mathrm{SH}_T\mathrm{AR}_R$. We performed a further experiment with much higher DA parameters, \ie $\mathrm{SH}_{200}\mathrm{AR}_{500}$, to show that the benefits provided by the DA are limited: as expected, too much deformation affects the CNN performances (indeed results obtained with $\mathrm{SH}_{200}\mathrm{AR}_{500}$ will be worse than those obtained with \eg $\mathrm{SH}_{40}\mathrm{AR}_{200}$).





The results we obtained are summarized in Table~\ref{table:results_all}. Case $\mathrm{SH}_0\mathrm{AR}_0$ corresponds to a training performed without DA technique, hence serves as a reference suffering from the overfitting phenomenon. It can be observed that as the DA parameters raise, the validation accuracy increases while the training accuracy decreases. This experimentally validates that the DA technique is efficient in reducing overfitting. Remarkably in some cases, for example in the \emph{DS\_low\_jitter} dataset case with $\mathrm{SH}_{100}\mathrm{AR}_{40}$, the best validation accuracy is higher than the best training accuracy. In Fig.~\ref{fig:high_acc} the training and validation accuracies achieved in this case epoch by epoch are depicted. It can be noticed that the unusual relation between the training and the validation accuracies does not only concern the maximal values, but is almost kept epoch by epoch. Observing the picture, we can be convinced that, since this fact occurs at many epochs, this is not a consequence of some unlucky inaccurate estimations. To interpret this phenomenon we observe that the training set contains both the original data and the augmented ones (\ie deformed by the DA) while the validation set only contains non-augmented data.  The fact that the achieved training accuracy  is lower than the validation one, indicates that the CNN does not succeed in learning how to classify the augmented data, but succeeds to extract the features of interest for the classification of the original data.  We judge this behaviour positively. Concerning the DA techniques we observe that they are efficient when applied independently and that their combination is still more efficient.

\begin{figure}[h]
\centering
\includegraphics[width=.5\textwidth]{../Figures/CHES2017/acc_DAadd_remove100_shift_40_deep_good_for_CW_shift_wo_DO.pdf} 
\caption[Excessive Data Augmentation example.]{Training of the CNN model with DA $\mathrm{SH}_{100}\mathrm{AR}_{40}$. The training classification problem becomes harder than the real classification problem, leading validation accuracy constantly higher than the training one.}\label{fig:high_acc}
\end{figure}

\begin{figure}
\includegraphics[width=.5\textwidth]{../Figures/CHES2017/results_low_jitter_new.pdf} 
\includegraphics[width=.5\textwidth]{../Figures/CHES2017/results_high_jitter_new.pdf} 
\caption[Comparison between a Gaussian template attack, with and without realignment, and our CNN strategy, over the  \emph{DS\_low\_jitter} and the  \emph{DS\_high\_jitter}.]{Comparison between a Gaussian template attack, with and without realignment, and our CNN strategy, over the  \emph{DS\_low\_jitter} (left) and the  \emph{DS\_high\_jitter} (right).}\label{fig:compareTA}
\end{figure}



\newcolumntype{C}{>{\centering\arraybackslash}p{3em}}
\begin{table}[t]
\centering
\caption[Results of our CNN in presence of artificially-generated jitter countermeasure, with different DA techniques.]{Results of our CNN in presence of artificially-generated jitter countermeasure, with different DA techniques. See the caption of Table \ref{tab:res_CW_shift} for a legend.}
\label{table:results_all}



\begin{tabular}{|C|C|CCCCCC|CC}
\hline
\multicolumn{10}{|C|}{\textbf{\emph{DS\_low\_jitter}}}\\
\hline
$a$                           & $b$                         & \multicolumn{2}{C|}{}                                                                                      & \multicolumn{2}{C|}{}                                                                                     & \multicolumn{2}{C|}{}                                                                                  & \multicolumn{2}{C|}{}                                      \\ \cline{1-2}
$c$                           & $d$                         & \multicolumn{2}{C|}{\multirow{-2}{*}{$\mathrm{SH}_{0}$}}                                                   & \multicolumn{2}{c|}{\multirow{-2}{*}{$\mathrm{SH}_{20}$}}                                                 & \multicolumn{2}{c|}{\multirow{-2}{*}{$\mathrm{SH}_{40}$}}                                              & \multicolumn{2}{c|}{\multirow{-2}{*}{$\mathrm{SH}_{200}$}} \\ \hline
\multicolumn{2}{|c|}{}                                      & \multicolumn{1}{c|}{\cellcolor[HTML]{EFEFEF}100.0\%} & \multicolumn{1}{c|}{\cellcolor[HTML]{EFEFEF}68.7\%} & \multicolumn{1}{c|}{99.8\%}                         & \multicolumn{1}{c|}{86.1\%}                         & \multicolumn{1}{c|}{98.9\%}                                  & 84.1\%                                  &                              &                             \\ \cline{3-8}
\multicolumn{2}{|c|}{\multirow{-2}{*}{$\mathrm{AR}_{0}$}}   & \multicolumn{1}{c|}{\cellcolor[HTML]{EFEFEF}57.4\%}  & \multicolumn{1}{c|}{\cellcolor[HTML]{EFEFEF}14}     & \multicolumn{1}{c|}{82.5\%}                         & \multicolumn{1}{c|}{6}                              & \multicolumn{1}{c|}{83.6\%}                                  & 6                                       &                              &                             \\ \cline{1-8}
\multicolumn{2}{|c|}{}                                      & \multicolumn{1}{c|}{87.7\%}                          & \multicolumn{1}{c|}{88.2\%}                         & \multicolumn{1}{c|}{82.4\%}                         & \multicolumn{1}{c|}{88.4\%}                         & \multicolumn{1}{c|}{81.9\%}                                  & 89.6\%                                  &                              &                             \\ \cline{3-8}
\multicolumn{2}{|c|}{\multirow{-2}{*}{$\mathrm{AR}_{100}$}} & \multicolumn{1}{c|}{86.0\%}                          & \multicolumn{1}{c|}{6}                              & \multicolumn{1}{c|}{87.0\%}                         & \multicolumn{1}{c|}{5}                              & \multicolumn{1}{c|}{87.5\%}                                  & 6                                       &                              &                             \\ \cline{1-8}
\multicolumn{2}{|c|}{}                                      & \multicolumn{1}{c|}{83.2\%}                          & \multicolumn{1}{c|}{88.6\%}                         & \multicolumn{1}{c|}{81.4\%} & \multicolumn{1}{c|}{86.9\%} & \multicolumn{1}{c|}{\textbf{80.6\%}} &\textbf{88.9\%} &                              &                             \\ \cline{3-8}
\multicolumn{2}{|c|}{\multirow{-2}{*}{$\mathrm{AR}_{200}$}} & \multicolumn{1}{c|}{86.6\%}                          & \multicolumn{1}{c|}{6}                              & \multicolumn{1}{c|}{85.7\%} & \multicolumn{1}{c|}{6}      & \multicolumn{1}{c|}{\textbf{87.7\%}} & \textbf{5}      &                              &                             \\ \hline
\multicolumn{2}{|c|}{}                                      &                                                      &                                                     &                                                     &                                                     &                                                              &                                         & \multicolumn{1}{c|}{85.0\%}  & \multicolumn{1}{c|}{88.6\%} \\ \cline{9-10} 
\multicolumn{2}{|c|}{\multirow{-2}{*}{$\mathrm{AR}_{500}$}} &                                                      &                                                     &                                                     &                                                     &                                                              &                                         & \multicolumn{1}{c|}{86.2\%}  & \multicolumn{1}{c|}{5}      \\ \cline{1-2} \cline{9-10}
\multicolumn{10}{|C|}{}\\
\hline
\multicolumn{10}{|C|}{\textbf{\emph{DS\_high\_jitter}}}\\
\hline
$a$                          & $b$                         & \multicolumn{2}{C|}{\multirow{2}{*}{$\mathrm{SH}_{0}$}}   & \multicolumn{2}{C|}{\multirow{2}{*}{$\mathrm{SH}_{20}$}}  & \multicolumn{2}{C|}{\multirow{2}{*}{$\mathrm{SH}_{40}$}} & \multicolumn{2}{C|}{\multirow{2}{*}{$\mathrm{SH}_{200}$}} \\ \cline{1-2}
$c$                          & $d$                         & \multicolumn{2}{C|}{}                                     & \multicolumn{2}{C|}{}                                     & \multicolumn{2}{C|}{}                                    & \multicolumn{2}{C|}{}                                     \\ \hline
\multicolumn{2}{|C|}{\multirow{2}{*}{$\mathrm{AR}_{0}$}}   & \multicolumn{1}{C|}{\cellcolor[HTML]{EFEFEF}100\%}  & \multicolumn{1}{l|}{\cellcolor[HTML]{EFEFEF}45.0\%} & \multicolumn{1}{C|}{100\%}  & \multicolumn{1}{C|}{60.0\%} & \multicolumn{1}{l|}{98.5\%}           & 67.6\%           &                             &                             \\ \cline{3-8}
\multicolumn{2}{|C|}{}                                     &  \multicolumn{1}{C|}{\cellcolor[HTML]{EFEFEF}40.6\%} & \multicolumn{1}{C|}{\cellcolor[HTML]{EFEFEF}35}  & \multicolumn{1}{C|}{51.1\%} & \multicolumn{1}{C|}{9}      & \multicolumn{1}{C|}{62.4\%}           & 11               &                             &                             \\ \cline{1-8}
\multicolumn{2}{|C|}{\multirow{2}{*}{$\mathrm{AR}_{100}$}} & \multicolumn{1}{C|}{90.4\%} & \multicolumn{1}{l|}{57.3\%} & \multicolumn{1}{C|}{76.6\%} & \multicolumn{1}{C|}{73.6\%} & \multicolumn{1}{C|}{78.5\%}           & 76.4\%           &                             &                             \\ \cline{3-8}
\multicolumn{2}{|C|}{}                                     & \multicolumn{1}{C|}{50.2\%} & \multicolumn{1}{C|}{15}     & \multicolumn{1}{C|}{72.4\%} & \multicolumn{1}{C|}{11}     & \multicolumn{1}{C|}{73.5\%}           & 9                &                             &                             \\ \cline{1-8}
\multicolumn{2}{|C|}{\multirow{2}{*}{$\mathrm{AR}_{200}$}} & \multicolumn{1}{C|}{83.1\%} & \multicolumn{1}{C|}{67.7\%} &\multicolumn{1}{C|}{\textbf{82.0\%}} & \multicolumn{1}{C|}{\textbf{77.1\%}} & \multicolumn{1}{l|}{82.6\%}           & 77.0\%           &                             &                             \\ \cline{3-8}
\multicolumn{2}{|C|}{}                                     & \multicolumn{1}{C|}{64.0\%} & \multicolumn{1}{C|}{11}     & \multicolumn{1}{C|}{\textbf{75.5\%}} & \multicolumn{1}{C|}{\textbf{8}}   & \multicolumn{1}{C|}{74.4\%}           & 8                &                             &                             \\ \hline
\multicolumn{2}{|C|}{\multirow{2}{*}{$\mathrm{AR}_{500}$}} &                             &                             &                             &                             &                                       &                  & \multicolumn{1}{C|}{83.6\%} & \multicolumn{1}{C|}{73.4\%} \\ \cline{9-10} 
\multicolumn{2}{|C|}{}                                     &                             &                             &                             &                             &                                       &                  & \multicolumn{1}{C|}{68.2\%} & \multicolumn{1}{C|}{11}     \\ \cline{1-2} \cline{9-10}  
\end{tabular}


\end{table}
According to our results in Table~\ref{table:results_all}, we selected the model issued using the $\mathrm{SH}_{200}\mathrm{AR}_{40}$ technique for the \emph{DS\_low\_jitter} dataset and the one issued using the $\mathrm{SH}_{200}\mathrm{AR}_{20}$ technique for the \emph{DS\_higher\_jitter}. In Fig.~\ref{fig:compareTA} we compare their performances with those of a Gaussian TA combined with a realignment technique. To tune this comparison, several state-of-the-art  Gaussian TA have been tested. Since in the experiment the leakage is concentrated in peaks that are easily detected by their relatively high amplitude, we use as realignment technique a simple method that consists in first detecting the peaks above a chosen threshold, then keeping all the samples in a window around these peaks. Then, for the selection of the PoIs, two approaches have been applied: first we selected from $3$ to $20$ points maximising the estimated instantaneous SNR, secondly we selected sliding windows of 3 to 20 consecutive points covering the region of interest. For the template processing, we tried (1) the classical approach \cite{Chari2003} where a mean and a covariance matrix are estimated for each class, (2) the \emph{pooled} covariance matrix strategy proposed in \cite{choudary2014efficient} and (3) the stochastic approach proposed in \cite{schindler2005stochastic}. The results plotted in Fig.~\ref{fig:compareTA} are the best ones we obtained (via the stochastic approach over some 5-sized windows). Results show that the performances of the CNN approach are much higher than those of the Gaussian templates, both with and without realignment. This confirms the robustness of the CNN approach with respect to the jitter effect:
the selection of PoIs and the realignment integrated in the training phase are effective.



\section{Experiments against Real-Case Hardware Countermeasures}\label{sec:AES}
As a last (but most challenging) experiment we deployed our CNN architecture to attack an AES hardware implementation over a modern secure smartcard (secure implementation on 90nm technology node). On this implementation, the architecture is designed to optimise the area, and the speed performances are not the major concern. The architecture is here minimal, implementing only one hardware instance of the SubByte module.  The AES SubByte operation is thus executed serially and one byte is processed per clock cycle. To protect the implementation, several countermeasures are implemented.  Among them, a hardware mechanism induces a strong jitter effect which produces an important traces' desynchronisation. The bench is set up to trig the acquisition of the trace on a peak which corresponds to the processing of the first byte. Consequently, the set of traces is aligned according to the processing of the first byte while the other bytes leakages are completely misaligned. To illustrate the effect of this misalignment, the SNR characterising the (aligned) first byte and the (misaligned) second byte are computed (according to \eqref{eq:SNR_formula}) using a set of $150,000$ traces labelled by the value of the SubByte output (256 labels). These SNRs are depicted in the top part of Fig.~\ref{fig:SNR}. The SNR of the first byte (in green) detects a quite high leakage, while the SNR of the second byte (in blue) is nullified. A zoom of the SNR of the second peak is proposed in the bottom part of Fig.~\ref{fig:SNR}. In order to confirm that the very low SNR corresponding to the second byte is only due to the desynchronisation, the patterns of the traces corresponding to the second byte have been resynchronised using a peak-detection-based algorithm, quite similar to the one applied for the experiments of Sec.~\ref{sec:artificial}. Then the SNR has been computed onto these new aligned traces and has been plot in red in the top-left part of Fig.~\ref{fig:SNR}; this SNR is very similar to that of the first byte. This clearly shows that (1) the leakage information is contained into the trace but is efficiently hidden by the jitter-based countermeasure, and that (2) the realignment technique we applied in this context is effective.

We applied the CNN approach onto the rough set of traces (without any alignement). First, a $2,500$-long window of the trace has been selected to input CNN. The window, identified by the vertical cursors in the bottom part of Fig.~\ref{fig:SNR}, has been selected to ensure that the pattern corresponding to the leakage of the second byte is inside the selection. At this step, it is important to notice that such a selection is not at all  as meticulous as the selection of PoIs required by a classical TA approach. The training phase has been performed using $98,000$ labelled traces; $1,000$ further traces have been used for the validation set. We performed the training phase over a desktop computer equipped with an Intel Xeon E5440 @2,83GHz processor, 24Gb of RAM and a GeForce GTS 450 GPU. Without data augmentation each epoch took about 200s.\footnote{raising to about $2,000$ seconds when $SH_{20}DA_{200}$ data augmentation is performed (data are augmented online during training)} The training stopped after 25 epochs. Considering that in this case we applied an early-stopping strategy that stopped training after 20 epochs without validation loss decrement, it means that the final trainable weights are obtained after 5 epochs (in about 15 minutes). The results that we obtained are summarized in Table~\ref{tab:res_AES}. They prove not only that our CNN is robust to the misalignment caused by the jitter but also that the DA technique is effective in raising its efficiency. A comparison between the CNN performances and the best results we obtained over the same dataset applying the realignment-TA strategy, is proposed in Fig.~\ref{fig:TA_smartcard}. Beyond the fact that the CNN approach slightly outperforms the realignment-TA one, and considering that both case-results shown here are surely non-optimal, what is remarkable is that the CNN approach is potentially suitable even in cases where realignment methods are impracticable or not satisfying. It is of particular interest in cases where sensitive information does not lie in proximity of peaks or of easily detectable patterns, since many resynchronisation techniques are based on pattern or peak detection. If the resynchronisation fails, the TA approach falls out of service, while the CNN one remains a further weapon in the hands of an attacker.

\begin{figure}
    \centering
    \includegraphics[width=\textwidth]{../Figures/CHES2017/snrs.png} 
     \caption[SNR values for an AES hardware implementation protected by jitter-based misalignment.]{AES hardware implementation protected by jitter-based misalignment. In green the SNR for the first byte; in blue the SNR for the second byte; in red the SNR for the second byte after a trace realignment.}\label{fig:SNR}
\end{figure}


\begin{figure}
    \centering
    \includegraphics[width=\textwidth]{../Figures/CHES2017/TA_CNN_smartcard.pdf} 
     \caption{Comparison between a Gaussian template attack with realignment, and our CNN strategy, over the modern smart card with jitter.}\label{fig:TA_smartcard}
\end{figure}

%
%    \includegraphics[width=.45\textwidth]{../Figures/CHES2017/TA_CNN_smartcard.pdf} 
%    \captionof{figure}{Top Left: in green the SNR for the first byte; in blue the SNR for the second byte; in red the SNR for the second byte after a trace realignment. Bottom Left: a zoom of the blue SNR trace. Right: comparison between a Gaussian template attack with realignment, and our CNN strategy, over the modern smart card with jitter.}\label{fig:SNR}

    

\begin{table}
\centering
\begin{tabular}{|c|c|c|c|c|c|c|c|}
\multicolumn{8}{c}{}\\
\hline
\multicolumn{2}{|c|}{} & \multicolumn{2}{c|}{$\mathrm{SH}_{0}\mathrm{AR}_{0}$} & \multicolumn{2}{c|}{$\mathrm{SH}_{10}\mathrm{AR}_{100}$} & \multicolumn{2}{c|}{$\mathrm{SH}_{20}\mathrm{AR}_{200}$} \\ \hline
$a$        & $b$       & 35.0\%                     & 1.1\%                    & 12.5\%                      & 1.5\%                      & \textbf{10.4\%}             & \textbf{2.2\%}             \\ \hline
$c$        & $d$       & 1.2\%                      & 137                      & 1.3\%                       & 89                         & \textbf{1.8\%}              & \textbf{54}                \\ \hline
\end{tabular}

\caption{Results of our CNN over the modern smart card with jitter.}\label{tab:res_AES}
\end{table}

  
\section{Conclusion}
In this chapter, we have proposed an end-to-end profiling attack approach, based on the CNNs. We claimed that such a strategy would keep effective even in presence of trace misalignment, and we successfully verified our claim by performing CNN-based attacks against different kinds of misaligned data. This property represents a great practical advantage compared to the state-of-the-art Template Attacks, that require a meticulous trace realignment in order to be efficient. Our strategy based over CNNs differs from classical TA for mainly two points. First, it makes use of a discriminative model, instead of a generative one. Second it takes in charge into a unique training phase all eventual preprocessing phases necessary for the successfulness of a TA. Indeed, beyond the trace realignment, that is not necessary for the CNN approach, it represents as well a solution to the problem of the selection of points of interest issue: CNNs efficiently manage high-dimensional data, allowing the attacker to simply  select large windows. In this sense, the experiments described in Sec.~\ref{sec:AES} are very representative: our CNN retrieves information from a large window of points instead of an almost null instantaneous SNR. To guarantee the robustness to trace misalignment, we used a quite complex architecture for our CNN, and we clearly identified the risk of overfitting phenomenon. To deal with this classical issue in machine learning, we proposed two Data Augmentation techniques adapted to misaligned side-channel traces. All the experimental results we obtained have proved that they provide a great benefit to the CNN strategy.  Attacks proposed in this chapter are performed against non-masked implementation. Nevertheless, since NNs are in general non-linear models, they naturally well-fit also the higher-order attack context, as discussed in \cite{maghrebi2016breaking} and \cite{DLwhitepaper}.

% Chapter Template

\chapter{Conclusions and Perspectives} % Main chapter title

\label{ChapterConclusions}

% use bayesian inference with 'key' as parameter (usa la chiave esplicitamente come parametro e ogni volta che aggiungi una traccia, aggiorna le distribuzioni a priori della variabile sensibile per calcolare la likelihood...)

%----------------------------------------------------------------------------------------
%	SECTION 1
%----------------------------------------------------------------------------------------

\section{Summary}

%----------------------------------------------------------------------------------------
%	SECTION 2
%----------------------------------------------------------------------------------------

\section{Strengthen Embedded Security: the Main Challenge for Machine Learning Applications}

%----------------------------------------------------------------------------------------
%	THESIS CONTENT - APPENDICES
%----------------------------------------------------------------------------------------

\appendix % Cue to tell LaTeX that the following "chapters" are Appendices


%\chapter*{Scenario 3 and 4 of CARDIS '15 paper}\label{Appendix_scenario3_4_cardis2015}

\subsubsection{Scenario 3.}
Let  $\newTraceLength$ be now variable and let the other parameters be fixed as follows: $\nbAttackTraces = 100, N_z=200, \numPoI = 3996$. Looking at Fig.~\ref{fig:3}, we might observe that the standard PCA might actually well perform in SCA context if provided with a larger number of kept components; on the contrary, a little number of components suffices to the LDA. Finally, keeping more of the necessary components does not worsen the efficiency of the attack, which allows the attacker to choose $\newTraceLength$ as the maximum value supported by his computational means.

\begin{remark}
In our experiments the ELV selection method only slightly outperforms the IPR. Nevertheless, since it relies on more sound and more general observations, {\em i.e.} the maximization of explained variance concentrated into few points, it is likely to be more robust and less case-specific. For example, in Fig.~\ref{fig:notSSS} it can be remarked that while the class-oriented PCA + ELV line keeps constant on the value 0 of guessing entropy, the class-oriented PCA + IPR is sometimes higher than 0.
\end{remark}

\todo{Is the table with results overview interesting?}
%
%\medskip
%
%\medskip
%
%  \begin{minipage}[c]{\textwidth}
%  \hspace*{-3mm}
%  \begin{minipage}[c]{0.45\textwidth}
%    \centering
%    \includegraphics[width=\textwidth]{figures/Criterion3.pdf}
%    \captionof{figure}{Guessing Entropy as function of the number of the traces size after reduction}\label{fig:3}
%  \end{minipage}
%\hspace{1mm}
%  \begin{minipage}[c]{0.45\textwidth}
%    \centering
%    \begin{tiny}
%\begin{tabular}{|c|c|c|c|c|c|}
%\hline
%&&\multicolumn{4}{|>{\columncolor[gray]{0.7}}c|}{Parameter to minimize}\\
%\hline
%\multicolumn{1}{|>{\columncolor[gray]{0.7}}c|}{Method}&\multicolumn{1}{|>{\columncolor[gray]{0.7}}c|}{Selection}& $N$ &  $N'$ (SSS) &  $N'$ ($\neg$SSS) &  $C$ \\
%\hline
%PCA standard & EGV & {\bf -} &  &{\bf -} &{\bf -} \\
%\hline
%PCA standard &\multicolumn{1}{|>{\columncolor[gray]{0.8}}c|}{ELV} & \multicolumn{1}{|>{\columncolor[gray]{0.9}}c|}{{\bf -}} & &\multicolumn{1}{|>{\columncolor[gray]{0.9}}c|}{{\bf -}} &\multicolumn{1}{|>{\columncolor[gray]{0.9}}c|}{{\bf -}} \\
%\hline
%PCA standard & IPR &{\bf -} & &{\bf -} &{\bf +} \\
%\hline
%PCA class & EGV & {\bf -} &{\bf -} &{\bf -} &{\bf -} \\
%\hline
%PCA class & \multicolumn{1}{|>{\columncolor[gray]{0.8}}c|}{ELV} &\multicolumn{1}{|>{\columncolor[gray]{0.9}}c|}{{\bf +}} &\multicolumn{1}{|>{\columncolor[gray]{0.9}}c|}{$\bigstar$}&\multicolumn{1}{|>{\columncolor[gray]{0.9}}c|}{$\bigstar$} &\multicolumn{1}{|>{\columncolor[gray]{0.9}}c|}{{\bf +}} \\
%\hline 
%PCA class & IPR & {\bf {\bf +}} &$\bigstar$ &{\bf +} &{\bf -} \\
%\hline 
%LDA & EGV &$\bigstar$ & & {\bf +} & $\bigstar$\\
%\hline 
%LDA & \multicolumn{1}{|>{\columncolor[gray]{0.8}}c|}{ELV} & \multicolumn{1}{|>{\columncolor[gray]{0.9}}c|}{{\bf +}} &  & \multicolumn{1}{|>{\columncolor[gray]{0.9}}c|}{{\bf +}} & \multicolumn{1}{|>{\columncolor[gray]{0.9}}c|}{$\bigstar$}\\
%\hline 
%LDA & IPR & {\bf +} & &{\bf +} & $\bigstar$ \\
%
%\hline 
%\multicolumn{1}{|>{\columncolor[gray]{0.8}}c|}{$\SW$ Null Space}  & EGV & &\multicolumn{1}{|>{\columncolor[gray]{0.9}}c|}{$\bigstar$ } & & \\
%\hline 
%\multicolumn{1}{|>{\columncolor[gray]{0.8}}c|}{$\SW$ Null Space}  & IPR & &\multicolumn{1}{|>{\columncolor[gray]{0.9}}c|}{{\bf +}} & & \\
%\hline 
%\multicolumn{1}{|>{\columncolor[gray]{0.8}}c|}{Direct LDA} & EGV & & \multicolumn{1}{|>{\columncolor[gray]{0.9}}c|}{$\bigstar$}& & \\
%\hline 
%\multicolumn{1}{|>{\columncolor[gray]{0.8}}c|}{Direct LDA} & IPR & &\multicolumn{1}{|>{\columncolor[gray]{0.9}}c|}{{\bf +}}& & \\
%\hline
%\multicolumn{2}{|>{\columncolor[gray]{0.8}}c|}{Fisherface} & &\multicolumn{1}{|>{\columncolor[gray]{0.9}}c|}{{\bf -}} & & \\
%\hline 
%\multicolumn{2}{|>{\columncolor[gray]{0.8}}c|}{$\ST$ Spanned Space}  & &\multicolumn{1}{|>{\columncolor[gray]{0.9}}c|}{{\bf -}} & & \\
%\hline
%\end{tabular}
%\end{tiny}
%\captionof{table}{Overview of extractors performances in tested situations.}\label{table:results}
%    \end{minipage}
%  \end{minipage}
%
%\begin{figure}
%\includegraphics[width=0.5\textwidth]{figures/Criterion4.pdf}
%\includegraphics[width=0.5\textwidth]{figures/Criterion4cutted.pdf} 
%\caption{Guessing Entropy as function of the number of time samples contributing to the extractor computation.}\label{fig:4}
%\end{figure}
%
%An overview of the results of our comparison in scenarios 1, 2 and 3 is depicted in Table~\ref{table:results}: depending on the adversary purpose, given by the parameter to minimize, a $\bigstar$ denotes the best method, a ${\bf +}$ denotes a method with performances close to those of the best one and a ${\bf -}$ is for methods that show lower performances. Techniques introduced in this paper are highlighted by a grey background.  For example we remark that the class-oriented PCA takes advantage of the association with our ELV selection of components, achieving optimal performances when the goal is to minimize the number of profiling traces $\numTraces[]'$. As expected, when there are no constraints over $\numTraces[]'$, the LDA outperforms the other methods; however, even in this case which is very favourable to the LDA, the class-oriented PCA equipped with the ELV selection has an efficiency which is close to that of the LDA.
%


\subsubsection{Scenario 4.}


This is the single scenario in which we allow the ELV selection method to not only select the components to keep but also to modify them, keeping only some coefficients within each component, setting the other ones to zero. We select pairs \textit{(component, time sample)} in decreasing order of the ELV values, allowing the presence of only $\newTraceLength = 3$ components and $\numPoI$ different times samples: {\em i.e.}, we impose that the matrix $A$ defining the extractor (see \eqref{eq:linearExtractor}) has $\newTraceLength = 3$ rows (storing the 3 chosen components, transposed) and exactly $\numPoI$ non-zero columns.
Looking at Fig.~\ref{fig:4} we might observe that the LDA allows to achieve the maximal guessing entropy with only 1 PoI in each of the 3 selected components. 
Actually, adding PoIs worsen its performances, which is coherent with the assumption that the vulnerable information leaks in only a few points. Such points are excellently detected by the LDA. Adding contribution from other points raises the noise, which is then compensated by the contributions of further noisy points, in a very delicate balance. Such a behaviour is clearly visible in standard PCA case: the first 10 points considered raise the level of noise, that is then balanced by the last 1000 points.


\chapter{Cross-Validation} 

\label{app:cross-validation} 

In the Machine Learning community, several evaluation frameworks are commonly
applied to assess the performances of a model or to select the best hyper-parameters
for a learning algorithm. These methods aim to provide an
estimator of the performance which does not
depend on the choice of the training set $\setDataTrain$ (on which the model is trained) and of the
test set $\setDataTest$ (on which the model is tested) but only on their size.

The so-called \emph{$t$-fold
cross-validation} \cite{friedman2001elements} is currently the preferred evaluation method. Let P be a
performance metric, $\hat{f}$ a model to evaluate, and $\setDataTrain=(\setLeak, \setTarget)$ a labelled dataset, the outline of the method is the
following: 
\begin{enumerate}
\item ~[optional] randomize the order of the labelled traces in $\setDataTrain$, 
\item ~split the samples and their corresponding labels into $t$ disjoint parts
of equal size $(\setLeak_1,\setTarget_1),\ldots,(\setLeak_t,\setTarget_t)$.
For each $i\in [1..t]$, do:
\begin{enumerate}
\item set $\setDataValidation \doteq (\setLeak_i, \setTarget_i)$ and
$\setDataTrain \doteq (\bigcup_{j\neq i} \setLeak_j, \bigcup_{j\neq
i} \setTarget_j)$,
\item (re-)train\footnote{The model is trained from scratch at each iteration of the loop over $t$.} the model $\hat{f}$ on $\setDataTrain$, 
\item compute the performance metric by evaluating the model on $\setDataValidation$:
\begin{equation}\label{equ:perMetric}
\text{P}_i = \text{P}(\hat{f}, \setDataValidation) \enspace ,
\end{equation}
\end{enumerate}
\item ~return the mean $\frac{1}{t}\sum_{i=1}^t \text{P}_i$.
\end{enumerate}

It is known that the $t$-fold cross-validation estimator is an unbiased
estimator of the generalization performance. Its main drawback is its variance
which may be large and difficult to estimate
\cite{breiman1996heuristics,bengio2005bias}. 
\chapter{Artificially Simulate Jitter}\label{appendix:artificial_jitter}

%\lstset{language=Python}

In order to analyse the behaviour of the techniques studied in this thesis over misaligned side-channel traces, we simulated sometimes a jitter effect to misalign some well-synchronized traces in a controlled way. When jittering is present, the clock stability is altered and clock cycles are sampled by a varying number of time samples. To simulate such effect, the windows containing clock patterns of an acquisition are selected one by one and passed as input to the following function, described in python code, in charge to enlarge or reduce them in a random way. The randomness depends on two parameters \texttt{sigma} and \texttt{B}, being the number of inserted or removed points be almost normally distributed, with standard deviation given by \texttt{sigma}, but bounded. The bound is controlled by \texttt{B} by the following rule: the final size of a window has to be at least $\frac{1}{\text{\texttt{B}}}$ times the original size and at most \texttt{B} times the original size. The value assigned to newly inserted points is the linear interpolation of the previous and the following points.

\begin{lstlisting}[frame=single]
def enlarge_reduce_window(window,sigma,B): 
    Npts = window.shape[0]
    new_window = np.copy(window)
    deltaPts = int(np.floor(np.random.randn(1)[0]*sigma))
    if (deltaPts >= 0):
        deltaPts = min(Npts*(B-1),deltaPts)
        for i  in range(deltaPts):
            curr_size = new_window.shape[0]
            pos = int(np.floor(np.random.rand(1)*curr_size))
            if pos==0 or pos==curr_size-1:
                new_window = np.insert(new_window,
                 pos,new_window[pos])
            else:
                new_window = np.insert(new_window,pos, 
                (new_window[pos-1]+
                new_window[pos])/2.0)
    else:
        deltaPts = max(-Npts*(1-1/B),deltaPts)
        for i in range(-deltaPts):
            curr_size = new_window.shape[0]
            pos = int(np.floor(np.random.rand(1)*curr_size))
            new_window = np.delete(new_window,pos)
    return new_window
    
\end{lstlisting}

This deformation is applied to each clock pattern independently. We remark that is implies that, for example, The 19th clock cycle of a deformed acquisition suffers from the cumulation of the previous 18 deformations. For the sake of visualizing the effect of such a jitter simulation, in Fig.~\ref{fig:jitter_traces} we depict some traces of  \emph{DS\_low\_jitter} \ref{fig:jitter_traces22} and of the \emph{DS\_high\_jitter} \ref{fig:jitter_traces66} datasets, used for experiments in Sec~.\ref{sec:hard}. They are obtained by perfectly synchronous acquisitions, with parameters set to \texttt{sigma} $= 2$, \texttt{B}$= 2$ for the \emph{DS\_low\_jitter}  dataset and \texttt{sigma} $= 6$, \texttt{B}$= 6$ for the \emph{DS\_high\_jitter} one.


\begin{figure}
\centering
\subfigure[]{\label{fig:jitter_traces22}
\includegraphics[width=\textwidth]{../Figures/CHES2017/jitter_2_2_framed.png} }
\subfigure[]{\label{fig:jitter_traces66}
\includegraphics[width=\textwidth]{../Figures/CHES2017/jitter_6_6_framed.png} }
\caption[Hardware misalignment: \emph{DS\_low\_jitter} and \emph{DS\_high\_jitter} datasets.]{\subref{fig:jitter_traces22}  some traces of the \emph{DS\_low\_jitter} dataset, a zoom of the part highlighted by the red rectangle is given in the bottom part. \subref{fig:jitter_traces22} some traces of the \emph{DS\_high\_jitter} dataset. The interesting clock cycle is highlighted by the grey rectangular area.}\label{fig:jitter_traces}
\end{figure}


\chapter{Kernel PCA construction}\label{app:KPCA}

Suppose that we want to perform PCA in the image space of a function $\Phi$ that is associated to a given kernel function $K$. The kernel version for PCA has been presented in \cite{scholkopf1998nonlinear}; as we said in Chapter~\ref{ChapterKernel}, the important step consists in expressing the operations needed for the PCA procedure in terms of the dot products between the mapped data.\\

Let us assume that data are centered in the feature space, {\em i.e.} $\sum_{i=1,\dots,\nbProfilingTraces}\Phi(\vLeakVec_{i})=0$.\footnote{Such a condition is not hard to achieve, even without explicitly pass through the feature space: it suffices substituting the kernel matrix $\kernelMatrix$ by the matrix $\tilde{\kernelMatrix} = \kernelMatrix - \boldsymbol{1}_{\nbProfilingTraces}\kernelMatrix - \kernelMatrix\boldsymbol{1}_{\nbProfilingTraces} + \boldsymbol{1}_{\nbProfilingTraces}\kernelMatrix\boldsymbol{1}_{\nbProfilingTraces}$, where $\boldsymbol{1}_{\nbProfilingTraces}$ denotes the matrix with each entry equal to $\frac{1}{\nbProfilingTraces}$. The same kind of matrix has to be computed in projecting phase, using the test data.} In this way the empirical covariance matrix $\covmat^\Phi$ of data in the feature space is given by:
\begin{equation}
\covmat^\Phi = \frac{1}{\nbProfilingTraces} \sum_{i=1}^{\nbProfilingTraces}\Phi(\vLeakVec_{i})\Phi(\vLeakVec_{i})^\intercal \mbox{ .}
\end{equation} 
We want to find eigenvalues $\lambda^\Phi \neq 0$ and eigenvectors $\AAlpha^\Phi\in \mathcal{F}\smallsetminus \{\boldsymbol{0}\}$ such that
\begin{equation}\label{eq:eigProblem}
\covmat^\Phi\AAlpha^\Phi = \lambda^\Phi \AAlpha^\Phi \mbox{ .}
\end{equation}
We remark that such an eigenvector satisfies
\begin{align}
\AAlpha^\Phi &= \frac{1}{\lambda^\Phi\nbProfilingTraces}\sum_{i=1}^{\nbProfilingTraces} \Phi(\vLeakVec_{i})\Phi(\vLeakVec_{i})^\intercal \AAlpha^\Phi\\
\label{eq:passaggio}
&=  \frac{1}{\lambda^\Phi\nbProfilingTraces}\sum_{i=1}^{\nbProfilingTraces} \left[\Phi(\vLeakVec_{i})^\intercal \AAlpha^\Phi\right] \Phi(\vLeakVec_{i}) =  \\
&= \sum_{i=1}^{\nbProfilingTraces}\underbrace{\frac{\Phi(\vLeakVec_{i})^\intercal \AAlpha^\Phi}{\lambda^\Phi\nbProfilingTraces}}_{\nu_i}\Phi(\vLeakVec_{i}) = \\
\label{eq:linearComb}
&= \sum_{i=1}^{\nbProfilingTraces}\nu_i \Phi(\vLeakVec_{i})\mbox{ ,}
\end{align}
where the step \eqref{eq:passaggio} makes use of the associativity of the matrix product and the commutativity of the scalar-matrix product. Eq.~\eqref{eq:linearComb} tells us that each eigenvector $\AAlpha^\Phi$ is expressible as a linear combination of the data mapped into the feature space $(\Phi(\vLeakVec_{i})_{i=1,\dots,\nbProfilingTraces}$, or equivalently each eigenvector $\AAlpha^\Phi$ lies in the span of $(\Phi(\vLeakVec_{i})_{i=1,\dots,\nbProfilingTraces})$. This observation authorizes to substitute to the problem \eqref{eq:eigProblem}, the following equivalent system:
\begin{equation}\label{eq:system}
\begin{cases}
\lambda^\Phi(\Phi(\vLeakVec_{1})\cdot  \AAlpha^\Phi) = \Phi(\vLeakVec_{1})\cdot \covmat^\Phi \AAlpha^\Phi \\
\vdots\\
\lambda^\Phi(\Phi(\vLeakVec_{\nbProfilingTraces})\cdot  \AAlpha^\Phi) = \Phi(\vLeakVec_{\nbProfilingTraces})\cdot \covmat^\Phi \AAlpha^\Phi
\end{cases}
\end{equation}


Joining \eqref{eq:linearComb} and \eqref{eq:system} we obtain, looking to the first equation of the system:
\begin{align}
\lambda^\Phi(\Phi(\vLeakVec_{1})\cdot \sum_{i=1}^{\nbProfilingTraces}\nu_i\Phi(\vLeakVec_{i})) &= \Phi(\vLeakVec_{1})\cdot\left[ \frac{1}{N}\sum_{i=1}^{\nbProfilingTraces}\Phi(\vLeakVec_{i})\Phi(\vLeakVec_{i})^\intercal (\sum_{i=1}^{\nbProfilingTraces}\nu_i \Phi(\vLeakVec_{i}))\right]\\
\lambda^\Phi \sum_{i=1}^{\nbProfilingTraces}\nu_i(\Phi(\vLeakVec_{1})\cdot \Phi(\vLeakVec_{i})) &=\Phi(\vLeakVec_{1})\cdot\left[ \sum_{j=1}^{\nbProfilingTraces}\frac{\nu_j}{N}\left(\sum_{i=1}^{\nbProfilingTraces}\Phi(\vLeakVec_{i})\Phi(\vLeakVec_{i})^\intercal\right)  \Phi(\vLeakVec_{j})\right]\\
\lambda^\Phi \sum_{i=1}^{\nbProfilingTraces}\nu_i(\Phi(\vLeakVec_{1})\cdot \Phi(\vLeakVec_{i})) &=\Phi(\vLeakVec_{1})\cdot\left[ \sum_{j=1}^{\nbProfilingTraces}\frac{\nu_j}{N}\sum_{i=1}^{\nbProfilingTraces}\underbrace{\Phi(\vLeakVec_{i})^\intercal \Phi(\vLeakVec_{j})}_{\Phi(\vLeakVec_{i}) \cdot \Phi(\vLeakVec_{j})}\Phi(\vLeakVec_{i})\right]\\
\lambda^\Phi \sum_{i=1}^{\nbProfilingTraces}\nu_i(\Phi(\vLeakVec_{1})\cdot \Phi(\vLeakVec_{i})) &= \sum_{j=1}^{\nbProfilingTraces}\frac{\nu_j}{N}\left[\Phi(\vLeakVec_{1})\cdot\sum_{i=1}^{\nbProfilingTraces}(\Phi(\vLeakVec_{i}) \cdot \Phi(\vLeakVec_{j}))\Phi(\vLeakVec_{i})\right]\\
\nbProfilingTraces\lambda^\Phi \sum_{i=1}^{\nbProfilingTraces}\nu_i\underbrace{(\Phi(\vLeakVec_{1})\cdot \Phi(\vLeakVec_{i}))}_{\kernelMatrix[1,i]} &= \sum_{j=1}^{\nbProfilingTraces}\nu_j\left[\sum_{i=1}^{\nbProfilingTraces}\underbrace{(\Phi(\vLeakVec_{i}) \cdot \Phi(\vLeakVec_{j}))}_{\kernelMatrix[i,j]}\underbrace{(\Phi(\vLeakVec_{1})\cdot\Phi(\vLeakVec_{i}))}_{\kernelMatrix[1,j]}\right]\mbox{ .}
\end{align}
Thus, the system \eqref{eq:system} is equivalent to the follow:

\begin{equation}
\begin{cases}
\nbProfilingTraces\lambda^\Phi \sum_{i=1}^{\nbProfilingTraces}\nu_i{\kernelMatrix[1,i]} &= \sum_{j=1}^{\nbProfilingTraces}\nu_j\left[\sum_{i=1}^{\nbProfilingTraces}{\kernelMatrix[1,j]}{\kernelMatrix[i,j]}\right]\\
\vdots \\
\nbProfilingTraces\lambda^\Phi \sum_{i=1}^{\nbProfilingTraces}\nu_i{\kernelMatrix[\nbProfilingTraces,i]} &= \sum_{j=1}^{\nbProfilingTraces}\nu_j\left[\sum_{i=1}^{\nbProfilingTraces}{\kernelMatrix[\nbProfilingTraces,j]}{\kernelMatrix[i,j]}\right]
\end{cases}
\end{equation}

%\begin{remark}
%The kernel matrix $\kernelMatrix$ is symmetric by construction: $\kernelMatrix[i,j] = \kernelMatrix[j,i]$ for each pair $i,j$.
%\end{remark}

Let $\nununu$ be the column vector containing the coefficients $\nu_i$ of \eqref{eq:linearComb}. The above system is expressible in matricial form as

\begin{equation}
\begin{cases}
\nbProfilingTraces\lambda^\Phi [\kernelMatrix\nununu][1] &= [\kernelMatrix^2\nununu][1]\\
\vdots \\
\nbProfilingTraces\lambda^\Phi [\kernelMatrix\nununu][\nbProfilingTraces] &= [\kernelMatrix^2\nununu][\nbProfilingTraces]\mbox{ ,}
\end{cases}
\end{equation}

which equals the following equation:

\begin{equation}
\nbProfilingTraces\lambda^\Phi\kernelMatrix\nununu = \kernelMatrix^2\nununu \mbox{ .}
\end{equation}

It can be shown that solving the last equation is equivalent to solve the following eigenvector problem
\begin{equation}\label{eq:eigProblemKPCA}
\gamma\nununu = \kernelMatrix\nununu \mbox{ .}
\end{equation}

Let $\gamma_1\geq\gamma_2\geq\dots\geq\gamma_{\nbProfilingTraces}$ denote the eigenvalues of $\kernelMatrix$, $\gamma_C$ being the last different from zero, and $\nununu_1,\dots, \nununu_{\nbProfilingTraces}$ the corresponding eigenvectors. For the sake of obtaining the corresponding normalized principal components in the feature space $\mathcal{F}$, denoted $\AAlpha^\Phi_1,\dots, \AAlpha^\Phi_C$, a normalization step is required, imposing for all $k=1,\dots, C$

\begin{equation}
\AAlpha^\Phi_k\cdot\AAlpha^\Phi_k = 1 \mbox{ ,}
\end{equation}

which can be translated into a condition for $\nununu_1,\dots, \nununu_C$, using \eqref{eq:linearComb} and \eqref{eq:eigProblemKPCA}:
\begin{equation}
1 = \sum_{i,j=1}^{\nbProfilingTraces}\nununu_k[i]\nununu_k[j](\Phi(\vLeakVec_{i})\cdot \Phi(\vLeakVec_{j})) = \nununu_k \cdot \kernelMatrix\nununu_k = \gamma_k(\nununu_k\cdot \nununu_k)
\end{equation}


Extracting the non-linear principal components of a datum $\vLeakVec_{}$ means projecting its image $\Phi(\vLeakVec_{})$ onto the eigenvectors $\AAlpha^\Phi_1,\dots, \AAlpha^\Phi_C$ in $\mathcal{F}$. To do so, we neither need to explicitly compute $\Phi(\vLeakVec_{})$ nor $\AAlpha^\Phi_i$. Indeed, using \eqref{eq:linearComb}:

\begin{equation}
\AAlpha^\Phi_k \cdot \Phi(\vLeakVec_{}) = \sum_{i=1}^{\nbProfilingTraces}\nununu_k[i](\Phi(\vLeakVec_{i}) \cdot \Phi(\vLeakVec_{})) =  \sum_{i=1}^{\nbProfilingTraces}\nununu_k[i]K(\vLeakVec_{i}, \vLeakVec_{}) \mbox{ .}
\end{equation}



\section{Kernel class-oriented PCA}

Suppose now that we want to perform a class-oriented PCA in the image space of a function $\Phi$ that is associated to a given kernel function $K$, i.e. we want to solve, using a kernel trick, the eigenvalue problem


\begin{equation}\label{eq:eigProblemSB}
\SB^\Phi\AAlpha^\Phi = \lambda^\Phi \AAlpha^\Phi \mbox{ ,}
\end{equation}

where $\SB^\Phi$ is the between-scatter matrix in the feature space: 

\begin{equation}
\SB^\Phi = \sum_{\sensVarGenValue\in\sensVarSet}\nbTracesPerClass(\mmmXclassPhi-\mmmXPhi)(\mmmXclassPhi-\mmmXPhi)^\intercal \mbox{ .}
\end{equation}

Here $\mmmXclassPhi = \frac{1}{\nbTracesPerClass}\sum_{i=1\colon \sensVar_i=\sensVarGenValue}\Phi(\vLeakVec_{i})$ and $\mmmXPhi = \frac{1}{\nbProfilingTraces}\sum_{i=1}^{\nbProfilingTraces}\Phi(\vLeakVec_{i})$.\\

As before, the eigenvectors $\AAlpha^\Phi_i$ are expressible as linear combination of the data images on $\mathcal{F}$, i.e. \eqref{eq:linearComb} is still true:

\begin{equation}
\AAlpha^\Phi = \sum_{i=1}^{\nbProfilingTraces}\nu_i \Phi(\vLeakVec_{i}) \mbox{ .}
\end{equation}

Moreover as before, the eigenvector problem \eqref{eq:eigProblemSB} can be translated in an eigenvector problem that gives the coefficients $\nununu$ as solutions. That is:

\begin{equation}\label{eq:eigProblemM}
\gamma\MMM = \MMM \nununu \mbox{ ,}
\end{equation}

where the matrix $\MMM$ is computed as

\begin{equation}
\MMM = \sum_{\sensVarGenValue\in\sensVarSet}\nbTracesPerClass(\MMMclass - \MMMT)(\MMMclass-\MMMT)\mbox{ ,}
\end{equation}

with $\MMMclass$ and $\MMMT$ being two $N$-sized vectors whose entries are given by:
\begin{align}
\MMMclass[\sensVarGenValue][j] = \frac{1}{\nbTracesPerClass}\sum_{i:\sensVar_i=\sensVarGenValue}K(\vLeakVec_j,\vLeakVec_i)\\
\MMMT[j] = \frac{1}{\nbTrainingTraces}\sum_{i=1}^{\nbTrainingTraces}K(\vLeakVec_{j},\vLeakVec_{i}) \mbox{ .}
\end{align}

Finally, one the eigenvector $\nununu$ are found, to project a datum $\vLeakVec_{}$ onto the corresponding principal component in the feature space we proceed as in the previous case:

\begin{equation} \label{eq:projection}
\AAlpha^\Phi_k \cdot \Phi(\vLeakVec_{}) =  \sum_{i=1}^{\nbProfilingTraces}\nununu_k[i]K(\vLeakVec_{i}, \vLeakVec_{}) \mbox{ .}
\end{equation}


%\include{Appendices/AppendixC}

%----------------------------------------------------------------------------------------
%	BIBLIOGRAPHY
%----------------------------------------------------------------------------------------

\printbibliography[heading=bibintoc]

%----------------------------------------------------------------------------------------

\end{document}  
